%        File: MethodOfManfSolutionForSWIRL.tex
%     Created: Wed Jun 01 07:00 PM 2022 E
% Last Change: Wed Jun 01 07:00 PM 2022 E
%
\documentclass[a4paper]{report}
\usepackage{mathtools}
\usepackage{amsmath}
\begin{document}


\begin{titlepage}

    \title{ Applying the Method of Manufactured Solutions to SWIRL}
    

    \author{ Jeffrey Severino \\
        University of Toledo \\
        Toledo, OH  43606 \\
    email: jseveri@rockets.utoledo.edu}


    \maketitle

\end{titlepage}
\section{Introduction}
The Method of Manufactured Solutions (MMS) is a process for generating an 
analytical solution for a code that provides the numerical solution for a 
given domain. The goal of MMS is to establish a manufactured solution that can 
be used to establish the accuracy of the code within question. For this study, 
SWIRL, a code used to calculate the radial modes within an infinitely long duct
is being validated through code verification. SWIRL accepts a given mean flow and 
uses numerical integration to obtain the speed of sound. The integration technique
is found to be the composite trapezoidal rule through asymptotic error analysis.

For SWIRL, the absolute bare minimum requirement is to define the corresponding
flow components for the domain of interest. SWIRL assumes no flow in the radial 
direction, leaving only two other components, axial and tangential for a 3D 
cylindrical domain. Since SWIRL is also non dimensionalized, the mean flow 
components are defined using the Mach number. SWIRL uses the tangential mach number
to obtain the speed of sound using numerical integration. The speed of sound
is then used to find the rest of the primative variables for the given flow. 

\section{Methods}

SWIRL is a linearized Euler equations of motion code that calculates the 
axial wavenumber and radial mode shapes from small unsteady disturbances in a mean flow. 
The mean flow varies along the axial and tangetial directons as a function of 
radius. The flow domain can either be a circular or annular duct, with or without
acoustic liner. SWIRL was originally written by Kousen [insert ref].

The SWIRL code requires two mean flow parameters as a function of radius, $M_x$
, and $M_{\theta}$. Afterwards, the speed of sound, $\widetilde{A}$ is calculated by 
integrating $M_{\theta}$ with respect to r. To verify that SWIRL is handling 
and returning the accompanying mean flow parameters, the error between the 
mean flow input and output variables are computed. Since the trapezoidal rule
is used to numerically integrate $M_{\theta}$, the discretization error and 
order of accuracy is computed. Since finite differencing schemes are to be used 
on the result of this integration, it is crucial to accompany the integration 
with methods of equal or less order of accuracy. This will be determined by 
applying another MMS on the eigenproblem which will also have an order of 
accuracy.

\subsection{Theory}

To relate the speed of sound to a given flow, the radial momentum equation
is used.  If the flow contains a swirling component,then the primitive variables 
are nonuniform through the flow, and mean flow assumptions are not valid. 
\begin{align*}
    \frac{\partial v_r}{\partial t} +
    v_r \frac{\partial v_r}{\partial r} + 
    \frac{v_{\theta}}{r} 
    \frac{\partial v_{\theta}}{\partial \theta} -
    \frac{v_{\theta}^2}{r} 
    v_x \frac{\partial v_r}{\partial x} =
    \frac{1}{\rho}\frac{\partial P}{\partial r}
\end{align*}

 To account to for this, the radial momentum is simplified by assuming the 
 flow is steady, the flow has no radial component. In addition, the viscous
 and body forces are neglected.  Then the radial pressure derivative term is
 set equal to the dynamic pressure term. Seperation of variables is applied.  

 \begin{align*}
     \frac{v_{\theta}^2}{r} = \frac{1}{\rho}\frac{\partial P}{\partial r} \\
 P = \int_{r}^{r_{max}} \frac{\rho V_{\theta}^2}{  r}
 \end{align*}

To show the work, we will start with the dimensional form of the equation and
differentiate both sides.  Applying separation of variables,
 \[
     \int_{r}^{r_{max}}
     \frac{\bar{\rho} v_{\theta}^2}{r} \partial r 
     =-\int_{P(r)}^{P(r_{max})}\partial p.
 \]

Since $\tilde{r} = r/r_{max}$,
\[r = \tilde{r}r_{max}.\]
Taking total derivatives (i.e. applying chain rule),
\[dr = d(\tilde{r}r_{max}) = d(\tilde{r})r_{max}, \]
Substituting these back in and evaluating the right hand side,
\[
    \int_{\tilde{r}}^{1} \frac{\bar{\rho} v_{\theta}^2}{\tilde{r}}\partial \tilde{r} 
    =P(1)-P(\tilde{r})
\]

For reference the minimum value of $\tilde{r}$ is,

\[\sigma = \frac{r_{max}}{r_{min}}\]

For the radial derivative, the definition of the speed of sound is utilized,
\[\frac{\partial A^2}{\partial r } =
\frac{\partial}{\partial r} \left( \frac{\gamma P}{\rho} \right).\]

Using the quotient rule, the definition of the speed of sound is extracted,

\begin{align*}
&= \frac{\partial P}{\partial r} \frac{\gamma \bar{\rho}}{\bar{\rho}^2} -
\left(
    \frac{\gamma P}{\bar{\rho}^2} 
\right) 
\frac{\partial \bar{\rho}}{\partial r}\\
&=  \frac{\partial P}{\partial r} \frac{\gamma }{\bar{\rho}} -
\left( \frac{A^2}{\bar{\rho}} \right) 
\frac{\partial \bar{\rho} }{\partial r}
\end{align*}

Using isentropic condition $ \partial P/A^2 = \partial \rho$, 

\begin{align*}
&= \frac{\partial P}{\partial r} \frac{\gamma }{\bar{\rho}} -
\left( \frac{1}{\bar{\rho}} \right) \frac{\partial  P }{\partial r}\\
\frac{\partial A^2}{\partial r} 
&= \frac{\partial P}{\partial r} \frac{\gamma - 1}{\bar{\rho}}  
\end{align*}

\begin{align*}
    \frac{\bar{\rho}}{\gamma -1}\frac{\partial A^2}{\partial r} &= \frac{\partial P}{\partial r} 
\end{align*}


Going back to the radial momentum equation, and rearranging the terms will simplify 
the expression. The following terms are defined to start the
nondimensionalization.  

\begin{align*}
    M_{\theta} &= \frac{V_{\theta}}{A} \\ 
    \widetilde{r} &= \frac{r}{r_{max}}  \\
    \widetilde{A} &= \frac{A}{A_{r,max}}  \\
    A &= \widetilde{A}{A_{r,max}} \\
    r &= \widetilde{r}{r_{max}}\\
    \frac{\partial}{\partial r} &=
    \frac{\partial \widetilde{r}}{\partial r} \frac{\partial}{\partial \widetilde{r}}\\
                                &= \frac{1}{r_{max}}\frac{\partial}{\partial \widetilde{r}}
\end{align*}
Dividing by $A$,
\begin{align*}
    \frac{M_{\theta}^2}{r}\left(\gamma - 1\right) 
 &= \frac{\partial A^2}{\partial r} \frac{1}{A^2}
\end{align*}

Now there is two options, either find the derivative of  $\bar{A}$ or the integral of
$M_{\theta}$ with respect to r.
\begin{enumerate}
    \item
%\begin{align*}
%\text{Integrating both sides } 
%\int_{r}^{r_{max}}\frac{M_{\theta}}{r}\left(\gamma - 1\right){\partial r}  &=\int_{A^2(r)}^{A^2(r_{max})}\frac{1}{A^2}  {\partial A^2}\\
%\int_{r}^{r_{max}}\frac{M^2_{\theta}}{r}\left(\gamma - 1\right){\partial r}  &=ln(A^2(r_{max})) - ln(A^2(r)) \\
%\int_{r}^{r_{max}}\frac{M^2_{\theta}}{r}\left(\gamma - 1\right){\partial r}  &=ln\left(\frac{A^2(r_{max})}{A^2(r)}\right) 
%\end{align*}
%
Defining non dimensional speed of sound $\tilde{A} = \frac{A(r)}{A(r_{max})}$
\begin{align*}
\int_{r}^{r_{max}}\frac{M_{\theta}}{r}\left(\gamma - 1\right){\partial r}  &=ln\left(\frac{1}{\tilde{A}^2}\right) \\
&= -2ln(\tilde{A})\\
\tilde{A}(r) &= exp\left[-\int_{r}^{r_{max}}\frac{M_{\theta}}{r}\frac{\left(\gamma - 1\right)}{2}{\partial r}\right] \\ \text{replacing r with }\tilde{r} \rightarrow \tilde{A}(r) &= exp\left[-\int_{r}^{r_{max}}\frac{M_{\theta}}{r}\frac{\left(\gamma - 1\right)}{2}{\partial r}\right]		\\
\tilde{A}(\tilde{r}) &= exp\left[\left(\frac{1 - \gamma}{2}\right)\int_{\tilde{r}}^{1}\frac{M_{\theta}}{\tilde{r}}{\partial \tilde{r}}\right]	
\end{align*}
\item Or we can differentiate
\end{enumerate}
Solving for $M_{\theta}$ ,
\begin{align*}
M_{\theta}^2 
&= \frac{\partial A^2}{\partial r} \frac{r}{A^2 \left(\gamma - 1\right)}
\end{align*}
Nondimensionalizing and substituting, 

\begin{align} 
    M_{\theta}^2
    \frac{\left( \gamma - 1 \right)}{\widetilde{r} r_{max}} &=
    \frac{1}{(\widetilde{A}A_{r,max})^2}\frac{A_{r,max}^2}{r_{max}}
    \frac{\partial \widetilde{A}^2}{\partial \widetilde{r}} \nonumber \\
    M_{\theta}^2     \frac{\left( \gamma - 1 \right)}{\widetilde{r} } &=
    \frac{1}{\widetilde{A}^2}
    \frac{\partial \widetilde{A}^2}{\partial \widetilde{r}} \nonumber \\
    M_{\theta} &= \sqrt{\frac{\widetilde{r}}{(\gamma-1) \widetilde{A}^2}
        \frac{\partial\widetilde{A}^2}{\partial \widetilde{r} }
    } \label{eq:Mthetabackcalculated}
\end{align}

\subsection{Procedure}
There are a few constraints and conditions that must be followed in order for the analytical 
function to work with SWIRL, 

\begin{itemize}
    \item The mean flow and speed of sound must be real and positive. This will 
        occur is a speed of sound is chosen such that the tangential mach number
        is imaginary
    \item The derivative of the speed of sound must be positive
    \item Any bounding constants used with the mean flow should not allow the 
        total Mach number to exceed one.
    \item the speed of sound should be one at the outer radius of the cylinder
\end{itemize}

Given these constraints, $tanh(r)$ is chosen as a function since it can be
modified to meet the conditions above. Literature (The tanh method: A tool for 
solving certain classes of nonlinear evolution and wave equations) 
is a paper than demonstrates the strength of using tanh functions.
One additonal benefit of tanh(r) is that it is bounded between one and negative one, i.e.

\begin{itemize}
    \item As r $\rightarrow$ $\infty$ tanh(r) $\rightarrow$ 1
    \item As -r $\rightarrow$ $-\infty$ tanh(r) $\rightarrow$ -1
\end{itemize}

To test the numerical integration method,  $M_{\theta}$ is defined as a result 
of differentiating the speed of sound, $A$. This is done opposed to integrating
$M_{\theta}$ analytically. However, an analytical function can be defined for
$M_{\theta}$, which can then be integrated to find what $\widetilde{A}$ should be. 
Instead, the procedure of choice is to back calculate what the appropriate 
$M_{\theta}$ is for a given expression for $\widetilde{A}$.  Since it is easier 
to take derivatives , we will solve for $M_{\theta}$ using Equation \ref{eq:Mthetabackcalculated} ,

\subsection{Tanh Summaion Formulation}
The goal is generate an MS with a number of ``stairs'' that is bounded between
zero and one. Here's what my focus group ideas are,

\begin{align*}
    1 = R + L 
\end{align*}
where, 1 is a constraint, and R and L are the two waves when summed need to cancel 
if it were the exact same amplitude \& opposite sign 

so ,

\begin{align*}
    R + L = \tanh(x) + -\tanh(x) = 0
\end{align*}
or in our case,

\begin{align*}
    R + L = \tanh(x) + -\tanh(x) = 1
\end{align*}

We can tweak this by adding knobs by adding ``knobs'' A and B. If we dont want 
the total to not exceed one then, $A_j + A_{j+1} \cdots A_{last} = 1$. $B_1$ changes
the steepnes of the kink that we want. In order to generalize this,


\begin{align*}
    \bar{A} = \sum_{j=1}^n R_{ij} + \sum_{j=1}^n L_{ij} 
\end{align*}
where,
\begin{align*}
    R_{ij} = A_j \tanh(B_j (x_i - x_j)) \\ 
    L_{ij} = A_j \tanh(B_j (x_j - x_n))  
\end{align*}

Letting $n = 3$\ldots

\begin{align*}
    \bar{A} &= S_{vert} + \sum_{j=1}^3 R_{ij} + \sum_{j=1}^3 L_{ij} \\
    \bar{A} &=
    A_1 \tanh(B_1 (x_i - x_1))  + 
    A_{2} \tanh(B_{2} (x_i - x_{2}))  + 
    A_{3} \tanh(B_{3} (x_i - x_3)) +  \\
    A_1 \tanh(B_1 (x_1 - x_n)) &+ 
    A_{2} \tanh(B_{2} (x_2 - x_{n}))  +
    A_{3} \tanh(B_{3} (x_3 - x_n))  
\end{align*}
and,
\begin{align*}
    A_1 = A_2 = A_3 = k_1 \\ 
    B_1 = B_2 = B_3 = k_2  
\end{align*}

A tanh summation method was constructed to make a manufactured solution with 
strong changes in slope. This ensures that the numerical approximation will not 
give trivial answers. 
then for some functions we need to impose boundary conditions. We will demonstrate
how the careless implementation of a boundary condition can lead to close approximations
on the interior.  The speed of sound is defined with the subscript $analytic$ to indicate that this is the analytical function of choice and has no physical relevance 
to the actual problem.

\begin{align*}
\widetilde{A}_{analytic} = \Lambda + k_1 \tanh \left( k_2 \left( \widetilde{r} - \widetilde{r}_{max} \right) \right),
\end{align*}

where, 

\begin{align*}
    \Lambda = 1 - k_1 \tanh(k_2 (1 - \widetilde{r}_{max})),
\end{align*}

When, $\widetilde{r}=\widetilde{r}_{max}$ , $\widetilde{A}_{analytic} = 1$.  
Taking the derivative with respect to $\widetilde{r}$,

\begin{align*}
    \frac{\partial \widetilde{A}_{analytic} }{\partial \widetilde{r}} &=
    \left(1 - \tanh^{2}{\left(\left(r - r_{max}\right) {k}_{2} \right)}\right) {k}_{1} {k}_{2}, \\ 
    &= \frac{ k_{1} k_{2}}{\cosh^{2}{\left(\left(r - r_{max}\right) {k}_{2} \right)}}.
\end{align*}

Substitute this into the expression for $M_{\theta}$ in Equation 
\ref{eq:Mthetabackcalculated},

\begin{align*}
    M_{\theta} = \sqrt{2}
    \sqrt{\frac{r {k}_{1} {k}_{2}}{\left(\kappa - 1\right) \left(\tanh{\left(\left(r - r_{max}\right) {k}_{2} \right)} {k}_{1} + \tanh{\left(\left(r_{max} - 1\right) {k}_{2} \right)} {k}_{1} + 1\right) \cosh^{2}{\left(\left(r - r_{max}\right) {k}_{2} \right)}}}
\end{align*} 

Now that the mean flow is defined, the integration method used to obtain the 
speed of sound

% What happens when $r = r_{max}$?

Initially the source terms were defined without mention of the indices of the 
matrices they make up. In other words, there was no fore sight on the fact that
these source terms are sums of the elements within A,B, and X. To investigate 
the source terms in greater detail, the FORTRAN code that calls the source 
terms will output the terms within the source term and then sum them, instead 


of just their sum.
i
$ [A]{x} = \lambda [B] {x} $

which can be rearranged as,

$ [A]{x} - \lambda [B] {x} = 0$

Here, $x$ is an eigenvector composed of the perturbation variables, $v_r,v_{\theta},v_x,p$ and $\lambda$ is the associated eigenvalue, (Note: $\lambda = -i \bar{\gamma}$)

Writing this out we obtain .

Linear System of Equations:
\begin{equation}
    -
    i \left(
        \frac{k}{A} - \frac{m}{r} M_{\theta}
    \right)
    v_r 
    -
    \frac{2}{r} M_{\theta} v_{\theta} 
    +
    \frac{dp}{dr} 
    +
    \frac{(\kappa - 1)}{r} M_{\theta}^2 p
    -
    \lambda M_x v_r =S_1
\end{equation}

Using matrix notation,

\begin{equation}
    A_{11}
    x_1 
    -
    A_{12} x_2 
    +
    A_{14} x_4
    -
    \lambda B_{11} x_1 = S_1
\end{equation}


But $A_{14}$ and $A_{41}$ in Kousen's paper only has the derivative operator.
Since I am currntly writing the matrix out term by term and not doing the matrix 
math to obtain the symbolic expressions, I will define $A_{14}$ with $dp/dr$ 
and $A_{41}$ with $dv_r/dr$
Similarly,
\begin{align}
    A_{21} x_1 &-
    A_{22} x_2 +
    A_{24} x_4 &-
    \lambda B_{22} x_2 &= S_2 \\
    A_{31} x_1 &-
    A_{33} x_3 &-
    \lambda (B_{33} x_3 + B_{34} x_4) &= S_3\\
    A_{41} x_1 &+
    A_{42} x_2 +
    A_{44} x_4 &- 
    \lambda (B_{33} x_3 + B_{44} x_4) &= S_4
\end{align}
Now we can begin looking at the source terms, term by term. They each should also
converge at a known rate





\subsection{Calculation of Observed Order-of-Accuracy}
The numerical scheme used to perform the integration of the tangential velocity
will have a theoretical order-of-accuracy. To find the theoretical 
order-of-accuracy, the discretization error must first be defined. The error, 
$\epsilon$, is a function of id spacing, $\Delta r$

\[ \epsilon = \epsilon(\Delta r) \]

The discretization error in the solution should be proportional to 
$\left( \Delta r \right)^{\alpha}$ where $\alpha > 0$ is the theoretical order
for the computational method.  The error for each grid is expressed as 
\[ \epsilon_{M_{\theta}}(\Delta r) = |M_{\theta,analytic}-M_{\theta,calc}|\]
where $M_{\theta,analytic}$ is the tangential mach number that is defined from the
speed of sound we also defined and the $M_{\theta,calc}$ is the result from 
SWIRL. The $\Delta r$ is to indicate that this is a discretization error for a
specific grid spacing. Applying the same concept to to the speed of sound,

If we define this error on various grid sizes and compute $\epsilon$ for
each grid, the observed order of accuracy can be estimated and compared to
the theoretical order of accuracy. For instance, if the numerical soution
is second-order accurate and the error is converging to a value, the L2 norm of
the error will decrease by a factor of 4 for every halving of the grid cell 
size. 

Since the input variables should remain unchanged (except from minor changes 
from the Akima interpolation), the error for the axial and tangential mach 
number should be zero. As for the speed of sound, since we are using an analytic
expression for the tangential mach number, we know what the theoretical result
would be from the numerical integration technique as shown above. 
Similarly we define the discretization error for the speed of sound.

\[ \epsilon_{A}(\Delta r) = |A_{analytic}-A_{calc}|\]

For a perfect answer, we expect $\epsilon$ to be zero. Since a Taylor series can 
be used to derive the numerical schemes, we know that the truncation of higher
order terms is what indicates the error we expect from using a scheme that 
is constructed with such truncated Taylor series.

The error at each grid point $j$ is expected to satisfy the following,

\begin{align*}
    0 &= |A_{analytic}(r_j) - A_{calc}(r_j)| \\
    \widetilde{A}_{analytic}(r_j) &= \widetilde{A}_{calc}(r_j) +
    (\Delta r)^{\alpha} \beta(r_j)  + H.O.T
\end{align*}

where the value of $\beta(r_j)$ does not change with grid spacing, and 
$\alpha$ is the asymptotic order of accuracy of the method. It is important to
note that the numerical method recovers the original equations as the grid 
spacing approached zero.  It is important to note that $\beta$ represents the
first derivative of the Taylor Series.  Subtracting $A_{analytic}$ from both
sides gives,

\begin{align*}
    A_{calc}(r_j) - A_{analytic}(r_j) &= A_{analytic}(r_j) - A_{analytic}(r_j)
    + \beta(r_j) (\Delta r)^{\alpha} \\
    \epsilon_A(r_j)(\Delta r) &= \beta(r_j) (\Delta r)^{\alpha}
\end{align*}

To estimate the order of accuracy of the accuracy, we define the global errors 
by calculating the L2 Norm of the error which is denoted as $\hat{\epsilon}_A$ 

\begin{align*}
    \hat{\epsilon}_A = \sqrt{\frac{1}{N}\sum_{j=1}^{N} \epsilon(r_j)^2  }
\end{align*}

\begin{align*}
    \hat{\beta}_A(r_j) = \sqrt{\frac{1}{N}\sum_{j=1}^{N} \beta(r_j)^2  }
\end{align*}

As the grid density increases, $\hat{\beta}$ should asymptote to a constant 
value. Given two grid densities, $\Delta r$ and $\sigma\Delta r$, and assuming
that the leading error term is much larger than any other error term,

\begin{align*}
    \hat{\epsilon}_{grid 1} &= \hat{\epsilon}(\Delta r) = \hat{\beta}(\Delta r)^{\alpha} \\
    \hat{\epsilon}_{grid 2} &= \hat{\epsilon}(\sigma \Delta r) = \hat{\beta}(\sigma \Delta r)^{\alpha} \\
                            &= \hat{\beta}(\Delta r)^{\alpha} \sigma^{\alpha}
\end{align*}

The ratio of two errors is given by,

\begin{align*}
    \frac{\hat{\epsilon}_{grid 2}}{\hat{\epsilon}_{grid 1}} &= 
    \frac{\hat{\beta}(\Delta r )^{\alpha}}{\hat{\beta}(\Delta r )^{\alpha}} \sigma^{\alpha} \\ &= \sigma^{\alpha}
\end{align*}

Thus, $\alpha$,the asymptotic rate of convergence is computed as follows 

\begin{align*}
    \alpha = \frac{
        \ln \frac{
            \hat{\epsilon}_{grid 2}
    }{\hat{\epsilon}_{grid 1} }}
    {\ln\left( \sigma \right) }
\end{align*}

Defining  for a doubling of grid points ,

\begin{align*}
    \alpha = \frac{\ln \left( \hat{\epsilon}\left( \frac{1}{2}\Delta  r\right)
            \right) -\ln \left( \hat{\epsilon}\left( \Delta  r\right)
    \right)}{\ln \left( \frac{1}{2} \right)}
\end{align*}


% 3.1 Guidelines for Creating Manufactured Solutions states:
% \begin{enumerate}
%     \item  The manufactured solutions should be composed of smooth analytic 
%         functions 
%     \item The manufactured solutions should exercize every term in the governing
%         equation that is being tested,
%     \item The solution should have non trivial derivatives.  
%     \item The solution derivatives should be bounded by a small constant. In this case
%         this constant should prevent the function from becoming greater than 
%         one.
%     \item The solution should not prevent the code from running 
%     \item The solution should be defined on a connected subset of two- or three-
%         dimensional space to allow flexibility in chosing the domain of the PDE.
%         Section 3.3.1 provides more information about this.
%     \item The solution should coincide with the differential operators of the PDE.
%         For example, the flux term in Fourier's law of conduction requires T to 
%         be differentiable.
% \end{enumerate}
% With these guidelines, a function is specified for the speed of sound to conduct
% a method of manufactured solutions on SWIRL's speed of sound numerical 
% integration. This is checked by observing the tangential mach number 
% produced from the speed of sound and comparing that to the tangential mach number
% that has been analytically defined (See Equation \ref{eq:Mthetabackcalculated}).
\end{document}


