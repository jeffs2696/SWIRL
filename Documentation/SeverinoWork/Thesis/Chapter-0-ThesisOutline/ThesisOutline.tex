
%        File: ThesisOutline.tex

%     Created: Tue Jun 21 01:00 AM 2022 E
% Last Change: Tue Jun 21 01:00 AM 2022 E
%
\documentclass[a4paper]{report}
\usepackage{outlines}
\usepackage{mathtools}
\begin{document}
\begin{titlepage}
    \title{Thesis Outline}
    \author{ Jeffrey Severino \\
        University of Toledo \\
        Toledo, OH  43606 \\
    email: jseveri@rockets.utoledo.edu}


    \maketitle

\end{titlepage}
\begin{outline}[enumerate]
    \1 Introduction:
    \1[1-A.] Overview: Briefly explain why the study in being undertaken and what 
    main questions of foreshadowed problems will be addressed. Do this in a 
    general manner, because it will be done more specifically in the 
    following sections.
    \2 Aircraft Noise
    \1[1-B.] Statement of the Problem: Discuss the problem to be addressed in the
    research - the gaps, perplexities, or inadequacies in existing theory, 
    empirical knowledge, practice, or policy that prompted the study. The problem may be
    a theory that appears inadequate to explain known phenomena, the lack of 
    empirical data on a potentially interesting relationship between X and Y,
    or a common practice that appears ineffective. First state the problem
    generally, and then state the specifics that your research will address. In 
    quantitative research, the specifics will include the constructs studied.
    What is the research gap?  
    The problem that researchers would want to see addressed in the researach:
    - ``What is missing'' or ``What is NOT ADDRESSED''
    - The question or problem that has not been answered in an area of specialization 
    - establishes the need or importance
    
    \1[1-C.] 
    The purpose of research is to acquire knowledge to address the problem or 
    certain aspects of it. Quantitative research tries to fulfill that purpose 
    by answering questions and/or testing hypotheses. Qualitative research 
    tries to fulfill that purpose by starting with foreshadowed problems,
    conjectures, or exploratory questions. Mixed-methods research may use both 
    approaches.
    \1[1-D.1] Research Questions or Hypotheses 
    Research questions address problems of the study. Each research question 
    seeks answers to a specific problem situation described in your study. 
    The type of the data and its availability determine the research quesitons.
    \textit{ For instance, research questions should relate to the conceptual 
    framework.} Each question should address and target a separate problem situation.

    A good hypothesis clearly states the expected relationship ( or difference) 
    between two variables and defines those variables in operational, measurable 
    terms. The hypothesis (or hypotheses) logically follows the review of related 
    literature and is based on the implications of previous research. A 
    well-developed hypothesis is testable, that is, can be confirmed or 
    dis-confirmed. 
    \1[1-D.2] Significance of the Study: Discuss the potential significance of
    the research. Significance comes from the uses that might be made of your
    results-how they might be of benefit to theory, knowledge, practice, policy,
    and future research. The potential significance should be based upon your
    literature revieW in Chapter 2. 

    \1[1-E.] Conceptual Framework: Briefly summarize the theoreotical foundation 
    or conceptual framework(s) 
    \2 Research Questions:

    \1 Literature Review:
    \1[a]] The problem to be addressed and its significance
    \1[b]] The theoretical foundation or conceptual framework
    \1[c]] The research questions, hypotheses, foreshadowed problems, or conjectures
    \1[d]] The research paradigm and the methodology

    \2 Intro - Topics, Purposes, and Methods of the Literature Review.
    \2 Description and Critique of Scholarly Literature
    \2 Inferences for Forthcoming study
    \3 The problem to be addressed in your research and its significance
    \4 The code verification presented in Maldando's work will be 
    improved by providing an observed order of accuracy
    \4 One key question that remains, is how well do unsteady linearized equations cap-
    ture the mechanisms of noise generation within realistic 
    turbomachinery flow? The numerical simulations for more complex 
    (axial shear + swirl) flows do not have analytical solutions and contain 
    different categories for the axial wavenumbers.  
    \3 possible research questions, hypotheses, foreshadowed problems, or 
    conjectures
    \4 What combination of spatial grid resolution and numerical scheme aid 
    the final result? The use of dissipation is apparent in literature, but the
    amount is not reported. 
    \4 Can the use of a desired damping rate help aid the sorting of axial wavenumbers
    \3 possible theoretical or conceptual framework to be used and
     possible research paradigms and methodologies used
    \4 MMS study 
    \4[-] Additional MMS framework will be explored (tanhSummation) and utilized
    to find MS' that meet the guidelines in Roaches work 
    \4[-] The use of MMS to test matrix construction for eigenvalue problem 
    \4 MES study
    \4[-] the analytical solution is known for uniform flow and is used for
    MES
   \4 Literature test case studies (some are shared with Maldonado)

    \begin{itemize}
        \item Kousen's Test Cases
            \subitem Cylinder, Uniform Flow with Liner (Table 4.3)
            \begin{align*}
                m &= 2 \\
                k &= \frac{\omega r_T}{A_T} = -1 \\
                M_x &= 0.5 \\
                \eta_T &= 0.72 + 0.42i\\
                \text{Confirm if 32 grid points is enough}
            \end{align*} 
            \subitem Cylinder, Shear Flow without Liner (Table 4.4)
            \begin{align*}
                m &= 0 \\
                kb &= \left(\frac{\omega r_T}{A_T}\right)b = 20 \\
                b &= r_{max} - r_{min} \\
                \tilde{r} = \frac{r}{b} \\
                M_x &= 0.3(1-\tilde{r})^{\frac{1}{7}} \\
                \eta_T &= 0\\
                \text{Confirm if 32 grid points is enough}
            \end{align*}
            \subitem Annulus, Shear Flow without Liner (Table 4.5)
            \begin{align*}
                m &= 0 \\
                kb &= \left(\frac{\omega r_T}{A_T}\right)b = 10 \\
                b &= r_{max} - r_{min}  = \frac{1}{7}\\
                k &= 70 \\
                \tilde{r} = \frac{r}{b} = 6.0 \\
                M_x &= 0.3\left(1 - 2 \left| \frac{r_{max}-r}{b} + 0.5 \right|  \right)^{\frac{1}{7}} \\
                \eta_T &= 0\\
                \text{Confirm if 32 grid points is enough}
            \end{align*}
            \subitem Annulus, Shear Flow with Liner (Table 4.6)
            \begin{align*}
                m &= 0 \\
                kb &= \left(\frac{\omega r_T}{A_T}\right)b = 10 \\
                b &= r_{max} - r_{min}  = \frac{1}{3}\\
                k &= 30 \\
                \tilde{r} = \frac{r}{b} = 2.0 \\
                M_x &= 0.3\left(1 - 2 \left| \frac{r_{max}-r}{b} + 0.5 \right|  \right)^{\frac{1}{7}} \\
                \eta_T &= 0.3 + 0.1i\\
                \text{Confirm if 32 grid points is enough}
            \end{align*}
    \end{itemize}

    \4[-] Kousen's test cases which were compared to P.N. Shankar's work and 
    then expanded for cases where data is needed . Maldonado has compared the same
    tests

    \2 Theoretical/Conceptual Framework for Forthcoming Study (May appear in 
    methods chapter)

    \1 Research Methodology
    \2 Conjectures, or Exploratory Questions:
    \2[-] What is the observed order of accuracy for SWIRL's second order and 
    fourth order solutions (MMS/MES)? How does the number of grid points 
    and manufactured solution effect the solution accuracy.  
    \2 Research Procedures - Describe in detail how the inquiry was undertaken.
    Generally the description should be thorough enough that other skilled
    researchers could approximately replicate your study from the description.
    \3 Introduce the epistemology that will guide the inquiry/
    Indicate the methodology used and why it was selected.
    \4 Aerodynamic Theoretical Background:
    The steady and unsteady aerodynamic models are presented here.
    \4 The MMS and tanh summation method will be used to define a sufficient test 
    case for the second/fourth order differencing schemes for the radial derivatives
    and for the approximated speed of sound obtained through numerical integration. To account 
    for acoustic liners as boundary conditons, fairing functions were used
    to impose boundary values for the radial velocity perturbation and for the 
    radial derivative of pressure.
    \3 Explain the theoretical perspective that will drive the research, and why it was selected. 
    \4 The need for a code verification study prompted the use of MMS and MES to
    investigate if SWIRL was coded correctly. In addition, the findings from these 
    studies inform the user how the numerical scheme is effecting the final solution which
    will offer insight for the more complicated flows.
    \3 Indicate the specific methods used and the justification for them. How were sites, cases, and
    informants selected? Why? What access did you unsuccessfully seek? Which people perhaps tried to
    minimize contact with you and which repeatedly sought it out? How did you collect your data? Why?
    The manufactured solutions were chosen such in a way that allows testing 
    for each term in SWIRL. Setting up the manufactured solutions this way 
    ensures that there is no variable unused.   
    \3 Indicate how you analyzed and interpreted your data, making sure the analysis was consistent with
    the selected methodology. If you inferred themes, explain how. If you coded the transcripts, explain
    the coding system and checks for coding reliability and validity. How did you analyze the data from
    the coding? How did you triangulate or otherwise verify findings? How did you interpret the full set
    of data?

    The 
    \1 Results of the Study
    \2 statement of the received results and their analysis
    \3 Results for MMS/MES/set of test cases will be presented. The MMS and MES
    will have errors and convergence rates associated with the schemes that are
    available in SWIRL.
    \2 Comparison of the obtained results and the initial goals/questions
    \1 Summmary
    \2 Discussion of the research results;
    \3 Discuss the convergence rates and make the case for fourth order schemes.
    \3 Discuss the effect on axial wavenumber categorization. 
    \2 Comparison of the obtained results with the findings of prior researchers
    \3 Discuss the appearance of nearly convected wavenumbers as the tests get more complex.
    \2 Suggestions regarding the use of the obtained findings for the further
    development of the topic and future investigation 
    \3 The use of GCI could give a grid which is closer to convergence
    \3 Limitations of the eigensolver
    \3 The use of a desired damping rate can be correlated to the length of a 
    Nacelle
    \3 Comparison against experimental data
    \1 Appendices
\end{outline}
\end{document}

