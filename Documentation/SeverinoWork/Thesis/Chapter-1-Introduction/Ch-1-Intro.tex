During the 1960s, the increased demand for commercialized aircraft transport
introduced jet engines to support large cargo and passengers. Consequently,
this rise in innovation resulted in high volume engine noise. After 1975, 
efforts to reduce aircraft noise eliminated the noise pollution for 90\% of
the population \cite{FAAPolicy}. However, since the early 2000s,
the advancement in noise reduction technologies has been moderately increasing,
leaving a requirement for drastic improvement in aeroacoustic modeling and treatment
strategies to compete with the demand for quiet subsonic flight.  
A turbomachine's general flow condition includes a series of axial, tangential,
and radial velocity components that vary depending on the location of concern.
The swirling flow between fan stages has been an area of interest due to the
potential for acoustic treatment in a location previously avoided for its flow complexity, among 
other reasons.

In general, jet engine designers can model flow within a turbomachine with 
the Navier Stokes Equations, a set 
of partial differential equations that describe the mass, momentum and energy
of a given viscous fluid. It is common in practice to utilize the Euler equations,
a closely related set of PDEs that model inviscid fluid, as they provide an 
approximation for higher Reynold number flows where viscosity 
does not play a critical role. A popular approach to modeling sound propagation 
within a flow is to ``linearize'' the Euler equations, which decomposes the 
flow solution into a mean and fluctuating component (insert refs). Another method
decomposes the flow into vortical and potential parts (ref Golubev \& Atassi). 
In either case, this presents an initial value problem and for certain flows and
domains, can obtain analytical solutions.  Swirling flow has been a difficult
problem to investigate in comparison to flows parallel to the wall domain of a duct \cite{COOPER2001}.   
. 
.




%
%- the problem that we're addressing
%
%-- being able to conduct component level code verification tests for the problem
%of characterizing the duct acoustics for flow using a LEE model.
%
%why is it a problem? there is multiple ways of arriving at the same solution. 
%This can be used to give a metric to either method for the various computational
%methods that may be needed to arrive at the final answer.
%
%In the mid 90's,an aeroacoustics model for swirling flow had been proposed and 
%has bypassed using a single PDE and has instead used an eigenvalue approach on the 
%four governing flow equations. (Why Is this better? does this capturE the problem
%differently? why not use the PDE alone instead\ldots)
%
%The proposed component verification from Kleb and Wood will be presented to addreess the characterization of 
%modes and the presence of numerical ones. Using higher accuracy methods should 
%further improve The result. another thing is to determine how many grid points are needed
%for each method
%
%
