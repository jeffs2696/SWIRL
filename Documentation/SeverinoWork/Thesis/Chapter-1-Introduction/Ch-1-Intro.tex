\section{Overview}
\subsection{Introduction of overall field and basic explaination of this topic (What)}
% I) Opening Section
%   1) Introduce the overall field 
Aircraft noise is one of the most environmentally detrimental consequences of
commercial flight. Studies suggest that exposure to aircraft noise leads
to dimished academic performance in youth, and could increase the risk of 
cardiovascular disease for populations close to airports \cite{Basner2017}. 
Although the COVID-19 pandemic reduced air panssenger trafic by 96\% between 2019 
and 2020, \cite{pandemicReport} the global aviation commmunity has proven 
to be resilliant during times of economic shock. The International Air Transport
Association (IATA) has studied the resilience of the global air passenger markets after four notable shocks to the aviation economy; 
The impact of four notable events, (1979 oil shock, 2000-2001 dot-com bust, 9/11, and the 2008 financial crisis) 
was studied by a statistical analysis of the estimated 'passenger gap' from 1950-2014. 
Data shows that approximately 72\% of the impact of the initial shock persists 
one year after the event and diminishes to just under one-fifth of the initial impact. 
Given these trends in air transportation, the need for  aviation noise  
regulation persists and will continue to be a concern so long as the aviation
continues this rate of growth.

% II) Background and overview
%      What is the current situation with aircraft noise
%      wht are acoustic modes and liners
%      Swirling flow codes
%      code verification
%   1) What is sound propagation and why is it relevant in ducted flows? 
%   2) What are the key theories and research
%   3) what is the current context and what does this mean
Since the dawn of commerical airlines in the early 20th centurty, the increased demand for aircraft transport
introduced jet engines to support large cargo and passengers. Consequently,
this rise in innovation resulted in high volume engine noise due to 
the frequency of flights. After 1975, 
efforts to reduce aircraft noise eliminated the noise pollution for 90\% of
the population \cite{FAAPolicy}.  However, given the rapid increase in
aircraft movements and consequently increase in noise exposure to larger 
populations, the advancement in noise reduction 
technologies has only been moderately increasing, leaving a requirement for 
in aeroacoustic modeling techniques and treatment strategies to compete with 
the demand for quiet subsonic flight \cite{icao2020}.  
%   2) Introduce the research problem
% The 
Between the 1950s and 1960's aero gas turbine designs shifted to higher by pass 
ratios with two or three shafts. The high by pass ratio (HBP) fan utilized multiple
stages of fans and  air streams \cite{smith1989aircraft}. The efficiency of these
engines rose with the availablility of materials that are able to cool flows
passing over the turbofan, thus slowing the overall jet velocity but maintaining the 
efficiency of the engine. One of the most popular configurations is the geared 
turbofan (GTF), the largest contributor to the noise on modern aircraft 
has been the fan upon take-off and landing.  Although the use of the GTF has
reduced the noise emissions by 75\% \cite{GTFinfo}, further noise reduction
technologies to mitigate the noise associated to the turbofans geometric
configuration and operation speeds. One of the most common techiques to reduce 
fan noise  is to include the use of acoustic treatment along the walls of the
turbomachine's nacelle. 

Due to the increase in high-bypass-ratio of turbomachines, the newest models of 
engines have a significantly larger diameter and a shorter nacelle, leaving less
room to place acoustic treatments in regions where it will be effective \cite{Kozaczuk2017}.
Figure \ref{fig:intro} shows the evolution of directivity for turbomachines as 
the use of HBR fans became more popular. As these engines continue 
to develop, an increased understanding of sound propagation within
the interstage of the engine is gong to be needed due to flow behavior (high compressibility
and rotational effects).  While a turbomachine's general flow condition includes 
a series of axial, tangential, and radial velocity components that vary
depending on the location of concern, the swirling flow between fan stages has 
been an area of interest due to the potential for acoustic treatment in a 
location previously avoided for its flow complexity. This work will explore how 
sound propagation is modeled and how the current state of code verification and 
validation currently stands. This introduction will describe how fluid mechanics
is utlized to establish an aeroacoustic model for various ducted flows. It will 
also discuss how code verification is used in the computational and numerical fields
but will show the need for the use of these code verification techniques for 
a frequency domain CAA code. 


%   3) State your research aims

%   4) Outline the introduction 

%3)
\begin{figure}
    \centering
    \includegraphics[width=\textwidth]{Chapter-1-Introduction/Figures/lowVhighBPRdirectivitySMITH2004.png}
    \caption{ The evolution of the directivity and the 
    relative levels of sources as a function of engine architecture (a)low bypass-ratio (b) high bypass ratio \cite{smith1989aircraft}}
    \label{fig:intro}
\end{figure}

\section{Statement of the of the research questions}
\subsection{What is already known?}
% III) Research Problem

%   1) What's already known?
%   2) What's missing?
%   3) Why is it a problem?
In general, jet engine designers can model flow within a turbomachine with 
the Navier Stokes (N-S) Equations, a set of partial differential equations that
describe the mass, momentum and energy of a given viscous fluid, however 
these equations can be computationally expensive as they are used in the most
general cases. For aeroacousticians, the N-S equations can be too complicated
to identify sound generation and propagation because acoustic waves are low 
amplitude (~ only a fraction of atmospheric pressure) and are not strongly
influenced by viscosity.   As a result, it is common in practice to 
utilize the Linearized Euler equations (LEE),
a closely related set of PDEs that model inviscid fluid, as they provide an 
approximation for higher Reynold number flows where viscosity 
does not play a critical role. A popular approach to modeling sound propagation 
within a flow is to ``linearize'' the Euler equations, which decomposes the 
flow solution into a mean and fluctuating component . The decomposition is done 
in a linear fashion because the sound propagation amplitude is small with respect
to the mean flow, and their presence does not appreciable change the mean flow 
field. The LEE provides a system of linear equations where for uniform flow, 
the solution is a family of wavenumbers and radial mode shapes that arise from 
unsteady disturbances for flows within a cylindrical duct.  Another method 
decomposes the flow into vortical and potential parts \cite{golubev1996sound}.
In either case, this presents an initial value problem which 
in limited cases can obtain analytical solutions for simplified mean flow. Once a mean flow 
is contains a tangential component, the LEE equations must be solved numerically.  
\subsection{What is missing?/Why is it a problem?}
For uniform flows in a hard wall duct , the waves are categorized as vortical 
,entropical, and acoustic waves. The vortical and entropic waves soley convect with 
the mean flow, where as the acoustic wave can propagate without damping or decay 
exponentially.  However, for swiriling flows, the waves are partially coupled
and are not easily categorized due to an additonal category of ``nearly convecting'' 
modes \cite{Kerrebrock2012},\cite{KERREBROCK1974} . 
Therefore, the families of waves must be found numerically \cite{Envia2004}
making the ducted acoustic propagtion in swirling flow a problem without
an analytical solution but has a framework for a numerical solution.
\subsection{The goal and significance of the investigation}
Swirling flow has been a difficult problem to investigate in comparison to 
flows parallel to the wall domain of a duct \cite{COOPER2001} because of the 
lack of an analytical solution and thus cannot be described from a single convective 
wave equation. However, the solution for sheared mean flows was first presented
by Goldstein \cite{Goldstein1978},\cite{Goldstein1979}. Various special cases of
swirling flow (free vortex and solid body swirl) was examined in \cite{KAPUR1973}
\cite{Kerrebrock2012}, \cite{KERREBROCK1974}. In recent years verification and
validation has been done given the rise in technologies capabale of experimentally
measuring the acoustic modes within a turbomachine \cite{Maldonado2016}. This
work aims offer additional insight to the verification and validation process 
by expanding on techniques used in this field.  
\section{Definition of the terms (if needed)}
\section{Organization of research (Structure)}
% IV) Research aims, objectives, and research question
% V) State benefit/significance
% - speed improvement 
% VI)
% VII)


This research aims to investigave the theoretical framework that is used to
model ducted sound propagation within various flow fields and apply the 
``gold-standard'' of code verification , the method of manufactured solutions 
(MMS) and method of exact solutions (MES) for the limited subset 
of equations where a solution is known, such as uniform mean flow. 


The literature review in the next chapter will discuss the governing equations that are
used to predict the acoustic behavior in a fluid flowing internally, followed 
by the research problem that arises in swirling flow, the objectives and 
questions, the significance and finally the limitations.  The proposed research aims to determine the impact of the numerical schemes used
in the swirling flow problem and how it effects the family of waves that are
produced from the problem formulation so a better understanding of the 
acoustic phenomena as the flow under goes a compressible rotational flow. The use
of the method of manufactured solutions is used as a means of ensuring the code is
correctly approximating the goverining equations and will check the effect of the numerical schemes
on the axial wavenumbers produced.
 
\section{Research Questions and Hypothesis}


- the problem that we're addressing

-- being able to conduct component level code verification tests for the problem
of characterizing the duct acoustics for flow using a LEE model.

why is it a problem? there is multiple ways of arriving at the same solution. 
This can be used to give a metric to either method for the various computational
methods that may be needed to arrive at the final answer.

In the mid 90's,an aeroacoustics model for swirling flow had been proposed and 
has bypassed using a single PDE and has instead used an eigenvalue approach on the 
four governing flow equations. (Why Is this better? does this capturE the problem
differently? why not use the PDE alone instead\ldots)

The proposed component verification from Kleb and Wood will be presented to addreess the characterization of 
modes and the presence of numerical ones. Using higher accuracy methods should 
further improve The result. another thing is to determine how many grid points are needed
for each method


