
\section{Introduction}
The Method of Manufactured Solutions (MMS) is a process for generating an 
analytical solution for a code that provides the numerical solution for a 
given domain. The goal of MMS is to establish a manufactured solution that can 
be used to establish the accuracy of the code within question. For this study, 
SWIRL, a code used to calculate the radial modes within an infinitely long duct
is being validated through code verification. SWIRL accepts a given mean flow and 
uses numerical integration to obtain the speed of sound. The integration technique
is found to be the composite trapezoidal rule through asymptotic error analysis.

For SWIRL, the absolute bare minimum requirement is to define the corresponding
flow components for the domain of interest. SWIRL assumes no flow in the radial 
direction, leaving only two other components, axial and tangential for a 3D 
cylindrical domain. Since SWIRL is also non dimensionalized, the mean flow 
components are defined using the Mach number. SWIRL uses the tangential mach number
to obtain the speed of sound using numerical integration. The speed of sound
is then used to find the rest of the primative variables for the given flow. 

