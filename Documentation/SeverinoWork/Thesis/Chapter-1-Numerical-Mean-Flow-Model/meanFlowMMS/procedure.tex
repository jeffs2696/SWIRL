
\subsection{Procedure}
There are a few constraints and conditions that must be followed in order for the analytical 
function to work with SWIRL, 

\begin{itemize}
    \item The mean flow and speed of sound must be real and positive. This will 
        occur is a speed of sound is chosen such that the tangential mach number
        is imaginary
    \item The derivative of the speed of sound must be positive
    \item Any bounding constants used with the mean flow should not allow the 
        total Mach number to exceed one.
    \item the speed of sound should be one at the outer radius of the cylinder
\end{itemize}

Given these constraints, $tanh(r)$ is chosen as a function since it can be
modified to meet the conditions above. Literature (The tanh method: A tool for 
solving certain classes of nonlinear evolution and wave equations) 
is a paper than demonstrates the strength of using tanh functions.
One additonal benefit of tanh(r) is that it is bounded between one and negative one, i.e.

\begin{itemize}
    \item As r $\rightarrow$ $\infty$ tanh(r) $\rightarrow$ 1
    \item As -r $\rightarrow$ $-\infty$ tanh(r) $\rightarrow$ -1
\end{itemize}

To test the numerical integration method,  $M_{\theta}$ is defined as a result 
of differentiating the speed of sound, $A$. This is done opposed to integrating
$M_{\theta}$ analytically. However, an analytical function can be defined for
$M_{\theta}$, which can then be integrated to find what $\widetilde{A}$ should be. 
Instead, the procedure of choice is to back calculate what the appropriate 
$M_{\theta}$ is for a given expression for $\widetilde{A}$.  Since it is easier 
to take derivatives , we will solve for $M_{\theta}$ using Equation \ref{eq:Mthetabackcalculated} ,

\subsection{Tanh Summaion Formulation}
The goal is generate an MS with a number of ``stairs'' that is bounded between
zero and one. Here's what my focus group ideas are,

\begin{align*}
    1 = R + L 
\end{align*}
where, 1 is a constraint, and R and L are the two waves when summed need to cancel 
if it were the exact same amplitude \& opposite sign 

so ,

\begin{align*}
    R + L = \tanh(x) + -\tanh(x) = 0
\end{align*}
or in our case,

\begin{align*}
    R + L = \tanh(x) + -\tanh(x) = 1
\end{align*}

We can tweak this by adding knobs by adding ``knobs'' A and B. If we dont want 
the total to not exceed one then, $A_j + A_{j+1} \cdots A_{last} = 1$. $B_1$ changes
the steepnes of the kink that we want. In order to generalize this,


\begin{align*}
    \bar{A} = \sum_{j=1}^n R_{ij} + \sum_{j=1}^n L_{ij} 
\end{align*}
where,
\begin{align*}
    R_{ij} = A_j \tanh(B_j (x_i - x_j)) \\ 
    L_{ij} = A_j \tanh(B_j (x_j - x_n))  
\end{align*}

Letting $n = 3$\ldots

\begin{align*}
    \bar{A} &= S_{vert} + \sum_{j=1}^3 R_{ij} + \sum_{j=1}^3 L_{ij} \\
    \bar{A} &=
    A_1 \tanh(B_1 (x_i - x_1))  + 
    A_{2} \tanh(B_{2} (x_i - x_{2}))  + 
    A_{3} \tanh(B_{3} (x_i - x_3)) +  \\
    A_1 \tanh(B_1 (x_1 - x_n)) &+ 
    A_{2} \tanh(B_{2} (x_2 - x_{n}))  +
    A_{3} \tanh(B_{3} (x_3 - x_n))  
\end{align*}
and,
\begin{align*}
    A_1 = A_2 = A_3 = k_1 \\ 
    B_1 = B_2 = B_3 = k_2  
\end{align*}

A tanh summation method was constructed to make a manufactured solution with 
strong changes in slope. This ensures that the numerical approximation will not 
give trivial answers. 
then for some functions we need to impose boundary conditions. We will demonstrate
how the careless implementation of a boundary condition can lead to close approximations
on the interior.  The speed of sound is defined with the subscript $analytic$ to indicate that this is the analytical function of choice and has no physical relevance 
to the actual problem.

\begin{align*}
\widetilde{A}_{analytic} = \Lambda + k_1 \tanh \left( k_2 \left( \widetilde{r} - \widetilde{r}_{max} \right) \right),
\end{align*}

where, 

\begin{align*}
    \Lambda = 1 - k_1 \tanh(k_2 (1 - \widetilde{r}_{max})),
\end{align*}

When, $\widetilde{r}=\widetilde{r}_{max}$ , $\widetilde{A}_{analytic} = 1$.  
Taking the derivative with respect to $\widetilde{r}$,

\begin{align*}
    \frac{\partial \widetilde{A}_{analytic} }{\partial \widetilde{r}} &=
    \left(1 - \tanh^{2}{\left(\left(r - r_{max}\right) {k}_{2} \right)}\right) {k}_{1} {k}_{2}, \\ 
    &= \frac{ k_{1} k_{2}}{\cosh^{2}{\left(\left(r - r_{max}\right) {k}_{2} \right)}}.
\end{align*}

Substitute this into the expression for $M_{\theta}$ in Equation 
\ref{eq:Mthetabackcalculated},

\begin{align*}
    M_{\theta} = \sqrt{2}
    \sqrt{\frac{r {k}_{1} {k}_{2}}{\left(\kappa - 1\right) \left(\tanh{\left(\left(r - r_{max}\right) {k}_{2} \right)} {k}_{1} + \tanh{\left(\left(r_{max} - 1\right) {k}_{2} \right)} {k}_{1} + 1\right) \cosh^{2}{\left(\left(r - r_{max}\right) {k}_{2} \right)}}}
\end{align*} 

Now that the mean flow is defined, the integration method used to obtain the 
speed of sound

% What happens when $r = r_{max}$?

Initially the source terms were defined without mention of the indices of the 
matrices they make up. In other words, there was no fore sight on the fact that
these source terms are sums of the elements within A,B, and X. To investigate 
the source terms in greater detail, the FORTRAN code that calls the source 
terms will output the terms within the source term and then sum them, instead 


of just their sum.
i
$ [A]{x} = \lambda [B] {x} $

which can be rearranged as,

$ [A]{x} - \lambda [B] {x} = 0$

Here, $x$ is an eigenvector composed of the perturbation variables, $v_r,v_{\theta},v_x,p$ and $\lambda$ is the associated eigenvalue, (Note: $\lambda = -i \bar{\gamma}$)

Writing this out we obtain \cdots.

Linear System of Equations:
\begin{equation}
    -
    i \left(
        \frac{k}{A} - \frac{m}{r} M_{\theta}
    \right)
    v_r 
    -
    \frac{2}{r} M_{\theta} v_{\theta} 
    +
    \frac{dp}{dr} 
    +
    \frac{(\kappa - 1)}{r} M_{\theta}^2 p
    -
    \lambda M_x v_r =S_1
\end{equation}

Using matrix notation,

\begin{equation}
    A_{11}
    x_1 
    -
    A_{12} x_2 
    +
    A_{14} x_4
    -
    \lambda B_{11} x_1 = S_1
\end{equation}


But $A_{14}$ and $A_{41}$ in Kousen's paper only has the derivative operator.
Since I am currntly writing the matrix out term by term and not doing the matrix 
math to obtain the symbolic expressions, I will define $A_{14}$ with $dp/dr$ 
and $A_{41}$ with $dv_r/dr$
Similarly,
\begin{align}
    A_{21} x_1 &-
    A_{22} x_2 +
    A_{24} x_4 &-
    \lambda B_{22} x_2 &= S_2 \\
    A_{31} x_1 &-
    A_{33} x_3 &-
    \lambda (B_{33} x_3 + B_{34} x_4) &= S_3\\
    A_{41} x_1 &+
    A_{42} x_2 +
    A_{44} x_4 &- 
    \lambda (B_{33} x_3 + B_{44} x_4) &= S_4
\end{align}
Now we can begin looking at the source terms, term by term. They each should also
converge at a known rate





