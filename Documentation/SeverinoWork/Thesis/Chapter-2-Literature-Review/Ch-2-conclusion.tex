
\section{Review of the assessment of the numerical techniques} 
Kousen assesed the accuracy of the numerical discretication technique for 
a series of test cases using three sources \cite{Shankar1972} \cite{Vo1978} and 
\cite{Astley1979} ([10] [15] and [45] in Kousen's work \cite{kousen1996pressure}
respectively).  The results were assesed by using various
literature comparisons. The methodology was presented of uniform mean axial flows,
but results for hard wall cases were presented by computing the order of accuracy for
the first four radial modes (See Figures 4.1-4.4), . (Explain why this is MES and offer MMS as a source
of verification and explain that MES is validation \cite{Roy2005} \cite{Salari2000} ). For a uniform axial flow, 
the axial wavenumbers can be computed from an analytical solution, where 
one of the key input parameters are the zero crossings of the derivative
of the Bessel Function of the first kind . The values are presented 
\cite{Kerrebrock1992} and are often refered to as separation constants 
for a circumfirential and radial mode pair ; which are needed to compute the solution 
of the convective wave equation.  

The axial wavenumber is found by using second-order differential convective 
wave equation for pressure using a fourth order accurate 
Runge-Kutta(RK) method \cite{kousen1995eigenmode} which was done to check 
against the results in \cite{Agarwal1989} (Table 4.1 and 4.2 in \cite{kousen1995eigenmode}). 
The output parameter was $\gamma/k$.  Each axial wave number can then be used 
to compute the analytical radial mode using the 
exponential assumption.  
Axial wavenumbers from annular and cylindrical ducts with lined walls were 
compared to findings from Astley and Eversman \cite{Astley1979} for uniform and 
sheared axial flow with liner (Table 4.3).  The results taken were from a ``high-order'' RK 
scheme used. Axial wavenumbers from cylindrical ducts with hard walls were 
compared to findings from Shankar \cite{Shankar1972} in Table 4.4 of \cite{kousen1995eigenmode}. 

In recent years  Maldonado et. al, \cite{Maldonado2016} has made significant 
contributions in solution verification given the recent improvements in 
experimental measurement techniques. The work has presented test cases for 
lined ducts that have been compared to Kousen \cite{kousen1995eigenmode}, Nijboer
\cite{Nijboer2001} and Peake \cite{Posson2013} and show excellent comparison. 
The goal of this work is to contribute these efforts by conducting the method
of manufactured solutions to offer clarity in using techniques often used in other
verification and validation (V\&V) studies. 
\subsection{The research questions, hypotheses, foreshadowed problems, or 
    conjectures}
While these results confirm the findings of SWIRL and other LEE codes, this does not check if the 
equations that were programmed were entered correctly. While an emphasis on 
solution verification is vital, it should be coupled with code verification to
determine the robustness and consistency of the algorithm.  The method of manufactured
solutions combined with order of accuracy verification is often used as a 
gold standard of code verification \cite{Roy2005} and has been shown to provide
an estimate of discretization error that can be computed before the final answer
is obtained. Since the mid 2000's, code verification literature has grown popular
in the field of computational mathematics and physics due to its ability to
conduct tests for numerical approximations of partial and ordinary differential 
equations. The MMS offers a means of ``manufacturing'' an arbitrary solution by
define functions for each term in the governing equation.  Common practices 
and guidelines are offered in \cite{Salari2000} to choose the fuunctions, but
are phrased such that MMS can be widely applied. Since various numerical problems
are unique in their treatment of spatial discretization, and boundary condtions ,
this work will describe the use of these guidelines and the nuances that have
been taken to check the boundary conditon and radial derivatives used in SWIRL.

Knupp in Code Verification by the MMS \cite{Salari2000} provides guidelines for creating
\section{Conclusion}
\begin{itemize}
    \item The conclusion summarizes 
        the key findings of the review in general terms. Notable commonalities between works, whether favorable or not
        , may be included here.
        \subitem This review discusses the development of the unsteady linearized equations, 
        and how improvements in the modal analysis capture more families of mechanisms of noise generation within non-uniform 
        swirling flow turbomachinery flow. 
    \item This section is the reviewer’s opportunity to justify a research 
        proposal. Therefore, the idea should be clearly re-stated and supported 
        according to the findings of the review.
        \subitem While the literature presented offers a measure of verification 
        and validation through the use of the Method of Exact Solutions, the Method 
        of Manufactured Solutions offers a level of code verification which allows
        for error checking by computing the approximate order of accuracy for 
        a given numerical scheme, which is independent of the final answer, which
        in this case is the axial wavenumber. The next chapter will outline the methodology
        and techniques used when applying MMS to SWIRL. The methods consist
        of unique treatment of boundary conditions using fairing functions as
        well as an example of using a summation to generate arbitrary functions 
        as manufactured solutions which has the dual benefit of giving a large
        number of derivatives but allows for high gradients in specific locations
        along the domain of the MS. The use of open-source widely available functions
        in Python were used to symbolically create the MS and then 
        used to generate FORTRAN code that will compute the MS for code comparison. 
        
        
        
        \subitem (Note: This literature review should be expanded to describe 
        specific details in the methodologies and validation techniques reported 
        by \cite{kousen1995eigenmode,kousen1996pressure,Golubev1997,Golubev1998}. 
\end{itemize}

