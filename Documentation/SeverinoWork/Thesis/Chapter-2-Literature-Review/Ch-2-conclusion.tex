
\section{Review of the assessment of the numerical techniques} 
Kousen assesed the accuracy of the numerical discretication technique for 
a series of test cases using three sources \cite{Shankar1972} \cite{Vo1978} and 
\cite{Astley1979} ([10] [15] and [45] in Kousen's work \cite{kousen1996pressure}
respectively).  The results were assesed by using various
literature comparisons. The methodology was presented of uniform mean axial flows,
but results for hard wall cases were presented by computing the order of accuracy for
the first four radial modes (See Figures 4.1-4.4), . (Explain why this is MES and offer MMS as a source
of verification and explain that MES is validation \cite{Roy2005} \cite{Salari2000} ). For a uniform axial flow, 
the axial wavenumbers can be computed from an analytical solution, where 
one of the key input parameters are the zero crossings of the derivative
of the Bessel Function of the first kind . The values are presented 
\cite{Kerrebrock1992} and are often refered to as separation constants 
for a circumfirential and radial mode pair ; which are needed to compute the solution 
of the convective wave equation.  

The axial wavenumber is found by using second-order differential convective 
wave equation for pressure using a fourth order accurate 
Runge-Kutta(RK) method \cite{kousen1995eigenmode} which was done to check 
against the results in \cite{Agarwal1989} (Table 4.1 and 4.2 in \cite{kousen1995eigenmode}). 
The output parameter was $\gamma/k$.  Each axial wave number can then be used 
to compute the analytical radial mode using the 
exponential assumption.  
Axial wavenumbers from annular and cylindrical ducts with lined walls were 
compared to findings from Astley and Eversman \cite{Astley1979} for uniform and 
sheared axial flow with liner (Table 4.3).  The results taken were from a ``high-order'' RK 
scheme used. Axial wavenumbers from cylindrical ducts with hard walls were 
compared to findings from Shankar \cite{Shankar1972} in Table 4.4 of \cite{kousen1995eigenmode}. 

In recent years  Maldonado et. al, \cite{Maldonado2016} has made significant 
contributions in solution verification given the recent improvements in 
experimental measurement techniques. The work has presented test cases for 
lined ducts that have been compared to Kousen \cite{kousen1995eigenmode}, Nijboer
\cite{Nijboer2001} and Peake \cite{Posson2013} and show excellent comparison. 
The goal of this work is to contribute these efforts by conducting the method
of manufactured solutions to offer clarity in using techniques often used in other
verification and validation (V\&V) studies. 
\subsection{The research questions, hypotheses, foreshadowed problems, or 
    conjectures}
While these results confirm the findings of SWIRL and other LEE codes, this does not check if the 
equations that were programmed were entered correctly. While an emphasis on 
solution verification is vital, it should be coupled with code verification to
determine the robustness and consistency of the algorithm.  The method of manufactured
solutions combined with order of accuracy verification is often used as a 
gold standard of code verification \cite{Roy2005} and has been shown to provide
an estimate of discretization error that can be computed before the final answer
is obtained. Since the mid 2000's, code verification literature has grown popular
in the field of computational mathematics and physics due to its ability to
conduct tests for numerical approximations of partial and ordinary differential 
equations. The MMS offers a means of ``manufacturing'' an arbitrary solution by
define functions for each term in the governing equation.  Common practices 
and guidelines are offered in \cite{Salari2000} to choose the fuunctions, but
are phrased such that MMS can be widely applied. Since various numerical problems
are unique in their treatment of spatial discretization, and boundary condtions ,
this work will describe the use of these guidelines and the nuances that have
been taken to check the boundary conditon and radial derivatives used in SWIRL.

Knupp in Code Verification by the MMS \cite{Salari2000} provides guidelines for creating
manufactured soultions which states,
\begin{enumerate}
    \item  The manufactured solutions should be composed of smooth analytic 
        functions 
    \item The manufactured solutions should exercize every term in the governing
        equation that is being tested,
    \item The solution should have non trivial derivatives.  
    \item The solution derivatives should be bounded by a small constant. In this case
        this constant should prevent the function from becoming greater than 
        one.
    \item The solution should not prevent the code from running 
    \item The solution should be defined on a connected subset of two- or three-
        dimensional space to allow flexibility in chosing the domain of the PDE.
    \item The solution should coincide with the differential operators of the PDE.
        For example, the flux term in Fourier's law of conduction requires T to 
        be differentiable.
\end{enumerate}

The goal is the to calculate an observed order of accuracy.
\subsection{Calculation of Observed Order-of-Accuracy}

To begin the method of manufactured solutions, the discretization error,
$\epsilon$ is defined as a function of radial grid spacing, $\Delta r$

\[ \epsilon = \epsilon(\Delta r) \]
The discretization error in the solution should be proportional to 
$\left( \Delta r \right)^{\alpha}$ where $\alpha > 0$ is the theoretical order
for the computational method.  The error for each grid is expressed as 
\[ \epsilon_{M_{\theta}}(\Delta r) = |M_{\theta,Analytic}-M_{\theta,calc}|\]
where $M_{\theta,Analytic}$ is the tangential mach number that is defined from the
speed of sound we also defined and the $M_{\theta,calc}$ is the result from 
SWIRL. The $\Delta r$ is to indicate that this is a discretization error for a
specific grid spacing. Applying the same concept to to the speed of sound,

If we define this error on various grid sizes and compute $\epsilon$ for
each grid, the observed order of accuracy can be estimated and compared to
the theoretical order of accuracy. For instance, if the numerical soution
is second-order accurate and the error is converging to a value, the L2 norm of
the error will decrease by a factor of 4 for every halving of the grid cell 
size. 
Since the input variables should remain unchanged, the error for the axial and tangential mach 
number should be zero. As for the speed of sound, since we are using an analytic
expression for the tangential mach number, we know what the theoretical result
would be from the numerical integration technique as shown above. 
Similarly we define the discretization error for the speed of sound.

\[ \epsilon_{A}(\Delta r) = |A_{Analytic}-A_{calc}|\]

For a perfect answer, we expect $\epsilon$ to be zero. Since a Taylor series can 
be used to derive the numerical schemes, we know that the truncation of higher
order terms is what indicates the error we expect from using a scheme that 
is constructed with such truncated Taylor series.

%\text{H.O.T}
The expectation is that the error at each grid point $j$ to satisfy the following,
\begin{align*}
    0 &= |A_{Analytic}(r_j) - A_{calc}(r_j)| \\
    \widetilde{A}_{Analytic}(r_j) &= \widetilde{A}_{calc}(r_j) +
    (\Delta r)^{\alpha} \beta(r_j)  + H.O.T
\end{align*}

where the value of $\beta(r_j)$ does not change with grid spacing, and 
$\alpha$ is the asymptotic order of accuracy of the method. It is important to
note that the numerical method recovers the original equations as the grid 
spacing approached zero.

Subtracting $A_{Analytic}$ from both sides gives

\begin{align*}
    A_{calc}(r_j) - A_{Analytic}(r_j) &= A_{Analytic}(r_j) - A_{Analytic}(r_j)
    + \beta(r_j) (\Delta r)^{\alpha} \\
    \epsilon_A(r_j)(\Delta r) &= \beta(r_j) (\Delta r)^{\alpha}
\end{align*}

To estimate the order of accuracy of the accuracy, the global errors 
are calculated by taking the L2 Norm of the error which is denoted as $\hat{\epsilon}_A$ 
\begin{align*}
    \hat{\epsilon}_A = \sqrt{\frac{1}{N}\sum_{j=1}^{N} \epsilon(r_j)^2  }
\end{align*}

\begin{align*}
    \hat{\beta}_A(r_j) = \sqrt{\frac{1}{N}\sum_{j=1}^{N} \beta(r_j)^2  }
\end{align*}
As the grid density increases, $\hat{\beta}$ should asymptote to a constant 
value. Given two grid densities, $\Delta r$ and $\sigma\Delta r$, and assuming
that the leading error term is much larger than any other error term,

\begin{align*}
    \hat{\epsilon}_{grid 1} &= \hat{\epsilon}(\Delta r) = \hat{\beta}(\Delta r)^{\alpha} \\
    \hat{\epsilon}_{grid 2} &= \hat{\epsilon}(\sigma \Delta r) = \hat{\beta}(\sigma \Delta r)^{\alpha} \\
                            &= \hat{\beta}(\Delta r)^{\alpha} \sigma^{\alpha}
\end{align*}

The ratio of two errors is given by,

\begin{align*}
    \frac{\hat{\epsilon}_{grid 2}}{\hat{\epsilon}_{grid 1}} &= 
    \frac{\hat{\beta}(\Delta r )^{\alpha}}{\hat{\beta}(\Delta r )^{\alpha}} \sigma^{\alpha} \\ &= \sigma^{\alpha}
\end{align*}

Thus, $\alpha$,the asymptotic rate of convergence is computed as follows 

\begin{align*}
    \alpha = \frac{
        \ln \frac{
            \hat{\epsilon}_{grid 2}
    }{\hat{\epsilon}_{grid 1} }}
    {\ln\left( \sigma \right) }
\end{align*}

For example ,a doubling of grid points has $\sigma=1/2$,

\begin{align*}
    \alpha = \frac{\ln \left( \hat{\epsilon}\left( \frac{1}{2}\Delta  r\right)
            \right) -\ln \left( \hat{\epsilon}\left( \Delta  r\right)
        \right)}{\ln \left( \frac{1}{2} \right)}
    \end{align*}



\section{Conclusion}
This review discusses the development of the unsteady linearized equations, 
and how improvements in the modal analysis capture more families of mechanisms of noise generation within non-uniform 
swirling flow turbomachinery flow. 
While the literature presented offers a measure of verification 
and validation through the use of the Method of Exact Solutions, the Method 
of Manufactured Solutions offers a level of code verification which allows
for error checking by computing the approximate order of accuracy for 
a given numerical scheme, which is independent of the final answer, which
in this case is the axial wavenumber. The next chapter will outline the methodology
and techniques used when applying MMS to SWIRL. The methods consist
of unique treatment of boundary conditions using fairing functions as
well as an example of using a summation to generate arbitrary functions 
as manufactured solutions which has the dual benefit of giving a large
number of derivatives but allows for high gradients in specific locations
along the domain of the MS. The use of open-source widely available functions
in Python were used to symbolically create the MS and then 
used to generate FORTRAN code that will compute the MS for code comparison.  

