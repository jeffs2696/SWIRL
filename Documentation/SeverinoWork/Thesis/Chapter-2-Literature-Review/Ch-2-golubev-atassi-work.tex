\section{Golubev and Atassi's work}
Similarly, a narrow annulus is once again 
studied but with a different theoretical and computational approach. The governing equations, similar to Wundrow \cite{Wundrow2019} were still then linearized in terms of potential and rotation. As suggested by Case \cite{Case2004}, a Fourier series analysis was used to conduct the normal mode analysis to find the corresponding wave numbers 
of the eigensystem. The findings show these fall into further classification which were previously denoted as purely convective and 
in part, nearly-convective wave modes. These two classifications of purely convective disturbance can be split into 
their ``nearly-convected vorticity dominated'' and ``nearly-sonic pressure dominated'' parts. These new 
results show the appearance of ``nearly-sonic pressure dominated modes'' which can propagate at varying phase speeds 
throughout the duct in both directions. The imposed Doppler shift from asymmetrical modes cause the sound propagate in the 
opposite direction of the mean flow swirl. A weak coupling relation relates the two and allows for the presence 
of vorticity - pressure mode coupling. Together, these nearly convected modes can be ``identified with the purely 
convective gusts in a non swilirling flow''. For the second group, these ``nearly convected'' vorticity dominated 
modes are further split as these disturbances approach the \textit{critical layer}, i.e. the 
location at which the viscous effects of the boundary layer begin to influence the coupling between modes. When both 
solid body and free vortex induced rotations are in the same direction, no instabilities arise from outside the critical 
layer. It was shown in later works that the influence of centrifugal and Coriolis forces created by the mean 
swirl prevent the decomposition of modes into their potential, rotational and entropic components. The paper ultimately proposes ``
a generalized definition for incident rotational waves(gusts) is proposed which accounts for both the eigenmodes and the 
initial value solutions'' 
%\section{Idea one. Central acoustic theory for unsteady ducted flow}
%\subsubsection{Main Idea - Uniform Mean Flow}
%In ducted turbomachinery, the appearance of unsteady flow has been first understood by characterizing the acoustic response of small disturbances. These small disturbances amount to modal content that govern the behavior of these responses. 
%\subsubsection{Evidence}
%It was shown by (Verdon:1989) that any small disturbance can be represented at a decomposition of acoustic, vortical, and entropic components. However, in the presence of sheared mean axial floww, the acoustic response of the mean flows that arise cannot be described in the same manner and cannot be solved analytically, calling for numerical approaches \cite{Kousen1996}. Preliminary results show that these responses fall into convective and ``non-convective'' classifications.
%\subsubsection{Analysis}
%Complex analysis allows us to observe the phase change in these mode shapes which can comprise the entire continua of the flow domain
%\subsubsection{Lead Out}

%\subsection{Concept}
%\subsubsection{Main Idea}
%The classification of modes into ``convected'', ``non-convected'' and ``nearly convected''
%\subsubsection{Evidence}
%\subsubsection{Analysis}
%\subsubsection{Lead Out}


%Tyler Sofrin selection rule serves well for rotor and stators that are evenly spaced. 
%- Utilizes the fourier transform

