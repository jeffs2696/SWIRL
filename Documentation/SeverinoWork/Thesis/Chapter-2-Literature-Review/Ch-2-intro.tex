\section{Introduction}
% Demonstrate knowledge Justify gap conceptual framework inform the methodology
\subsection{The problem to be addressed and its significance}
The impact of swirling flow has been , there has
been a concerted effort to model sound propagation under such conditions. 
This has lead to a number of publications that attempt to discern the acoustic
signature of swirling flow through a cylindrical domain \cite{Kerrebrock2012} ,
\cite{KERREBROCK1974},  \cite{kousen1996pressure}, 
\cite{kousen1995eigenmode}, \cite{golubev1996sound} ,\cite{Golubev1998},\cite{Golubev1997},
\cite{Tam1998}, \cite{Nijboer2001}, \cite{Guan2009}, \cite{COOPER2001}, 
\cite{Cooper2006}, \cite{Posson2013}, \cite{Heaton2006}. 

\
\subsection{Establish reasoning - i.e. point - of - view for reviewing the literature}


A large amount of aircraft noise was reduced from 1975-2000, 
effectively eliminating the noise pollution for 90\% of the population \cite{FAAPolicy}. 
Since the early 2000's, the advancement in noise reduction technologies has been gradual, 
leaving a requirement for drastic improvement in aeroacoustic treatment 
strategies to compete with the demand of quiet subsonic  flight. 
A previous theoretical review by Envia has been suggested that a non linear 
time domain computation could capture the source generation (incident turbulence) 
in addition to the broadband noise. Such a process would have the capabilities 
of  of solving all components of noise generation in an individual calculation \cite{Envia2004}. 
This would at minimum, require an ``LES-type'' fidelity code, which can tend 
to be computationally expensive.  A promising option is NASA GRC's Broadband 
Aeroacoustic Stator Simulation (BASS).  BASS is a high-order, high 
accuracy computational aeroacoustics (CAA) code which has been used to study 
non linear provides mean an extensive study of non linear phenomena in 
turbomachinery flow, and in particular, mechanisms of noise generation that are 
produced from unsteady disturbances. This code allows for a wide variety of 
finite differencing and time marching schemes as well as artificial dissipation methods. This effective computational tool allows for an in-depth modeling of realistic velocity profiles that would be representative of flow produced from a rotor blade row. In recent work, a new method of implementing realistic, three-dimensional rotor wakes free from acoustics was validated  \cite{Hixon2010}. 
This provides a means of studying acoustic responses non linear swirling flows within pragmatic geometric configurations, while simultaneously allow for the modeling of sources generation produced from the incident turbulence. 

Over the last 40 years, several studies have investigated the impact of 
swirling flow on the sound propagation with the use of the 3D linearized Euler
equations, directly solving the unsteady equations as opposed to approximating 
the solutions to both steady and unsteady flow problems. Another NASA code, 
LINFLUX predict acoustic disturbances from blade movements in subsonic flows. 
While LINFLUX demostrates capabillities to model three dimensional steady and 
unsteady flows, the computational time is larger compared to lower fidelity codes
that only compute the unsteady portion of the problem. The benefit of such a 
code is beneficial to engine and liner designers who are interested in 
a wide range of configurations, requiring a parametric study. SWIRL \cite{kousen1995eigenmode}, 
is a code that coducts an eigenmode analysis by assuming a constant radius annular or 
cylindrical duct with acoustically lined walls using the linearized unsteady Euler
equations. Such a code is needed to identify the modal content and can be 
used as a boundary condition with LINFLUX by using a mode matching technique. 
The goal of this theoretical review is to see the current state of low-fidelity
frequency domain LEE codes and where the foundational model differs depending
on the problem formulation. 

\subsection{Explain the order/sequence of the review}
This review will compare studies that have considered the applicability of 
unsteady linearized Euler equations on cylindrical and annular ducts. First the
work of Kousen will breifly summarized \cite{kousen1995eigenmode,kousen1996pressure}. Secondly, 
the alternate approached shown by Golubev and Atassi used in \cite{Golubev1998,Golubev1997} 
were also compared. These works have utilized a standard normal mode approach 
to determine the modal response of inviscid, compressible, swirling flow 
within a cylindrical and annular ducts. Results and findings have revealed 
three categories of associated wave - modes, acoustic, nearly convected, 
and nearly sonic. Detailed examination of the literature indicate that the 
hierarchical system was not initially apparent. A qualitative description of 
the numerical methods used to evaluate the eigensystem of solutions will be
described.
The use of the Method of Manufactured Solutions will also be described. While 
it's use is widespead in the verification and validation community, very little
aeroacoustic codes utlize the MMS to apply code verification. This review will
discuss the procedures and guidelines associated with the MMS, and some specific
measures that were taken to ensure that the guidelines were met and that the 
suggestions used in literature were expanded upon. 

\subsection{The theoretical foundation of conceptual framework}





 One key question that remains, is how well do unsteady linearized equations capture 
 the mechanisms of noise generation within realistic turbomachinery flow? 
 There exists a copious amount of published work describing the use of wave 
 equations to describe the governing behavior of ducted sound propagation. 
 However, turbomachines inherently rely on high velocities, high temperatures 
 to maximize efficiencies. Such mechanisms need a more thorough formulation so 
 that a complete acoustic description of the flow can be provided. 


To begin exploring various geometries, we must first ask: what are the central 
theories that have been used to explain the acoustic behavior of unsteady swirling flows?
Do the linearized Euler equations provide an adequate means of capturing the mechanisms 
of noise generation within realistic turbomachinery flow?
% Tam proposed a theoretical framework where the modes 
% of a flow are characterized by an initial boundary value problem, 
% allowing for the appearance of ''continuum modes`` \cite{Tarn1998}
