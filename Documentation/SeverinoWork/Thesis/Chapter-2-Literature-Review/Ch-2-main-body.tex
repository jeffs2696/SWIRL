\subsection{Work by Kousen}
The study was expanded by \cite{KERREBROCK1974,YURKOVICH1975} who
included cases of solid body swirl. Wundrow studied the swirling potential flows using Goldstein's disturbance velocity
decomposition \cite{Goldstein1978} using a numerical approach and found that the solutions
were more accurate and found more efficiently \cite{KERREBROCK1974}. Kousen expanded these efforts by including \cite{kousen1996pressure,kousen1995eigenmode} the effect unsteady disturbances in the presence of 
forced solid body-swirl and free-vortex flow without the use of potential flow theory. Using normal 
mode analysis along with a radial spatial differencing scheme, the wave modes produced within cylindrical and annular ducts was 
reported. Results show (figure 4.7) two distinct families of modes, purely convected and acoustic 
modes. The presence of axial shear in combination solid body flows can endue coupling between these modes, which 
in theory would not appear unless viscous terms were present in the eigenvalue analysis. Lack of available swirl flow 
results ``hampered'' validation attempts. However, this eigenvalue approach was utilized with a new quasi-3D 
formulation to find the axial wavenumbers from solid body flow was proposed in his next work . 

In \cite{kousen1995eigenmode}, A quasi 3D formulation was validated and used to predict 
the appearance of modes due to the interaction of a rotor with spatially 
uniform steady and unsteady flow. These modes were first classified by 
\cite{Tyler1962} as ``spinning modes''. In addition, further investigation was 
done on the results shown in \cite{kousen1995eigenmode}. The axial wavenumbers that 
were previously found to be purely convective were shown to be in part
,``nearly convective'' (shear) pressure modes. These were found to propagate in the 
axial direction with no loss in amplitude, thus never satisfying the cut-off condition. 

