%        File: AnalyticalDuctModes.tex
%     Created: Fri Feb 11 10:00 PM 2022 E
% Last Change: Fri Feb 11 10:00 PM 2022 E
%
%\documentclass[a4paper]{report}
%\usepackage{mathtools}
%\begin{document}
%
\section{No Flow}



Starting with equation 2.28 (Wave Equation) in Kousen's paper,



\begin{equation}
    \frac{1}{A^2}\frac{D^2\tilde{p}}{Dt^2} -
    \nabla^2 \tilde{p} =
    2 \bar{\rho} \frac{d V_x}{d x} \frac{\partial  \tilde{v}_r}{ \partial x} 
    \label{eqn:KousensWaveEquation}
\end{equation}


lets look at the no flow case. In the case of sheared flow, $dV_x/dx = 0$ the right hand side will be zero 



\begin{align*}
    \frac{1}{A^2}\left(
        \frac{\partial^2 \tilde{p}}{\partial t^2} + 
        \vec{V}\cdot \vec {\nabla} (\tilde{p}) 
    \right) -
    \nabla^2
    \tilde{p} &=
    0 \\
\end{align*}

Substituting the definitions for $\nabla$ and $\nabla^2$ in cylindrical 
coordinates gives,

\begin{align*} 
    \frac{1}{A^2}\left(
        \frac{\partial^2 \tilde{p}}{\partial t^2}
    + 
        \vec{V}\cdot \left(
            \frac{\partial\tilde{p}}{\partial t} + 
            \frac{1}{\tilde{r}}\frac{\partial \tilde{p} }{\partial \tilde{r}} +
            \frac{\partial \tilde{p}}{\partial \theta} +
            \frac{\partial \tilde{p}}{\partial x}  
        \right)  \right)-
        \left(
            \frac{\partial^2 \tilde{p}}{\partial t^2} + 
            \frac{1}{\tilde{r}}\frac{\partial \tilde{p}}{\partial r} +
            \frac{1}{\tilde{r}^2} \frac{\partial^2 \tilde{p}}{\partial \theta^2} + 
            \frac{\partial^2 \tilde{p}}{\partial x^2} 
        \right) &= 0  
\end{align*} 
Setting $\vec{V} = 0$,

\begin{align*} 
    \frac{1}{A^2}\left(
        \frac{\partial^2 \tilde{p}}{\partial t^2}
    \right) - 
        \left(
            \frac{\partial^2 \tilde{p}}{\partial t^2} + 
            \frac{1}{\tilde{r}}\frac{\partial \tilde{p}}{\partial  r}  +
            \frac{1}{\tilde{r}^2} \frac{\partial^2 \tilde{p}}{\partial \theta^2} + 
            \frac{\partial^2 \tilde{p}}{\partial x^2} 
        \right) &= 0  
\end{align*} 
Recall, $\tilde{p} = p/\bar{\rho} A^2$. To dimensionalize the equation, this is
substituted and both sides are multiplied by $\bar{\rho}A^2$,


\begin{align*} 
    \frac{1}{A^2}\left(
        \frac{\partial^2 {p}}{\partial t^2}
    \right) - 
        \left(
            \frac{\partial^2 {p}}{\partial t^2} + 
            \frac{1}{\tilde{r}}\frac{\partial p}{\partial r} +
            \frac{1}{\tilde{r}^2} \frac{\partial^2 p}{\partial \theta^2} + 
            \frac{\partial^2 p}{\partial x^2} 
        \right) &= 0  
\end{align*} 

The process of separation of variables(seperation indeterminatarum)
was first written and formalized by John Bernoulli in a letter to Leibniz. The method
of separation of variables requires an assumed solution as well as initial and boundary 
conditions. For a partial differential equation, the assumed solution can be a 
linear combination of solutions to a system of ordinary differential equations that
comprises the partial differential equation. Since $p$ is a function of four
variables, the solution is assumed to be a linear combination of four solutions.
Each solution is assumed to be Euler's identity, a common ansant for linear partial 
differential equations and boundary conditions.

Defining,

\begin{equation}
    p(x,r,\theta,t) = X(x) R(r) \Theta(\theta) T(t)
\end{equation}

where, 

\begin{align*}
    X(x) &=
    A_1 e^{ik_x x} +
    B_1 e^{-ik_x x }\\
    \Theta(\theta) &=
    A_2 e^{i k_{\theta} \theta } +
    B_2 e^{-ik_{\theta} \theta }\\
    T(t) &=
    A_3 e^{i \omega t } +
    B_3 e^{-i\omega t  }
\end{align*}

The next step is to rewrite the wave equation in terms of $X$, $R$, $\Theta$,
and $T$. To further simplify the result, each term is divided by $p$.
Before the substitution, the derivatives of the assumed solutions need to be
evaluated.


\subsubsection{Temporal Derivatives}

\begin{align*}
    \frac{\partial p}{\partial t} 
    &=
    \frac{\partial }{\partial t}  \left( XR\Theta T \right) \\
    &=
    XR\Theta\frac{\partial T}{\partial t}  
\end{align*}


\begin{align*}
    \frac{1}{p}\frac{\partial p}{\partial t} 
    &=
    \frac{ 1}{X R \Theta T}  \left( XR\Theta\frac{\partial T}{\partial t} \right) \\
    &=\frac{ 1}{ T}\frac{\partial T}{\partial t}  
\end{align*}

\begin{align*}
    \frac{\partial^2 p}{\partial t^2} 
    &=
    \frac{\partial^2 }{\partial t^2}  \left( XR\Theta T \right) \\
    &=
    XR\Theta\frac{\partial^2 T}{\partial t^2}  
\end{align*}


\begin{align*}
    \frac{1}{p}\frac{\partial^2 p}{\partial t^2} 
    &=
    \frac{ 1}{X R \Theta T}  \left( XR\Theta\frac{\partial^2 T}{\partial t^2} \right) \\
    &=\frac{ 1}{ T}\frac{\partial^2 T}{\partial t^2}  
\end{align*}

\begin{align*}
    \frac{\partial T}{\partial t} &=
    \frac{\partial}{\partial t}
        \left( 
        A_3 e^{i \omega t} + B_3 e^{-i \omega t}
    \right)  \\
    &=
    \frac{\partial}{\partial t} \left(A_3 e^{i \omega t}  \right) +
    \frac{\partial}{\partial t} \left(B_3 e^{-i \omega t}  \right)\\ 
    &= i \omega A_3 e^{i \omega t} - i \omega B_3 e^{i \omega t} 
\end{align*}

\begin{align*}
    \frac{\partial^2 T}{\partial t^2} &=
    \frac{\partial^2}{\partial t^2}
        \left( 
        i \omega A_3 e^{i \omega t} + i \omega B_3 e^{-i \omega t}
    \right)  \\
    &=
    \frac{\partial^2}{\partial t^2} \left(i \omega A_3 e^{i \omega t}  \right) +
    \frac{\partial^2}{\partial t^2} \left(- i \omega B_3 e^{-i \omega t}  \right)\\ 
    &= (i \omega)^2 A_3 e^{i \omega t} - (i \omega)^2 B_3 e^{i \omega t} 
\end{align*}

\begin{align*}
    \frac{1}{T}\frac{\partial^2 T}{\partial t^2} 
    &=
    (i\omega)^2 \\
    &= -\omega^2
\end{align*}


\subsubsection{Radial Derivatives}
\begin{align*}
    \frac{\partial p}{\partial r} 
    &=
    \frac{\partial }{\partial r}  \left( XR\Theta T \right) \\
    &=
    X\Theta T\frac{\partial R}{\partial r}  
\end{align*}


\begin{align*}
    \frac{1}{p}\frac{\partial p}{\partial r} 
    &=
    \frac{ 1}{X R \Theta T}  \left( X\Theta T\frac{\partial R}{\partial r} \right) \\
    &=\frac{ 1}{ R}\frac{\partial R}{\partial r}  
\end{align*}

\begin{align*}
    \frac{\partial^2 p}{\partial r^2} 
    &=
    \frac{\partial^2 }{\partial r^2}  \left( XR\Theta T \right) \\
    &=
    X\Theta T\frac{\partial^2 R}{\partial r^2}  
\end{align*}


\begin{align*}
    \frac{1}{p}\frac{\partial^2 p}{\partial r^2} 
    &=
    \frac{ 1}{X R \Theta T}  \left( X\Theta T \frac{\partial^2 R}{\partial r^2} \right) \\
    &=\frac{ 1}{ R}\frac{\partial^2 R}{\partial r^2}  
\end{align*}
The radial derivatives will be revisited once the remaining derivatives are evaluated,

\subsubsection{Tangential Derivatives}

\begin{align*}
    \frac{\partial p}{\partial \theta } 
    &=
    \frac{\partial }{\partial t}  \left( XR\Theta T \right) \\
    &=
    XRT\frac{\partial \Theta}{\partial \theta}  
\end{align*}


\begin{align*}
    \frac{1}{p}\frac{\partial p}{\partial \theta} 
    &=
    \frac{ 1}{X R \Theta T}  \left( XR\Theta\frac{\partial T}{\partial \theta} \right) \\
    &=\frac{ 1}{ \Theta}\frac{\partial \Theta}{\partial \theta}  
\end{align*}

\begin{align*}
    \frac{\partial^2 p}{\partial \theta^2} 
    &=
    \frac{\partial^2 }{\partial \theta^2}  \left( XR\Theta T \right) \\
    &=
    XRT\frac{\partial^2 \Theta }{\partial \theta^2}  
\end{align*}


\begin{align*}
    \frac{1}{p}\frac{\partial^2 p}{\partial \theta^2} 
    &=
    \frac{ 1}{X R \Theta T}  \left( XRT\frac{\partial^2 \Theta}{\partial \theta^2} \right) \\
    &=\frac{ 1}{ \Theta}\frac{\partial^2 \Theta}{\partial \theta^2}  
\end{align*}

\begin{align*}
    \frac{\partial \Theta}{\partial \theta} &=
    \frac{\partial}{\partial \theta}
        \left( 
            A_2 e^{i k_{\theta} \theta} + B_2 e^{-i k_{\theta} \theta}
        \right)  \\
    &=
    \frac{\partial}{\partial \theta} \left(A_2 e^{i k_{\theta} \theta}  \right) +
    \frac{\partial}{\partial \theta} \left(B_2 e^{-i k_{\theta} \theta}  \right)\\ 
    &= i k_{\theta} A_2 e^{i k_{\theta} \theta} - i k_{\theta} B_2 e^{i k_{\theta} \theta} 
\end{align*}

\begin{align*}
    \frac{\partial^2 \Theta }{\partial \theta^2} &=
    \frac{\partial^2}{\partial \theta^2}
        \left( 
        i k_{\theta} A_2 e^{i k_{\theta} \theta} - i k_{\theta} B_2 e^{i k_{\theta} \theta} 
    \right)  \\
    &=
    \frac{\partial^2}{\partial \theta^2} \left(i k_{\theta} A_2 e^{i k_{\theta} \theta}  \right) +
    \frac{\partial^2}{\partial \theta^2} \left(- i k_{\theta} B_2 e^{-i k_{\theta} \theta}  \right)\\ 
    &= (i k_{\theta})^2 A_2 e^{i k_{\theta} \theta } - (i k_{\theta})^2 B_2 e^{i k_{\theta} \theta} 
\end{align*}

\begin{align*}
    \frac{1}{\Theta}\frac{\partial^2 \Theta}{\partial \theta^2} 
    &=
    (ik_{\theta})^2 \\
    &= -k_{\theta}^2
\end{align*}

\subsubsection{Axial Derivatives}

\begin{align*}
    \frac{\partial p}{\partial x} 
    &=
    \frac{\partial }{\partial x}  \left( XR\Theta T \right) \\
    &=
    R\Theta T \frac{\partial X}{\partial x}  
\end{align*}


\begin{align*}
    \frac{1}{p}\frac{\partial p}{\partial x} 
    &=
    \frac{ 1}{X R \Theta T}  \left( R\Theta\frac{\partial X}{\partial x} \right) \\
    &=\frac{ 1}{ X}\frac{\partial X}{\partial x}  
\end{align*}

\begin{align*}
    \frac{\partial^2 p}{\partial x^2} 
    &=
    \frac{\partial^2 }{\partial x^2}  \left( XR\Theta T \right) \\
    &=
    R\Theta T \frac{\partial^2 X}{\partial x^2}  
\end{align*}


\begin{align*}
    \frac{1}{p}\frac{\partial^2 p}{\partial x^2} 
    &=
    \frac{ 1}{X R \Theta T}  \left( R\Theta T \frac{\partial^2 X}{\partial x^2} \right) \\
    &=\frac{ 1}{ X}\frac{\partial^2 X}{\partial x^2}  
\end{align*}

\begin{align*}
    \frac{\partial X}{\partial x} &=
    \frac{\partial}{\partial t}
        \left( 
        A_3 e^{i k_x t} + B_3 e^{-i \omega t}
    \right)  \\
    &=
    \frac{\partial}{\partial t} \left(A_1 e^{i k_x x}  \right) +
    \frac{\partial}{\partial t} \left(B_1 e^{-i k_x x }  \right)\\ 
    &= i k_x A_1 e^{i k_x x } - i k_x B_1 e^{i k_x x} 
\end{align*}

\begin{align*}
    \frac{\partial^2 X}{\partial x^2} &=
    \frac{\partial^2}{\partial x^2}
        \left( 
        i k_x A_1 e^{i k_x x} + i k_x B_1 e^{-i k_x x}
    \right)  \\
    &=
    \frac{\partial^2}{\partial x^2} \left(i k_x A_1 e^{i k_x x}  \right) +
    \frac{\partial^2}{\partial x^2} \left(- i k_x B_1 e^{-i k_x x}  \right)\\ 
    &= (i k_x)^2 A_1 e^{i k_x x} - (i k_x)^2 B_1 e^{i k_x x} 
\end{align*}

\begin{align*}
    \frac{1}{X}\frac{\partial^2 X}{\partial x^2} 
    &=
    (i k_x)^2 \\
    &= -k_x^2
\end{align*}

Substituting this back into the wave equation yields ,



\begin{align*} 
    \frac{1}{A^2}\left(
        \frac{\partial^2 {p}}{\partial t^2}
    \right) &= 
        \left(
            \frac{\partial^2 {p}}{\partial t^2} + 
            \frac{1}{\tilde{r}}\frac{\partial p}{\partial r} +
            \frac{1}{\tilde{r}^2} \frac{\partial^2 p}{\partial \theta^2} + 
            \frac{\partial^2 p}{\partial x^2} 
        \right) 
\end{align*} 

\begin{equation}
    \frac{1}{A^2} \frac{1}{T}\frac{\partial^2 T}{\partial t^2} = 
    \frac{1}{R}\frac{\partial^2 R}{\partial r^2 } +
    \frac{1}{r}\frac{1}{R}\frac{\partial R}{\partial r}  + 
    \frac{1}{r^2}\frac{1}{\Theta}\frac{\partial \Theta}{\partial \theta} + 
    \frac{1}{X}\frac{\partial^2 X}{\partial x^2}
    \label{eqn:waveode}
\end{equation}

Notice that each term is only a function of its associated independent variable.
So, if we vary the time, only the term on the left-hand side can vary. However,
since none of the terms on the right-hand side depend on time, that means the
right-hand side cannot vary, which means that the ratio of time with its second
derivative is independent of time. The practical upshot is that each of these 
terms is constant, which has been shown. The wave numbers are the \textit{separation constants} 
that allow the PDE to be split into four separate ODE's. Substituting the separation constants 
into Equation (\ref{eqn:waveode}) gives, 


\begin{equation}
    -\frac{\omega^2}{A^2}  = 
    \frac{1}{R}
    \left(      
    \frac{\partial^2 R}{\partial r^2 } +
    \frac{1}{r}\frac{\partial R}{\partial r}  
\right) -
    \frac{k_{\theta}^2}{r^2}-  
    k_x^2
    \label{eqn:waveode2}
\end{equation}
Note that the dispersion relation states $\omega = k A$

\begin{equation}
    \frac{1}{R}
    \left(      
    \frac{\partial^2 R}{\partial r^2 } +
    \frac{1}{r}\frac{\partial R}{\partial r}  
\right) -
    \frac{k_{\theta}^2}{r^2}-  
    k_x^2 + k^2 = 0
    \label{eqn:waveode3}
\end{equation}
The remaining terms are manipulated to follow the same form as \textit{Bessel's Differntial 
Equation} ,

\begin{equation}
    x^2 \frac{d^2 y}{dx^2} + x \frac{dy }{dx } + (x^2 - n^2) y = 0
    \label{eqn:besselODE}
\end{equation}

The general solution to Bessel's differential equation is a linear combination of
the Bessel functions of the first kind, $J_n(x)$ and of the second kind, $Y_n(x)$ 
\cite{wolphram:bessel}. The subscript $n$ refers to the order of Bessel's equation.

\begin{equation}
    y(x) = AJ_n(x) + BY_n(x)
    \label{eqn:besselsolution}
\end{equation}

By rearranging Equation (\ref{eqn:waveode3}), a comparison can be made to Equation
(\ref{eqn:besselODE}) to show that the two equations are of the same form. 

The first step is to revisit the radial derivatives that have not been addressed.
As was done for the other derivative terms, the radial derivatives will also 
be set equal to a separation constant, $-k_r^2$. 

\begin{align}
    \underbrace{\frac{1}{R}
    \left(      
    \frac{\partial^2 R}{\partial r^2 } +
    \frac{1}{r}\frac{\partial R}{\partial r}  
\right) -
    \frac{k_{\theta}^2}{r^2}}_{-k_r^2}-  
    k_x^2 + k^2 = 0
    \label{eqn:wavenumber_without_kr}
\end{align}

The reader may be curious as to why the tangential separation constant $k_{\theta}$ is 
included within the definition of the radial separation constant. 

Recall the ODE for the tangential direction, 

\begin{align*}
    \frac{\partial \Theta}{\partial \theta} \frac{1}{\Theta} = - k_{\theta}^2\\
    \frac{\partial \Theta}{\partial \theta} \frac{1}{\Theta} + \Theta k_{\theta}^2 = 0 
\end{align*}

where the solution is more or less,

\begin{align*}
    \Theta(\theta) = e^{i k_{\theta} \theta}
\end{align*}

In order to have non trivial, sensible solutions, the value of $\Theta(0)$ and
$\Theta(2\pi)$ need to be the same, and this needs to be true for any multiple 
of $2\pi$ for a fixed r. Taking $\Theta$ to be one, a unit circle, it can be shown that the domain
is only going to be an integer multiple. Therefore, there is an implied periodic
azimuthal boundary condition, i.e. $0<\theta\leq 2 \pi$ and $k_{\theta}=m$. 

Continuing with the radial derivatives\ldots


\begin{align*}
    -k_r^2 =\frac{1}{R}
    \left(      
    \frac{\partial^2 R}{\partial r^2 } +
    \frac{1}{r}\frac{\partial R}{\partial r}  
\right) -
    \frac{m^2}{r^2} 
\end{align*}
To further simplify, the chain rule is used to do a change of variables, $x = k_r r$
\begin{align*}
    \frac{\partial R}{\partial r} &= \frac{dR}{dx}\frac{dx}{dr}\\
    &=
    \frac{dR}{dx}\frac{d}{dr}\left( k_r r \right) \\
    &=
    \frac{dR}{dx} k_r 
\end{align*} 


\begin{align*}
    \frac{\partial^2 R}{\partial r^2} &= \frac{d^2R}{dx^2}\left(\frac{dx}{dr}\right)^2 + 
    \frac{dR}{dr}\frac{d^2x}{dr^2}\\
    &=
    \frac{d^2R}{dx^2}\frac{d}{dr} k_r^2 + k_r \frac{d^2r}{dr^2}\\
    &=
    \frac{d^2R}{dx^2}\frac{d}{dr} k_r^2
\end{align*} 

Substituting this into Equation (\ref{eqn:waveode3}),
\begin{equation}
    \left(\frac{d^2R}{dx^2}k_r^2 +
    \frac{1}{r}\frac{d^2R}{dx^2}k_r\right) +
    \left(k_r^2 - \frac{m^2}{r^2}\right)R
    \label{eqn:waveode4}
\end{equation}
Dividing Equation \ref{eqn:waveode4} by $k_r^2$,

\begin{equation}
    \left(\frac{d^2R}{dx^2} +
    \frac{1}{k_r r}\frac{d^2R}{dx^2}\right) +
    \left(1  - \frac{m^2}{k_r^2 r^2}\right)R
    \label{eqn:waveode5}
\end{equation}

\begin{equation}
    \left(\frac{d^2R}{dx^2} +
    \frac{1}{x^2}\frac{d^2R}{dx^2}\right) +
    \left(1  - \frac{m^2}{x^2}\right)R
    \label{eqn:waveode6}
\end{equation}

Multiplying Equation (\ref{eqn:waveode6}) by $x^2$ gives,

\begin{equation}
    \frac{d^2R}{dr^2}x^2 + 
    \frac{dR}{dr}x + 
    \left( x^2 - m^2 \right)R
    \label{eqn:finalradialode}
\end{equation}
which matches the form of Bessel's equation

Therefore, the solution goes from this,
\begin{equation}
    y(x) = AJ_n(x) + BY_n(x)
    \label{eqn:besselsolution}
\end{equation}
to this,


\begin{equation}
    R(r) = (AJ_n(k_r r) + BY_n(k_r r)) 
    \label{eqn:besselsolution}
\end{equation}
where the coefficients $A$ and $B$ are found after applying radial
boundary conditions. %and there is an exponential dependence. 




\subsubsection{Hard Wall boundary condition}
\begin{align*}
    \frac{\partial p}{\partial r}|_{r = r_{min}}  =\frac{\partial p}{\partial r}|_{r = r_{max}} = 0 \rightarrow 
    \frac{\partial}{\partial r} \left( X\Theta T R \right) &= 0 \\
    X \Theta T\frac{\partial R}{\partial r}  &= 0 \\
    \frac{\partial R}{\partial r}  &= 0 
\end{align*}

where,


\begin{align*} 
    \frac{ \partial R}{\partial r}|_{r_{min}} &= AJ_n'(k_r r_{min}) + B Y_n' (k_r r_{min}) = 0 
    \rightarrow B = -A \frac{J_n'(k_r r_{min})}{Y_n'(k_r r_{min})}
\end{align*}


\begin{align*} 
    \frac{ \partial R}{\partial r} &= AJ_n'(k_r r_{max}) + B Y_n' (k_r r_{max}) = 0 \\
                                   &= AJ_n'(k_r r_{max}) - A\frac{J_n' (k_r r_{min})}{Y_n'(k_r r_{min})} Y_n' (k_r r_{max}) = 0 \\
                                   &= \frac{J_n'(k_r r_{min})}{J_n' (k_r r_{max})} - \frac{Y_n'(k_r r_{min})}{Y_n' (k_r r_{max})} = 0 
\end{align*}
where $k_r r$ are the zero crossings for the derivatives of the Bessel functions of the first and second kind.

In summary, the wave equation for no flow in a hollow duct with hard walls is obtained 
from Equation (\ref{eqn:wavenumber_without_kr}).
\begin{equation}
    k^2 = k_r^2 + k_x^2
    \label{eqn:wavenumber_equation}
\end{equation}


Solving for the axial wavenumber gives,
\section{Uniform Flow}

To get the same equation but for uniform flow, the same procedure can be followed.

Starting with Equation 2.27 redimensionalized, 

\begin{align*}
    \frac{ d^2 \tilde{p}}{d \tilde{r}^2} +
    \frac{1}{\tilde{r}} 
    \frac{d \tilde{p}}{d \tilde{r}} + 
    \frac{2 \bar{\gamma} \left( \frac{d m_x}{d \tilde{r}} \right)}
    {\left( k - \bar{\gamma} m_x \right)}\frac{d \tilde{p}}{d \tilde{r}}+
    \left[ \left( k - \bar{\gamma} m_x \right)^2 - \frac{m^2}{\tilde{r}^2}- 
    \bar{\gamma}^2 \right] \tilde{p}
\end{align*}

Let's separate the new terms from the old ones, 

\begin{align*}
    \frac{ d^2 \tilde{p}}{d \tilde{r}^2} +
    \frac{1}{\tilde{r}} 
    \frac{d \tilde{p}}{d \tilde{r}} + 
    \frac{2 \bar{\gamma} \left( \frac{d m_x}{d \tilde{r}} \right)}
    {\left( k - \bar{\gamma} m_x \right)}\frac{d \tilde{p}}{d \tilde{r}}+
    \left[ \left( k - \bar{\gamma} m_x \right)^2 - \frac{m^2}{\tilde{r}^2}- 
    \bar{\gamma} \right] \tilde{p}
\end{align*}


Recalling the non-dimensional definitions,
\begin{align*}
    \tilde{p} &= \frac{p}{\bar{\rho} A^2} \\
    \tilde{r} &= \frac{r}{r_T} \\
    \frac{\partial \tilde{p}}{\partial \tilde{r}} &= 
    \frac{ \partial \tilde{p}}{\partial r} \frac{\partial r}{ \partial \tilde{r}}  \\ 
    &= \frac{ \partial \tilde{p}}{\partial r} \frac{\partial }{ \partial \tilde{r}} \left( \tilde{r} r_T \right) \\
    &= 
    \frac{ \partial \tilde{p}}{\partial r}  r_T \\
    \frac{\partial^2 \tilde{p}}{\partial \tilde{r}^2} &= 
    \frac{ \partial^2 \tilde{p}}{\partial r^2}  (r_T)^2+ 
    \frac{ \partial \tilde{p}}{\partial r} \frac{\partial^2 r}{ \partial \tilde{r}^2} \\
    &= \frac{ \partial^2 \tilde{p}}{\partial r^2}  (r_T)^2 
\end{align*}

\begin{align*}
    \frac{\partial}{\partial r} \left( \frac{p}{\bar{\rho} A^2} \right) 
    &=
    \frac{\left(\frac{\partial}{\partial r} \left(  p\right) \bar{\rho} A^2 - 
    \underbrace{\frac{\partial \bar{\rho}A^2}{\partial r}}_0 p \right)}{\left( \bar{\rho} A^2 \right)^2}\\ 
    &= \frac{1}{\bar{\rho}A^2} \frac{\partial p}{\partial r}
\end{align*}

\begin{align*}
    \frac{ d^2 \tilde{p}}{d \tilde{r}^2} +
    \frac{1}{\tilde{r}} 
    \frac{d \tilde{p}}{d \tilde{r}}- 
    \frac{m^2}{\tilde{r}^2}\tilde{p}- 
    \bar{\gamma}^2  \tilde{p}
 + 
    \frac{2 \bar{\gamma} \left( \frac{d M_x}{d \tilde{r}} \right)}
    {\left( k - \bar{\gamma} M_x \right)}\frac{d \tilde{p}}{d \tilde{r}}+
    \left( k - \bar{\gamma} M_x \right)^2\tilde{p} 
\end{align*}

If there is only uniform flow, then $dM_x/dr = 0$,

\begin{align*}
    \frac{ d^2 \tilde{p}}{d \tilde{r}^2} +
    \frac{1}{\tilde{r}} 
    \frac{d \tilde{p}}{d \tilde{r}}- 
    \frac{m^2}{\tilde{r}^2}\tilde{p}- 
    \bar{\gamma}^2  \tilde{p}
 + 
    \left( k - \bar{\gamma} M_x \right)^2\tilde{p} 
\end{align*}

Re-dimensionalizing,

\begin{align*}
    \frac{1}{\bar{\rho} A^2}\left[
    \frac{ d^2 p}{d r} r_T^2+
    \frac{r_T}{r} 
    \frac{d p}{d r} r_T - 
    \frac{m^2}{r^2}r_T^2 p - k_x^2r_T^2  p\right]
    + \left( \frac{\omega }{A}r_T - k_x r_T M_x \right)^2p 
\end{align*}

Expanding the last term and substituting $\omega/A = k$

\begin{align*}
    \frac{1}{\bar{\rho} A^2}\left[
    \frac{ d^2 p}{d r} r_T^2+
    \frac{r_T}{r} 
    \frac{d p}{d r} r_T - 
    \frac{m^2}{r^2}r_T^2 p - k_x^2r_T^2  p\right]
    +\left( r_T^2\left(
        k^2 - 2 k k_x M_x + k_x^2 M_x^2 \right)
    \right)p 
\end{align*}
Canceling out $r_T/\bar{\rho}A$ in every term


\begin{align*}
    \frac{ d^2 p}{d r} +
    \frac{1}{r} 
    \frac{d p}{d r} + \left[ 
    k^2 - 2 k k_x M_x + k_x^2 M_x^2- \frac{m^2}{r^2}  - k_x^2\right]p 
\end{align*}

Continue here,


Defining 

$$- N^2 = k_x^2 M_x^2 - 2 k k_x M_x - k_x^2 $$
$$-N^2 = -(1 -  M_x^2)k_x^2 - 2 k k_x M_x  $$
$$-N^2 =  -\beta^2 k_x^2 - 2 k k_x M_x  $$


\begin{align*}
    \frac{ d^2 p}{d r} +
    \frac{1}{r} 
    \frac{d p}{d r} + \left[ 
    k^2 - N^2 - \frac{m^2}{r^2}  \right]p 
\end{align*}

Let $k_r^2 = k^2 - N^2$


\begin{align*}
    \frac{ d^2 p}{d r} +
    \frac{1}{r} 
    \frac{d p}{d r} + \left[ 
    k_r^2  - \frac{m^2}{r^2}  \right]p 
\end{align*}

Looking at the radial wavenumber,

\begin{align*}
    k_r^2 &= k^2 - N^2 \\
          &= k^2-\beta^2 k_x^2 - 2 k k_x M_x \\
    0 &=  -\beta ^2 k_x ^2 -  \left( 2M_x k \right)k_x +(k^2 - k_r^2)
\end{align*}

Where the roots to this equation are the axial wavenumber,


Applying the quadratic formula and taking 

\begin{align*}
    A &= - \beta^2 \\
    B &= - 2M_x k\\
    C &= k^2 - k_2^2
\end{align*} 

Note B is negative when $M_x$ is positive,

(I feel like N should change based on $M_x's$ sign)

\begin{align*}
    k_x &= \frac{2M_x k \pm \sqrt{4 M_x^2 k^2 + 4 \beta^2 \left( k^2 - k_r^2 \right)}}{-2\beta^2}\\
        &= \frac{-M_x k \pm \sqrt{k^2 - k_r^2}}{\beta^2}
\end{align*}

\section{Annular Duct Axial Wavenumber solution}



This needs to be proven,


In \cite{Amr2001}, the axial wavenumber for annular ducts is reported,

\begin{equation*}
    \frac{-(\omega - m M_{\theta})M_x \pm \sqrt{\left( \omega - m M_{\theta}^2 \right) - \beta\left( m^2 + \Gamma_{m,n} ^2 \right)}}{beta^2}
\end{equation*}

where 

$\Gamma_{m,n}= \frac{n^2 \pi^2}{\left( r_{max} - r_{min} \right)^2}$

%$\bibliographystyle{plain}
%\bibliography{references}
%\end{document}


