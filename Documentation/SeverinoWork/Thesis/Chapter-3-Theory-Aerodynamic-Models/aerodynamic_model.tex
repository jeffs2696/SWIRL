\section{Introduction}

This chapter will outline the steady and unsteady aerodynamic
models used for this study. The MMS procedure as it is used in this study will 
be described. The summation method used to generate symbolic expression will 
be breifly described. This chapter will also present the use of fairing functions 
to impose the equivallent boundary conditions used in the numerical approximation.
The procedure for calculating the approximated rate of convergence for a system
of equations is also presented.

\section{Modal Propagation Theroetical Background}
\subsection{Aerodynamic Model}

The governing equations for an isentropic ideal gas are the conservation of 
mass, momentum and energy respectively along with the the constitutive relation 
for the speed of sound. For a cylindrical duct, the coordiate system consists
of a radial, tangential and axial components.  

% The following assumptions are utlized to simplify the model 

% \begin{itemize}
%     \item Steady flow does not fluctuate in time $\partial / \partial t = 0$
% \end{itemize}

% \begin{equation}
%     \vec{V} \bar{\rho} = 0
%     \label{eqn:consOfMass}
% \end{equation}

\begin{align}
\frac{\partial \rho}{\partial t} + %Conservation of mass
v_r \frac{\partial \rho}{\partial r} +
\frac{v_{\theta}   }{r}
\frac{\partial \rho}{\partial \theta} +
v_x \frac{\partial \rho}{\partial \theta} + 
\rho 
\left(
\frac{1}{r} \frac{\partial (rv_r)	}{\partial r} +
\frac{1}{r}	\frac{\partial v_{\theta}}{\partial \theta} +
\frac{\partial v_x}{\partial x}
\right) 
&= 0 \\% \label{ConservationOfMass} %%%%%%%%%%%%%%%%%%%%%%%%%%%%%%%%%%%%%%
\frac{\partial v_r}{\partial t} + 
v_r \frac{\partial v_r}{\partial r} +
\frac{v_{\theta}  }{r}
\frac{\partial v_r}{\partial \theta}- \frac{v_{\theta}^2}{r}+ 
v_x \frac{\partial v_r}{\partial x} 
&= -\frac{1}{\rho} \frac{\partial p}{\partial r}\\  
\frac{\partial v_{\theta}}{\partial t} + 
v_r \frac{\partial v_{\theta}}{\partial r} +
\frac{v_{\theta}}{r}
\frac{\partial v_{\theta}}{\partial \theta} +
\frac{v_r v_{\theta}}{r}+ 
v_x \frac{\partial v_{\theta}}{\partial x} 
&= -\frac{1}{\rho r} \frac{\partial p}{\partial \theta}\\ 
\frac{\partial v_{x}}{\partial t} + 
v_r 
\frac{\partial v_x}{\partial r} +
\frac{v_{\theta}}{r}
\frac{\partial v_x}{\partial \theta}+ 
v_x \frac{\partial v_x}{\partial x} 
&= 
-\frac{1}{\rho } 
\frac{\partial p}{\partial x}\\  
\frac{\partial p }{\partial t} +
v_r 
\frac{\partial p}{\partial r} +
\frac{v_{\theta}}{r}
\frac{\partial p}{\partial \theta} +
v_x \frac{\partial p}{\partial \theta} + 
\gamma p 
\left(
\frac{1}{r}\frac{\partial (rv_r)}{\partial r} +
\frac{1}{r}\frac{v_{\theta}}{\partial \theta} +
\frac{\partial v_x}{\partial x}
\right) &= 0
\end{align}
The following assumptions to simplify the aerodynamic model for the steady mean
flow case,

\begin{itemize}
    \item No flow in the radial direction. Consequentially, the flow is 
        axisymmetric along the downstream direction.
    \item No surface or body forces
    \item Isentropic conditions 
\end{itemize}

\subsubsection{Steady Flow}
For steady flow, the continuity, momentum and entropy equations are

\begin{align}
    \nabla (\vec{V} \bar{\rho}) &=  0 \\
(\vec{V}\cdot \nabla) \vec{V} &=  0\\
\nabla S = 0
\end{align}
\[\]

If the radial velocity is neglected, the velocity vector in cylindrical coordinates 
become,
\[\vec{V}(r,\theta,x) = V_x(r) \hat{e}_x + V_{\theta} (r) \hat{e}_{\theta} \] 
where $\hat{e}_x$ and $\hat{e}_{\theta}$ are unit vectors for the axial and 
tangential directions. The next section will present general mean flows that 
follow the form of the velocity vector shown. 

 
%Starting with the radial momentum equation, 
%\begin{align*}
%\frac{\partial v_r}{\partial t} + 
%v_r \frac{\partial v_r}{\partial r} +
%\frac{v_{\theta}  }{r}
%\frac{\partial v_r}{\partial \theta}- \frac{v_{\theta}^2}{r}+ 
%v_x \frac{\partial v_r}{\partial x} 
%&= -\frac{1}{\rho} \frac{\partial p}{\partial r} 
%\end{align*}
%See Appendix for speed of sound derivation

