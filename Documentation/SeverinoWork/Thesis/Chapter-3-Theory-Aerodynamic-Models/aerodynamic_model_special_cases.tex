% \section{Applying model to various flows}
% The LEE for flows ith axial sheared flow, solid body and free vortex swirl
% were reviewed by \cite{kousen1995eigenmode}, and most recently studied by \cite{Maldonado2016}.
\section{Nonuniformities from swirling mean flow}

% \[P = \int_{\}^{1} \frac{\bar{\rho} V_{\theta}^2}{\tilde{r}} d\tilde{r}\] 
% % \subsection{Axial Shear Flow}
% % In \cite{kousen1995eigenmode}, axial sheared flows through a constant area duct was 
% % investigated.
% Axially sheared flows are different than axial uniform flow in that a velocity gradient is
% present along the x axis. 
% All other primitive variables (pressure and density which is $\propto$ speed of
% sound) are constant. As a result, the only changes that occur are in the x
% direction. This implies that $\partial / \partial \theta = 0$. 
% For the conservation of mass,

% \[ \nabla (\vec{V}\bar{\rho}) =  \left( 
% \underbrace{
% 	\cancel{
% 		\frac{\partial (\bar{\rho}v_r)	}{\partial r}
% 	}
% }_{v_r = 0} +
% \underbrace{\cancel{\frac{1}{r}	\frac{\partial \bar{\rho}v_{\theta}}{\partial \theta}}}_{\frac{\partial }{\partial \theta}} +
% \frac{\partial \bar{\rho}v_x}{\partial x}
% \right) = \frac{\partial \bar{\rho}v_x}{\partial x}\] 

% The conservation of momentum in the radial direction becomes,
% % \[(\vec{V}\cdot \nabla) \vec{V} =
% % \cancel{v_r \frac{\partial v_r}{\partial r}} +
% % \cancel{\frac{v_{\theta}  }{r}
% % 	\frac{\partial v_r}{\partial \theta}}- \frac{v_{\theta}^2}{r}+ 
% % \cancel{v_x \frac{\partial v_r}{\partial x} }
% % = -\frac{1}{\rho} \frac{\partial P}{\partial r}
% % \]
% % \[
% % \frac{v_{\theta}^2}{r}
% % = \frac{1}{\rho} \frac{\partial P}{\partial r}
% % \] 

% \[
% \frac{{\rho} v_{\theta}^2}{r} 
% =\frac{\partial P}{\partial r}
% \]
% For the tangential direction,
% \[(\vec{V}\cdot \nabla) \vec{V} = \cancel{v_r \frac{\partial v_{\theta}}{\partial r} +
% 	\frac{v_{\theta}}{r}
% 	\frac{\partial v_{\theta}}{\partial \theta} +
% 	\frac{v_r v_{\theta}}{r}}+ 
% v_x \frac{\partial v_{\theta}}{\partial x} 
% = \cancel{-\frac{1}{\rho r} \frac{\partial P}{\partial \theta}}\]
% Dividing $v_x$ to the other side,
% \[ \frac{\partial v_{\theta}}{\partial x}  = 0\]

% Similarly for the axial direction,

% \[ \frac{\partial v_x}{\partial x} = 0 \]
% Since the flow is isentropic , $\nabla S = 0$ the relation, $A^2= \frac{\nabla \bar{P}}{ \nabla \bar{\rho}}$ can be
% used to account for the change in speed of sound with radius as the mean
% flow contains a tangential (swirling) component.


% \section{Accounting for solid body swirl}

If the mean flow contains a swirling component, i.e. a velocity vector in the 
tangential direction, the mean quantities, pressure , density are non-uniform, 
thus also changing the speed of sound. By integrating the radial momentum
equation, an expression for the speed of sound was established to account for 
the resulting nonuniformities due to rotations in the flow. 


% If the flow contains a swirling component, then the primitive variables are 
% nonuniform through the flow, and mean flow assumptions are not valid. 
% To account to this, we integrate the momentum equation in the radial direction 
% with respect to the radius. 


\begin{equation}
    p = \int_{r_{min}}^{r_{max}} \frac{\rho v_{\theta}^2}{r} dr 
    \label{eqn:radialmomentum_integrated}
\end{equation}

where $r_{min}$ and $r_{max}$ are the bounds of the radius. Since the flow
is isentropic, the pressure is related to the speed of sound through $\nabla p =
A^2 \nabla \rho$; which is used to compute $\rho$. With the relationship 
$A^2 = \gamma p/\rho$, the speed of sound is found to be,

% The dimensional form is,
% \[
% \frac{\bar{\rho} v_{\theta}^2}{r} 
% =\frac{\partial P}{\partial r}
% \]
% By appplying separation of variables, the expression for $P$ can be found,
% \[
% \int_{r}^{r_{max}} \frac{\bar{\rho} v_{\theta}^2}{r}\partial r 
% =-\int_{P(r)}^{P(r_{max})}\partial P
% \]

% Since $\tilde{r} = r/r_{max}$ then,
% \[r = \tilde{r}r_{max}\]
% by taking total derivatives and applying chain rule,

% \[dr = d(\tilde{r}r_{max}) = d(\tilde{r})r_{max}\]
% Substituting these terms back in and evaluating the right hand side,
% \[
% \int_{\tilde{r}}^{1} \frac{\bar{\rho} v_{\theta}^2}{\tilde{r}}\partial \tilde{r} 
% =P(1)-P(\tilde{r})
% \]
% For reference the minimum value of $\tilde{r}$ is

% \[\sigma = \frac{r_{max}}{r_{min}}\]

% The radial derivative of the speed of sound squared is then used to find the 
% speed of sound in the cases where there is mean tangetial component regardless
% of there being axial flow,

% \[\frac{\partial A^2}{\partial r } = \frac{\partial}{\partial r} \left( \frac{\gamma P}{\rho} \right)\]
% Using the quotient rule, we can extract the definition of the speed of sound.
% \begin{align*}
% &= \frac{\partial P}{\partial r} \frac{\gamma \bar{\rho}}{\bar{\rho}^2} - \left( \frac{\gamma P}{\bar{\rho}^2} \right) \frac{\partial \bar{\rho}}{\partial r}\\
% &=  \frac{\partial P}{\partial r} \frac{\gamma }{\bar{\rho}} - \left( \frac{A^2}{\bar{\rho}} \right) \frac{\partial \bar{\rho} }{\partial r}\\ \text{Using } \partial P/A^2 = \partial \rho \rightarrow &= \frac{\partial P}{\partial r} \frac{\gamma }{\bar{\rho}} - \left( \frac{1}{\bar{\rho}} \right) \frac{\partial \bar{ P} }{\partial r}\\
% \frac{\partial A^2}{\partial r} &= \frac{\partial P}{\partial r} \frac{\gamma - 1}{\bar{\rho}}  \\ \text{or..}
% \frac{\bar{\rho}}{\gamma -1}\frac{\partial A^2}{\partial r} &= \frac{\partial P}{\partial r} 
% \end{align*}


% \begin{align*}
% \frac{\bar{\rho} v_{\theta}^2}{r} 
% &=\frac{\partial P}{\partial r}\\
% \frac{\cancel{\bar{\rho}} v_{\theta}^2}{r} 
% &=\frac{\cancel{\bar{\rho}}}{\gamma -1}\frac{\partial A^2}{\partial r}\\
% \frac{v_{\theta}^2}{r}\left(\gamma -1\right) &= \frac{\partial A^2}{\partial r}\\ \text{Dividing both sides by } A^2 \rightarrow \frac{M_{\theta}}{r}\left(\gamma - 1\right) &= \frac{\partial A^2}{\partial r} \frac{1}{A^2}
% \end{align*}
% \begin{align*}
% \text{Integrating both sides } \int_{r}^{r_{max}}\frac{M_{\theta}}{r}\left(\gamma - 1\right){\partial r}  &=\int_{A^2(r)}^{A^2(r_{max})}\frac{1}{A^2}  {\partial A^2}\\
% \int_{r}^{r_{max}}\frac{M^2_{\theta}}{r}\left(\gamma - 1\right){\partial r}  &=ln(A^2(r_{max})) - ln(A^2(r)) \\
% \int_{r}^{r_{max}}\frac{M^2_{\theta}}{r}\left(\gamma - 1\right){\partial r}  &=ln\left(\frac{A^2(r_{max})}{A^2(r)}\right) 
% \end{align*}

% Defining non dimensional speed of sound $\tilde{A} = \frac{A(r)}{A(r_{max})}$
% \begin{align*}
% \int_{r}^{r_{max}}\frac{M_{\theta}}{r}\left(\gamma - 1\right){\partial r}  &=ln\left(\frac{1}{\tilde{A}^2}\right) \\
% &= -2ln(\tilde{A})\\
% \tilde{A}(r) &= exp\left[-\int_{r}^{r_{max}}\frac{M_{\theta}}{r}\frac{\left(\gamma - 1\right)}{2}{\partial r}\right] \\ \text{replacing r with }\tilde{r} \rightarrow \tilde{A}(r) &= exp\left[-\int_{r}^{r_{max}}\frac{M_{\theta}}{r}\frac{\left(\gamma - 1\right)}{2}{\partial r}\right]		\\
% \end{align*}
% (See appendix for full derivation)
\begin{align}
\tilde{A}(\tilde{r}) &= exp\left[\left(\frac{1 - \gamma}{2}\right)\int_{\tilde{r}}^{1}\frac{M_{\theta}}{\tilde{r}}{\partial \tilde{r}}\right]	
\end{align}

The appendix shows how the speed of sound was extracted.
For special cases of swirling flow, the relation to between the speed 
of sound and the tangential velocity can be found. Expressions can be derived 
for free vortex , and/or solid body swirl. The non-dimesionalization is shown in the
next section.


