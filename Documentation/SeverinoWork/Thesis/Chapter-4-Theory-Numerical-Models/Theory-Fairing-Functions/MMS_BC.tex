:%        File: MMS_BC.tex
%     Created: Wed Oct 06 09:00 AM 2021 E
% Last Change: Wed Oct 06 09:00 AM 2021 E
%
\documentclass[a4paper]{article}
\usepackage{amsmath}
\usepackage{multicol}
\usepackage{biblatex}
\addbibresource{refs.bib}
\begin{document}

% creating a modified manufactured solution}


\section{Setting Boundary Condition Values Using a Fairing Function}

\subsection{Using $\beta$ as a scaling parameter}

Defining the nondimensional radius in the same way that SWIRL does:

\begin{align*}
    \widetilde{r} = \frac{r}{r_T}
\end{align*}
where $r_T$ is the outer radius of the annulus.

The hub-to-tip ratio is defined as:

\begin{align*}
    \sigma = \frac{r_H}{r_T}
     &= \widetilde{r}_H
\end{align*}
where $\widetilde{r}_H$ is the inner radius of the annular duct. The hub-to-tip
ratio can also be zero indicating the duct is hollow.

A useful and similar parameter is introduced, $\beta$, where $ 0 \leq \beta \leq 1$

\begin{align*}
    \beta &=
    \frac{r - r_H}{r_T - r_H}
\end{align*}
Dividing By $r_T$
\begin{align*}
    \beta &= 
    \frac{
        \frac{r}{r_T} - \frac{r_H}{r_T}
}{
        \frac{r_T}{r_T} - \frac{r_H}{r_T}
}\\
&= \frac{
    \widetilde{r} - \widetilde{r}_H
}{
1 - \sigma
}
\end{align*}

Suppose a manufactured solution $f_{MS}$ with boundaries $f_{MS}(r= \sigma)$ 
and $f_{MS}(\widetilde{r}= 1)$ is the specified analytical solution. The goal
is to change the boundary conditions of the manufactured solution in such  way 
that allows us to adequately check the boundary conditions imposed on SWIRL.
Defining the manufactured solution, $f_{MS}(\widetilde{r})$,   where
$\sigma \leq \widetilde{r} \leq 1$ and there are desired values of $f$ at the 
boundaries desired values are going to be denoted as $f_{minBC}$ and $f_{maxBC}$.
The desired changes in $f$ are defined as:

\begin{align*}
    \Delta f_{minBC} = f_{minBC} - f_{MS}(\widetilde{r} = \sigma)\\
    \Delta f_{maxBC} = f_{maxBC} - f_{MS}(\widetilde{r} = 1) 
\end{align*}
We'd like to impose these changes smoothly on the manufactured solution function.
To do this,the fairing functions, $A_{min}(\widetilde{r})$ and $A_{max}(\widetilde{r})$
where:
\begin{align*}
    f_{BCsImposed}(\widetilde{r}) = f_{MS}(\widetilde{r}) +
    A_{min}(\widetilde{r}) \Delta f_{minBC}  +  
    A_{max}(\widetilde{r}) \Delta f_{maxBC}  
\end{align*}
Then, in order to set the condition at the appropriate boundary, the following 
conditions are set,


\begin{align*}
    A_{min}(\widetilde{r} = \sigma) &= 1\\
    A_{min}(\widetilde{r} = 1) &= 0 \\
    A_{max}(\widetilde{r} = 1) &= 1 \\
    A_{max}(\widetilde{r} = \sigma) &= 0 
\end{align*}
If $A_{min}(\widetilde{r})$ is defined as a function of $A_{max}(\widetilde{r})$
then only $A_{max}(\widetilde{r})$ needs to be defined, therefore 
\begin{align*}
    A_{min}(\widetilde{r}) = 1 - A_{max}(\widetilde{r}) 
\end{align*}

It is also desirable to set the derivatives for the fairing function at the 
boundaries incase there are boundary conditions imposed on the derivatives of 
the fairing function.

\begin{align*}
    \frac{\partial A_{max}}{\partial \widetilde{r}}|_{\widetilde{r}= \sigma} &= 0\\
    \frac{\partial A_{max}}{\partial \widetilde{r}}|_{\widetilde{r}= 1} &= 0    
\end{align*}

\begin{align*}
    \frac{\partial A_{min}}{\partial \widetilde{r}}|_{\widetilde{r}= \sigma} &= 0\\
    \frac{\partial A_{min}}{\partial \widetilde{r}}|_{\widetilde{r}= 1} &= 0    
\end{align*}



\subsection{Minimum Boundary Fairing Function}

Looking at $A_{min}$ first, the polynomial is:

\begin{align*}
    A_{min} \left( \beta \right) &= 
    a + b \beta + c \beta^2 + d \beta^3                   \\
    A_{min} \left( \widetilde{r} \right) &= 
    a + b \left( \frac{\widetilde{r} - \sigma}{1 - \sigma} \right)+
    c\left( \frac{\widetilde{r} - \sigma}{1 - \sigma} \right)  ^2+
    d\left( \frac{\widetilde{r} - \sigma}{1 - \sigma} \right)^3                    \\
\end{align*}

Taking the derivative,
\begin{align*}
    A'_{min} \left( \widetilde{r} \right) &= 
    b \left( \frac{1}{1 - \sigma} \right)+
    2 c\left( \frac{1}{1 - \sigma} \right)\left( \frac{\widetilde{r} - \sigma}{1 - \sigma} \right)  +
    3 d\left( \frac{1}{1-\sigma} \right)\left( \frac{\widetilde{r} - \sigma}{1 - \sigma} \right)^2\\
    A'_{min} \left( \beta \right) &= 
    \left( \frac{1}{1 - \sigma} \right)
    \left[
    b +
    2 c \beta + 
    3 d \beta^2
    \right]
\end{align*}
Now we will use the conditons mentioned earlier as constraints to this system 
of equations
Using the possible values of $\widetilde{r}$,

\begin{align*}
    A_{min}(\sigma) &= a &= 1 \\
    A_{min}(1) &= a + b + c + d  &= 0 \\
    A'_{min}(\sigma) &=  b    &= 0 \\
    A'_{min}(1) &=  b + 2 c + 3 d    &= 0 \\
\end{align*}


which has the solution,

\begin{align*}
    a &= 1 \\
    b &= 0 \\
    c &= -3 \\
    d &= 2 
\end{align*}

giving the polynomial as: 

\begin{align*}
    A_{min} (\widetilde{r}) = 1 - 3 \left(  \frac{\widetilde{r} - \sigma }{ 1 - \sigma}\right)^2 +
    2 \left( \frac{\widetilde{r} - \sigma}{1 - \sigma} \right)^3 
\end{align*}



\subsection{Max boundary polynomial}
Following the same procedure for $A_{max}$ gives 
\begin{align*}
    A_{min} (\widetilde{r}) = 3 \left(  \frac{\widetilde{r} - \sigma }{ 1 - \sigma}\right)^2 -
    2 \left( \frac{\widetilde{r} - \sigma}{1 - \sigma} \right)^3
\end{align*}
\subsection{Corrected function} 

The corrected function is then, 

\begin{align*}
    f_{BCsImposed} (\widetilde{r}) &= 
    f_{MS}(\widetilde{r}) &+ A_{min} \Delta f_{minBC} + A_{max} \Delta f_{maxBC} \\
                                   &= 
    f_{MS}(\widetilde{r}) &+
    \left(
        1 - 3 \left(  \frac{\widetilde{r} - \sigma }{ 1 - \sigma}\right)^2 +
    2 \left( \frac{\widetilde{r} - \sigma}{1 - \sigma} \right)^3 
\right)
    \left[ \Delta f_{minBC} \right]\\ 
           & &+
    \left(
         3 \left(  \frac{\widetilde{r} - \sigma }{ 1 - \sigma}\right)^2- 
    2 \left( \frac{\widetilde{r} - \sigma}{1 - \sigma} \right)^3 
\right)
    \left[ \Delta f_{maxBC} \right] \\ 
              f_{BCsImposed} (\beta) &= f_{MS}(\beta) &+ \Delta f_{minBC} + \left(  3 \beta^2 - 2 \beta^3  \right)
    \left[  \Delta f_{maxBC} - \Delta f_{minBC} \right]
\end{align*}
Note that we're carrying the correction throughout the domain, as opposed to 
limiting the correction at a certain distance away from the boundary. The 
application of this correction ensures that there is no discontinuous derivatives
inside the domain; as suggested in Roach's MMS guidelines (insert ref) 


What is meant by ``just because $A_{min}$ and its first derivative go to zero
doesn't mean that the second derivatives''




\subsection{Symbolic Sanity Checks}
We want to ensure that $f_{BCsImposed}$ has the desired boundary conditions, 
$f_{minBC/maxBC}$ instead of the original boundary values that come along
for the ride in the manufactured solutions, $f_{MS}(\widetilde{r}=\sigma /1)$. 
In another iteration of this method, we will be changing the derivative values,
so let's check the values of $\frac{\partial f_{BCsImposed}}{\partial \widetilde{r}}$ 
to make sure those aren't effected unintentionally.

\subsubsection*{Symbolic Sanity Check 1}
The modified manufactured solution, $f_{BCsImposed}$ with the fairing functions
$A_{min}$ and $A_{max}$ substituted in is,
\begin{align*}
    f_{BCsImposed}(\widetilde{r}) =
    \left(
        3 \left(  \frac{\widetilde{r} - \sigma }{ 1 - \sigma}\right)^2- 
        2 \left( \frac{\widetilde{r} - \sigma}{1 - \sigma} \right)^3 
    \right)
    \left[ \Delta f_{maxBC} \right] .
\end{align*} 
Further simplification yields,
\begin{align*}
    f_{BCsImposed}(\widetilde{r} = \sigma) 
    &=
    \left(
        f_{MS}(\widetilde{r} = \sigma) +
        \Delta f_{minBC} +
        \left( 3\left(  \frac{\sigma - \sigma}{1 - \sigma} \right)^2- 
          2\left(  \frac{\sigma - \sigma}{1 - \sigma} \right)^3- 
        \right)
        \left[ \Delta f_{maxBC} - \Delta f_{minBC}  \right] 
    \right)\\
    &=  f_{MS}(\widetilde{r} = \sigma) + \Delta f_{minBC}\\
    &=  f_{MS}(\widetilde{r} = \sigma) + (f_{minBC} - f_{MS}(\widetilde{r} = \sigma)) \\
    &=  f_{minBC}
\end{align*} 


\begin{align*}
    f_{BCsImposed}(\widetilde{r} = 1 ) 
    &=
    \left(
        f_{MS}(\widetilde{r} = 1) +
        \Delta f_{minBC} +
        \left( 3\left(  \frac{1 - \sigma}{1 - \sigma} \right)^2- 
          2\left(  \frac{1 - \sigma}{1 - \sigma} \right)^3- 
        \right)
        \left[ \Delta f_{maxBC} - \Delta f_{minBC}  \right] 
    \right)\\
    &=  f_{MS}(\widetilde{r} = 1) + \Delta f_{maxBC}\\
    &=  f_{MS}(\widetilde{r} = 1) + (f_{maxBC} - f_{MS}(\widetilde{r} = 1)) \\
    &=  f_{maxBC}
\end{align*} 


\begin{align*}
    \frac{\partial}{\partial \widetilde{r}}\left(  f_{BCsImposed}(\widetilde{r}) =
    \left(
        3 \left(  \frac{\widetilde{r} - \sigma }{ 1 - \sigma}\right)^2- 
        2 \left( \frac{\widetilde{r} - \sigma}{1 - \sigma} \right)^3 
    \right)
    \left[ \Delta f_{maxBC} \right]\right) \\
    \frac{\partial f_{MS}}{\partial \widetilde{r}} + 
    \left( \frac{6}{1-\sigma} \right)
    \left( 
        \left( \frac{\widetilde{r} - \sigma}{1 - \sigma} \right) -
    \left( \frac{r - \sigma}{1 - \sigma} \right)^2 \right)
    \left( \Delta f_{maxBC} - \Delta f_{minBC} \right)
\end{align*} 


At $\widetilde{r} = \sigma$, the derivative is: 

\begin{align*}
    \frac{\partial f_{MS}}{\partial \widetilde{r}}|_{\sigma} \\
    \frac{\partial f_{MS}}{\partial \widetilde{r}}|_{1} 
\end{align*}



\subsection{Min boundary derivative polynomial}

The polynomial is of the form: 

\begin{align*}
    B_{min} \left( \beta \right) &= 
    a + b \beta + c \beta^2 + d \beta^3                   \\
    B_{min} \left( \widetilde{r} \right) &= 
    a + b \left( \frac{\widetilde{r} - \sigma}{1 - \sigma} \right)+
    c\left( \frac{\widetilde{r} - \sigma}{1 - \sigma} \right)  ^2+
    d\left( \frac{\widetilde{r} - \sigma}{1 - \sigma} \right)^3                    \\
\end{align*}
Taking the derivative,
\begin{align*}
    B'_{min} \left( \widetilde{r} \right) &= 
    b \left( \frac{1}{1 - \sigma} \right)+
    2 c\left( \frac{1}{1 - \sigma} \right)\left( \frac{\widetilde{r} - \sigma}{1 - \sigma} \right)  +
    3 d\left( \frac{1}{1-\sigma} \right)\left( \frac{\widetilde{r} - \sigma}{1 - \sigma} \right)^2\\
    B'_{min} \left( \beta \right) &= 
    \left( \frac{1}{1 - \sigma} \right)
    \left[
    b +
    2 c \beta + 
    3 d \beta^2
    \right]
\end{align*}


Applying the four constraints gives:
\begin{align*}
    a &= 0\\
    b &= \left( 1 - \sigma \right) \\
    a + b + c + d &= 0\\
    2 + 2c + 3d &= 0
\end{align*}
\begin{align*}
    c + d &= -b  \\
    2c + 3d &= -b
\end{align*}
\begin{align*}
    c &= -2b \\
    d &= b
\end{align*}

and the min boundary derivative polynomial is: 
\begin{align*}
    B_{min}\left( \widetilde{r} \right) &= 
    b \left( \frac{\widetilde{r} - \sigma }{1 - \sigma}\right) -
    2b\left( \frac{\widetilde{r} - \sigma }{1 - \sigma}\right) ^2 +
    b \left( \frac{\widetilde{r} - \sigma }{1 - \sigma}\right)^3 \\
    &=  \left( 1 - \sigma \right)
    \left( \left( \frac{\widetilde{r} - \sigma }{1 - \sigma}\right)  - 
    2\left( \frac{\widetilde{r} - \sigma }{1 - \sigma}\right)^2 +
    \left( \frac{\widetilde{r} - \sigma }{1 - \sigma}\right)^3\right)
\end{align*} 
\subsection{Polynomial function, max boundary derivative}
The polynomial is of the form:

The polynomial is of the form: 

\begin{align*}
    B_{max} \left( \beta \right) &= 
    a + b \beta + c \beta^2 + d \beta^3                   \\
    B_{max} \left( \widetilde{r} \right) &= 
    a + b \left( \frac{\widetilde{r} - \sigma}{1 - \sigma} \right)+
    c\left( \frac{\widetilde{r} - \sigma}{1 - \sigma} \right)  ^2+
    d\left( \frac{\widetilde{r} - \sigma}{1 - \sigma} \right)^3                    \\
\end{align*}
which has the derivative,


\begin{align*}
    B'_{max} \left( \widetilde{r} \right) &= 
    b \left( \frac{1}{1 - \sigma} \right)+
    2 c\left( \frac{1}{1 - \sigma} \right)\left( \frac{\widetilde{r} - \sigma}{1 - \sigma} \right)  +
    3 d\left( \frac{1}{1-\sigma} \right)\left( \frac{\widetilde{r} - \sigma}{1 - \sigma} \right)^2\\
    B'_{max} \left( \beta \right) &= 
    \left( \frac{1}{1 - \sigma} \right)
    \left[
    b +
    2 c \beta + 
    3 d \beta^2
    \right]
\end{align*}
Applying the four constraints gives:
\begin{align*}
    a &= 0 \\
    b &= 0 \\
    a + b + c + d &= 0 \\
    b + 2c + 3d &= (1-\sigma)
\end{align*} 

working this out: 

\begin{align*}
    c + d &=  0 \\
    2c + 3d &= (1 - \sigma)
\end{align*} 

gives 

\begin{align*}
    c &= -\left( 1 - \sigma\right)
    d &= \left( 1 - \sigma\right)
\end{align*}
and the max boundary derivative polynomial is:

\begin{align*}
    B_{max} \left( \widetilde{r} \right) &= 
    \left( 1 - \sigma \right) \left( 
        - \left( \frac{\widetilde{r}-\sigma}{1 - \sigma} \right)^2 +
        \left( \frac{\widetilde{r}-\sigma}{1-\sigma} \right)^3
    \right)
\end{align*}
\subsection{Putting it together}

The corrected function is then: 
\begin{align*}
    f_{BCsImposed} \left( \widetilde{r} \right) &= 
    f_{MS} + 
    B_{min}\left( \widetilde{r} \right) \Delta f'_{minBC} +
    B_{max}\left( \widetilde{r} \right) \Delta f'_{maxBC}
\end{align*}
\begin{align*}
    &= 
    f_{MS} + \\
    \left( 1 - \sigma \right) \left( 
        \left( \frac{\widetilde{r} - \sigma}{1 - \sigma} \right) 
        - \left( \frac{\widetilde{r} - \sigma}{1 - \sigma} \right) ^2
\right) \Delta f'_{minBC} + \\
\left( 1 - \sigma \right) \left( - \left( \frac{\widetilde{r} - \sigma}{1 - \sigma} \right)^2 + 
    \left( \frac{\widetilde{r} - \sigma}{1 - \sigma}  \right)^3
\right)
\left( \Delta f'_{minBC} + \Delta f'_{maxBC} \right) 
\end{align*}
%
Now here we will do sanity checks for the derivative boundaries.
%
%
%
%\begin{align*}
%    \widetilde{r} = \frac{r - r_{min}}{r_{max} - r_{min}} 
%\end{align*}
%
%Substituting $r_{min}$ $r_{max}$ for $r$ gives,
%
%\begin{align*}
%    \widetilde{r}_{min} &= \frac{r_{min} - r_{min}}{r_{max} - r_{min}} = 0\\
%    \widetilde{r}_{max} &= \frac{r_{max} - r_{min}}{r_{max} - r_{min}} = 1
%\end{align*}
%
%The goal is to set desired values at the boundaries of the specified analytical
%function. First we define the values at the boundaries, i.e.
%\begin{align*}
%    f(\widetilde{r} &= \widetilde{r}_{min}) = f_{min}     \\
%    f(\widetilde{r} &= \widetilde{r}_{max}) = f_{max}     
%\end{align*}
%
%Then, the change between our desired boundary condition value and the actual is,
%To do so, a desired change in the boundary condition must be defined. 
%
%\begin{align*}
%    \Delta f_{min} =  (f_{min}) - (f_{min})_{desired}   \\
%    \Delta f_{max} =  (f_{max}) - (f_{max})_{desired}   
%\end{align*}
%
%To ensure that the desired changes are imposed \textit{smoothly}. The smoothness 
%of a function is measured by the number of continuous derivatives the desired function
%has over the domain of the function. At the very minimum, a smooth function will be continuous and
%hence differentiable everywhere. When generating manufactured solutions, smoothness
%of the solution is often times assumed but is not guarenteed 
%\cite{oberkampf2002verification}. (transition sentance)
%
%Defining the faring function:
%
%\begin{align*}
%    f_{imposed}(\widetilde{r}) = 
%    f(\widetilde{r}) + 
%    A_{min}(\widetilde{r}) \Delta f_{min} +
%    A_{max}(\widetilde{r}) \Delta f_{max}  
%\end{align*}
%
%In order for the imposed boundary conditions to work, the desired values must be 
%such that,
%
%\begin{align*}
%    A_{min}(\widetilde{r}_{min}) &= 1 &A_{min}(\widetilde{r}_{max}) &= 0  \\
%    A_{max}(\widetilde{r}_{max}) &= 1 &A_{max}(\widetilde{r}_{min}) &= 1 
%\end{align*}
%
%This assured that the opposite boundaries are not affected.(How?)
%
%For simplicity lets define:
%
%\begin{align*}
%    A_{min}(\widetilde{r}) =
%    1-
%    A_{max}(\widetilde{r}) 
%\end{align*}
%
%so now only $A_{max}$ needs to be defined. 
%
%As mentioned, the desired boundary condition need to allow the analytical function
%to be differentiable, and as a consequence, it would be wise to also set those. 
%In addition, different types of boundary conditions (such as Neumann) that would 
%require this. 
%
%\begin{align*}
%    \frac{\partial A_{max} }{\partial \widetilde{r}}|_{\widetilde{r}_{min}} = 0 \\
%    \frac{\partial A_{max} }{\partial \widetilde{r}}|_{\widetilde{r}_{max}} = 0 \\
%\end{align*}
%
%A straight forward choice would be 
%
%\begin{align*}
%    A_{max}(\widetilde{r}) = 
%    3 \widetilde{r}^2 - 2 \widetilde{r}^3
%\end{align*}
%
%Note that the correction is carried from boundary to boundary, as opposed to 
%applying the correction to only to a region near the boundaries. This ensures smooth 
%derivatives in the interior domain.
    
\end{document}


