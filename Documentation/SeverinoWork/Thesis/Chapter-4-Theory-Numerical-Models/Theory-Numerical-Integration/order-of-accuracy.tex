
\subsection{Calculation of Observed Order-of-Accuracy}
The numerical scheme used to perform the integration of the tangential velocity
will have a theoretical order-of-accuracy. To find the theoretical 
order-of-accuracy, the discretization error must first be defined. The error, 
$\epsilon$, is a function of id spacing, $\Delta r$

\[ \epsilon = \epsilon(\Delta r) \]

The discretization error in the solution should be proportional to 
$\left( \Delta r \right)^{\alpha}$ where $\alpha > 0$ is the theoretical order
for the computational method.  The error for each grid is expressed as 
\[ \epsilon_{M_{\theta}}(\Delta r) = |M_{\theta,analytic}-M_{\theta,calc}|\]
where $M_{\theta,analytic}$ is the tangential mach number that is defined from the
speed of sound we also defined and the $M_{\theta,calc}$ is the result from 
SWIRL. The $\Delta r$ is to indicate that this is a discretization error for a
specific grid spacing. Applying the same concept to to the speed of sound,

If we define this error on various grid sizes and compute $\epsilon$ for
each grid, the observed order of accuracy can be estimated and compared to
the theoretical order of accuracy. For instance, if the numerical soution
is second-order accurate and the error is converging to a value, the L2 norm of
the error will decrease by a factor of 4 for every halving of the grid cell 
size. 

Since the input variables should remain unchanged (except from minor changes 
from the Akima interpolation), the error for the axial and tangential mach 
number should be zero. As for the speed of sound, since we are using an analytic
expression for the tangential mach number, we know what the theoretical result
would be from the numerical integration technique as shown above. 
Similarly we define the discretization error for the speed of sound.

\[ \epsilon_{A}(\Delta r) = |A_{analytic}-A_{calc}|\]

For a perfect answer, we expect $\epsilon$ to be zero. Since a Taylor series can 
be used to derive the numerical schemes, we know that the truncation of higher
order terms is what indicates the error we expect from using a scheme that 
is constructed with such truncated Taylor series.

The error at each grid point $j$ is expected to satisfy the following,

\begin{align*}
    0 &= |A_{analytic}(r_j) - A_{calc}(r_j)| \\
    \widetilde{A}_{analytic}(r_j) &= \widetilde{A}_{calc}(r_j) +
    (\Delta r)^{\alpha} \beta(r_j)  + H.O.T
\end{align*}

where the value of $\beta(r_j)$ does not change with grid spacing, and 
$\alpha$ is the asymptotic order of accuracy of the method. It is important to
note that the numerical method recovers the original equations as the grid 
spacing approached zero.  It is important to note that $\beta$ represents the
first derivative of the Taylor Series.  Subtracting $A_{analytic}$ from both
sides gives,

\begin{align*}
    A_{calc}(r_j) - A_{analytic}(r_j) &= A_{analytic}(r_j) - A_{analytic}(r_j)
    + \beta(r_j) (\Delta r)^{\alpha} \\
    \epsilon_A(r_j)(\Delta r) &= \beta(r_j) (\Delta r)^{\alpha}
\end{align*}

To estimate the order of accuracy of the accuracy, we define the global errors 
by calculating the L2 Norm of the error which is denoted as $\hat{\epsilon}_A$ 

\begin{align*}
    \hat{\epsilon}_A = \sqrt{\frac{1}{N}\sum_{j=1}^{N} \epsilon(r_j)^2  }
\end{align*}

\begin{align*}
    \hat{\beta}_A(r_j) = \sqrt{\frac{1}{N}\sum_{j=1}^{N} \beta(r_j)^2  }
\end{align*}

As the grid density increases, $\hat{\beta}$ should asymptote to a constant 
value. Given two grid densities, $\Delta r$ and $\sigma\Delta r$, and assuming
that the leading error term is much larger than any other error term,

\begin{align*}
    \hat{\epsilon}_{grid 1} &= \hat{\epsilon}(\Delta r) = \hat{\beta}(\Delta r)^{\alpha} \\
    \hat{\epsilon}_{grid 2} &= \hat{\epsilon}(\sigma \Delta r) = \hat{\beta}(\sigma \Delta r)^{\alpha} \\
                            &= \hat{\beta}(\Delta r)^{\alpha} \sigma^{\alpha}
\end{align*}

The ratio of two errors is given by,

\begin{align*}
    \frac{\hat{\epsilon}_{grid 2}}{\hat{\epsilon}_{grid 1}} &= 
    \frac{\hat{\beta}(\Delta r )^{\alpha}}{\hat{\beta}(\Delta r )^{\alpha}} \sigma^{\alpha} \\ &= \sigma^{\alpha}
\end{align*}

Thus, $\alpha$,the asymptotic rate of convergence is computed as follows 

\begin{align*}
    \alpha = \frac{
        \ln \frac{
            \hat{\epsilon}_{grid 2}
    }{\hat{\epsilon}_{grid 1} }}
    {\ln\left( \sigma \right) }
\end{align*}

Defining  for a doubling of grid points ,

\begin{align*}
    \alpha = \frac{\ln \left( \hat{\epsilon}\left( \frac{1}{2}\Delta  r\right)
            \right) -\ln \left( \hat{\epsilon}\left( \Delta  r\right)
    \right)}{\ln \left( \frac{1}{2} \right)}
\end{align*}

Similarly for the eigenvalue problem, 
Initially the source terms were defined without mention of the indicies of the 
matricies they make up. In other words, there was no fore sight on the fact 
that these source terms are sums of the elements within A,B, and X. 
To investigate the source terms in greater detail, the FORTRAN code 
that calls the source terms will output the terms within the source term 
and then sum them, instead of just their sum.

\[ [A]x - \lambda [B]x = 0 \]



