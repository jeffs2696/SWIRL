\subsection{Introduction}
The frequency domain, linearized Euler equation computer code, SWIRL, was 
verified through the Methods of Manufactured and Exact solutions. 
However, while the Method of Exact Solutions (MES) can be used to validate the
ouput of SWIRL, there is a limitation on the number of exact solutions available 
based on the flow and domain.  Any changes in a flow configuration (uniform axial flow vs. sheared axial flow, without any tangential 
component) require a recalculation of the analytical solution. Code validation 
through MES is for the case of uniform flow in a cylindrical 
duct since the exact solution requires implementing special Bessel functions
. In addition, the exact solution needs to be recomputed for changes in radii 
and the use of acoustic liner (i.e., boundary conditions). For this reason, 
the more comprehensive MMS was first used for code verification to test each 
variable in the governing equations and gain measurable acceptance criteria 
without relying on expert judgment.

MMS was implemented on a component level since SWIRL consists
of two main numerical approximations for flows with axial and tangential 
components. First, the numerical integration technique required for radial 
change in sound speed was verified. Then, the four governing equations that make
up the matrices required for the eigenvalue problem were tested. Note that the 
output of the eigenvalue problem was tested with MMS by recomputing the expression
$[A] {x} - \lambda [B] {x} = 0$ with a given eigenvalue/vector pair. 
The calculated $L_{2,norm}$ (i.e. error and the order of accuracy for both numerical methods are discussed below.
The MMS was then used to compare against validation cases in literature to outline 
the methodolody for test cases for which there is no solution.
