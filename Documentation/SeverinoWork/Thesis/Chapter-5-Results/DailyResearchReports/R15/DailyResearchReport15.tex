
%        File: WeeklyResearchReport_4_19_21.tex
%     Created: Mon Apr 19 08:00 AM 2021 E
% Last Change: Mon Apr 19 08:00 AM 2021 E
%
\documentclass[a4paper]{article}
\usepackage{mathtools}
\usepackage{verbatim}
\usepackage{graphicx}
\usepackage{tabularx}
\usepackage{pgfplots}
\usepackage{adjustbox}
\usepackage{booktabs}
\makeatletter
\let\latex@xfloat=\@xfloat
\def\@xfloat #1[#2]{%
    \latex@xfloat #1[#2]%
    \def\baselinestretch{1}
    \@normalsize\normalsize
    \normalsize
}
\makeatother
\usepackage{amsmath}
\usepackage{mathtools}
\usepackage{epigraph}
\usepackage{cancel}
\usepackage{xcolor}
\newcommand\Ccancel[2][black]{\renewcommand\CancelColor{\color{#1}}\cancel{#2}}
\usepackage{algorithm}
\usepackage{graphicx}
\usepackage[noend]{algpseudocode}
\usepackage{gnuplot-lua-tikz}
\usepackage[utf8]{inputenc}
\usepackage{pgfplots}
\usepackage{tabularx}
\DeclareUnicodeCharacter{2212}{−}
\usepgfplotslibrary{groupplots,dateplot}
\usetikzlibrary{patterns,shapes.arrows}
\pgfplotsset{compat=newest}
\begin{document}
\begin{titlepage}

    \title{
    Daily Research Report}

    \author{ Jeffrey Severino \\
        University of Toledo \\
        Toledo, OH  43606 \\
    email: jseveri@rockets.utoledo.edu}


    \maketitle

\end{titlepage}
\section{Current Research Direction}
The goal is to find the analytical solution for a uniform flow case without
liner. It isn't clear how the analytical solution changes with liner or if
it changes at all. I suspect the boundary conditions have to be implemented because
my current results do not match Table 4.3. There is a case that has shear flow
in a cylinder but then the question becomes, what is the analytical case for sheared
flow? If I could find a test case for uniform 

\section{Research Performed}
The literature on duct modes in turbomachinery was reviewed to find test cases
that can be used outside of Kousen's references but there is not any ``readily''
available. The best reference I could find was Maldanado's Figure 10 $m = 2$
, $"Mach = 0.3"$, $"He = 10"$. The quotes are to indicate that a different variable
was used. Since the case says it is uniform flow, the Mach number must be axial.
A non dimensional frequency is often indicated as the Helmholtz number. There is
no definiton offered, but Rienstra often uses this and defines it just like we 
defined $k$. MaldanAdo also shows the effect of liner on the same case. After
scanning literature, I think the best step is to replicate these results.

The analytical cut-on mode was calculated for this case and it was validated, 
but this doesn't provide the axial wavenumbers for this case. Perhaps I need to 
grab the values from this Figure!

\section{Issues and concerns}
It is unfortunate how cryptic it is to replicate these results easily. The steps 
are not clearly outlined from step to step and the conflicts between variable
naming conventions can make it easy to make a mistake. Kerrebrocks and Rienstra's reference is 
a good starting point to replicate as they both cited Pridmore-Brown (1958) who
cited Moore (1930s). It seems thats how far back this theory goes, but test cases
are rarely provided. Maldanado's work helps in this endevor alot.


\section{Planned Research}
After reviewing the literature, Maldanado's Figure 10 in Sound Propagation in Lined Annular Ducts with MeanSwirling Flow
has the best case to replicate because it has uniform flow in a cylindrical
duct that is NOT lined
\end{document}


