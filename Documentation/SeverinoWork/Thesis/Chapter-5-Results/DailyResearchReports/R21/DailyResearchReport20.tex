
%        File: WeeklyResearchReport_4_19_21.tex
%     Created: Mon Apr 19 08:00 AM 2021 E
% Last Change: Mon Apr 19 08:00 AM 2021 E
%
\documentclass[a4paper]{article}
\usepackage{mathtools}
\usepackage{verbatim}
\usepackage{graphicx}
\usepackage{tabularx}
\usepackage{pgfplots}
\usepackage{adjustbox}
\usepackage{booktabs}
\makeatletter
\let\latex@xfloat=\@xfloat
\def\@xfloat #1[#2]{%
    \latex@xfloat #1[#2]%
    \def\baselinestretch{1}
    \@normalsize\normalsize
    \normalsize
}
\makeatother
\usepackage{amsmath}
\usepackage{mathtools}
\usepackage{epigraph}
\usepackage{cancel}
\usepackage{xcolor}
\newcommand\Ccancel[2][black]{\renewcommand\CancelColor{\color{#1}}\cancel{#2}}
\usepackage{algorithm}
\usepackage{graphicx}
\usepackage[noend]{algpseudocode}
\usepackage{gnuplot-lua-tikz}
\usepackage[utf8]{inputenc}
\usepackage{pgfplots}
\usepackage{tabularx}
\DeclareUnicodeCharacter{2212}{−}
\usepgfplotslibrary{groupplots,dateplot}
\usetikzlibrary{patterns,shapes.arrows}
\pgfplotsset{compat=newest}
\begin{document}
\begin{titlepage}

    \title{
    Daily Research Report}

    \author{ Jeffrey Severino \\
        University of Toledo \\
        Toledo, OH  43606 \\
    email: jseveri@rockets.utoledo.edu}


    \maketitle

\end{titlepage}
\section{Current Research Direction}
This week the modal content was looked at for the anaylytical solution.
\section{Research Performed}
The axial wavenumbers for a uniform flow case where $m = 2$ , $M = 0.3$ and 
$k = -1$ is presented. 

 \begin{figure}[h!]
     \centering
     \includegraphics[width=\textwidth]{/home/jeff-severino/SWIRL/CodeRun/03-plotReport/tex-outputs/T1.pdf}
 \end{figure}

 The acoustic modes are along the cut off line (parallel to the imaginary axis)
 and the convective modes are to the left or right of the line. 

 The analytical mode shape is going to be 

 \begin{equation}
     p' = p(r) e^{i(m \theta + k_x x - \omega t)}
 \end{equation}
 If we only want to look at things with respect to $x$, let's set $\theta = 0$
 $t = 0$, $p(r) = 1 (?)$, then


 \begin{equation}
     p' = e^{\pm i k_x x }
 \end{equation}

 When plotting this I get the following:
 \begin{figure}[h!]
     \centering
     \includegraphics{k_x_0_re.pdf}
     \caption{Analytical Mode Shapes for $k_x,0$}
     \label{fig:kx0}
 \end{figure}

 \begin{figure}[h!]
     \centering
     \includegraphics{k_x_1_re.pdf}
     \caption{Analytical Mode Shapes for $k_x,1$}
     \label{fig:kx0}
 \end{figure}

 \begin{figure}[h!]
     \centering
     \includegraphics{k_x_2_re.pdf}
     \caption{Analytical Mode Shapes for $k_x,2$}
     \label{fig:kx0}
 \end{figure}

 \begin{figure}[h!]
     \centering
     \includegraphics{k_x_0_im.pdf}
     \caption{Analytical Mode Shapes for $k_x,0$}
 \end{figure}
 \begin{figure}[h!]
     \centering
     \includegraphics{k_x_1_im.pdf}
     \caption{Analytical Mode Shapes for $k_x,1$}
 \end{figure}
 \begin{figure}[h!]
     \centering
     \includegraphics{k_x_2_im.pdf}
     \caption{Analytical Mode Shapes for $k_x,2$}
 \end{figure}

\begin{tiny}
 \begin{verbatim}
     \input{gam.nonconv.0128}
 \end{verbatim}
\end{tiny}
\section{Issues and concerns}


 \begin{figure}
     \centering
     \includegraphics[width=\textwidth]{/home/jeff-severino/SWIRL/CodeRun/03-plotReport/tex-outputs/egv_prop_re.pdf}
 \end{figure}


 \begin{figure}
     \centering
     \includegraphics[width=\textwidth]{/home/jeff-severino/SWIRL/CodeRun/03-plotReport/tex-outputs/egv_prop_im.pdf}
 \end{figure}


 \begin{figure}
     \centering
     \includegraphics[width=\textwidth]{/home/jeff-severino/SWIRL/CodeRun/03-plotReport/tex-outputs/egv_decay_re.pdf}
 \end{figure}

 \begin{figure}
     \centering
     \includegraphics[width=\textwidth]{/home/jeff-severino/SWIRL/CodeRun/03-plotReport/tex-outputs/egv_decay_im.pdf}
 \end{figure}


\section{Planned Research}
The manual sorting of modes by index number is very stressful and prone to error.
For cases where we have extra modes it is hard to identify which one is physically meaning ful 
and which isnt.
\end{document}


