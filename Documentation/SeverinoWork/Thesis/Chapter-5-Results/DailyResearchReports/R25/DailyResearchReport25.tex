
%        File: WeeklyResearchReport_4_19_21.tex
%     Created: Mon Apr 19 08:00 AM 2021 E
% Last Change: Mon Apr 19 08:00 AM 2021 E
%
\documentclass[a4paper]{article}
\begin{document}
\begin{titlepage}

    \title{
    Daily Research Report}

    \author{ Jeffrey Severino \\
        University of Toledo \\
        Toledo, OH  43606 \\
    email: jseveri@rockets.utoledo.edu}


    \maketitle

\end{titlepage}
\section{Current Research Direction}
The current research direction is to
\begin{enumerate}
    \item The MMS was used to create a closed analytic form for the solution
        that I want for the ``final'' test problem. The issue is that one MMS
        cannot simultaneously test all conditions.
\subitem The first solution tests an annular duct with liners and with arbitrary
axial and swirling flow. This was done to exhaust each variable in the LEE. The 
manufactured solutions were generated using a summation of tangents for the 
mean flow profile and for the perturbation variables. Although each variable
is set to a non zero value, this does not test the implementation of the L'Hopital's rule.
\item The second solution tests a cylindrical duct with hard walls with
    a uniform mean flow profile. The objective here is to use the MMS along 
    with the actual analytical solution. This will set up the framework for a 
    solution validation. The reason for doing this is because there are aspects 
    of SWIRL that do not get tested by calculating a rate of convergence. The first
    component that needs to be tested is the radial mode zero crossing method. 
\subitem The main question that needs to be addressed is when and how should the modes 
that do not have a physical meaning be removed. The question of when is referring 
to which combination of grid points and numerical schemes give a sufficient answer?
If a sufficient answer is decided when we have a certain order of accuracy, this 
does not necessarily mean we will have the correct modes. 
\subitem If we know that a certain scheme (2nd order or 4th order) needs a certain
number of grid points per wavelength, then there is an inherent mode limit that comes 
from the numerical scheme of choice. This can determine the bounds of the ``reduced mode
problem''\ldots 
\end{enumerate}


\section{Research Performed}
The L2 and Error was added to the output for all four LEE equations and speed of
sound at each grid. The Error better shows a comparison between the actual and expected 
solutions. 

\section{Issues and Concerns}
I'd like to start using second order or fourth order dissipation but by finishing 
my first attempt at a solution validation, I'll get a better idea of what I would
need for dissipation if I need it at all for the scheme and grid I choose. 

\section{Planned Research}
Make is going to be used to call the different modules for the two (or more MMS 
tests). I have been switching back and fourth and Make will just make it so the 
files can be run as needed. The MMS results for the second solution will be 
reported.
After a certain mode, the modes begin to get noisy and the hypothesis is because
there is not enough resolution (in one of the numerical methods or derivatives) 
so the number of modes to look at at once will be decided. If we are only looking
at 4 modes for example, then we need to have enough resolution for that test.






\end{document}


