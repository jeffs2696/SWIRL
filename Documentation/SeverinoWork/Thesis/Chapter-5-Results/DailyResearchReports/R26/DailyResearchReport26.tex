
%        File: WeeklyResearchReport_4_19_21.tex
%     Created: Mon Apr 19 08:00 AM 2021 E
% Last Change: Mon Apr 19 08:00 AM 2021 E
%
\documentclass[a4paper]{article}
\begin{document}
\begin{titlepage}

    \title{
    Daily Research Report}

    \author{ Jeffrey Severino \\
        University of Toledo \\
        Toledo, OH  43606 \\
    email: jseveri@rockets.utoledo.edu}


    \maketitle

\end{titlepage}
\section{Current Research Direction}
The current research direction is to investigate the Taylor Series central spatial
differencing for the radial derivatives needed in SWIRL. The investigation will 
be done by looking at the wavenumber performance.
\section{Research Performed}
Defining the numerical wavenumber $(k \Delta x )*$ as 

\begin{equation}
    \left( \frac{\partial f}{\partial x} \right)|_{numerical}=
    ik \left( \frac{\left( k  \Delta x \right)*}{k \Delta x} \right) \exp^{ikx_0}
\end{equation}
If we plot the parameter $\left[ k \Delta x \right]*$ as a function of 
$k \Delta x $ from $0 \leq k \Delta x \leq \pi$ we could compare second order
fourth order and the RDRP schemes used in SWIRL. 
The second order scheme was analyzed, but the fourth order and RDRP.

The next two plots should be the numerical wavespeed and its error

\begin{equation}
    \tilde{c} = \left( \frac{\left( k \Delta x  \right)^*}{k \Delta x} \right) 
\end{equation}


\begin{equation}
    \epsilon = \left( \frac{\left( k \Delta x  \right)^*}{k \Delta x} - 1 \right) 
\end{equation}
\section{Issues and Concerns}
None at the moment. The question is how many grid points per wavelength is needed
to attain $90\%$?$99\%$?$99.9\%$?


\section{Planned Research}
Complete the analysis for these schemes and provide plots that show their performance.




\end{document}


