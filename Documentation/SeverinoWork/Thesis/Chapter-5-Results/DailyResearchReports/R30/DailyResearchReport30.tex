%        File: WeeklyResearchReport_4_19_21.tex
%     Created: Mon Apr 19 08:00 AM 2021 E
% Last Change: Mon Apr 19 08:00 AM 2021 E
%
\documentclass[a4paper]{article}
\usepackage{mathtools}
\usepackage{verbatim}
\usepackage{graphicx}
\usepackage{tabularx}
\usepackage{pgfplots}
\usepackage{adjustbox}
\usepackage{booktabs}
\makeatletter
\let\latex@xfloat=\@xfloat
\def\@xfloat #1[#2]{%
    \latex@xfloat #1[#2]%
    \def\baselinestretch{1}
    \@normalsize\normalsize
    \normalsize
}
\makeatother
\usepackage{amsmath}
\usepackage{mathtools}
\usepackage{epigraph}
\usepackage{cancel}
\usepackage{xcolor}
\newcommand\Ccancel[2][black]{\renewcommand\CancelColor{\color{#1}}\cancel{#2}}
\usepackage{algorithm}
\usepackage{graphicx}
\usepackage[noend]{algpseudocode}
\usepackage{gnuplot-lua-tikz}
\usepackage[utf8]{inputenc}
\usepackage{pgfplots}
\usepackage{tabularx}
\usepackage{hyperref}
\DeclareUnicodeCharacter{2212}{−}
\usepgfplotslibrary{groupplots,dateplot}
\usetikzlibrary{patterns,shapes.arrows}
\pgfplotsset{compat=newest}
\begin{document}
\begin{titlepage}

    \title{
    Daily Research Report}

    \author{ Jeffrey Severino \\
        University of Toledo \\
        Toledo, OH  43606 \\
    email: jseveri@rockets.utoledo.edu}


    \maketitle

\end{titlepage}
\section{Current Research Direction}
The goal here is to stop running SWIRL on Tuesday, June 14th. All the data that 
will be needed for the thesis needs to be final.
\section{Research Performed}
The Fortran F/OSS Programmers Group has released a utility called ford that 
automatically generates FORTRAN documentation. The installation details (2 commands)
are provided in the README file on the github page \href{https://github.com/Fortran-FOSS-Programmers/ford}{here}. 
The output is an HTML link that serves as user manual for the code. The workflow is 
quite simple. After installation of \verb|ford|, a \verb|project-file.md| is used
as a way of formatting the output of \verb|ford|. Initally, it can just be created 
using a \verb|touch| command in the source directory. Right now it is blank. Then,
from within the source directory the command,

\begin{verbatim}
ford -d . project-file.md
\end{verbatim}
will generate a \verb|doc/| directory with the HTML file and other goodies. To make
comments register to \verb|ford| from within your FORTRAN code, simply comment with
\verb|!!| instead of \verb|!|. The output will be \verb|index.html| and can be 
opened in linux using \verb|xdg-open|. The home page is included in this directory.  

As I wrap up my SWIRL code work, I now know a method of documentation that will
guide the user in using the code moving forward.



\section{Issues and Concerns}
This does count as refactoring and should not be done simultaneously. Comments will
be added as I finish up data analysis. It is more important that the framework is in
place. The comments can be improved after the thesis is complete unless It is okay 
to put my code in the Thesis itself.

One issue I am having is that LAPACK gives an odd error at high number of iterations
and the error is a \verb|SIGFPE| from within the \verb|f90_zggev.f90 INTERFACE SUBROUTINE| .
Catching this error has been a bit brutal.

\section{Planned Research}
Finish write up on MMS, the data is gathered, but writing about it is a challenge. The doubling of 
grid points per iteration makes the code quite long to run\dots

\end{document}


