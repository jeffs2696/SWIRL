%        File: WeeklyResearchReport_4_19_21.tex
%     Created: Mon Apr 19 08:00 AM 2021 E
% Last Change: Mon Apr 19 08:00 AM 2021 E
%
\documentclass[a4paper]{article}
\usepackage{mathtools}
\usepackage{verbatim}
\usepackage{graphicx}
\usepackage{tabularx}
\usepackage{pgfplots}
\usepackage{adjustbox}
\usepackage{booktabs}
\makeatletter
\let\latex@xfloat=\@xfloat
\def\@xfloat #1[#2]{%
    \latex@xfloat #1[#2]%
    \def\baselinestretch{1}
    \@normalsize\normalsize
    \normalsize
}
\makeatother
\usepackage{amsmath}
\usepackage{mathtools}
\usepackage{epigraph}
\usepackage{cancel}
\usepackage{xcolor}
\newcommand\Ccancel[2][black]{\renewcommand\CancelColor{\color{#1}}\cancel{#2}}
\usepackage{algorithm}
\usepackage{graphicx}
\usepackage[noend]{algpseudocode}
\usepackage{gnuplot-lua-tikz}
\usepackage[utf8]{inputenc}
\usepackage{pgfplots}
\usepackage{tabularx}
\DeclareUnicodeCharacter{2212}{−}
\usepgfplotslibrary{groupplots,dateplot}
\usetikzlibrary{patterns,shapes.arrows}
\pgfplotsset{compat=newest}
\begin{document}
\begin{titlepage}

    \title{
    Weekly Research Report}


    \author{ Jeffrey Severino \\
        University of Toledo \\
        Toledo, OH  43606 \\
    email: jseveri@rockets.utoledo.edu}


    \maketitle

\end{titlepage}
\section{Current Research Direction}
\section{Research Performed In the Past 24 hours}

\subsection{Current Validation Work}

A comparison was conducted for a hollow cylinder undergoing uniform flow with
acoustic liners along the outer duct perimeter. The azimuthal mode number, reduced 
frequency, mach number and duct liner admittance is reported below,
\begin{align*}
    m &= 2 \\
    k &= \frac{\omega r_T}{A_T} = -1 \\
    M_x &= 0.5 \\
    \eta_T &= 0.72 + 0.42i
\end{align*} 
%\begin{figure}[h!]
%    \centering
%    \includegraphics[width=\textwidth]{tex-outputs/gam.acc.scatter.Table4.3.pdf}
%\end{figure}
%






The results shown in \ref{Table43} are in moderately good agreement. The 
results were obtained by visually comparing the output in \verb|gam.acc| for 32 
grid points. Note that the indicies for the SWIRL deliverable are different that 
the ones obtained for the most recent version of the code. While the 
convective axial wavenumbers show agreement to machine precision, this is not 
particularly insightful given that there are an infinite number of possible solutions 
that could satisfy the eigenvalue problem. The results that are of concern 
are propagating modes that are not convecting with the mean flow.  The scatter plot
of the axial wavenumbers show some sporadic behaviour around the imaginary axis.
The results from the MMS along with this plot indicate that more grid points are going 
to be needed if a finite difference technique is to be used. It should be 
noted that a spectral differencing method were using for Kousen's report and for
srcF2008. Using a higher order scheme would also improve accuracy.


\begin{table}
 \centering
 \begin{adjustbox}{width=1\textwidth}
     \small
 \begin{tabular}{c | r | r | r | r | r | r}
 \hline
 $\gamma^{\pm}_n$ & Kousen Ref. [15] & Kousen report & srcF2008 & index  & current & index\\
 \hline
 $\gamma_0^{+}$ & $ 0.620 - 5.014  i $ & $ 0.6195 - 5.0139 i$ & $ 0.61954  - 5.01386 i$ & 60  & 0.620755853112 - 5.00592416941i& 34 \\
 $\gamma_1^{+}$ & $-5.820 - 3.897  i $ & $-5.8195 - 3.8968 i$ & $-5.81953  - 3.89677 i$ & 58  &-0.581267772517 - 3.90050864568i& 33 \\
 $\gamma_2^{+}$ & $ 0.445 - 9.187  i $ & $ 0.4453 - 9.1868 i$ & $ 0.44533  - 9.18684 i$ & 59  &0.451569491142 -  9.12191317214i & 31 \\
 $\gamma_3^{+}$ & $ 0.453 - 13.062 i $ & $ 0.4539 - 13.062 i$ & $ 0.45389  - 13.0615 i$ & 57  &0.464247902898 - 12.8487472519i & 29 \\ 
 $\gamma_4^{+}$ & $ 0.480 - 16.822 i $ & $ 0.4795 - 16.822 i$ & $ 0.47952  - 16.8216 i$ & 55  &0.492340380223 - 16.3292825150i & 27 \\
 $\gamma_5^{+}$ & $ 0.503 - 20.531 i $ & $ 0.5029 - 20.531 i$ & $ 0.50287  - 20.5307 i$ & 51  &0.514522630594 -19.5817182568i& 25 \\
 $\gamma_6^{+}$ & $ 0.522 - 24.213 i $ & $ 0.5220 - 24.213 i$ & $ 0.52202  - 24.2129 i$ & 50  &0.516658239854 -22.5715880605i& 23 \\
 $\gamma_7^{+}$ & $ 0.538 - 27.880 i $ & $ 0.5376 - 27.880 i$ & $ 0.53754  - 27.8800 i$ & 48  & - & - \\
 $\gamma_8^{+}$ & $ 0.550 - 31.537 i $ & $ 0.5502 - 31.537 i$ & $ 0.55024  - 31.5368 i$ & 47  & - & - \\
 $\gamma_9^{+}$ & $ 0.589 - 49.75  i $ & $ 0.5891 - 49.754 i$ & $ 0.58745  - 49.7669 i$ & 33  &- &- \\ \hline
 $\gamma_0^{-}$ & $ 0.410 + 1.290  i $ & $ 0.4101 + 1.2904 i$ & $ 0.41009  + 1.29037 i$ & 64  &0.409973310292  + 1.29020083859i& 64 \\
 $\gamma_1^{-}$ & $ 1.259 + 6.085  i $ & $ 1.2595 + 6.0852 i$ & $ 1.25949  + 6.08517 i$ & 63  &1.25530612217  + 6.07214375548i & 62 \\
 $\gamma_2^{-}$ & $ 1.146 + 9.668  i $ & $ 1.1457 + 9.6679 i$ & $ 1.14567  + 9.66787 i$ & 62  &1.13696444935  + 9.59622801724i &  60\\
$\gamma_3^{-}$ & $ 1.022 + 13.315 i $ & $ 1.0218 + 13.315 i$ & $ 1.02183  + 13.3150 i$ & 61  &1.00950576515 + 13.0957277529i & 58  \\
 $\gamma_4^{-}$ & $ 0.943 + 16.977 i $ & $ 0.9425 + 16.977 i$ & $ 0.94250  + 16.9767 i$ & 56  &0.928059983039 +  16.4791343118i& 56  \\
 $\gamma_5^{-}$ & $ 0.891 + 20.635 i $ & $ 0.8908 + 20.635 i$ & $ 0.89075  + 20.6353 i$ & 54  &0.856678172769 +  22.6544943903i & 52 \\
 $\gamma_6^{-}$ & $ 0.855 + 24.288 i $ & $ 0.8549 + 24.288 i$ & $ 0.85490  + 24.2883 i$ & 53  &0.941762848775 +  25.3460188358i & 50 \\
 $\gamma_7^{-}$ & $ 0.829 + 27.937 i $ & $ 0.8288 + 27.937 i$ & $ 0.82877  + 27.9369 i$ & 52  &- & - \\
 $\gamma_8^{-}$ & $ 0.809 + 31.581 i $ & $ 0.8089 + 31.581 i$ & $ 0.80891  + 31.5812 i$ & 49  &- & - \\
 $\gamma_9^{-}$ & $ 0.755 + 49.77  i $ & $ 0.7547 + 49.772 i$ & $ 0.75658  + 49.7851 i$ & 39  &- & - \\ \hline
 \end{tabular}
\end{adjustbox}
 \caption{Table 4.3 data}
 \label{Table43}
\end{table}


\begin{table}
 \centering
 \begin{adjustbox}{width=1\textwidth}
     \small
 \begin{tabular}{c | r | r | r | r | r | r }
 \hline
 $\gamma^{\pm}_n$ & srcF2008 & index  & 32 points & index & 64 points& index\\
 \hline
 $\gamma_0^{+}$ &  $ 0.61954  - 5.01386 i$ &$ 60$  &$ 0.620755853112 - 5.00592416941i$ & $34 $ & $0.619830466387E+00 -0.501195898338E+01i$ &68 \\
 $\gamma_1^{+}$ &  $-5.81953  - 3.89677 i$ &$ 58$  &$-0.581267772517 - 3.90050864568i$& $33 $ & $-0.581874144252E+01 -0.389719406459E+01 i$ &67\\
 $\gamma_2^{+}$ &  $ 0.44533  - 9.18684 i$ &$ 59$  &$ 0.451569491142 - 9.12191317214i$ &$ 31$&$0.446784510254E+00 -0.917151486382E+01 i$& 65 \\
 $\gamma_3^{+}$ &  $ 0.45389  - 13.0615 i$ &$ 57$  &$ 0.464247902898 - 12.8487472519i $& $29 $&$  0.456385619609E+00 -0.130115227368E+02i$& 63   \\ 
 $\gamma_4^{+}$ &  $ 0.47952  - 16.8216 i$ &$ 55$  &$ 0.492340380223 - 16.3292825150i $& $27 $&$ 0.482906458331E+00 -0.167063669858E+02i$ &61\\
 $\gamma_5^{+}$ &  $ 0.50287  - 20.5307 i$ &$ 51$  &$ 0.514522630594 -19.5817182568i$& $25$ & $0.506963241913E+00 -0.203096267281E+02i$ &59 \\
 $\gamma_6^{+}$ &  $ 0.52202  - 24.2129 i$ &$ 50$  &$ 0.516658239854 -22.5715880605i$& $23$ & $0.526558860613E+00 -0.238358532167E+02i$ & 55 \\
 $\gamma_7^{+}$ &  $ 0.53754  - 27.8800 i$ &$ 48$  & - & -&$0.542123590089E+00 -0.272859574060E+02i$ & 53 \\                       
 $\gamma_8^{+}$ &  $ 0.55024  - 31.5368 i$ &$ 47$  & - & - & $0.554191417366E+00 -0.306549838386E+02i $ & 51 \\ 
 $\gamma_9^{+}$ &  $ 0.58745  - 49.7669 i$ &$ 33$  &- &- \\ \hline  
 $\gamma_0^{-}$ &  $ 0.41009  + 1.29037 i$ &$ 64$  &$0.409973310292  + 1.29020083859i$& $64 $ & $0.410069261267E+00  0.129033632980E+01i $ &128 \\
 $\gamma_1^{-}$ &  $ 1.25949  + 6.08517 i$ &$ 63$  &$1.25530612217  + 6.07214375548i $& $62 $&$0.125845417744E+01  0.608210427128E+01i$ & 126\\
 $\gamma_2^{-}$ &  $ 1.14567  + 9.66787 i$ &$ 62$  &$1.13696444935  + 9.59622801724i $& $ 60$ &$0.114350845928E+01  0.965104848780E+01i$& 124 \\
 $\gamma_3^{-}$ & $ 1.02183  + 13.3150 i$ & $61  $&$1.00950576515 + 13.0957277529i $&$ 58 $ & $0.101870775473E+01  0.132634680645E+02i $& 122\\
 $\gamma_4^{-}$ &  $ 0.94250  + 16.9767 i$ &$ 56$  &$0.928059983039 +  16.4791343118i$& $56 $&$0.938535330387E+00  0.168600097406E+02i $& 120\\
 $\gamma_5^{-}$ &  $ 0.89075  + 20.6353 i$ &$ 54$  &$0.856678172769 +  22.6544943903i$ &$ 52$ &$0.886130853983E+00  0.204129469402E+02i$ &118 \\
 $\gamma_6^{-}$ &  $ 0.85490  + 24.2883 i$ &$ 53$  &$0.941762848775 +  25.3460188358i$ &$ 50$ & $0.849927767573E+00  0.239101576350E+02i$ & 116 \\
 $\gamma_7^{-}$ &  $ 0.82877  + 27.9369 i$ &$ 52$  &- & -& $0.823875448687E+00  0.273421537697E+02i$ & 114 \\
 $\gamma_8^{-}$ &  $ 0.80891  + 31.5812 i$ & 49  &- & -& $0.804827463709E+00  0.306993188999E+02i$ & 112 \\
 $\gamma_9^{-}$ &  $ 0.75658  + 49.7851 i$ & 39  &- & -& $$ & \\ \hline
 \end{tabular}
\end{adjustbox}
\caption{Table 4.3 data with higher resolution}
 \label{Table43_64}
\end{table}

Using more grid points improved the comparison between wavenumbers, however 
$\gamma_9^{\pm}$ were difficult to identify, however with 32 grid points, $\gamma_{7-9}^{\pm}$ were 
also difficult to identify. In general, there is okay agreement, but this shows 
that more grid is needed or a higher order scheme. Note the dissimilarity between
indicies, but the wavenumbers are relatively near each other, being only 2-5
indicies away from each other. 

\begin{table}
 \centering
 \begin{adjustbox}{width=1\textwidth}
     \small
 \begin{tabular}{c | r | r | r | r | r | r }
 \hline
 $\gamma^{\pm}_n$ & srcF2008 & index  & 128 points & index & 216 points& index\\
 \hline
 $\gamma_0^{+}$ &  $ 0.61954  - 5.01386 i$ & $60$ & $0.619612254146E+00 -0.501339663670E+01i $ & $234$ & $0.619559288719E+00 -0.501374773990E+01$ & $494$ \\
 $\gamma_1^{+}$ &  $-5.81953  - 3.89677 i$ &$ 58$  &$-0.581944739477E+01 -0.389682536102E+01i$& $233 $ & $-0.581952656873E+01 -0.389678398044E+01 i$ & $493$\\
 $\gamma_2^{+}$ &  $ 0.44533  - 9.18684 i$ &$ 59$  &$0.445682448956E+00 -0.918312004621E+01 i$ &$ 231$ & $$ & \\
 $\gamma_3^{+}$ &  $ 0.45389  - 13.0615 i$ & $ 57$  & $0.454498201155E+00 -0.130494173106E+02i $& $229 $&$  $&    \\ 
 $\gamma_4^{+}$ &  $ 0.47952  - 16.8216 i$ &$ 55$  &$0.480366242866E+00 -0.167938280084E+02 i $& $227 $&$ $ &\\
 \end{tabular}
\end{adjustbox}
 \caption{Table 4.3 data}
 \label{Table43}
\end{table}
It is apparent that the solution is approaching a known answer, but by 216 grid points,
we are getting about 5-6 digits of precision. 


\section{Issues and Concerns}
The convective wavenumbe couldn't be calculated when the radius was zero so I 
used L'Hopital's rule:

\begin{verbatim}

        do i = 1,np
            r  =  rr(i)
            rx =  rmx(i)
            rt =  rmt(i)
            as =  snd(i)
            if (r.eq. 0.0_rDef ) then
                cv = (omega/as - mode*rt)/rx 
            else 
            ! WRITE(0,*) rx, r
                cv = (omega/as -mode/r*rt)/rx
            endif
!            WRITE(0,*) cv 
            if (abs(cv).gt.cvcmax) cvcmax = abs(cv)
            if (abs(cv).lt.cvcmin) cvcmin = abs(cv)
        enddo
\end{verbatim}
\section{Planned Research}
Go back and obtain the L2 plots for the MMS to put a stamp on it and move on. 
It'll serve as a guideline And I will only use the points that i \textit{know } will 
give me decent answers. I realized this is pointless when we know what it takes 
to get a good answer.

\end{document}


