%        File: WeeklyResearchReport_4_19_21.tex
%     Created: Mon Apr 19 08:00 AM 2021 E
% Last Change: Mon Apr 19 08:00 AM 2021 E
%
\documentclass[a4paper]{article}
\usepackage{mathtools}
\usepackage{verbatim}
\usepackage{graphicx}
\usepackage{tabularx}
\usepackage{pgfplots}
\usepackage{adjustbox}
\usepackage{booktabs}
\makeatletter
\let\latex@xfloat=\@xfloat
\def\@xfloat #1[#2]{%
    \latex@xfloat #1[#2]%
    \def\baselinestretch{1}
    \@normalsize\normalsize
    \normalsize
}
\makeatother
\usepackage{amsmath}
\usepackage{mathtools}
\usepackage{epigraph}
\usepackage{cancel}
\usepackage{xcolor}
\newcommand\Ccancel[2][black]{\renewcommand\CancelColor{\color{#1}}\cancel{#2}}
\usepackage{algorithm}
\usepackage{graphicx}
\usepackage[noend]{algpseudocode}
\usepackage{gnuplot-lua-tikz}
\usepackage[utf8]{inputenc}
\usepackage{pgfplots}
\usepackage{tabularx}
\DeclareUnicodeCharacter{2212}{−}
\usepgfplotslibrary{groupplots,dateplot}
\usetikzlibrary{patterns,shapes.arrows}
\pgfplotsset{compat=newest}
\begin{document}
\begin{titlepage}

    \title{
    Thesis Introduction}


    \author{ Jeffrey Severino \\
        University of Toledo \\
        Toledo, OH  43606 \\
    email: jseveri@rockets.utoledo.edu}


    \maketitle

\end{titlepage}
\section{Thesis introduction}
During the 1960s, the increased demand for commercialized aircraft transport
introduced jet engines to support large cargo and passengers. Consequently,
this rise in innovation resulted in high volume engine noise. After 1975, 
efforts to reduce aircraft noise eliminated the noise pollution for 90\% of
the population \cite{FAAPolicy}. However, since the early 2000s,
the advancement in noise reduction technologies has been moderately increasing,
leaving a requirement for drastic improvement in aeroacoustic modeling and treatment
strategies to compete with the demand for quiet subsonic flight.  
A turbomachine's general flow condition includes a series of axial, tangential,
and radial velocity components that vary depending on the location of concern.
The swirling flow between fan stages has been an area of interest due to the
potential for acoustic treatment in a location previously avoided for its flow complexity, among 
other reasons.

In general, jet engine designers can model flow within a turbomachine with 
the Navier Stokes Equations, a set 
of partial differential equations that describe the mass, momentum and energy
of a given viscous fluid. It is common in practice to utilize the Euler equations,
a closely related set of PDEs that model inviscid fluid, as they provide an 
approximation for higher Reynold number flows where viscosity 
does not play a critical role. A popular approach to modeling sound propagation 
within a flow is to ``linearize'' the Euler equations, which decomposes the 
flow solution into a mean and fluctuating component (insert refs). Another method
decomposes the flow into vortical and potential parts [\cite{golubev1996sound}] 
In either case, this presents an initial value problem and for certain flows and
domains, can obtain analytical solutions.  The LEE provides a system of linear 
equations where the solution is a family of wavenumbers and radial mode shapes 
that arise from unsteady disturbances.  For uniform flows in a hard wall duct , 
the waves are categorized as vortical and entropical waves that soley convect 
with the mean flow, where as the acoustic wave can propagate without damping or 
decay exponentially.  However, for swiriling flows, the waves are partially coupled
and are not easily categorized due to an additonal category of ``nearly convecting'' 
modes (Kerrebrock) and do not form a complete basis system for all wave types.
Therefore, the families of waves must be found numerically \cite{Envia2004}
making the ducted acoustic propagtion in swirling flow a problem without
an analytical solution but has a framework for a numerical solution.

Swirling flow has been a difficult problem to investigate in comparison to 
flows parallel to the wall domain of a duct \cite{COOPER2001} because of the 
lack of an analytical solution and thus cannot be described from a single convective 
wave equation. Various methods of code validation have been
conducted, however, a gold standard method of code verification offers a measure
of ``goodness'' that would strengthen the results of a numerical solution by 
providing a ``psudo-analytical'' solution which is can then be compared to the 
numerical solution.   

The proposed research aims to determine the impact of the numerical schemes used
in the swirling flow problem and how it effects the family of waves that are
produced from the problem formulation so a better understanding of the 
acoustic phenomena as the flow under goes a compressible rotational flow. The use
of the method of manufactured solutions is used as a means of ensuring the code is
approximating the right equations and will check the effect of the numerical schemes
on the axial wavenumbers produced 
\section{Issues and Concerns}
Is the research gap truly stated in the introduction?

\section{Planned Research} 

Ensure that the research problem is clearly stated and that the need for this
work is explained.
\bibliography{references}
\bibliographystyle{unsrt}
\end{document}


