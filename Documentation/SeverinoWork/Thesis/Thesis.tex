\documentclass[12pt]{uthesis-v12}  %---> DO NOT ALTER THIS COMMAND
\usepackage{amsmath}
\usepackage{mathtools}
\usepackage{epigraph}
\usepackage{cancel}
\usepackage{xcolor}
\newcommand\Ccancel[2][black]{\renewcommand\CancelColor{\color{#1}}\cancel{#2}}
\usepackage{algorithm}
\usepackage{graphicx}
\usepackage[noend]{algpseudocode}
\usepackage{gnuplot-lua-tikz}
\begin{document} %---> %---> %---> %---> DO NOT ALTER THIS COMMAND

%--------+----------------------------------------------------------+
%        |  \title{}                                    (REQUIRED)  |
%        |  \author{}                                   (REQUIRED)  |
%        |                                                          |
%        |  See section 3.1 of "Read_Me_First_(v12).pdf"            |
%        |                                                          |
%        |  Also see section 2.2 of above "Read Me" file for the    |
%        |  proper use of the invisible tilde ("~") character when  |
%        |  entering a middle initial in the \author command.       |
%        +----------------------------------------------------------+

\title{ Verification and Validation Method for 
\protect\\an Acoustic Mode Prediction Code for Turbomachinery Noise}

\author{Jeffrey Severino} 
%--------+----------------------------------------------------------+
%        |  \copyrightpage{}                            (REQUIRED)  |
%        |                                                          |
%        |  See section 3.2 of "Read_Me_First_(v12).pdf"            |
%        |                                                          |
%        |  1) You must enter either "yes" or "no" in this          |
%        |      command.  Inputting "yes" produces a copyright      |
%        |      notification page as the second page and inputting  |
%        |      "no" produces a blank second page.                  |
%        |  2) Input to this command is case sensitive.             |
%        |  3) Default: the "yes" option.                           |
%        +----------------------------------------------------------+

\copyrightpage{yes}

%--------+----------------------------------------------------------+
%        |  \mydocument{}                               (REQUIRED)  |
%        |                                                          |
%        |  See section 3.3 of "Read_Me_First_(v12).pdf"            |
%        |                                                          |
%        |  1) Input to this command is limited to the following    |
%        |     three options: a) Dissertation                       |
%        |                    b) Thesis                             |
%        |                    c) Project                            |
%        |  2) Input to this command is case-sensitive.             |
%        +----------------------------------------------------------+

\mydocument{Thesis}

%--------+----------------------------------------------------------+
%        |  \degree{}{}                                 (REQUIRED)  |
%        |                                                          |
%        |  See section 3.4 of "Read_Me_First_(v12).pdf"            |
%        |                                                          |
%        |  You need to provide two distinct inputs into this       |
%        |  command:                                                |
%        |     1) In the first set of braces you need to specify    |
%        |        the *exact* degree you will receive. Some         |
%        |        examples are: -) Masters of Arts                  |
%        |                      -) Masters of Science               |
%        |                      -) Doctor of Philosophy             |
%        |     2) In the second set of braces you need to state the |
%        |        *specific* discipline or area for that degree     |
%        |        (e.g., Economics, Education, Engineering, etc.).  |
%        |  Students should consult their advisor if they have any  |
%        |  questions about this information.                       |
%        +----------------------------------------------------------+

\degree{Masters of Science}{Mechanical Engineering}

%--------+----------------------------------------------------------+
%        |  \conferraldate{}{}                          (REQUIRED)  |
%        |                                                          |
%        |  See section 3.5 of "Read_Me_First_(v12).pdf"            |
%        |                                                          |
%        |  In the two set of braces enter the month and then the   |
%        |  year your degree will be *conferred* by the university. |
%        +----------------------------------------------------------+

\conferraldate{May}{2022}

%--------+----------------------------------------------------------+
%        |  \advisor{}                                  (REQUIRED)  |
%        |                                                          |
%        |  See section 3.6.2 of "Read_Me_First_(v12).pdf"          |
%        |                                                          |
%        |  1) Also see section 2.2 of "Read_Me_First_(v12).pdf"    |
%        |     for the proper use of the invisible tilde ("~")      |
%        |     character when entering a middle initial or the      |
%        |     abbreviation of an academic title (e.g., Dr.) in     |
%        |     the \advisor{} command.                              |
%        |  2) Also see section 3.6.1. for consistent presentation  |
%        |     of title page signature lines.                       |
%        +----------------------------------------------------------+

\advisor{Dr.~Ray Hixon}

%--------+----------------------------------------------------------+
%        |  Committee Member Signature Commands         (OPTIONAL)  |
%        |                                                          |
%        |  See section 3.6.3 of "Read_Me_First_(v12).pdf"          |
%        |                                                          |
%        |  1) Use the commands below to provide signature lines    |
%        |     for your "other" committee members;                  |
%        |        --> you must list your other committee members    |
%        |            in alphabetic order by last name              |
%        |        --> to do this, use the commands below in the     |
%        |            order presented below.                        |
%        |  2) You can choose to include none, some, or all of the  |
%        |     "XXXmember" commands below --- based on the number   |
%        |     committee members you have; simply delete (or        |
%        |     comment-out) any of the commands below that are not  |
%        |     needed.                                              |
%        |  3) Do not include the name of your committee chair or   |
%        |     the Graduate Dean in the commands listed below.      |
%        |     Their signature lines are generated by the           |
%        |     \advisor{} and \graduatedean{}{} commands.           |
%        |  4) You cannot use any of the commands below more than   |
%        |     once. (For details on this issue, see section 3.6.3  |
%        |     of "Read_Me_First_(v12).pdf".)                       |
%        |  5) Also see section 2.2 of "Read_Me_First_(v12).pdf"    |
%        |     for the proper use of the invisible tilde ("~")      |
%        |     character when entering a middle initial or the      |
%        |     abbreviation of an academic title (e.g., Dr.) in     |
%        |     the commands below.                                  |
%        |  6) See section 3.6.1. for consistent presentation of    |
%        |     title page signature lines.                          |
%        |                                                          |
%        |  I know I shouldn't have to say this, but enough         |
%        |  students over the years have made the same mistake      |
%        |  that I'm forced to state:                               |
%        |                                                          |
%        |      THE NAMES USED IN THE FOLLOWING COMMANDS ARE        |
%        |      SILLY NAMES I'VE USED AS EXAMPLES ONLY.  THEY       |
%        |      ARE NOT THE ACTUAL NAMES OF YOUR COMMITTEE          |
%        |      MEMBERS.  REPLACE THE SILLY NAMES BELOW WITH        |
%        |      THE NAMES OF YOUR ACTUAL COMMITTEE MEMBERS.         |
%        |                                                          |
%        +----------------------------------------------------------+

  \secondmember{Dr.~Chinhua Sheng}
%   \thirdmember{Dr.~Chris P.~Bacon}
%  \fourthmember{Dr.~Adam Baum}
%   \fifthmember{Dr.~Corey O.~Graff}
%   \sixthmember{Dr.~Hugh Jass}
% \seventhmember{Dr.~Noah Lott}
%  \eighthmember{Dr.~Jean Poole}
%
%--------+----------------------------------------------------------+
%        |  \graduatedean{}{}                           (REQUIRED)  |
%        |                                                          |
%        |  See section 3.6.4 of "Read_Me_First_(v12).pdf"          |
%        |                                                          |
%        |  1) THE NAME AND TITLE PROVIDED BELOW ARE THOSE OF THE   |
%        |     ACTUAL GRADUATE DEAN AT THE TIME THIS DOCUMENT WAS   |
%        |     CONSTRUCTED (January 2012). Contact the Graduate     |
%        |     College to determine whether this information is     |
%        |     correct at the time you submit your document.        |
%        |  2) Section 2.2 of "Read_Me_First_(v12).pdf" describes   |
%        |     the proper use of the invisible tilde ("~")          |
%        |     character when entering a middle initial or the      |
%        |     abbreviation of an academic title (e.g., Dr.) in     |
%        |     the \graduatedean{} command.                         |
%        |  3) See section 3.6.1. for consistent presentation of    |
%        |     title page signature lines.                          |
%        +----------------------------------------------------------+

\graduatedean{Dr.~Patricia R.~Komuniecki}{Dean}

%--------+----------------------------------------------------------+
%        |  \maketitle                                  (REQUIRED)  |
%        |                                                          |
%        |  See section 3.7 of "Read_Me_First_(v12).pdf"            |
%        |                                                          |
%        |  This is a required LaTeX command; to be brief, bad      |
%        |  things will happen if this command is not included      |
%        |  in your document at this particular location.           |
%        +----------------------------------------------------------+

\maketitle  %---->  ----->  ---->  ---->   DO NOT ALTER THIS COMMAND

%--------+----------------------------------------------------------+
%        |  Abstract Page Environment                   (REQUIRED)  |
%        |                                                          |
%        |  See section 3.8 of "Read_Me_First_(v12).pdf"            |
%        +----------------------------------------------------------+

\begin{abstractpage}
Over the last 20 years, there has been an increase in computational fluid dynamic 
codes that have made numerical analysis more and more readily available, allowing
turbomachine designers to create more novel designs. However, as airport noise
limitations become more restrictive over time, reducing aircraft 
takeoff and landing noise remains a prominent issue in the aviation community. 
One popular method to reduce aircraft noise is using acoustic liners placed on 
the walls of the engine inlet and exhaust ducts. These liners are designed to 
reduce the amplitude of acoustic modes emanating from the bypass fan as they 
propagate through the engine. The SWIRL code is a frequency-domain linearized 
Euler equation solver that is designed to predict the effect of acoustic liners
on acoustic modes propagating in realistic sheared and swirling mean flows, guiding
the design of more efficient liner configurations. The purpose of this study is
to validate SWIRL using the Method Of Manufactured Solutions (MMS). This study 
also investigated the effect of the integration and spatial differencing methods 
on the convergence for a given Manufactured Solution. In addition, the effect 
of boundary condition implementation was tested.  The improved MMS convergence
rates shown for these tests suggest that the revised SWIRL code provides more 
accurate solutions with less computational effort than the original formulation.



\end{abstractpage}

%--------+----------------------------------------------------------+
%        |  Dedication Page Environment                 (OPTIONAL)  |
%        |                                                          |
%        |  See section 3.9 of "Read_Me_First_(v12).pdf"            |
%        |                                                          |
%        |  If both a dedication page and an acknowledgements page  |
%        |  are included in the document, the dedication page must  |
%        |  proceed the acknowledgements page.                      |
%        +----------------------------------------------------------+

\begin{dedication}
\noindent For my friends and family, who have always believed in my potential 
when I did not believe it myself. 
\end{dedication}

%--------+----------------------------------------------------------+
%        |  Acknowledgments Page Environment            (OPTIONAL)  |
%        |                                                          |
%        |  See section 3.10 of "Read_Me_First_(v12).pdf"           |
%        |                                                          |
%        |  If both a dedication page and an acknowledgements page  |
%        |  are included in the document, the dedication page must  |
%        |  proceed the acknowledgements page.                      |
%        +----------------------------------------------------------+

\begin{acknowledgments}
\noindent This work is supported by the NASa Advanced Air Transport Technologies
(AATT) Project. I would like to thank Edmane Envia(?) of the NASA Glenn Research Center, who
is the technical monitor of this work. 
A very special thanks goes to Dr. Ray Hixon who supervised and guided me through
out my course work and Master's Thesis. His rigor and tenacity in his profession has 
been the model example for an aspiring aeroacoustician. 
I would like to also thank all of my committee members, Dr. Chunhua Sheng and 
Dr. Sorin Cioc. Their contributions have been instrumental. Thanks to Dr. Clifford
Brown for his programatic insights. 

I would also like to thank my focus group peers, Zaid Sabri, and Matthew Gibbons 
for their patience and support over the years. I wish them the best in all of 
their endevours.
I would also like to thank Copper Moon Coffee Shop for the endless supply of caffine 
and conversations.


\end{acknowledgments}

%--------+----------------------------------------------------------+
%        |  \tableofcontents                            (REQUIRED)  |
%        |  \listoftables                            (CONDITIONAL)  |
%        |  \listoffigures                           (CONDITIONAL)  |
%        |                                                          |
%        |  See sections 3.11 & 3.12 of "Read_Me_First_(v12).pdf"   |
%        |                                                          |
%        |  1) You must include the \tableofcontents command in     |
%        |     your document: the UT Manual requires every          |
%        |     dissertation/thesis to have a detailed table of      |
%        |     contents.                                            |
%        |  2) Including the \listoftables and \listoffigures       |
%        |     commands is "conditional."  See sections 3.12 of     |
%        |     "Read_Me_First_(v12).pdf" for additional details.    |
%        +----------------------------------------------------------+

\tableofcontents  %----->  ----->  ---->  DO NOT ALTER THIS COMMAND
\listoftables \listoffigures
\begin{listofsymbols}
    \emblem{$A$        }{mean flow speed of sound}
    \emblem{$A_T      $}{speed of sound at the duct radius}
    \emblem{$\tilde{A}$}{dimensionless speed of sound, $\frac{A}{A_T}$}
    \emblem{$D/Dt$     }{material derivative, $\partial/\partial t + V \cdot \nabla $}
    \emblem{$D_N$      }{derivative matrix using $N$ points}
    \emblem{$\textbf{e}_x$,$\textbf{e}_{\theta}$ }{s}
\end{listofsymbols}
%--------+----------------------------------------------------------+
%        |  \captionformat{}                            (REQUIRED)  |
%        |                                                          |
%        |  See section 3.12.2 of "Read_Me_First_(v12).pdf"         |
%        |                                                          |
%        |  1) You are required to choose between the "hang" or     |
%        |     "align" option for this command.                     |
%        |  2) Input to this command is case sensitive.             |
%        |  3) Default: ``hang'' option.                            |
%        +----------------------------------------------------------+

\captionformat{hang}

%--------+----------------------------------------------------------+
%        |  List of Abbreviations Environment           (OPTIONAL)  |
%        |                                                          |
%        |  See section 3.13 of "Read_Me_First_(v12).pdf"           |
%        |                                                          |
%        |  1) This is an optional section; consult your advisor    |
%        |     to determine whether you need/want to include this   |
%        |     section in your document.                            |
%        |  2) If you do not want a List of Abbreviations simply    |
%        |     delete the material below (and these instructions).  |
%        |  3) If you do want a List of Abbreviations simply        |
%        |     replace the silly material below with the            |
%        |     information relevant to your document.               |
%        |     a. Within the "listofabbreviations" environment      |
%        |        below you must use a separate \abbreviation{}{}   |
%        |        command for each entry in your List of            |
%        |        Abbreviations.                                    |
%        |     b. As the examples below demonstrate, the            |
%        |        information within the first set of braces is     |
%        |        the abbreviation and the information in the       |
%        |        second set of braces is the definition of that    |
%        |        abbreviation.                                     |
%        +----------------------------------------------------------+

\begin{listofabbreviations}

    \abbreviation{CFD}{Computational Fluid Dynamics}
    \abbreviation{GLE}{Gauss' law for electricity: $\nabla\cdot E
                       = \displaystyle\frac{\rho}{\varepsilon_0}
                       = 4\pi k \rho$}
    \abbreviation{HHS}{Department of Health and Human Services}
    \abbreviation{IaR}{I am root}

\end{listofabbreviations}

%--------+----------------------------------------------------------+
%        |  List of Symbols Environment                 (OPTIONAL)  |
%        |                                                          |
%        |  See section 3.14 of "Read_Me_First_(v12).pdf"           |
%        |                                                          |
%        |  1) This is an optional section; consult your advisor    |
%        |     to determine whether you need/want to include this   |
%        |     section in your document.                            |
%        |  2) If you do not want a List of Symbols simply delete   |
%        |     the material below (and these instructions).         |
%        |  3) If you do want a List of Symbols simply replace the  |
%        |     silly material below with the information relevant   |
%        |     to your document.                                    |
%        |       a. Within the "listofsymbols" environment below    |
%        |          you must use a separate \emblem{}{} command     |
%        |          for each entry in your List of Symbols.         |
%        |       b. As the examples below show, insert your symbol  |
%        |          within the first set of braces in the           |
%        |          \emblem{}{} command, and its definition within  |
%        |          the second set of braces.                       |
%        |       c. Use the \emblemskip command to insert a blank   |
%        |          line between different categories of symbols:   |
%        |          -) such additional spacing is required between  |
%        |             different categories of symbols;             |
%        |          -) see "Read_Me_First_(v12).pdf" for details.   |
%        +----------------------------------------------------------+

%--------+----------------------------------------------------------+
%        |  Preface Environment                         (OPTIONAL)  |
%        |                                                          |
%        |  See section 3.15 of "Read_Me_First_(v12).pdf"           |
%        +----------------------------------------------------------+

\begin{preface}
\end{preface}

%XXXXXXXXXXXXXXXXXXXXXXXXXXXXXXXXXXXXXXXXXXXXXXXXXXXXXXXXXXXXXXXXXXXX
%XXXXXXXXXXXXXXXXXXXXXXXXXXXXXXXXXXXXXXXXXXXXXXXXXXXXXXXXXXXXXXXXXXXX
%XXXXXXXXXXXXXXXXXXXXXXXXXXXXXXXXXXXXXXXXXXXXXXXXXXXXXXXXXXXXXXXXXXXX
%XXXXXXXXXXXXXXXXXXXXXXXXXXXXXXXXXXXXXXXXXXXXXXXXXXXXXXXXXXXXXXXXXXXX

%--------+----------------------------------------------------------+
%        |  \makebody                                   (REQUIRED)  |
%        |                                                          |
%        |  See section 3.16 of "Read_Me_First_(v12).pdf"           |
%        |                                                          |
%        |  This is a *required* UThesis command; again, bad        |
%        |  things will happen if this command is not included in   |
%        |  your document at this particular location --- see the   |
%        |  file "Read_Me_First_(v12).pdf" for details.             |
%        +----------------------------------------------------------+

\makebody   %------->  ------->  ------->  DO NOT ALTER THIS COMMAND

%XXXXXXXXXXXXXXXXXXXXXXXXXXXXXXXXXXXXXXXXXXXXXXXXXXXXXXXXXXXXXXXXXXXX
%XXXXXXXXXXXXXXXXXXXXXXXXXXXXXXXXXXXXXXXXXXXXXXXXXXXXXXXXXXXXXXXXXXXX
%XXXXXXXXXXXXXXXXXXXXXXXXXXXXXXXXXXXXXXXXXXXXXXXXXXXXXXXXXXXXXXXXXXXX
%XXXXXXXXXXXXXXXXXXXXXXXXXXXXXXXXXXXXXXXXXXXXXXXXXXXXXXXXXXXXXXXXXXXX

%--------+----------------------------------------------------------+
%        |  \chapter{}                                  (REQUIRED)  |
%        |                                                          |
%        |  See section 3.17 of "Read_Me_First_(v12).pdf"           |
%        |                                                          |
%        |  For guidance on using the commands \chapter{},          |
%        |  \section{}, \subsection{}, \subsubsection{}, etc., see  |
%        |  Leslie Lamport's "LaTeX: A Document Preparation         |
%        |  System." Addison Wesley: Reading Massachusetts, 1985.   |
%        +----------------------------------------------------------+
\chapter{Background}
\section{Introduction}
During the 1960s, the increased demand for commercialized aircraft transport
introduced jet engines to support large cargo and passengers. Consequently,
this rise in innovation resulted in high volume engine noise. After 1975, 
efforts to reduce aircraft noise eliminated the noise pollution for 90\% of
the population \cite{FAAPolicy}. However, since the early 2000s,
the advancement in noise reduction technologies has been moderately increasing,
leaving a requirement for drastic improvement in aeroacoustic modeling and treatment
strategies to compete with the demand for quiet subsonic flight.  
A turbomachine's general flow condition includes a series of axial, tangential,
and radial velocity components that vary depending on the location of concern.
The swirling flow between fan stages has been an area of interest due to the
potential for acoustic treatment in a location previously avoided for its flow complexity, among 
other reasons.

In general, jet engine designers can model flow within a turbomachine with 
the Navier Stokes Equations, a set 
of partial differential equations that describe the mass, momentum and energy
of a given viscous fluid. It is common in practice to utilize the Euler equations,
a closely related set of PDEs that model inviscid fluid, as they provide an 
approximation for higher Reynold number flows where viscosity 
does not play a critical role. A popular approach to modeling sound propagation 
within a flow is to ``linearize'' the Euler equations, which decomposes the 
flow solution into a mean and fluctuating component (insert refs). Another method
decomposes the flow into vortical and potential parts (ref Golubev \& Atassi). 
In either case, this presents an initial value problem and for certain flows and
domains, can obtain analytical solutions.  Swirling flow has been a difficult
problem to investigate in comparison to flows parallel to the wall domain of a duct \cite{COOPER2001}.   
. 
.




%
%- the problem that we're addressing
%
%-- being able to conduct component level code verification tests for the problem
%of characterizing the duct acoustics for flow using a LEE model.
%
%why is it a problem? there is multiple ways of arriving at the same solution. 
%This can be used to give a metric to either method for the various computational
%methods that may be needed to arrive at the final answer.
%
%In the mid 90's,an aeroacoustics model for swirling flow had been proposed and 
%has bypassed using a single PDE and has instead used an eigenvalue approach on the 
%four governing flow equations. (Why Is this better? does this capturE the problem
%differently? why not use the PDE alone instead\ldots)
%
%The proposed component verification from Kleb and Wood will be presented to addreess the characterization of 
%modes and the presence of numerical ones. Using higher accuracy methods should 
%further improve The result. another thing is to determine how many grid points are needed
%for each method
%
%

\chapter{Chapter 2: Literature Review}
\section{Introduction}
\subsection{Define your topic and provide an appropriate context for reviewing the literature}
During the 1960's, the increased demand for commercialized aircraft transport 
introduced jet engines capable of supporting large amounts of cargo and passengers. 
On the contrary, this rise in innovation came with a consequence, high volume 
engine noise.  At the time, thee most significant source of noise within an 
aircraft turbo fan engine was attributed to flow interactions between the 
inlet rotor and adjacent stator. 
It is common in turbomachinery to have a stator behind the inlet rotor (or fan) 
to counteract the rotation in the flow due to the rotor. More importantly, 
this configuration causes wakes to form after the rotor, which come into contact 
with the stator. This is denoted as rotor/stator interaction noise \cite{Tyler1962}. 
The ``Tyler-Sofrin selection rule'' (TSSR) was a fundamental contribution 
to the aerospace community, because it is a powerful yet simple 
tool used to quickly determine whether an engine was likely to have 
observable noise emission from the rotor and stator before the engine was manufactured. 
Despite the fact, these preliminary studies were focused on axial compressor noise, the 
TSSR was a pivotal stepping stone. Its application towards modern 
turbomachinery flow has been attempted by modeling the flow as steady in the 
rotating reference frame of the rotor's blade row. However, turbomachinery by 
its very nature produces unsteady flow, and the TSSR in 
not suitable \cite{Holmes2011}. Despite these dissimilarities, the work of 
\cite{Tyler1962} set a strong foundation for the aerospace community, 
allowing for further advances in turbomachinery noise prediction.

\subsection{Establish reasoning - i.e. point - of - view for reviewing the literature}
\subsubsection{Prior work in non-reflecting, non-linear VGBC}
A large amount of aircraft noise was reduced from 1975-2000, 
effectively eliminating the noise pollution for 90\% of the population \cite{Administration}. 
Since the early 2000's, the advancement in noise reduction technologies has been gradual, 
leaving a requirement for drastic improvement in aeroacoustic treatment 
strategies to compete with the demand of quiet subsonic  flight. 
A previous theoretical review by Envia has been suggested that a non linear 
time domain computation could capture the source generation (incident turbulence) 
in addition to the broadband noise. Such a process would have the capabilities 
of  of solving all components of noise generation in an individual calculation \cite{Envia2004}. 
This would at minimum, require an ``LES-type'' fidelity code, which can tend 
to be computationally expensive.  A promising option is NASA GRC's Broadband 
Aeroacoustic Stator Simulation (BASS).  BASS is a high-order, high 
accuracy computational aeroacoustics (CAA) code which has been used to study 
non linear provides mean an extensive study of non linear phenomena in 
turbomachinery flow, and in particular, mechanisms of noise generation that are 
produced from unsteady disturbances. This code allows for a wide variety of 
finite differencing and time marching schemes as well as artificial dissipation methods. This effective computational tool allows for an in-depth modeling of realistic velocity profiles that would be representative of flow produced from a rotor blade row. In recent work, a new method of implementing realistic, three-dimensional rotor wakes free from acoustics was validated  \cite{Hixon2011}. This provides a means of studying acoustic responses non linear swirling flows within pragmatic geometric configurations, while simultaneously allow for the modeling of sources generation produced from the incident turbulence. 

\subsection{Explain the order/sequence of the review}
This review will compare studies that have considered the applicability of unsteady linearized Euler equations on cylindrical and annular ducts. First the work of Kousen will breifly summarized \cite{Kousen1996,Kousen1999}. Secondly, the alternate approached shown by Golubev and Atassi used in \cite{Golubev1996,Golubev1998} were also compared. These works have utilized a standard normal mode approach to determine the modal response of inviscid, compressible, swirling flow within a cylindrical and annular ducts. Results and findings have revealed three categories of associated wave - modes, acoustic, nearly convected, and nearly sonic. Detailed examination of the literature indicate that the hierarchical system was not initially apparent. A qualitative description of the numerical methods used to evaluate the eigensystem of solutions will be described.

One key question that remains, is how well do unsteady linearized equations capture the mechanisms of noise generation within realistic turbomachinery flow? There exists a copious amount of published work describing the use of wave equations to describe the governing behavior of ducted sound propagation(See \cite{Michel2008} for extensive preliminary review). However, turbomachines inherently rely on high velocities, high temperatures to maximize efficiencies. Such mechanisms need a more thorough formulation so that a complete acoustic description of the flow can be provided. Large contributions were first made by Kousen and Atassi and as a result they will be used for comparison.

%To begin exploring various geometries, we must first ask: what are the central theories that have been used `````````````````````````````````````````````````````````````````````````````````````````````````````````````````````````````112\sqrt{`312` explain the acoustic behavior of unsteady swirling flows? Do the linearized Euler equations provide an adequate means of capturing the mechanisms of noise generation within realistic turbomachinery flow? This paper will explore prior theories used to explain this and their methodologies. Large contributions were first made by Koussen and Atassi and as a result they will be used for comparison. Tam proposed a theoretical framework where the modes of a flow are characterized by an initial boundary value problem, allowing for the appearance of ''continuum modes`` \cite{Tarn1998}
\subsection{State the scope (what is included and what is not)}
This review will be a qualitative report of the findings, but a quantitative comparison ''meta-analysis`` should be done for potentially ideal test cases.

\subsection{Work by Kousen}
The study was expanded by \cite{KERREBROCK1974,YURKOVICH1975} who
included cases of solid body swirl. Wundrow studied the swirling potential flows using Goldstein's disturbance velocity
decomposition \cite{Goldstein1978} using a numerical approach and found that the solutions
were more accurate and found more efficiently \cite{KERREBROCK1974}. Kousen expanded these efforts by including \cite{kousen1996pressure,kousen1995eigenmode} the effect unsteady disturbances in the presence of 
forced solid body-swirl and free-vortex flow without the use of potential flow theory. Using normal 
mode analysis along with a radial spatial differencing scheme, the wave modes produced within cylindrical and annular ducts was 
reported. Results show (figure 4.7) two distinct families of modes, purely convected and acoustic 
modes. The presence of axial shear in combination solid body flows can endue coupling between these modes, which 
in theory would not appear unless viscous terms were present in the eigenvalue analysis. Lack of available swirl flow 
results ``hampered'' validation attempts. However, this eigenvalue approach was utilized with a new quasi-3D 
formulation to find the axial wavenumbers from solid body flow was proposed in his next work . 

In \cite{kousen1995eigenmode}, A quasi 3D formulation was validated and used to predict 
the appearance of modes due to the interaction of a rotor with spatially 
uniform steady and unsteady flow. These modes were first classified by 
\cite{Tyler1962} as ``spinning modes''. In addition, further investigation was 
done on the results shown in \cite{kousen1995eigenmode}. The axial wavenumbers that 
were previously found to be purely convective were shown to be in part
,``nearly convective'' (shear) pressure modes. These were found to propagate in the 
axial direction with no loss in amplitude, thus never satisfying the cut-off condition. 


\section{Golubev and Atassi's work}
Similarly, a narrow annulus is once again 
studied but with a different theoretical and computational approach. The governing equations, similar to Wundrow \cite{
Wundrow2019} were still then linearized in terms of potential and rotation. As suggested by Case \cite{Case1960}, a Fourier series analysis was used to conduct the normal mode analysis to find the corresponding wave numbers 
of the eigensystem. The findings show these fall into further classification which were previously denoted as purely convective and 
in part, nearly-convective wave modes. These two classifications of purely convective disturbance can be split into 
their ``nearly-convected vorticity dominated'' and ``nearly-sonic pressure dominated'' parts. These new 
results show the appearance of ``nearly-sonic pressure dominated modes'' which can propagate at varying phase speeds 
throughout the duct in both directions. The imposed Doppler shift from asymmetrical modes cause the sound propagate in the 
opposite direction of the mean flow swirl. A weak coupling relation relates the two and allows for the presence 
of vorticity - pressure mode coupling. Together, these nearly convected modes can be ``identified with the purely 
convective gusts in a non swilirling flow''. For the second group, these ``nearly convected'' vorticity dominated 
modes are further split as these disturbances approach the \textit{critical layer}, i.e. the 
location at which the viscous effects of the boundary layer begin to influence the coupling between modes. When both 
solid body and free vortex induced rotations are in the same direction, no instabilities arise from outside the critical 
layer. It was shown in later works that the influence of centrifugal and Coriolis forces created by the mean 
swirl prevent the decomposition of modes into their potential, rotational and entropic components. The paper ultimately proposes ``
a generalized definition for incident rotational waves(gusts) is proposed which accounts for both the eigenmodes and the 
initial value solutions'' 
%\section{Idea one. Central acoustic theory for unsteady ducted flow}
%\subsubsection{Main Idea - Uniform Mean Flow}
%In ducted turbomachinery, the appearance of unsteady flow has been first understood by characterizing the acoustic response of small disturbances. These small disturbances amount to modal content that govern the behavior of these responses. 
%\subsubsection{Evidence}
%It was shown by (Verdon:1989) that any small disturbance can be represented at a decomposition of acoustic, vortical, and entropic components. However, in the presence of sheared mean axial floww, the acoustic response of the mean flows that arise cannot be described in the same manner and cannot be solved analytically, calling for numerical approaches \cite{Kousen1996}. Preliminary results show that these responses fall into convective and ``non-convective'' classifications.
%\subsubsection{Analysis}
%Complex analysis allows us to observe the phase change in these mode shapes which can comprise the entire continua of the flow domain
%\subsubsection{Lead Out}

%\subsection{Concept}
%\subsubsection{Main Idea}
%The classification of modes into ``convected'', ``non-convected'' and ``nearly convected''
%\subsubsection{Evidence}
%\subsubsection{Analysis}
%\subsubsection{Lead Out}


%Tyler Sofrin selection rule serves well for rotor and stators that are evenly spaced. 
%- Utilizes the fourier transform


\section{Conclusion}
\begin{itemize}
    \item The conclusion summarizes 
        the key findings of the review in general terms. Notable commonalities between works, whether favorable or not
        , may be included here.
        \subitem This review discusses the development of the unsteady linearized equations, 
        and how improvements in the modal analysis capture more families of mechanisms of noise generation within non-uniform 
        swirling flow turbomachinery flow. 
    \item This section is the reviewer’s opportunity to justify a research 
        proposal. Therefore, the idea should be clearly re-stated and supported according to the findings of 
        the review.
        \subitem This literature review should be expanded to describe specific details in the methodologies and validation techniques reported by \cite{Kousen1996,Kousen1999,Golubev1996,Golubev1998}. 
\end{itemize}


\chapter{Chapter 3: Theory}
\include{Chapter-3-Theory-Aerodynamic-Models/divergence-in-cylindrical}
\subsection{Setting up SWIRL's Aerodynamic Model}
The Euler Equations in Cylindrical Form are,
\begin{align*}
\frac{\partial \rho}{\partial t} + %Conservation of mass
v_r \frac{\partial \rho}{\partial r} +
\frac{v_{\theta}   }{r}
\frac{\partial \rho}{\partial \theta} +
v_x \frac{\partial \rho}{\partial \theta} + 
\rho 
\left(
\frac{1}{r} \frac{\partial (rv_r)	}{\partial r} +
\frac{1}{r}	\frac{\partial v_{\theta}}{\partial \theta} +
\frac{\partial v_x}{\partial x}
\right) 
&= 0 \\% \label{ConservationOfMass} %%%%%%%%%%%%%%%%%%%%%%%%%%%%%%%%%%%%%%
\frac{\partial v_r}{\partial t} + 
v_r \frac{\partial v_r}{\partial r} +
\frac{v_{\theta}  }{r}
\frac{\partial v_r}{\partial \theta}- \frac{v_{\theta}^2}{r}+ 
v_x \frac{\partial v_r}{\partial x} 
&= -\frac{1}{\rho} \frac{\partial p}{\partial r}\\  
\frac{\partial v_{\theta}}{\partial t} + 
v_r \frac{\partial v_{\theta}}{\partial r} +
\frac{v_{\theta}}{r}
\frac{\partial v_{\theta}}{\partial \theta} +
\frac{v_r v_{\theta}}{r}+ 
v_x \frac{\partial v_{\theta}}{\partial x} 
&= -\frac{1}{\rho r} \frac{\partial p}{\partial \theta}\\ 
\frac{\partial v_{x}}{\partial t} + 
v_r 
\frac{\partial v_x}{\partial r} +
\frac{v_{\theta}}{r}
\frac{\partial v_x}{\partial \theta}+ 
v_x \frac{\partial v_x}{\partial x} 
&= 
-\frac{1}{\rho } 
\frac{\partial p}{\partial x}\\  
\frac{\partial p }{\partial t} +
v_r 
\frac{\partial p}{\partial r} +
\frac{v_{\theta}}{r}
\frac{\partial p}{\partial \theta} +
v_x \frac{\partial p}{\partial \theta} + 
\gamma p 
\left(
\frac{1}{r}\frac{\partial (rv_r)}{\partial r} +
\frac{1}{r}\frac{v_{\theta}}{\partial \theta} +
\frac{\partial v_x}{\partial x}
\right) &= 0
\end{align*}

SWIRL utilizes the following assumptions to simplify the aerodynamic model

\begin{itemize}
    \item No flow in the radial direction. Consequentially, the flow is 
        axisymmetric along the downstream direction.
    \item No surface or body forces
    \item Isentropic conditions 
\end{itemize}

For steady flow, the continuity, momentum and entropy equations are

\[\nabla (\vec{V} \bar{\rho}) = 0\]
\[(\vec{V}\cdot \nabla) \vec{V}\]
\[\nabla S = 0\]

If we neglect radial velocity, the velocity vector in cylindrical coordinates are

\[\vec{V}(r,\theta,x) = V_x(r) \hat{e}_x + V_{\theta} (r) \hat{e}_{\theta} \]
%Starting with the radial momentum equation, 
%\begin{align*}
%\frac{\partial v_r}{\partial t} + 
%v_r \frac{\partial v_r}{\partial r} +
%\frac{v_{\theta}  }{r}
%\frac{\partial v_r}{\partial \theta}- \frac{v_{\theta}^2}{r}+ 
%v_x \frac{\partial v_r}{\partial x} 
%&= -\frac{1}{\rho} \frac{\partial p}{\partial r} 
%\end{align*}
%See Appendix for speed of sound derivation

% \section{Applying model to various flows}
% The LEE for flows ith axial sheared flow, solid body and free vortex swirl
% were reviewed by \cite{kousen1995eigenmode}, and most recently studied by \cite{Maldonado2016}.
\subsection{Nonuniformities from swirling mean flow}

% \[P = \int_{\}^{1} \frac{\bar{\rho} V_{\theta}^2}{\tilde{r}} d\tilde{r}\] 
% % \subsection{Axial Shear Flow}
% % In \cite{kousen1995eigenmode}, axial sheared flows through a constant area duct was 
% % investigated.
% Axially sheared flows are different than axial uniform flow in that a velocity gradient is
% present along the x axis. 
% All other primitive variables (pressure and density which is $\propto$ speed of
% sound) are constant. As a result, the only changes that occur are in the x
% direction. This implies that $\partial / \partial \theta = 0$. 
% For the conservation of mass,

% \[ \nabla (\vec{V}\bar{\rho}) =  \left( 
% \underbrace{
% 	\cancel{
% 		\frac{\partial (\bar{\rho}v_r)	}{\partial r}
% 	}
% }_{v_r = 0} +
% \underbrace{\cancel{\frac{1}{r}	\frac{\partial \bar{\rho}v_{\theta}}{\partial \theta}}}_{\frac{\partial }{\partial \theta}} +
% \frac{\partial \bar{\rho}v_x}{\partial x}
% \right) = \frac{\partial \bar{\rho}v_x}{\partial x}\] 

% The conservation of momentum in the radial direction becomes,
% % \[(\vec{V}\cdot \nabla) \vec{V} =
% % \cancel{v_r \frac{\partial v_r}{\partial r}} +
% % \cancel{\frac{v_{\theta}  }{r}
% % 	\frac{\partial v_r}{\partial \theta}}- \frac{v_{\theta}^2}{r}+ 
% % \cancel{v_x \frac{\partial v_r}{\partial x} }
% % = -\frac{1}{\rho} \frac{\partial P}{\partial r}
% % \]
% % \[
% % \frac{v_{\theta}^2}{r}
% % = \frac{1}{\rho} \frac{\partial P}{\partial r}
% % \] 

% \[
% \frac{{\rho} v_{\theta}^2}{r} 
% =\frac{\partial P}{\partial r}
% \]
% For the tangential direction,
% \[(\vec{V}\cdot \nabla) \vec{V} = \cancel{v_r \frac{\partial v_{\theta}}{\partial r} +
% 	\frac{v_{\theta}}{r}
% 	\frac{\partial v_{\theta}}{\partial \theta} +
% 	\frac{v_r v_{\theta}}{r}}+ 
% v_x \frac{\partial v_{\theta}}{\partial x} 
% = \cancel{-\frac{1}{\rho r} \frac{\partial P}{\partial \theta}}\]
% Dividing $v_x$ to the other side,
% \[ \frac{\partial v_{\theta}}{\partial x}  = 0\]

% Similarly for the axial direction,

% \[ \frac{\partial v_x}{\partial x} = 0 \]
% Since the flow is isentropic , $\nabla S = 0$ the relation, $A^2= \frac{\nabla \bar{P}}{ \nabla \bar{\rho}}$ can be
% used to account for the change in speed of sound with radius as the mean
% flow contains a tangential (swirling) component.


% \section{Accounting for solid body swirl}

If the mean flow contains a swirling component, i.e. a velocity vector in the 
tangential direction, the mean quantities, pressure , density are non-uniform, 
thus also changing the speed of sound. By integrating the radial momentum
equation, an expression for the speed of sound was established to account for 
the resulting nonuniformities due to rotations in the flow. 


% If the flow contains a swirling component, then the primitive variables are 
% nonuniform through the flow, and mean flow assumptions are not valid. 
% To account to this, we integrate the momentum equation in the radial direction 
% with respect to the radius. 


\begin{equation}
    p = \int_{r_{min}}^{r_{max}} \frac{\rho v_{\theta}^2}{r} dr 
    \label{eqn:radialmomentum_integrated}
\end{equation}

where $r_{min}$ and $r_{max}$ are the bounds of the radius. Since the flow
is isentropic, the pressure is related to the speed of sound through $\nabla p =
A^2 \nabla \rho$; which is used to compute $\rho$. With the relationship 
$A^2 = \kappa p/\rho$, the speed of sound is found to be,

% The dimensional form is,
% \[
% \frac{\bar{\rho} v_{\theta}^2}{r} 
% =\frac{\partial P}{\partial r}
% \]
% By appplying separation of variables, the expression for $P$ can be found,
% \[
% \int_{r}^{r_{max}} \frac{\bar{\rho} v_{\theta}^2}{r}\partial r 
% =-\int_{P(r)}^{P(r_{max})}\partial P
% \]

% Since $\tilde{r} = r/r_{max}$ then,
% \[r = \tilde{r}r_{max}\]
% by taking total derivatives and applying chain rule,

% \[dr = d(\tilde{r}r_{max}) = d(\tilde{r})r_{max}\]
% Substituting these terms back in and evaluating the right hand side,
% \[
% \int_{\tilde{r}}^{1} \frac{\bar{\rho} v_{\theta}^2}{\tilde{r}}\partial \tilde{r} 
% =P(1)-P(\tilde{r})
% \]
% For reference the minimum value of $\tilde{r}$ is

% \[\sigma = \frac{r_{max}}{r_{min}}\]

% The radial derivative of the speed of sound squared is then used to find the 
% speed of sound in the cases where there is mean tangetial component regardless
% of there being axial flow,

% \[\frac{\partial A^2}{\partial r } = \frac{\partial}{\partial r} \left( \frac{\gamma P}{\rho} \right)\]
% Using the quotient rule, we can extract the definition of the speed of sound.
% \begin{align*}
% &= \frac{\partial P}{\partial r} \frac{\gamma \bar{\rho}}{\bar{\rho}^2} - \left( \frac{\gamma P}{\bar{\rho}^2} \right) \frac{\partial \bar{\rho}}{\partial r}\\
% &=  \frac{\partial P}{\partial r} \frac{\gamma }{\bar{\rho}} - \left( \frac{A^2}{\bar{\rho}} \right) \frac{\partial \bar{\rho} }{\partial r}\\ \text{Using } \partial P/A^2 = \partial \rho \rightarrow &= \frac{\partial P}{\partial r} \frac{\gamma }{\bar{\rho}} - \left( \frac{1}{\bar{\rho}} \right) \frac{\partial \bar{ P} }{\partial r}\\
% \frac{\partial A^2}{\partial r} &= \frac{\partial P}{\partial r} \frac{\gamma - 1}{\bar{\rho}}  \\ \text{or..}
% \frac{\bar{\rho}}{\gamma -1}\frac{\partial A^2}{\partial r} &= \frac{\partial P}{\partial r} 
% \end{align*}


% \begin{align*}
% \frac{\bar{\rho} v_{\theta}^2}{r} 
% &=\frac{\partial P}{\partial r}\\
% \frac{\cancel{\bar{\rho}} v_{\theta}^2}{r} 
% &=\frac{\cancel{\bar{\rho}}}{\gamma -1}\frac{\partial A^2}{\partial r}\\
% \frac{v_{\theta}^2}{r}\left(\gamma -1\right) &= \frac{\partial A^2}{\partial r}\\ \text{Dividing both sides by } A^2 \rightarrow \frac{M_{\theta}}{r}\left(\gamma - 1\right) &= \frac{\partial A^2}{\partial r} \frac{1}{A^2}
% \end{align*}
% \begin{align*}
% \text{Integrating both sides } \int_{r}^{r_{max}}\frac{M_{\theta}}{r}\left(\gamma - 1\right){\partial r}  &=\int_{A^2(r)}^{A^2(r_{max})}\frac{1}{A^2}  {\partial A^2}\\
% \int_{r}^{r_{max}}\frac{M^2_{\theta}}{r}\left(\gamma - 1\right){\partial r}  &=ln(A^2(r_{max})) - ln(A^2(r)) \\
% \int_{r}^{r_{max}}\frac{M^2_{\theta}}{r}\left(\gamma - 1\right){\partial r}  &=ln\left(\frac{A^2(r_{max})}{A^2(r)}\right) 
% \end{align*}

% Defining non dimensional speed of sound $\tilde{A} = \frac{A(r)}{A(r_{max})}$
% \begin{align*}
% \int_{r}^{r_{max}}\frac{M_{\theta}}{r}\left(\gamma - 1\right){\partial r}  &=ln\left(\frac{1}{\tilde{A}^2}\right) \\
% &= -2ln(\tilde{A})\\
% \tilde{A}(r) &= exp\left[-\int_{r}^{r_{max}}\frac{M_{\theta}}{r}\frac{\left(\gamma - 1\right)}{2}{\partial r}\right] \\ \text{replacing r with }\tilde{r} \rightarrow \tilde{A}(r) &= exp\left[-\int_{r}^{r_{max}}\frac{M_{\theta}}{r}\frac{\left(\gamma - 1\right)}{2}{\partial r}\right]		\\
% \end{align*}
% (See appendix for full derivation)
\begin{align*}
\tilde{A}(\tilde{r}) &= exp\left[\left(\frac{1 - \gamma}{2}\right)\int_{\tilde{r}}^{1}\frac{M_{\theta}}{\tilde{r}}{\partial \tilde{r}}\right]	
\end{align*}

For special cases of swirling flow, the relation to between the speed 
of sound and the tangential velocity can be found. Expressions can be derived 
for free vortex , and/or solid body swirl. 




\subsection{Linearizing the governing equations}
\subsubsection{Linearizing Conservation of Mass}
To linearize the Euler equations, we substitute each flow variable with its equivalent mean and perturbation components. Note that the mean term is only a function of space whereas the perturbation component is a dependent on both space and time (functional dependence is not explicity written with each variable). Assuming that we can divide the variable into a known laminar flow solution to the Navier-Stokes equations and a small amplitude perturbation solution:

\begin{align}
v_r 		&= V_r(x) + v_r'\\
v_{\theta} 	&= V_{\theta} + v_{\theta}'\\
v_x 		&= V_x + v_x'\\
p 			&= \bar{p} + p'\\
\rho 		&= \bar{\rho} + \rho'
\end{align}
\newpage
Starting with continuity,
%\begin{flushleft}
\begin{equation*} 
\end{equation*}
\begin{align*}
\frac{\partial \rho}{\partial t} + 
v_r \frac{\partial \rho}{\partial r} +
\frac{v_{\theta}   }{r}
\frac{\partial \rho}{\partial \theta} +
v_x \frac{\partial \rho}{\partial \theta} + 
\rho 
\left(
\frac{1}{r} \frac{\partial (rv_r)	}{\partial r} +
\frac{1}{r}	\frac{\partial v_{\theta}}{\partial \theta} +
\frac{\partial v_x}{\partial x}
\right) 
= 0\\
\frac{\partial \bar{\rho} + \rho' }{\partial t} +
(V_r + v_r') 
\frac{\partial \bar{\rho} + \rho'}{\partial r} +
\frac{V_{\theta} + v_{\theta}'}{r}
\frac{\partial \bar{\rho} + \rho'}{\partial \theta} +
(V_x + v_x') 
\frac{\partial \bar{\rho} + \rho'}{\partial \theta} + \\ 
(\bar{\rho} + \rho') 
\left(
\frac{1}{r}\frac{\partial (r(V_r +v_r'))}{\partial r} +
\frac{1}{r}\frac{\partial V_{\theta}+v_{\theta}'}{\partial \theta} +
\frac{\partial (V_x + v_x')}{\partial x}
\right) = 0\\
\frac{\partial \bar{\rho}}{\partial t} + 
\frac{\partial \rho'     }{\partial t} +\\
V_r \frac{\partial \bar{\rho}}{\partial r} +  
v_r'\frac{\partial \bar{\rho}}{\partial r} + 
V_r \frac{\partial \rho'}{\partial r} +
v_r'\frac{\partial \rho'}{\partial r} + \\
\frac{1}{r}
\left(
V_{\theta} \frac{\partial \bar{\rho}}{\partial \theta} +
v_{\theta}'\frac{\partial \bar{\rho}}{\partial \theta} + 
V_{\theta} \frac{\partial \rho'		}{\partial \theta} + 
v_{\theta}'\frac{\partial \rho'		}{\partial \theta}
\right) + \\ 
V_x \frac{\partial \bar{\rho}}{\partial x} + 
v_x'\frac{\partial \bar{\rho}}{\partial x} +
V_x \frac{\partial \rho'	 }{\partial x} 	+
v_x'\frac{\partial \rho'    }{\partial x}		+\\
\bar{\rho} 
\left(
\frac{1}{r}
\left(
\frac{\partial (rV_r  )    }{\partial r} +
\frac{\partial (r(v_r')    }{\partial r} +
\frac{\partial V_{\theta}  }{\partial \theta} +
\frac{\partial v_{\theta}' }{\partial \theta}
\right) +
\frac{\partial V_x }{\partial x} +
\frac{\partial v_x'}{\partial x}
\right) + \\
\rho'
\left(
\frac{1}{r}
\left(
\frac{\partial (rV_r  )    }{\partial r} +
\frac{\partial (r(v_r')    }{\partial r} +
\frac{\partial V_{\theta}  }{\partial \theta} +
\frac{\partial v_{\theta}' }{\partial \theta}
\right) +
\frac{\partial V_x }{\partial x} +
\frac{\partial v_x'}{\partial x}
\right)
= 0	\\
\end{align*}
%\end{flushleft}
Important things to note
\begin{itemize}
	\item The small disturbances are infinitesimal (thus linearized)
	\item Neglect second order terms.
	\item The continuity equation is comprised of mean velocity components. This is subtracted off in each of the governing equations
\end{itemize}
Blue will be used for terms that are removed after subtracting in the original continuity equation, green will be used to cancel higher(2nd) order terms.Red will be used if we take the radial velocity to be zero.

\begin{align*}
=
\Ccancel[blue]
{\frac{\partial \bar{\rho}}{\partial t}} + 
\frac{\partial \rho'     }{\partial t} + \\
\Ccancel[blue]  {V_r \frac{\partial \bar{\rho}}{\partial r}} +  
\Ccancel[blue] {v_r'\frac{\partial \bar{\rho}}{\partial r}} + 
\Ccancel[red]{V_r \frac{\partial \rho'}{\partial r}} +
\Ccancel[green]{v_r'\frac{\partial \rho'}{\partial r}} + \\
\frac{1}{r}
\left(
\Ccancel[blue]{V_{\theta} \frac{\partial \bar{\rho}}{\partial \theta}} +
\Ccancel[blue]{v_{\theta}'\frac{\partial \bar{\rho}}{\partial \theta}} + 
V_{\theta} \frac{\partial \rho'		}{\partial \theta} + 
\Ccancel[green]{v_{\theta}'\frac{\partial \rho'		}{\partial \theta}}
\right) + \\ 
\Ccancel[blue]{V_x \frac{\partial \bar{\rho}}{\partial x}} + 
\Ccancel[blue]{v_x'\frac{\partial \bar{\rho}}{\partial x}} +
V_x \frac{\partial \rho'	 }{\partial \theta} 	+
\Ccancel[green]{v_x'\frac{\partial \rho'    }{\partial x}}		+\\
\bar{\rho} 
\left(
\frac{1}{r}
\left(
\Ccancel[blue]{\frac{\partial (rV_r  )    }{\partial r}} +
\frac{\partial (r(v_r')    }{\partial r} +
\Ccancel[blue]{\frac{\partial V_{\theta}  }{\partial \theta}} +
\frac{\partial v_{\theta}' }{\partial \theta}
\right) +
\Ccancel[green]{\frac{\partial V_x }{\partial x}} +
\frac{\partial v_x'}{\partial x}
\right) + \\
\rho'
\left(
\frac{1}{r}
\left(
\Ccancel[blue]{\frac{\partial (rV_r  )    }{\partial r}} +
\Ccancel[green]{\frac{\partial (r(v_r')    }{\partial r} }+
\Ccancel[blue]{\frac{\partial V_{\theta}  }{\partial \theta} }+
\Ccancel[green]{\frac{\partial v_{\theta}' }{\partial \theta}}
\right) +
\Ccancel[blue]{\frac{\partial V_x }{\partial x}} +
\Ccancel[green]{\frac{\partial v_x'}{\partial x}} = 0
\right)
\end{align*}

\begin{align*}
\boxed{
	\frac{\partial \rho'}{\partial t} +
	\frac{V_{\theta}}{r}
	\frac{\partial \rho'}{\partial \theta} + 
	V_x
	\frac{\partial \rho'}{\partial x} +
	\bar{\rho}
	\left(
	\frac{1}{r}
	\left(
	\frac{\partial r v_r'}{\partial r} + \frac{\partial v_{\theta}'}{\partial \theta}		 
	\right) +
	\frac{\partial v_x'}{\partial x}
	\right)= 0} 
\end{align*}
\newpage
\subsubsection{Linearizing the Conservation of Momentum\\ in the \textit{r} direction}
Starting with the mean momentum equation 
\begin{align*}
\frac{\partial v_r}{\partial t} + 
v_r \frac{\partial v_r}{\partial r} +
\frac{v_{\theta}  }{r}
\frac{\partial v_r}{\partial \theta}- \frac{v_{\theta}^2}{r}+ 
v_x \frac{\partial v_r}{\partial x} 
&= -\frac{1}{\rho} 
\frac{\partial p}{\partial r}
\end{align*}
Looking at the left hand side first
\begin{align*} 
\frac{\partial (V_r + v_r') }{\partial t} + 
( V_r + v_r' ) 
\frac{\partial( V_r + v_r')}{\partial r} +
\frac{V_{\theta} + v_{\theta}'}{r}
\frac{\partial(V_r + v_r')}{\partial \theta} -
\frac{ (V_{\theta} + v_{\theta}')^2}{ r} + 
(V_x + v_x') 
\frac{\partial (V_r + v_r')}{\partial x} 	
&= -\frac{1}{\rho} \frac{\partial p}{\partial r}\\
\Ccancel[blue]  {\frac{\partial  V_r  }{\partial t}}	+
\frac{\partial  v_r' }{\partial t} + \\
\Ccancel[blue]  {V_r  \frac{\partial  V_r  }{\partial r}}  +
\Ccancel[blue] {v_r' \frac{\partial  V_r  }{\partial r}} + 
\Ccancel[red] {V_r  \frac{\partial  v_r' }{\partial r}} + 
\Ccancel[green]{v_r' \frac{\partial  v_r' }{\partial r}} +\\
\frac{1}{r}
\left(
\Ccancel[blue]  {V_{\theta} \frac{\partial V_r}{\partial \theta}} +
\Ccancel[blue] {v_{\theta}'\frac{\partial V_r}{\partial \theta}} +
V_{\theta} \frac{\partial v'_r}{\partial \theta} +
\Ccancel[green]{v_{\theta}'\frac{\partial v'_r}{\partial \theta}}
\right)- \\
\frac{1}{r}\left(
\Ccancel[blue]{V_{\theta}^2} + 
2V_{\theta}v'_{\theta} + 	
\Ccancel[green]{v^{'2}_{\theta}}\right)+\\
\Ccancel[blue]{V_x \frac{\partial V_r }{\partial x}} +
\Ccancel[blue]{v_x'\frac{\partial V_r }{\partial x}} +  
V_x \frac{\partial v_r'}{\partial x} +
\Ccancel[green]{v_x' \frac{\partial v_r'}{\partial x}} 
&= -\frac{1}{\rho} 
\frac{\partial p}{\partial r}\\
\frac{\partial  v_r' }{\partial t} +
V_r  \frac{\partial  v_r' }{\partial r} + 
\frac{V_{\theta}}{r} \frac{\partial v'_r}{\partial \theta} -
\frac{2V_{\theta}v'_{\theta}}{r} +
V_x \frac{\partial v_r'}{\partial x} 
&= -\frac{1}{\rho} 
\frac{\partial p}{\partial r}
\end{align*}
\newpage
Now looking at the right side,
Expanding the $1/\rho $ using a Taylor series approximation

\begin{align*}
\frac{1}{\bar{\rho} + \rho'} 
&= \frac{1}{\bar{\rho}} 
+ \left(
\frac{1}{\bar{\rho} 
	+ \rho'} 
- \frac{1}{\rho}
\right) \\
&= \frac{1}{\bar{\rho}} 
+ \left(
\frac{\bar{\rho}}{\bar{\rho}(\bar{\rho} 
	+ \rho')} 
- \frac{1}{\rho} \frac{\bar{\rho} + \rho'}{\bar{\rho} + \rho'}
\right) \\
&= \frac{1}{\bar{\rho}} 
- \left(
\frac{\bar{\rho} - \bar{\rho} + \rho'}{\bar{\rho}(\bar{\rho} 
	+ \rho')}
\right)	\\
&= \frac{1}{\bar{\rho}} 
- \frac{\rho'}{\bar{\rho}}
\underbrace{\left(
	\frac{1}{\bar{\rho} + \rho}
	\right)}_\text{This is what we're solving for!}	\\
&= \frac{1}{\bar{\rho}} 
- \frac{\rho'}{\bar{\rho}}
\underbrace{
	\left[\frac{1}{\bar{\rho}} 
	+ \left(
	\frac{1}{\bar{\rho} 
		+ \rho'} 
	- \frac{1}{\rho}
	\right) \right] }_\text{This is from step 1}	\\
&= \frac{1}{\bar{\rho}} 
- \frac{\rho'}{\bar{\rho}^2} +
\underbrace{
	\left[ \left(\frac{\rho'}{\bar{\rho}}\right)^2
	\frac{1}{\bar{\rho} 
		+ \rho'} 
	\right] }_\text{These are higher order terms that will go to $\infty$}	\\	
\end{align*}
\begin{align*} % \text{Plugging this back in to the right hand side, (don't forget the negative!)} \rightarrow
\frac{1}{\rho}\frac{\partial p}{\partial r} = \left( -\frac{1    }{\bar{\rho}} +
\frac{\rho'}{\bar{\rho}^2}\right) \left(\frac{\partial \bar{p} + p'}{\partial r}\right)\\
\frac{1}{\rho}\frac{\partial p}{\partial r} =  -\Ccancel[blue]{\frac{1    }{\bar{\rho}}  \frac{\partial \bar{p}}{\partial r}} -  
\frac{1    }{\bar{\rho}}  \frac{\partial p'}{\partial r} +
\frac{\rho'}{\bar{\rho}^2}\frac{\partial \bar{p}}{\partial r} +
\Ccancel[green]{\frac{\rho'}{\bar{\rho}^2}\frac{\partial p'}{\partial r}}\\
\frac{1}{\rho}\frac{\partial p}{\partial r} =  -\frac{1    }{\bar{\rho}}  \frac{\partial p'}{\partial r} +
\frac{\rho'}{\bar{\rho}^2}\frac{\partial \bar{p}}{\partial r} 
\end{align*}

\begin{align*}
\boxed{
	\frac{\partial  v_r' }{\partial t} +
	\frac{V_{\theta}}{r} \frac{\partial v'_r}{\partial \theta} -
	\frac{2V_{\theta}v'_{\theta}}{r} +
	V_x \frac{\partial v_r'}{\partial x} =\frac{1    }{\bar{\rho}}  \frac{\partial p'}{\partial r} +
	\frac{\rho'}{\bar{\rho}^2}\frac{\partial \bar{p}}{\partial r} 
}
\end{align*}
\newpage
\subsubsection{Linearizing the Conservation of Momentum\\ in the \textit{$\theta$} direction}
Starting with the mean momentum equation 
\begin{align*}
\frac{\partial v_{\theta}}{\partial t} + 
v_r \frac{\partial v_{\theta}}{\partial r} +
\frac{v_{\theta}  }{r}
\frac{\partial v_{\theta}}{\partial \theta}+
\frac{v_r v_{\theta}}{r}+ 
v_x \frac{\partial v_{\theta}}{\partial x} 
&= -\frac{1}{\rho r} 
\frac{\partial p}{\partial \theta}
\end{align*}

Looking at the left hand side first
\begin{align*} 
\frac{\partial (V_{\theta} + v_{\theta}') }{\partial t} + 
( V_r + v_r' ) 
\frac{\partial( V_{\theta} + v_{\theta}')}{\partial r} + \\
\frac{V_{\theta} + v_{\theta}'}{r}
\frac{\partial(V_{\theta} + v_{\theta}')}{\partial \theta} +
\frac{ (V_r + v_r')(V_{\theta} + v_{\theta}')}{ r} + 
(V_x + v_x') 
\frac{\partial (V_{\theta} + v_{\theta}')}{\partial x} 	
&= 
-\frac{1}{\rho r} \frac{\partial p}{\partial \theta}\\
\Ccancel[blue]  {\frac{\partial  V_{\theta}  }{\partial t}}	+
\frac{\partial  v_{\theta}' }{\partial t} + \\
\Ccancel[blue]  {V_r  \frac{\partial  V_{\theta}  }{\partial r}}  +
\underbrace{v_r' \frac{\partial  V_{\theta}  }{\partial r}}_{v_r'=0} + 
\Ccancel[red]{V_r  \frac{\partial  v_{\theta}' }{\partial r}} + 
\Ccancel[green]{v_r' \frac{\partial  v_{\theta}' }{\partial r}} +\\
\frac{1}{r}
\left(
\Ccancel[blue]  {V_{\theta} \frac{\partial V_{\theta}}{\partial \theta}} +
\Ccancel[blue] {v_{\theta}'\frac{\partial V_{\theta}}{\partial \theta}} +
V_{\theta} \frac{\partial v'_{\theta}}{\partial \theta} +
\Ccancel[green]{v_{\theta}'\frac{\partial v'_{\theta}}{\partial \theta}}
\right)+ \\
\frac{1}{r}\left(
\Ccancel[blue]{V_r V_{\theta}} + 
v_r'V_{\theta} +
\Ccancel[red]{V_r v'_{\theta}}  + 	
\Ccancel[green]{v_r'v_{\theta}'}\right)+\\
\Ccancel[blue]{V_x \frac{\partial V_{\theta} }{\partial x}} +
\Ccancel[blue]{v_x'\frac{\partial V_{\theta} }{\partial x}} +  
V_x \frac{\partial v_{\theta}'}{\partial x} +
\Ccancel[green]{v_x' \frac{\partial v_{\theta}'}{\partial x}} 
&= -\frac{1}{\rho r} 
\frac{\partial p}{\partial \theta}\\
\frac{\partial  v_{\theta}' }{\partial t} +
v_r' \frac{\partial  V_{\theta}  }{\partial r} +
\frac{V_{\theta}}{r} \frac{\partial v'_{\theta}}{\partial \theta} +
\frac{v'_rV_{\theta}}{r} +
V_x \frac{\partial v_r'}{\partial x} 
&= -\frac{1}{\rho r} 
\frac{\partial p}{\partial \theta}
\end{align*}
Now looking at the right side,
Expanding the $1/\rho $ using a Taylor series approximation

\begin{align*}
&
-\frac{1}{\rho r}\frac{\partial p}{\partial \theta} = \left( -\frac{1    }{\bar{\rho}} +
\frac{\rho'}{\bar{\rho}^2}\right) \left(\frac{\partial \bar{p} + p'}{\partial \theta}\right)\\
&\frac{1}{\rho r}\frac{\partial p}{\partial \theta} =  -\Ccancel[blue]{\frac{1    }{\bar{\rho}}  \frac{\partial \bar{p}}{\partial \theta}} -  
\frac{1    }{\bar{\rho} r}  \frac{\partial p'}{\partial \theta} +
\Ccancel[blue]{\frac{\rho'}{\bar{\rho}^2 r}\frac{\partial \bar{p}}{\partial \theta}} +
\Ccancel[green]{\frac{\rho'}{\bar{\rho}^2 r}\frac{\partial p'}{\partial \theta}}\\
&\frac{1}{\rho r}\frac{\partial p}{\partial \theta} =  -\frac{1    }{\bar{\rho}}  \frac{\partial p'}{\partial \theta} 
\end{align*}

\begin{align*}
\boxed{
	\frac{\partial  v_{\theta}' }{\partial t} +
	v_r' \frac{\partial  V_{\theta}  }{\partial r} +
	\frac{V_{\theta}}{r} \frac{\partial v'_{\theta}}{\partial \theta} +
	\frac{v'_rV_{\theta}}{r} +
	V_x \frac{\partial v_r'}{\partial x} 
	= -\frac{1}{\bar{\rho} r}	\frac{\partial p'}{\partial \theta}
}
\end{align*}
\newpage
\subsubsection{Linearizing the Conservation of Momentum\\ in the \textit{x} direction}
Starting with the mean momentum equation 
\begin{align*}
\frac{\partial v_{x}}{\partial t} + 
v_r 
\frac{\partial v_x}{\partial r} +
\frac{v_{\theta}}{r}
\frac{\partial v_x}{\partial \theta}+ 
v_x \frac{\partial v_x}{\partial x} 
&= 
-\frac{1}{\rho } 
\frac{\partial p}{\partial x}
\end{align*}

\begin{align*} 
\frac{\partial (V_x + v_x') }{\partial t} + 
( V_r + v_r' ) 
\frac{\partial( V_x + v_x')}{\partial r} +
\frac{V_{\theta} + v_{\theta}'}{r}
\frac{\partial(V_x + v_x')}{\partial \theta} + 
(V_x + v_x') 
\frac{\partial (V_x + v_x')}{\partial x} 	
&= -\frac{1}{\rho } \frac{\partial p}{\partial x}\\
\Ccancel[blue]  {\frac{\partial  V_x  }{\partial t}}	+
\frac{\partial  v_x' }{\partial t} + \\
\Ccancel[blue]  {V_r  \frac{\partial  V_x  }{\partial r}}  +
v_r' \frac{\partial  V_x  }{\partial r} + 
\Ccancel[red]{V_r  \frac{\partial  v_x' }{\partial r}} + 
\Ccancel[green]{v_r' \frac{\partial  v_x' }{\partial r}} +\\
\frac{1}{r}
\left(
\Ccancel[blue]  {V_{\theta} \frac{\partial V_x}{\partial \theta}} +
\Ccancel[blue] {v_{\theta}'\frac{\partial V_x}{\partial \theta}} +
V_{\theta} \frac{\partial v'_x}{\partial \theta} +
\Ccancel[green]{v_{\theta}'\frac{\partial v'_x}{\partial \theta}}
\right)+ \\
\Ccancel[blue]{V_x \frac{\partial V_x }{\partial x}} +
\Ccancel[blue]{v_x'\frac{\partial V_x }{\partial x}} +  
V_x \frac{\partial v_x'}{\partial x} +
\Ccancel[green]{v_x' \frac{\partial v_x'}{\partial x}} 
&= -\frac{1}{\rho } 
\frac{\partial p}{\partial x} \\
\boxed{\frac{\partial  v_x' }{\partial t} +
	v_r' \frac{\partial  V_x  }{\partial r} +
	\frac{V_{\theta}}{r} \frac{\partial v'_x}{\partial \theta} +
	V_x \frac{\partial v_x'}{\partial x} 
	= -\frac{1}{\rho } 
	\frac{\partial p}{\partial x}}
\end{align*}
\[
\]
\newpage
\begin{align*}
\text{} \rightarrow&
-\frac{1}{\rho }\frac{\partial p}{\partial x} = \left( -\frac{1    }{\bar{\rho}} +
\frac{\rho'}{\bar{\rho}^2}\right) \left(\frac{\partial \bar{p} + p'}{\partial x}\right)\\
&\frac{1}{\rho }\frac{\partial p}{\partial x} =  -\Ccancel[blue]{\frac{1    }{\bar{\rho}}  \frac{\partial \bar{p}}{\partial x}} -  
\frac{1    }{\bar{\rho} }  \frac{\partial p'}{\partial x} +
\Ccancel[blue]{\frac{\rho'}{\bar{\rho}^2 r}\frac{\partial \bar{p}}{\partial x}} +
\Ccancel[green]{\frac{\rho'}{\bar{\rho}^2 r}\frac{\partial p'}{\partial x}}\\
&\frac{1}{\rho }\frac{\partial p}{\partial x} =  -\frac{1    }{\bar{\rho}}  \frac{\partial p'}{\partial x} 
\end{align*}

\begin{align*}
\boxed{
	\frac{\partial  v_x' }{\partial t} +
	v_r' \frac{\partial  V_x  }{\partial r} +
	\frac{V_{\theta}}{r} \frac{\partial v'_x}{\partial \theta} +
	V_x \frac{\partial v_x'}{\partial x} 
	= -\frac{1    }{\bar{\rho}}  \frac{\partial p'}{\partial x} 
}
\end{align*}

\newpage

\subsubsection{Linearizing the Energy Equation}
\begin{align*}
\frac{\partial p}{\partial t} + v_r \frac{\partial p}{\partial r} + \frac{v_{\theta}}{r}\frac{\partial p}{\partial x} + v_x \frac{\partial p}{\partial x} 
+ \gamma p \left(\frac{1}{r} \frac{\partial r v_r}{\partial r} + \frac{1}{r} \frac{\partial v_{\theta}}{\partial \theta} + \frac{\partial v_x}{\partial x}\right) = 0
\end{align*}

\begin{align*}
\frac{\partial (\bar{P}+P')}{\partial t} + 
(V_r + v_r')
\frac{ \partial (\bar{P}+P')}{\partial r} + 
\frac{  (V_{\theta} + v_{\theta}') }{r}\frac{\partial (\bar{ P} +p') }{\partial x} + 
(V_x + v_x') 
\frac{\partial  (\bar{P}+P')}{\partial x} + ...\\
\gamma (\bar{P}+P') 
\left(
\frac{1}{r} \frac{\partial r (V_r + v_r')}{\partial r} + 
\frac{1}{r} \frac{\partial   (V_{\theta} + v_{\theta}')}{\partial \theta} + 
\frac{\partial (V_x + v_x')}{\partial x}
\right) = 0  	\\
\\
\frac{\partial \bar{P} }{\partial t} +
\frac{\partial      P' }{\partial t} +\\
V_r  \frac{\partial \bar{P}}{\partial r} + 
V_r  \frac{\partial    P'}{\partial r} + 
V_r' \frac{\partial \bar{P}}{\partial r} + 
V_r' \frac{\partial      P'}{\partial r} + \\     
\frac{V_{\theta}}{r} \frac{\partial \bar{ P}}{\partial x} + 
\frac{V_{\theta}}{r} \frac{\partial       P'}{\partial x} +
\frac{v_{\theta}'}{r} \frac{\partial \bar{ P}}{\partial x} + 
\frac{v_{\theta}'}{r} \frac{\partial       P'}{\partial x} + \\
V_x  \frac{\partial \bar{P}}{\partial x} + 
V_x  \frac{\partial    P'}{\partial x} + 
v_x' \frac{\partial \bar{P}}{\partial x} + 
v_x' \frac{\partial      P'}{\partial x} + \\ 
\gamma \bar{ P}  \left(
\frac{1}{r} \frac{\partial r V_r}{\partial r} + 
\frac{1}{r} \frac{\partial r v_r'}{\partial r} +
\frac{1}{r} \frac{\partial V_{\theta}}{\partial \theta} + 
\frac{\partial v_{\theta}'}{\partial \theta}+ 
\frac{\partial V_x}{\partial x} + 
\frac{\partial v_x'}{\partial x} 
\right) \\
\gamma P' \left(
\frac{1}{r} \frac{\partial r V_r}{\partial r} + 
\frac{1}{r} \frac{\partial r v_r'}{\partial r} + 
\frac{1}{r} \frac{\partial V_{\theta}}{\partial \theta} + 
\frac{\partial v_{\theta}'}{\partial \theta}+ \frac{\partial V_x}{\partial x} +
\frac{\partial v_x'}{\partial x}\right) 
\end{align*}

\begin{equation*}
\boxed{\frac{\partial p '}{\partial t} + v_r'\frac{\partial P}{\partial r} + \frac{V_{\theta}}{r}\frac{\partial p'}{\partial \theta} + V_x\frac{\partial p'}{x} + \gamma P \left(\frac{1}{r}\frac{\partial (r v_r')}{\partial r} + \frac{1}{r} \frac{\partial v_{\theta}'}{\partial \theta} + \frac{\partial v_x'}{\partial x}\right) = 0}
\end{equation*}
\newpage

The linearized Euler equations are,

\begin{align*}
\frac{\partial \rho'}{\partial t} +
\frac{V_{\theta}}{r}
\frac{\partial \rho'}{\partial \theta} + 
V_x
\frac{\partial \rho'}{\partial x} +
\bar{\rho}
\left(
\frac{1}{r}
\left(
\frac{\partial r v_r'}{\partial r} + \frac{\partial v_{\theta}'}{\partial \theta}		 
\right) +
\frac{\partial v_x'}{\partial x}
\right)= 0\\
\frac{\partial  v_r' }{\partial t} +
\frac{V_{\theta}}{r} \frac{\partial v'_{\theta}}{\partial \theta} -
\frac{2V_{\theta}v'_{\theta}}{r} +
V_x \frac{\partial v_r'}{\partial x} = -\frac{1    }{\bar{\rho}}  \frac{\partial p'}{\partial r} +
\frac{\rho'}{\bar{\rho}^2}\frac{\partial \bar{p}}{\partial r}\\
\frac{\partial  v_{\theta}' }{\partial t} +
v_r' \frac{\partial  V_{\theta}  }{\partial r} +
\frac{V_{\theta}}{r} \frac{\partial v'_{\theta}}{\partial \theta} +
\frac{v'_rV_{\theta}}{r} +
V_x \frac{\partial v_r'}{\partial x} 
= -\frac{1}{\bar{\rho} r}	\frac{\partial p'}{\partial \theta}\\
\frac{\partial  v_x' }{\partial t} +
v_r' \frac{\partial  V_x  }{\partial r} +
\frac{V_{\theta}}{r} \frac{\partial v'_x}{\partial \theta} +
V_x \frac{\partial v_x'}{\partial x} 
= -\frac{1    }{\bar{\rho}}  \frac{\partial p'}{\partial x} \\
\frac{\partial p '}{\partial t} +
v_r'\frac{\partial P}{\partial r} + 
\frac{V_{\theta}}{r}\frac{\partial p'}{\partial \theta} + 
V_x\frac{\partial p'}{x} + 
\gamma P \left(\frac{1}{r}\frac{\partial (r v_r')}{\partial r} + 
\frac{1}{r} \frac{\partial v_{\theta}'}{\partial \theta} + \frac{\partial v_x'}{\partial x}\right) = 0
\end{align*}


Recalling:

\[\frac{\partial P}{\partial r} = \frac{\bar{\rho} V_{\theta}^2}{r} \]
\[\gamma P  = \bar{\rho}A^2\]
\[\rho' = \frac{1}{A^2} p'\]

We can rearrange the equations to reflect Equations 2.33-2.36. Note that the momentum equation in the $\theta$ and $x$ directions remain unchanged. The term $ \frac{\partial(rv_r')}{\partial r}  = \frac{\partial (r)}{\partial r}v_r' + \frac{\partial v_r'}{\partial r} r$ in the Energy equation
\begin{align*}
\frac{1}{\bar{\rho} A^2}\left(
\frac{\partial p'}{\partial t} +
\frac{V_{\theta}}{r}
\frac{\partial p'}{\partial \theta} + 
V_x
\frac{\partial p'}{\partial x}
\right) +
\frac{V_{\theta}^2}{A^2 r}v_r'+
\frac{\partial v_r'}{\partial r} + \frac{v_r'}{r} +
\frac{1}{r}
\frac{\partial v_{\theta}'}{\partial \theta}		 
 +
\frac{\partial v_x'}{\partial x}
&= 0\\
\frac{\partial  v_r' }{\partial t} +
\frac{V_{\theta}}{r} \frac{\partial v'_r}{\partial \theta} -
\frac{2V_{\theta}v'_{\theta}}{r} +
V_x \frac{\partial v_r'}{\partial x} &= \frac{1}{\bar{\rho}} \frac{\partial p'}{\partial r}+\frac{V_{\theta}}{\bar{\rho} r A^2}   p'
\\
\frac{\partial  v_{\theta}' }{\partial t} +
v_r' \frac{\partial  V_{\theta}  }{\partial r} +
\frac{V_{\theta}}{r} \frac{\partial v'_{\theta}}{\partial \theta} +
\frac{v'_rV_{\theta}}{r} +
V_x \frac{\partial v_{\theta}'}{\partial x} 
= -\frac{1}{\bar{\rho} r}	\frac{\partial p'}{\partial \theta}\\
\frac{\partial  v_x' }{\partial t} +
v_r' \frac{\partial  V_x  }{\partial r} +
\frac{V_{\theta}}{r} \frac{\partial v'_x}{\partial \theta} +
V_x \frac{\partial v_x'}{\partial x} 
&= -\frac{1    }{\bar{\rho}}  \frac{\partial p'}{\partial x} 	
\end{align*}

\newpage
\section{Substituting Pertubation Variables}
\[v'_r = v_r (r) e^{i\left(k_x x + m \theta - \omega t \right)} \]
\[v'_{\theta} = v_{\theta} (r) e^{i\left(k_x x + m \theta - \omega t \right)} \]	
\[v'_x = v_x (r) e^{i\left(k_x x + m \theta - \omega t\right)} \]	
\[p' = p(r) e^{i\left(k_x x + m \theta - \omega t\right)} \]	

Conservation of Energy
\begin{align*}
\frac{1}{\bar{\rho} A^2}\left(
\frac{\partial p'}{\partial t} +
\frac{V_{\theta}}{r}
\frac{\partial p'}{\partial \theta} + 
V_x
\frac{\partial p'}{\partial x}
\right) +
\frac{V_{\theta}^2}{A^2 r}v_r'+
\frac{\partial v_r'}{\partial r} + 
\frac{1}{r}
\frac{\partial v_{\theta}'}{\partial \theta}		 
+
\frac{\partial v_x'}{\partial x}
&= 0
\end{align*}


LHS derivatives:
\[\frac{\partial p'}{\partial t} =
\underbrace{\frac{\partial p(r)}{\partial t}}_{0} e^{i\left(k_x x + m \theta - \omega t \right)} +
p(r) \left(-i\omega e^{i\left(k_x x + m \theta - \omega t \right)}\right) \]

\[\frac{\partial p'}{\partial \theta} = 
\underbrace{\frac{\partial p(r)}{\partial \theta}}_{0} e^{i\left(k_x x + m \theta - \omega t \right)} + 
p(r) \left(im e^{i\left(k_x x + m \theta - \omega t \right)}\right) \]

\[\frac{\partial p'}{\partial x} = \underbrace{\frac{\partial p(r)}{\partial x}}_{0} e^{i\left(k_x x + m \theta - \omega t \right)} +
p(r) \left(ik_x e^{i\left(k_x x + m \theta - \omega t \right)}\right) \]

\[\frac{\partial v_r'}{\partial r} =
\frac{\partial v_r(r)}{\partial r} e^{i\left(k_x x + m \theta - \omega t \right)} +
v_r(r) \underbrace{\frac{\partial }{\partial r}\left( e^{i\left(k_x x + m \theta - \omega t \right)}\right)}_0 \]

\[\frac{\partial v_{\theta}'}{\partial \theta} = 
\frac{\partial v_{\theta}(r)}{\partial \theta} e^{i\left(k_x x + m \theta - \omega t \right)} + 
v_{\theta}(r) \left(im e^{i\left(k_x x + m \theta - \omega t \right)}\right) \]

\[\frac{\partial v_x'}{\partial x} = \underbrace{\frac{\partial v_x(r)}{\partial x}}_{0} e^{i\left(k_x x + m \theta - \omega t \right)} + v_x(r) \left(ik_x e^{i\left(k_x x + m \theta - \omega t \right)}\right) \]

After substituting and canceling common terms,

\begin{align*}
\frac{1}{\bar{\rho} A^2}\left(
p(r) \left(-i\omega e^{i\left(k_x x + m \theta - \omega t \right)}\right) +
\frac{V_{\theta}}{r}
p(r) \left(im e^{i\left(k_x x + m \theta - \omega t \right)}\right) + 
V_x
p(r) \left(ik_x e^{i\left(k_x x + m \theta - \omega t \right)}\right)
\right) + \\
\frac{V_{\theta}^2}{A^2 r}v_r'+
\frac{\partial v_r(r)}{\partial r} e^{i\left(k_x x + m \theta - \omega t \right)} + 
\frac{1}{r}
\left(v_{\theta}(r) \left(im e^{i\left(k_x x + m \theta - \omega t \right)}\right)\right) 
+
v_x(r) \left(ik_x e^{i\left(k_x x + m \theta - \omega t \right)}\right) 
&= 0
\end{align*}

\begin{align*}
\frac{1}{\bar{\rho} A^2} \left(-i\omega + \frac{imV_{\theta}}{r} + ik_xV_x  +
\right)p(r)  \\
\frac{V_{\theta}^2}{A^2 r}v_r'+ \frac{v_r'}{r} +
\frac{\partial v_r(r)}{\partial r}+ 
\frac{im}{r} v_{\theta}(r)
+
ik_xv_x(r) 
&= 0
\end{align*}



\newpage
conservation of momentum in the r direction,
\[
\frac{\partial v_r'}{\partial t} +
\frac{V_{\theta}}{r} \frac{\partial v_r'}{\partial \theta} - \frac{2V_{\theta}}{r}v_r' + V_x\frac{\partial v_r'}{\partial x} = \frac{\partial p'}{\partial r} + \frac{\rho'}{\bar{\rho}^2}\frac{\partial P}{\partial r}
\]

LHS derivatives:
\[\frac{\partial v_r'}{\partial t} =
 \underbrace{\frac{\partial v_r(r)}{\partial t}}_{0} e^{i\left(k_x x + m \theta - \omega t \right)} +
  v_r(r) \left(-i\omega e^{i\left(k_x x + m \theta - \omega t \right)}\right) \]
\[\frac{\partial v_r'}{\partial \theta} = 
\underbrace{\frac{\partial v_r(r)}{\partial \theta}}_{0} e^{i\left(k_x x + m \theta - \omega t \right)} + v_r(r) \left(im e^{i\left(k_x x + m \theta - \omega t \right)}\right) \]
\[\frac{\partial v_r'}{\partial x} = \underbrace{\frac{\partial v_r(r)}{\partial x}}_{0} e^{i\left(k_x x + m \theta - \omega t \right)} + v_r(r) \left(ik_x e^{i\left(k_x x + m \theta - \omega t \right)}\right) \]

RHS:
\[\frac{\partial p'}{\partial r} = \frac{\partial P(r)}{\partial r} e^{i\left(k_x x + m \theta - \omega t \right)} + P(r)\underbrace{\frac{\partial}{\partial  r} e^{i\left(k_x x + m \theta - \omega t \right)}}_0 \]

Recalling $ p'/\rho' = A^2 \rightarrow \rho' = \frac{1}{A^2}p' $ $ \frac{\partial \bar{P}}{\partial r} = \frac{\bar{\rho} v_{\theta}^2}{r} $

\[\frac{\rho ' }{\bar{\rho}^2} \frac{ \partial \bar{P}}{\partial r} = \frac{V_{\theta}^2}{A^2}\frac{1}{\bar{\rho} r}p'\]


After substituting and canceling common terms,
\begin{align*}
 v_r \left(-i\omega \cancel{e^{i\left(k_x x + m \theta - \omega t \right)}}\right)+
\frac{V_{\theta}}{r} v_r \left(im \cancel{e^{i\left(k_x x + m \theta - \omega t \right)}}\right) -
\frac{2V_{\theta}}{r}v_r  \cancel{e^{i\left(k_x x + m \theta - \omega t \right)}} + V_x \left(v_r \left(ik_x \cancel{e^{i\left(k_x x + m \theta - \omega t \right)}}\right)\right) \\
=   \left(-\frac{1}{\bar{\rho}} \frac{\partial P(r)}{\partial r}\cancel{e^{i\left(k_x x + m \theta - \omega t \right)} } +  \frac{V_{\theta}^2}{A^2}\frac{1}{\bar{\rho} r} P(r)\cancel{e^{i\left(k_x x + m \theta - \omega t \right)} }\right) 
\end{align*}

\[\left(-i\omega + \frac{i m V_{\theta}}{r} + i k_x V_x \right) v_r - \frac{2 V_{\theta}}{r}v_{\theta}  = -\frac{1}{\bar{\rho}} \frac{\partial P}{\partial r}+ \frac{V_{\theta^2}}{A^2}\frac{1}{\bar{\rho} r}p\]

Continuing with conservation of momentum in the $\theta$ direction,

\[\frac{\partial  v_{\theta}' }{\partial t} +
v_r' \frac{\partial  V_{\theta}  }{\partial r} +
\frac{V_{\theta}}{r} \frac{\partial v'_{\theta}}{\partial \theta} +
\frac{v'_rV_{\theta}}{r} +
V_x \frac{\partial v_{\theta}'}{\partial x} 
= -\frac{1}{\bar{\rho} r}	\frac{\partial p'}{\partial \theta}\]

\[\frac{\partial v'_{\theta}}{\partial t} = 
\underbrace{\frac{\partial v_{\theta}(r)}{\partial t}}_{0} e^{i\left(k_x x + m \theta - \omega t \right)} + 
v_{\theta}(r) \left(-i\omega e^{i\left(k_x x + m \theta - \omega t \right)}\right)\]

\[\frac{\partial v'_{\theta}}{\partial \theta} = \underbrace{\frac{\partial v'_{\theta}(r)}{\partial \theta}}_{0} e^{i\left(k_x x + m \theta - \omega t \right)} + v_{\theta}(r) \left(im e^{i\left(k_x x + m \theta - \omega t \right)}\right)\]

\[\frac{\partial v'_{\theta}}{\partial x} = \underbrace{\frac{\partial v_{\theta}(r)}{\partial x}}_{0} e^{i\left(k_x x + m \theta - \omega t \right)} + v_{\theta}(r) \left(ik_x e^{i\left(k_x x + m \theta - \omega t \right)}\right)\]

\[\frac{\partial p'}{\partial \theta} = \underbrace{\frac{\partial P(r)}{\partial \theta}}_{0} e^{i\left(k_x x + m \theta - \omega t \right)} + P(r)\underbrace{\frac{\partial}{\partial  \theta} e^{i\left(k_x x + m \theta - \omega t \right)}}_{m  e^{i\left(k_x x + m \theta - \omega t \right)}}\]



After substituting and canceling common terms 

\[\left(-i\omega + \frac{i m V_{\theta}}{r} + i k_x V_x \right) v_{\theta} + \left(\frac{V_{\theta}}{r} +  \frac{\partial V_{\theta}}{\partial r}\right)v_\theta= -\frac{m}{\bar{\rho}r}p\]
Next, the conservation of momentum in the x direction,
\[\frac{\partial  v_x' }{\partial t} +
v_r' \frac{\partial  V_x  }{\partial r} +
\frac{V_{\theta}}{r} \frac{\partial v'_x}{\partial \theta} +
V_x \frac{\partial v_x'}{\partial x} 
= -\frac{1    }{\bar{\rho}}  \frac{\partial p'}{\partial x} \]


\[\frac{\partial v_x'}{\partial t} = \underbrace{\frac{\partial v_x(r)}{\partial t}}_{0} e^{i\left(k_x x + m \theta - \omega t \right)} + 
v_x(r) \left(-i\omega e^{i\left(k_x x + m \theta - \omega t \right)}\right)\]

\[\frac{\partial v_x'}{\partial \theta} = \underbrace{\frac{\partial v_x(r)}{\partial \theta}}_{0} e^{i\left(k_x x + m \theta - \omega t \right)} + 
v_x(r) \left(im e^{i\left(k_x x + m \theta - \omega t \right)}\right)\]


\[\frac{\partial v_x'}{\partial x} = \frac{\partial v_x(r)}{\partial x} e^{i\left(k_x x + m \theta - \omega t \right)} + 
v_x(r) \left(i k_x e^{i\left(k_x x + m \theta - \omega t \right)}\right)\]

\[\frac{\partial p'}{\partial x} = 0 + ik_xpe^{i\left(k_x x + m \theta - \omega t \right)} \]

\[\left(-i\omega + \frac{imV_{\theta}}{r} + i k_xV_x\right)v_x + \frac{\partial V_x}{\partial r} v_r = - \frac{i
	k_x}{\bar{\rho}}p\]

The Linearized Euler equations now become

\begin{align*}
r\text{-direction: }& i\left(-\omega + \frac{ m}{r} +  k_x V_x \right) v_r - \frac{2 \bar{v_{\theta}}}{r}v_{\theta}  = -\frac{1}{\bar{\rho}} \frac{\partial P}{\partial r}+ \frac{V_{\theta^2}}{A^2}\frac{1}{\bar{\rho} r}p\\
\theta\text{-direction: }& i\left(-\omega + \frac{ m}{r} +  k_x V_x \right) v_{\theta} + \left(\frac{V_{\theta}}{r} +  \frac{\partial V_{\theta}}{\partial r}\right)v_\theta = -\frac{m}{\bar{\rho}r}p \\ 
x\text{-direction: }&i\left(-\omega + \frac{mV_{\theta}}{r} +  k_xVx\right)v_x + \frac{\partial V_x}{\partial r} v_r = - \frac{i
	k_x}{\bar{\rho}}p\\ 
\text{Energy: }&\frac{1}{\bar{\rho} A^2} \left(-i\omega + \frac{imV_{\theta}}{r} + ik_xV_x  +
\right)p(r)  +
\frac{V_{\theta}^2}{A^2 r}v_r'+ \frac{v_r'}{r} +
\frac{\partial v_r(r)}{\partial r}+ 
\frac{im}{r} v_{\theta}(r)
+
ik_xv_x(r) 
&= 0
\end{align*}


\section{Non-Dimensionalization}
Defining 

\[r_T = r_{max}\]
\[A_T = A(r_{max})\]
\[k = \frac{\omega r_T}{A_T}\]
\[\bar{\gamma} = k_x r_T\]
\[\tilde{r} = \frac{r}{r_T}\]
\[\frac{\partial }{\partial r} = \frac{\partial \tilde{r}}{\partial r} \frac{\partial }{\partial \tilde{r}} = \frac{1}{r_T} \frac{\partial }{\partial \tilde{r}}\]
\[V_{\theta} = M_{\theta} A\]
\[V_{x} = M_{x} A\]
\[\tilde{A} = \frac{A}{A_T}\]
\[v_{x} =\tilde{v}_x A\]
\[v_{r} =\tilde{v}_r A\]
\[v_{\theta} =\tilde{v}_{\theta} A\]
\[p = \tilde{p} \bar{\rho} A^2\]

\begin{align*}
r\text{-direction: }& i\left(-\omega + \frac{ m V_{\theta}}{r} +  k_x V_x \right) v_r - \frac{2 \bar{v}_{\theta}}{r}v_{\theta}  = -\frac{1}{\bar{\rho}} \frac{\partial P}{\partial r}+ \frac{V_{\theta^2}}{A^2}\frac{1}{\bar{\rho} r}p\\
\theta\text{-direction: }& i\left(-\omega + \frac{ m}{r} +  k_x V_x \right) v_{\theta} + \left(\frac{V_{\theta}}{r} +  \frac{\partial V_{\theta}}{\partial r}\right)v_\theta= -\frac{m}{\bar{\rho}r}p \\ 
x\text{-direction: } &i\left(-\omega + \frac{mV_{\theta}}{r} +  k_xVx\right)v_x + \frac{\partial V_x}{\partial r} v_r = - \frac{i
	k_x}{\bar{\rho}}p\\
\text{Energy: }&\frac{1}{\bar{\rho} A^2} i\left(-\omega + \frac{mV_{\theta}}{r} + k_xV_x  
\right)p(r)  +
\frac{V_{\theta}^2}{A^2 r}v_r'+ \frac{v_r'}{r} +
\frac{\partial v_r(r)}{\partial r}+ 
\frac{im}{r} v_{\theta}(r)
+
ik_xv_x(r) 
= 0
\end{align*}

Substituting the non dimensional quantities, and noting $r_T$ and $A^2$ in each term

\[ i\left[ 
- \frac{k}{\tilde{A}} + 
\frac{m M_{\theta}}{\tilde{r}} + 
\bar{\gamma} M_x \right] 
\tilde{v}_r - 
\frac{2 M_{\theta} \tilde{v}_{\theta}}{\tilde{r}} = 
\frac{1}{\bar{\rho} A^2}\frac{\partial \tilde{p}\bar{\rho} A^2}{\partial \tilde{r}} + 
M_{\theta}\frac{\tilde{P}}{\tilde{r}}\]

Similarly for the $\theta$and $x$ directions:

\begin{align*}
	 i\left[ - \frac{k}{\tilde{A}} + \frac{m M_{\theta}}{\tilde{r}} + \bar{\gamma} M_x \right] \tilde{v}_{\theta} + \left(\frac{ M_{\theta}}{\tilde{r}}  + \frac{1}{A} \frac{\partial M_{\theta}A}{\partial \tilde{r}}\right)\tilde{v}_r = \frac{i m}{\tilde{r}}\tilde{P}\\
	 i\left[ - \frac{k}{\tilde{A}} + \frac{m M_{\theta}}{\tilde{r}} + \bar{\gamma} M_x \right] \tilde{v}_x  + \frac{1}{A} \frac{\partial M_x A}{\partial \tilde{r}}\tilde{v}_r = -i \bar{\gamma}\tilde{P}
\end{align*}

and for the Energy equation:
\begin{align*}
	 i \left[ \frac{k}{\tilde{A}} + 
	 \frac{mM_{\theta}}{\tilde{r}} - 
	 \bar{\gamma}M_x \right] \tilde{p} + 
	 \frac{M_{\theta}^2}{\tilde{r}}\tilde{v}_r + 
	 \frac{1}{A}\frac{\partial ( \tilde{v}_r A)}{\tilde{r}}+ 
	 \frac{\tilde{v}_r  }{\tilde{r}} + \frac{im}{\tilde{r}}\tilde{v}_\theta + i \bar{\gamma} \tilde{v}_x = 0
\end{align*}
\[\]
Expanding mean derivatives (using product rule) $\frac{\partial \tilde{p}\bar{\rho} A^2}{\partial \tilde{r}} $

\[ i\left[ 
- \frac{k}{\tilde{A}} + 
\frac{m M_{\theta}}{\tilde{r}} + 
\bar{\gamma} M_x 
\right] \tilde{v}_r - 
\frac{2 M_{\theta} \tilde{v}_{\theta}}{\tilde{r}} = 
\frac{1}{\bar{\rho} A^2}\left( \frac{\partial \tilde{p}}{\partial \tilde{r}} +
\frac{\partial \bar{\rho} A^2}{\partial \tilde{r}}  \right)+ 
M_{\theta}\frac{\tilde{P}}{\tilde{r}}\]


\[i\left[ - \frac{k}{\tilde{A}} + \frac{m M_{\theta}}{\tilde{r}} + \bar{\gamma} M_x \right] \tilde{v}_{\theta} + \left(\frac{ M_{\theta}}{\tilde{r}}  + \frac{1}{A} \frac{\partial M_{\theta}A}{\partial \tilde{r}}\right)\tilde{v}_r = \frac{i m}{\tilde{r}}\tilde{P}\]
\[i\left[ - \frac{k}{\tilde{A}} + \frac{m M_{\theta}}{\tilde{r}} + \bar{\gamma} M_x \right] \tilde{v}_x  + \frac{1}{A} \frac{\partial M_x A}{\partial \tilde{r}}\tilde{v}_r = -i \bar{\gamma}\tilde{P} \]
Need energy equation still


then put in matrix form to complete theory.

\chapter{Chapter 4: Numerical Models}
\section{Numerical Integration}

\section{Introduction}
The Method of Manufactured Solutions (MMS) is a process for generating an 
analytical solution for a code that provides the numerical solution for a 
given domain. The goal of MMS is to establish a manufactured solution that can 
be used to establish the accuracy of the code within question. For this study, 
SWIRL, a code used to calculate the radial modes within an infinitely long duct
is being validated through code verification. SWIRL accepts a given mean flow and 
uses numerical integration to obtain the speed of sound. The integration technique
is found to be the composite trapezoidal rule through asymptotic error analysis.


% !TeX root = ../defense.tex
\section{Steady Flow Aerodynamic Model}
\frame{\sectionpage}
\begin{frame} 
For an ideal polytropic gas with density, $\rho$, velocity, $\vec{V}$, and 
pressure, $p$ , the Euler equations are,
\begin{align}
    \frac{D\rho}{Dt} = - \rho \nabla \cdot \vec{V} 
    \label{eqn:CompressibleConservationOfMass} \\
    \frac{D\vec{V}}{Dt} = - \frac{\nabla p}{\rho} + \vec{g} 
    \label{eqn:CompressibleConservationOfMomentum} \\
    \frac{Dp}{Dt} = - \gamma p \nabla \cdot \vec{V} 
    \label{eqn:CompressibleConservationOfEnergy} \\
\end{align}
where Equations (\ref{eqn:CompressibleConservationOfMass},
\ref{eqn:CompressibleConservationOfMomentum},
\ref{eqn:CompressibleConservationOfEnergy}) are the conservation of mass, 
momentum, and energy equations respectively. 
\end{frame}
\begin{frame}
If the flow is assumed to be asymmetric, then the radial velocity component is 
zero. With this considered, the velocity vector ,$\vec{V}$ in cylindrical coordinates 
become,
\begin{align}
    \vec{V}(r,\theta,x) &= v_x(r) \hat{e}_x + v_{\theta} (r) \hat{e}_{\theta} 
    \label{eqn:VelocityVector}
\end{align}
  
where $\hat{e}_x$ and $\hat{e}_{\theta}$ are unit vectors for the axial and 
tangential directions.The gradient operator ,$\nabla$ in cylindrical
coordinates, is 

\begin{align}
	\vec{\nabla} 
	&= \hat{e}_r \frac{\partial} {\partial r} + 
	\frac{1}{r} \hat{e}_{\theta}   
	\frac{\partial} {\partial \theta} + 
	\frac{\partial }{\partial z} \hat{e}_z = 0
    \label{eqn:NablaInCylindrical}
\end{align}


    
\end{frame}
\begin{frame}
For a steady flow, ($\partial/\partial t = 0$) , the compressible flow equations
can be further reduced to,

\begin{align}
    \nabla (\vec{V} \rho) &=  0 \\
    (\vec{V}\cdot \nabla) \vec{V} &=  0\\
    \nabla S &= 0
\end{align}

where $S$ represents the entropy in the mean flow.


\end{frame}
\begin{frame}{Nonuniformities from swirling mean flow}
    

If the mean flow contains a swirling component, i.e. a velocity vector in the 
tangential direction, the mean quantities, pressure , density are non-uniform, 
thus also changing the speed of sound. By integrating the radial momentum
equation, an expression for the speed of sound was established to account for 
the resulting nonuniformities due to rotations in the flow. 
\begin{equation}
    p = \int_{r_{min}}^{r_{max}} \frac{\rho v_{\theta}^2}{r} dr 
    \label{eqn:radialmomentum_integrated}
\end{equation}

where $r_{min}$ and $r_{max}$ are the bounds of the radius. Since the flow
is isentropic, the pressure is related to the speed of sound through $\nabla p =
A^2 \nabla \rho$; which is used to compute $\rho$. 

\end{frame}
\begin{frame}

With the relationship 
$A^2 = \kappa p/\rho$, the speed of sound is found to be,

    
\begin{align*}
\tilde{A}(\tilde{r}) &= exp\left[\left(\frac{1 - \gamma}{2}\right)\int_{\tilde{r}}^{1}\frac{M_{\theta}}{\tilde{r}}{\partial \tilde{r}}\right]	
\end{align*}

\end{frame}
\begin{frame}

    
\end{frame}
\section{Unsteady Flow Aerodynamic Model}
\frame{\sectionpage}
\begin{frame}
    
Goldstein demonstrated the linearized momentum and continuity PDE  can be 
combined to derive the convective wave equation by taking the divergence of
the momentum equation and taking the difference of the substantial derivative
of the conservation of mass equation to yield,

\begin{equation}
    \frac{1}{A^2}\frac{D^2\tilde{p}}{Dt^2} -
    \nabla^2 \tilde{p} =
    2 \bar{\rho} \frac{d V_x}{d x} \frac{\partial  \tilde{v}_r}{ \partial x} 
    \label{eqn:KousensWaveEquation}
\end{equation}

\end{frame}
\begin{frame}
    

In the case of sheared flow, $dV_x/dx = 0$ the right hand side will be zero 



\begin{align*}
    \frac{1}{A^2}\left(
        \frac{\partial^2 \tilde{p}}{\partial t^2} + 
        \vec{V}\cdot \vec {\nabla} (\tilde{p}) 
    \right) -
    \nabla^2
    \tilde{p} &=
    0 \\
\end{align*}

\end{frame}
\begin{frame}

Utilizing the relation, $\tilde{p} = p/\bar{\rho} A^2$,substituting the 
definitions for $\nabla$ and $\nabla^2$ and setting $\vec{V} =0$
in cylindrical coordinates gives,



\begin{align*} 
    \frac{1}{A^2}\left(
        \frac{\partial^2 {p}}{\partial t^2}
    \right) - 
        \left(
            \frac{\partial^2 {p}}{\partial t^2} + 
            \frac{1}{\tilde{r}}\frac{\partial p}{\partial r} +
            \frac{1}{\tilde{r}^2} \frac{\partial^2 p}{\partial \theta^2} + 
            \frac{\partial^2 p}{\partial x^2} 
        \right) &= 0  
\end{align*} 

    
\end{frame}
\begin{frame}
    

 The method of separation of variables requires an assumed solution as well as initial and boundary 
conditions. 

For a partial differential equation, the assumed solution can be a 
linear combination of solutions to a system of ordinary differential equations that
comprises the partial differential equation. 

Since $p$ is a function of four variables, the solution is 
assumed to be a linear combination of four solutions.

Each solution is assumed to be Euler's identity, a common ansant for linear partial 
differential equations and boundary conditions.

 Defining,

\begin{equation}
    p(x,r,\theta,t) = X(x) R(r) \Theta(\theta) T(t)
\end{equation}
\end{frame}
\begin{frame}
where, 

\begin{align*}
    X(x) &=
    A_1 e^{ik_x x} +
    B_1 e^{-ik_x x }\\
    \Theta(\theta) &=
    A_2 e^{i k_{\theta} \theta } +
    B_2 e^{-ik_{\theta} \theta }\\
    T(t) &=
    A_3 e^{i \omega t } +
    B_3 e^{-i\omega t  }
\end{align*}



The next step is to rewrite the wave equation in terms of $X$, $R$, $\Theta$,
and $T$. To further simplify the result, each term is divided by $p$.

\begin{equation}
    \frac{1}{A^2} \frac{1}{T}\frac{\partial^2 T}{\partial t^2} = 
    \frac{1}{R}\frac{\partial^2 R}{\partial r^2 } +
    \frac{1}{r}\frac{1}{R}\frac{\partial R}{\partial r}  + 
    \frac{1}{r^2}\frac{1}{\Theta}\frac{\partial \Theta}{\partial \theta} + 
    \frac{1}{X}\frac{\partial^2 X}{\partial x^2}
    \label{eqn:waveode}
\end{equation}
\end{frame}
    

\begin{frame}
    

\begin{equation}
    \frac{1}{R}
    \left(      
    \frac{\partial^2 R}{\partial r^2 } +
    \frac{1}{r}\frac{\partial R}{\partial r}  
\right) -
    \frac{k_{\theta}^2}{r^2}-  
    k_x^2 + k^2 = 0
    \label{eqn:waveode3}
\end{equation}
The remaining terms are manipulated to follow the same form as \textit{Bessel's Differntial 
Equation} ,

\begin{equation}
    x^2 \frac{d^2 y}{dx^2} + x \frac{dy }{dx } + (x^2 - n^2) y = 0
    \label{eqn:besselODE}
\end{equation}

The general solution to Bessel's differential equation is a linear combination of
the Bessel functions of the first kind, $J_n(k_r r)$ and of the second kind, $Y_n(k_r r)$ 
\cite{wolphram:bessel}. The subscript $n$ refers to the order of Bessel's equation.
    
\begin{equation}
    R(r) = (AJ_n(k_r r) + BY_n(k_r r)) 
    \label{eqn:besselsolution}
\end{equation}
where the coefficients $A$ and $B$ are found after applying radial
boundary conditions. %and there is an exponential dependence. 
\end{frame}
\begin{frame}
By rearranging Equation (\ref{eqn:waveode3}), a comparison can be made to Equation
(\ref{eqn:besselODE}) to show that the two equations are of the same form. 

The first step is to revisit the radial derivatives that have not been addressed.
As was done for the other derivative terms, the radial derivatives will also 
be set equal to a separation constant, $-k_r^2$. 

\begin{align}
    \underbrace{\frac{1}{R}
    \left(      
    \frac{\partial^2 R}{\partial r^2 } +
    \frac{1}{r}\frac{\partial R}{\partial r}  
\right) -
    \frac{k_{\theta}^2}{r^2}}_{-k_r^2}-  
    k_x^2 + k^2 = 0
    \label{eqn:wavenumber_without_kr}
\end{align}

\end{frame}
\begin{frame}
    

The reader may be curious as to why the tangential separation constant $k_{\theta}$ is 
included within the definition of the radial separation constant. 

Recall the ODE for the tangential direction, 

\begin{align*}
    \frac{\partial \Theta}{\partial \theta} \frac{1}{\Theta} = - k_{\theta}^2\\
    \frac{\partial \Theta}{\partial \theta} \frac{1}{\Theta} + \Theta k_{\theta}^2 = 0 
\end{align*}

where the solution is more or less,

\begin{align*}
    \Theta(\theta) = e^{i k_{\theta} \theta}
\end{align*}

In order to have non trivial, sensible solutions, the value of $\Theta(0)$ and
$\Theta(2\pi)$ need to be the same, and this needs to be true for any multiple 
of $2\pi$ for a fixed r. Taking $\Theta$ to be one, a unit circle, it can be shown that the domain
is only going to be an integer multiple. Therefore, there is an implied periodic
azimuthal boundary condition, i.e. $0<\theta\leq 2 \pi$ and $k_{\theta}=m$. 

\end{frame}

\begin{frame}
There is a unique treatment for the radial derivatives.


\begin{align*}
    -k_r^2 =\frac{1}{R}
    \left(      
    \frac{\partial^2 R}{\partial r^2 } +
    \frac{1}{r}\frac{\partial R}{\partial r}  
\right) -
    \frac{m^2}{r^2} 
\end{align*}
To further simplify, the chain rule is used to do a change of variables, $x = k_r r$
\begin{align*}
    \frac{\partial R}{\partial r} &= \frac{dR}{dx}\frac{dx}{dr}\\
    &=
    \frac{dR}{dx}\frac{d}{dr}\left( k_r r \right) \\
    &=
    \frac{dR}{dx} k_r 
\end{align*} 


    
\end{frame}
\begin{frame}
    
\begin{align*}
    \frac{\partial^2 R}{\partial r^2} &= \frac{d^2R}{dx^2}\left(\frac{dx}{dr}\right)^2 + 
    \frac{dR}{dr}\frac{d^2x}{dr^2}\\
    &=
    \frac{d^2R}{dx^2}\frac{d}{dr} k_r^2 + k_r \frac{d^2r}{dr^2}\\
    &=
    \frac{d^2R}{dx^2}\frac{d}{dr} k_r^2
\end{align*} 

Substituting this into Equation (\ref{eqn:waveode3}),
\begin{equation}
    \left(\frac{d^2R}{dx^2}k_r^2 +
    \frac{1}{r}\frac{d^2R}{dx^2}k_r\right) +
    \left(k_r^2 - \frac{m^2}{r^2}\right)R
    \label{eqn:waveode4}
\end{equation}
Dividing Equation \ref{eqn:waveode4} by $k_r^2$,

\end{frame}
\begin{frame}
\begin{equation}
    \left(\frac{d^2R}{dx^2} +
    \frac{1}{k_r r}\frac{d^2R}{dx^2}\right) +
    \left(1  - \frac{m^2}{k_r^2 r^2}\right)R
    \label{eqn:waveode5}
\end{equation}

\begin{equation}
    \left(\frac{d^2R}{dx^2} +
    \frac{1}{x^2}\frac{d^2R}{dx^2}\right) +
    \left(1  - \frac{m^2}{x^2}\right)R
    \label{eqn:waveode6}
\end{equation}
\end{frame}
\begin{frame}
    
Multiplying Equation (\ref{eqn:waveode6}) by $x^2$ gives,

\begin{equation}
    \frac{d^2R}{dr^2}x^2 + 
    \frac{dR}{dr}x + 
    \left( x^2 - m^2 \right)R
    \label{eqn:finalradialode}
\end{equation}
which matches the form of Bessel's equation
\end{frame}
\begin{frame}

Therefore, the solution goes from this,
to this,


\begin{equation}
    y(x) = AJ_n(x) + BY_n(x)
    \label{eqn:besselsolution}
\end{equation}
to this,

\begin{equation}
    R(r) = AJ_n(k_r r) + BY_n(k_r r)
    \label{eqn:besselsolution}
\end{equation}
\end{frame}
\begin{frame}
    
\subsubsection{Hard Wall boundary condition}
\begin{align*}
    \frac{\partial p}{\partial r}|_{r = r_{min}}  =\frac{\partial p}{\partial r}|_{r = r_{max}} = 0 \rightarrow 
    \frac{\partial}{\partial r} \left( X\Theta T R \right) &= 0 \\
    X \Theta T\frac{\partial R}{\partial r}  &= 0 \\
    \frac{\partial R}{\partial r}  &= 0 
\end{align*}

where,


\begin{align*} 
    \frac{ \partial R}{\partial r}|_{r_{min}} &= AJ_n'(k_r r_{min}) + B Y_n' (k_r r_{min}) = 0 
    \rightarrow B = -A \frac{J_n'(k_r r_{min})}{Y_n'(k_r r_{min})}
\end{align*}

\end{frame}
\begin{frame}
\begin{align*} 
    \frac{ \partial R}{\partial r} &= AJ_n'(k_r r_{max}) + B Y_n' (k_r r_{max}) = 0 \\
                                   &= AJ_n'(k_r r_{max}) - A\frac{J_n' (k_r r_{min})}{Y_n'(k_r r_{min})} Y_n' (k_r r_{max}) = 0 \\
                                   &= \frac{J_n'(k_r r_{min})}{J_n' (k_r r_{max})} - \frac{Y_n'(k_r r_{min})}{Y_n' (k_r r_{max})} = 0 
\end{align*}
where $k_r r$ are the zero crossings for the derivatives of the Bessel functions of the first and second kind.

In summary, the wave equation for no flow in a hollow duct with hard walls is obtained 
from Equation (\ref{eqn:wavenumber_without_kr}).
\begin{equation}
    k^2 = k_r^2 + k_x^2
    \label{eqn:wavenumber_equation}
\end{equation}

\end{frame}
% \section{Results Update}
% \frame{\sectionpage}

% \begin{frame}{Motivation}
%     \begin{alertblock}{How is the analytical solution computed for Sound Prop. in Uniform Axial Flow}
%     \begin{enumerate}[<+->]\itemsep9pt
%         \item The analytical solution are the axial wavenumbers and propagating
%             modes for a given frequency, axial velocity and azimuthal mode order.
%         \item For a given azimuthal mode, there is a range of radial modes.
%         \item These radial modes can be categorized based on the sign of the axial wavenumber and if it is
%             complex in value. 
%         \item Propagating modes are defined by axial wavenumbers, $k_x$, that have a real-part only, yielding 
%             the assumed fluctuation to resemble Euler's Formula ($e^{ik_x x}$). 
%         \item On the other hand, if the $k_x$ is complex, then the mode will resemble an exponentially decaying
%             function since the imaginary number cancels, leaving a minus sign in front of
%             the axial wavenumber.
%     \end{enumerate}
%     \end{alertblock}

%     \tiny
%     % \hspace{3.75em}\url{http://www.klimaschutzplan2050.de/en/action-areas/energy-sector/}
% \end{frame}
% \begin{frame}

% \begin{equation}
%     k_x  = \frac{- M_x k \pm \sqrt{k^2 - ( 1 - M_x^2) J_{m,n}'^2 }}{\left( 1 - M_x^2 \right)}.
%     \label{eqn:ax_wavenumb}
% \end{equation}

% where $M_x$ is the axial Mach number, $k$ is the temporal (referred to as reduced)
% frequency, and $J_{m,n}'$ is the derivative of the Bessel function of the first kind.  
% The $\pm$ accounts for both upstream and downstream modes.

% The condition for propagation is such that the axial wavenumber is larger than 
% a ``cut-off'' value

% \begin{equation}
%     k_{x,real}  = \frac{\pm M_x k }{\left( M_x^2 - 1 \right)}.
%     \label{eqn:cuton}
% \end{equation}


% \end{frame}
% \begin{frame}
    
% Every term that is being raised to the one half i.e. square rooted must 
% be larger than zero to keep axial wavenumber from being imaginary. The mode 
% will propagate or decay based on this condition. Recall thaT the mode is of the 
% form 
% \begin{equation}
%     e^{i k_x x}
%     \label{eqn:fluctuationexample}
% \end{equation}
% if $k_x$ has a real part, $k_{x,real}$ and an imaginary part $i k_{x,imag}$ 
% then,

% \begin{align}
%     &= e^{i k_x x} \\
%     &= e^{i (k_{x,real}+ i k_{x,imag}) x} \\
%     &= \underbrace{e^{i k_{x,real}x}}_{\textit{amplitude}} \underbrace{e^{- k_{x,imag} x}}_{\textit{exponential decay}} 
% \end{align}


% \end{frame}
% \begin{frame}
    

% Although the ``cut-off'' decay to nearly zero rapidly, the rate at which this occured
% was not much of a concern earlier on in turbomachinery design. As nacelles 
% continue to grow shorter, a mode that is ``cut-off'' may make it outside the duct.

% For this work a desired amplitude was arbitrarily chosen for a mode, $y_{desired}$
% and then the axial location at which this occurred, $x_{desired}$ which 
% can be compared against a desired length for a nacelle.  
% Since SWIRL assumes an infinitely long duct, there is nothing limiting the 
% modes propagation with respect to nacelle length. For example, if the 
% desired amplitude is one percent, then $x_{desired}$,

% \begin{align*}
%     0.01 &=  e^{-10 x_{desired}},\\
%     -\frac{ln|0.01|}{10} &=  x_{desired},\\
%     -\frac{ln|0.01|}{10} &= 0.4605170185988091 .
% \end{align*}

% \end{frame}
% \begin{frame}
    


%  \begin{figure}
%      \centering
%      \includegraphics[width=0.7\textwidth]{/home/jeff-severino/SWIRL/CodeRun/04-plotReport/tex-outputs/desired_cut_off_location_1_percent_of_max2.pdf}
%      \caption{Decaying mode with $k_x = 0 + 10j$ and unit amplitude. One percent
%      of the maximum amplitude is identified for nacelle length comparison}
%      \label{fig:decaying_mode_with_1_percent_amp}
%  \end{figure}
 
 
% In general,
% \begin{align*}
%     y_{desired} &=  e^{-k_{x,imag} x_{desired} }\\
%     -\frac{ln|y_{desired}|}{k_{x,imag}} &=  x_{desired}
% \end{align*}
% \end{frame}
% \begin{frame}
% \section{Analytical Test Case 1}
% \begin{table}[h!]
%     \centering
%     \begin{tabular}{|l|l|}
%         \hline
%         $\sigma$ & \textit{0.0} \\ \hline
%         $k$      & \textit{10}   \\ \hline
%         $m$      & \textit{2}    \\ \hline
%         $M_x$    & \textit{0.3}  \\ \hline
%     \end{tabular}
%     \caption{Validation test case parameters, Uniform Flow Annular Duct} 
% \end{table}

% \end{frame}

% \begin{frame}
    
%  \begin{figure}
%      \centering
%      \includegraphics[width=0.7\textwidth]{/home/jeff-severino/SWIRL/CodeRun/04-plotReport/tex-outputs/bessel_analytical_test_case.pdf}
%      \caption{The Bessel function with the values of $J'_{m,\mu}$ labeled}
%      \label{fig:decaying_mode_with_1_percent_amp}
%  \end{figure}
% \end{frame}



% \begin{frame}
    
%  \begin{figure}
%      \centering
%      \includegraphics[width=0.7\textwidth]{/home/jeff-severino/SWIRL/CodeRun/04-plotReport/tex-outputs/axial_wavenumber_analytical_test_case.pdf}
%      \caption{The Bessel function with the values of $J'_{m,\mu}$ labeled}
%      \label{fig:decaying_mode_with_1_percent_amp}
%  \end{figure}
% \end{frame}


% \begin{frame}{Normalized Mode}

    
%  \begin{figure}
%      \centering
%      \includegraphics[width=0.7\textwidth]{/home/jeff-severino/SWIRL/CodeRun/04-plotReport/tex-outputs/Normalized_Mode_test_case.pdf}
%      \caption{The Bessel function with the values of $J'_{m,\mu}$ labeled}
%      \label{fig:decaying_mode_with_1_percent_amp}
%  \end{figure}
% \end{frame}
% \begin{frame}
    
% \end{frame}




% \begin{frame}{Future Work}
%     \begin{alertblock}{}
%     \begin{enumerate}[<+->]\itemsep9pt
%         \item I need to use sanity checks to ensure that the normalization provided
%             by Rienstra is implemented correctly.
%         \item While the numerical normalized mode has an integral of one, the 
%             analytical mode is not.
%         \item a few things that have been checked along the way but need to be 
%             reported are,
%         \item Zero's of $J'_m$ 
%         \item Value of $J_m$ at the zero location
%         \item Relations involving integrals If it so not too cumbersome.
%             There are some simplifications that could be checked\ldots 
%         \item Check the analytical test case that has been reported in literature.
%            The difference is that $\sigma = 0.25$
%     \end{enumerate}
%     \end{alertblock}

%     \tiny
%     % \hspace{3.75em}\url{http://www.klimaschutzplan2050.de/en/action-areas/energy-sector/}
% \end{frame}

% % \begin{frame}{Perspective}{Disciplines for investigating energy topics}
% %     \begin{center}
% %     \begin{tikzpicture}[
% %       node distance=4.5em and .75cm,font=\small
% %     ]
% %     % flowboxes
% %     \node[flowbox] (physik) {
% %         \fbtitle{Physics}\vphantom{yÖ}
% %     \nodepart{two}
% %         \begin{minipage}{.16\textwidth}
% %             \centering
% %             Theoretical\\ feasibility\\
% %             \scriptsize (Natural laws)
% %         \end{minipage}
% %     };

% %     \node[flowbox,right=of physik] (technik) {
% %         \fbtitle{Engineering}\vphantom{yÖ}
% %     \nodepart{two}
% %         \begin{minipage}{.16\textwidth}
% %             \centering
% %             Technical\\ feasibility\\
% %             \scriptsize (Technologies)
% %         \end{minipage}
% %     };

% %     \node[flowbox,right=of technik] (econ) {
% %         \fbtitle{Economy}\vphantom{yÖ}
% %     \nodepart{two}
% %         \begin{minipage}{.16\textwidth}
% %             \centering
% %             Economic\\ feasibility\\
% %             \scriptsize (Funding)
% %         \end{minipage}
% %     };

% %     \node[flowbox,right=of econ] (society) {
% %         \fbtitle{Society}\vphantom{yÖ}
% %     \nodepart{two}
% %         \begin{minipage}{.16\textwidth}
% %             \centering
% %             Social\\ feasibility\\
% %             \scriptsize (Decision space)
% %         \end{minipage}
% %     };

% %     \uncover<2->{%
% %         \draw [decorate,decoration={brace,amplitude=10pt,mirror},ultra thick,jdblue]
% %             ($(technik.south west) + (-.2em,-1em)$) --
% %             ($(econ.south east)    + (+.2em,-1em)$) coordinate[midway,yshift=-3.8em] (midpoint-below);
    
% %         \node[flowbox] at (midpoint-below) (tech-econ)  {
% %             \fbtitle{Techno-economic modelling}\vphantom{yÖ}
% %         \nodepart{two}
% %             \begin{minipage}{.4\textwidth}
% %                 \centering
% %                 How much energy?
% %                 For how much?\vphantom{yÖ}
% %             \end{minipage}
% %         };
% %     }
% %     \end{tikzpicture}
% %     \end{center}
% % \end{frame}

\include{Chapter-4-Theory-Numerical-Models/Theory-Numerical-Integration/theory}
%
\subsection{Procedure}
There are a few constraints and conditions that must be followed in order for the analytical 
function to work with SWIRL, 

\begin{itemize}
    \item The mean flow and speed of sound must be real and positive. This will 
        occur is a speed of sound is chosen such that the tangential mach number
        is imaginary
    \item The derivative of the speed of sound must be positive
    \item Any bounding constants used with the mean flow should not allow the 
        total Mach number to exceed one.
    \item the speed of sound should be one at the outer radius of the cylinder
\end{itemize}

Given these constraints, $tanh(r)$ is chosen as a function since it can be
modified to meet the conditions above. Literature (The tanh method: A tool for 
solving certain classes of nonlinear evolution and wave equations) 
is a paper than demonstrates the strength of using tanh functions.
One additonal benefit of tanh(r) is that it is bounded between one and negative one, i.e.

\begin{itemize}
    \item As r $\rightarrow$ $\infty$ tanh(r) $\rightarrow$ 1
    \item As -r $\rightarrow$ $-\infty$ tanh(r) $\rightarrow$ -1
\end{itemize}

To test the numerical integration method,  $M_{\theta}$ is defined as a result 
of differentiating the speed of sound, $A$. This is done opposed to integrating
$M_{\theta}$ analytically. However, an analytical function can be defined for
$M_{\theta}$, which can then be integrated to find what $\widetilde{A}$ should be. 
Instead, the procedure of choice is to back calculate what the appropriate 
$M_{\theta}$ is for a given expression for $\widetilde{A}$.  Since it is easier 
to take derivatives , we will solve for $M_{\theta}$ using Equation \ref{eq:Mthetabackcalculated} ,

\subsection{Tanh Summaion Formulation}
The goal is generate an MS with a number of ``stairs'' that is bounded between
zero and one. Here's what my focus group ideas are,

\begin{align*}
    1 = R + L 
\end{align*}
where, 1 is a constraint, and R and L are the two waves when summed need to cancel 
if it were the exact same amplitude \& opposite sign 

so ,

\begin{align*}
    R + L = \tanh(x) + -\tanh(x) = 0
\end{align*}
or in our case,

\begin{align*}
    R + L = \tanh(x) + -\tanh(x) = 1
\end{align*}

We can tweak this by adding knobs by adding ``knobs'' A and B. If we dont want 
the total to not exceed one then, $A_j + A_{j+1} \cdots A_{last} = 1$. $B_1$ changes
the steepnes of the kink that we want. In order to generalize this,


\begin{align*}
    \bar{A} = \sum_{j=1}^n R_{ij} + \sum_{j=1}^n L_{ij} 
\end{align*}
where,
\begin{align*}
    R_{ij} = A_j \tanh(B_j (x_i - x_j)) \\ 
    L_{ij} = A_j \tanh(B_j (x_j - x_n))  
\end{align*}

Letting $n = 3 \ldots$

\begin{align*}
    \bar{A} &= S_{vert} + \sum_{j=1}^3 R_{ij} + \sum_{j=1}^3 L_{ij} \\
    \bar{A} &=
    A_1 \tanh(B_1 (x_i - x_1))  + 
    A_{2} \tanh(B_{2} (x_i - x_{2}))  + 
    A_{3} \tanh(B_{3} (x_i - x_3)) +  \\
    A_1 \tanh(B_1 (x_1 - x_n)) &+ 
    A_{2} \tanh(B_{2} (x_2 - x_{n}))  +
    A_{3} \tanh(B_{3} (x_3 - x_n))  
\end{align*}
and,
\begin{align*}
    A_1 = A_2 = A_3 = k_1 \\ 
    B_1 = B_2 = B_3 = k_2  
\end{align*}

A tanh summation method was constructed to make a manufactured solution with 
strong changes in slope. This ensures that the numerical approximation will not 
give trivial answers. 
then for some functions we need to impose boundary conditions. We will demonstrate
how the careless implementation of a boundary condition can lead to close approximations
on the interior.  The speed of sound is defined with the subscript $analytic$ to indicate that this is the analytical function of choice and has no physical relevance 
to the actual problem.

\begin{align*}
\widetilde{A}_{analytic} = \Lambda + k_1 \tanh \left( k_2 \left( \widetilde{r} - \widetilde{r}_{max} \right) \right),
\end{align*}

where, 

\begin{align*}
    \Lambda = 1 - k_1 \tanh(k_2 (1 - \widetilde{r}_{max})),
\end{align*}

When, $\widetilde{r}=\widetilde{r}_{max}$ , $\widetilde{A}_{analytic} = 1$.  
Taking the derivative with respect to $\widetilde{r}$,

\begin{align*}
    \frac{\partial \widetilde{A}_{analytic} }{\partial \widetilde{r}} &=
    \left(1 - \tanh^{2}{\left(\left(r - r_{max}\right) {k}_{2} \right)}\right) {k}_{1} {k}_{2}, \\ 
    &= \frac{ k_{1} k_{2}}{\cosh^{2}{\left(\left(r - r_{max}\right) {k}_{2} \right)}}.
\end{align*}

Substitute this into the expression for $M_{\theta}$ in Equation 
\ref{eq:Mthetabackcalculated},

\begin{align*}
    M_{\theta} = \sqrt{2}
    \sqrt{\frac{r {k}_{1} {k}_{2}}{\left(\kappa - 1\right) \left(\tanh{\left(\left(r - r_{max}\right) {k}_{2} \right)} {k}_{1} + \tanh{\left(\left(r_{max} - 1\right) {k}_{2} \right)} {k}_{1} + 1\right) \cosh^{2}{\left(\left(r - r_{max}\right) {k}_{2} \right)}}}
\end{align*} 

Now that the mean flow is defined, the integration method used to obtain the 
speed of sound

% What happens when $r = r_{max}$?

Initially the source terms were defined without mention of the indices of the 
matrices they make up. In other words, there was no fore sight on the fact that
these source terms are sums of the elements within A,B, and X. To investigate 
the source terms in greater detail, the FORTRAN code that calls the source 
terms will output the terms within the source term and then sum them, instead 


of just their sum.
i
$ [A]{x} = \lambda [B] {x} $

which can be rearranged as,

$ [A]{x} - \lambda [B] {x} = 0$

Here, $x$ is an eigenvector composed of the perturbation variables, $v_r,v_{\theta},v_x,p$ and $\lambda$ is the associated eigenvalue, (Note: $\lambda = -i \bar{\gamma}$)

Writing this out we obtain $\cdots$.

Linear System of Equations:
\begin{equation}
    -
    i \left(
        \frac{k}{A} - \frac{m}{r} M_{\theta}
    \right)
    v_r 
    -
    \frac{2}{r} M_{\theta} v_{\theta} 
    +
    \frac{dp}{dr} 
    +
    \frac{(\kappa - 1)}{r} M_{\theta}^2 p
    -
    \lambda M_x v_r =S_1
\end{equation}

Using matrix notation,

\begin{equation}
    A_{11}
    x_1 
    -
    A_{12} x_2 
    +
    A_{14} x_4
    -
    \lambda B_{11} x_1 = S_1
\end{equation}


But $A_{14}$ and $A_{41}$ in Kousen's paper only has the derivative operator.
Since I am currntly writing the matrix out term by term and not doing the matrix 
math to obtain the symbolic expressions, I will define $A_{14}$ with $dp/dr$ 
and $A_{41}$ with $dv_r/dr$
Similarly,
\begin{align}
    A_{21} x_1 &-
    A_{22} x_2 +
    A_{24} x_4 &-
    \lambda B_{22} x_2 &= S_2 \\
    A_{31} x_1 &-
    A_{33} x_3 &-
    \lambda (B_{33} x_3 + B_{34} x_4) &= S_3\\
    A_{41} x_1 &+
    A_{42} x_2 +
    A_{44} x_4 &- 
    \lambda (B_{33} x_3 + B_{44} x_4) &= S_4
\end{align}
Now we can begin looking at the source terms, term by term. They each should also
converge at a known rate







\subsection{Calculation of Observed Order-of-Accuracy}
The numerical scheme used to perform the integration of the tangential velocity
will have a theoretical order-of-accuracy. To find the theoretical 
order-of-accuracy, the discretization error must first be defined. The error, 
$\epsilon$, is a function of id spacing, $\Delta r$

\[ \epsilon = \epsilon(\Delta r) \]

The discretization error in the solution should be proportional to 
$\left( \Delta r \right)^{\alpha}$ where $\alpha > 0$ is the theoretical order
for the computational method.  The error for each grid is expressed as 
\[ \epsilon_{M_{\theta}}(\Delta r) = |M_{\theta,analytic}-M_{\theta,calc}|\]
where $M_{\theta,analytic}$ is the tangential mach number that is defined from the
speed of sound we also defined and the $M_{\theta,calc}$ is the result from 
SWIRL. The $\Delta r$ is to indicate that this is a discretization error for a
specific grid spacing. Applying the same concept to to the speed of sound,

If we define this error on various grid sizes and compute $\epsilon$ for
each grid, the observed order of accuracy can be estimated and compared to
the theoretical order of accuracy. For instance, if the numerical soution
is second-order accurate and the error is converging to a value, the L2 norm of
the error will decrease by a factor of 4 for every halving of the grid cell 
size. 

Since the input variables should remain unchanged (except from minor changes 
from the Akima interpolation), the error for the axial and tangential mach 
number should be zero. As for the speed of sound, since we are using an analytic
expression for the tangential mach number, we know what the theoretical result
would be from the numerical integration technique as shown above. 
Similarly we define the discretization error for the speed of sound.

\[ \epsilon_{A}(\Delta r) = |A_{analytic}-A_{calc}|\]

For a perfect answer, we expect $\epsilon$ to be zero. Since a Taylor series can 
be used to derive the numerical schemes, we know that the truncation of higher
order terms is what indicates the error we expect from using a scheme that 
is constructed with such truncated Taylor series.

The error at each grid point $j$ is expected to satisfy the following,

\begin{align*}
    0 &= |A_{analytic}(r_j) - A_{calc}(r_j)| \\
    \widetilde{A}_{analytic}(r_j) &= \widetilde{A}_{calc}(r_j) +
    (\Delta r)^{\alpha} \beta(r_j)  + H.O.T
\end{align*}

where the value of $\beta(r_j)$ does not change with grid spacing, and 
$\alpha$ is the asymptotic order of accuracy of the method. It is important to
note that the numerical method recovers the original equations as the grid 
spacing approached zero.  It is important to note that $\beta$ represents the
first derivative of the Taylor Series.  Subtracting $A_{analytic}$ from both
sides gives,

\begin{align*}
    A_{calc}(r_j) - A_{analytic}(r_j) &= A_{analytic}(r_j) - A_{analytic}(r_j)
    + \beta(r_j) (\Delta r)^{\alpha} \\
    \epsilon_A(r_j)(\Delta r) &= \beta(r_j) (\Delta r)^{\alpha}
\end{align*}

To estimate the order of accuracy of the accuracy, we define the global errors 
by calculating the L2 Norm of the error which is denoted as $\hat{\epsilon}_A$ 

\begin{align*}
    \hat{\epsilon}_A = \sqrt{\frac{1}{N}\sum_{j=1}^{N} \epsilon(r_j)^2  }
\end{align*}

\begin{align*}
    \hat{\beta}_A(r_j) = \sqrt{\frac{1}{N}\sum_{j=1}^{N} \beta(r_j)^2  }
\end{align*}

As the grid density increases, $\hat{\beta}$ should asymptote to a constant 
value. Given two grid densities, $\Delta r$ and $\sigma\Delta r$, and assuming
that the leading error term is much larger than any other error term,

\begin{align*}
    \hat{\epsilon}_{grid 1} &= \hat{\epsilon}(\Delta r) = \hat{\beta}(\Delta r)^{\alpha} \\
    \hat{\epsilon}_{grid 2} &= \hat{\epsilon}(\sigma \Delta r) = \hat{\beta}(\sigma \Delta r)^{\alpha} \\
                            &= \hat{\beta}(\Delta r)^{\alpha} \sigma^{\alpha}
\end{align*}

The ratio of two errors is given by,

\begin{align*}
    \frac{\hat{\epsilon}_{grid 2}}{\hat{\epsilon}_{grid 1}} &= 
    \frac{\hat{\beta}(\Delta r )^{\alpha}}{\hat{\beta}(\Delta r )^{\alpha}} \sigma^{\alpha} \\ &= \sigma^{\alpha}
\end{align*}

Thus, $\alpha$,the asymptotic rate of convergence is computed as follows 

\begin{align*}
    \alpha = \frac{
        \ln \frac{
            \hat{\epsilon}_{grid 2}
    }{\hat{\epsilon}_{grid 1} }}
    {\ln\left( \sigma \right) }
\end{align*}

Defining  for a doubling of grid points ,
%
%\begin{align*}
%    \alpha = \frac{\ln \left( \hat{\epsilon}\left( \frac{1}{2}\Delta  r\right)
%            \right) -\ln \left( \hat{\epsilon}\left( \Delta  r\right)
%    \right)}{\ln \left( \frac{1}{2} \right)}
%\end{align*}
%
Similarly for the eigenvalue problem, 

\[ [A]x = \lambda [B]x \]


\section{Fairing Functions}
\input{Chapter-4-Theory-Numerical-Models/Theory-Fairing-Functions/intro}

\section{Setting Boundary Condition Values Using a Fairing Function}

\subsection{Using $\beta$ as a scaling parameter}

Defining the nondimensional radius in the same way that SWIRL does:

\begin{align*}
    \widetilde{r} = \frac{r}{r_T}
\end{align*}
where $r_T$ is the outer radius of the annulus.

The hub-to-tip ratio is defined as:

\begin{align*}
    \sigma = \frac{r_H}{r_T}
     &= \widetilde{r}_H
\end{align*}
where $\widetilde{r}_H$ is the inner radius of the annular duct. The hub-to-tip
ratio can also be zero indicating the duct is hollow.

A useful and similar parameter is introduced, $\beta$, where $ 0 \leq \beta \leq 1$

\begin{align*}
    \beta &=
    \frac{r - r_H}{r_T - r_H}
\end{align*}
Dividing By $r_T$
\begin{align*}
    \beta &= 
    \frac{
        \frac{r}{r_T} - \frac{r_H}{r_T}
}{
        \frac{r_T}{r_T} - \frac{r_H}{r_T}
}\\
&= \frac{
    \widetilde{r} - \widetilde{r}_H
}{
1 - \sigma
}
\end{align*}

Suppose a manufactured solution $f_{MS}$ with boundaries $f_{MS}(r= \sigma)$ 
and $f_{MS}(\widetilde{r}= 1)$ is the specified analytical solution. The goal
is to change the boundary conditions of the manufactured solution in such  way 
that allows us to adequately check the boundary conditions imposed on SWIRL.
Defining the manufactured solution, $f_{MS}(\widetilde{r})$,   where
$\sigma \leq \widetilde{r} \leq 1$ and there are desired values of $f$ at the 
boundaries desired values are going to be denoted as $f_{minBC}$ and $f_{maxBC}$.
The desired changes in $f$ are defined as:

\begin{align*}
    \Delta f_{minBC} = f_{minBC} - f_{MS}(\widetilde{r} = \sigma)\\
    \Delta f_{maxBC} = f_{maxBC} - f_{MS}(\widetilde{r} = 1) 
\end{align*}
We'd like to impose these changes smoothly on the manufactured solution function.
To do this,the fairing functions, $A_{min}(\widetilde{r})$ and $A_{max}(\widetilde{r})$
where:
\begin{align*}
    f_{BCsImposed}(\widetilde{r}) = f_{MS}(\widetilde{r}) +
    A_{min}(\widetilde{r}) \Delta f_{minBC}  +  
    A_{max}(\widetilde{r}) \Delta f_{maxBC}  
\end{align*}
Then, in order to set the condition at the appropriate boundary, the following 
conditions are set,


\begin{align*}
    A_{min}(\widetilde{r} = \sigma) &= 1\\
    A_{min}(\widetilde{r} = 1) &= 0 \\
    A_{max}(\widetilde{r} = 1) &= 1 \\
    A_{max}(\widetilde{r} = \sigma) &= 0 
\end{align*}
If $A_{min}(\widetilde{r})$ is defined as a function of $A_{max}(\widetilde{r})$
then only $A_{max}(\widetilde{r})$ needs to be defined, therefore 
\begin{align*}
    A_{min}(\widetilde{r}) = 1 - A_{max}(\widetilde{r}) 
\end{align*}

It is also desirable to set the derivatives for the fairing function at the 
boundaries incase there are boundary conditions imposed on the derivatives of 
the fairing function.

\begin{align*}
    \frac{\partial A_{max}}{\partial \widetilde{r}}|_{\widetilde{r}= \sigma} &= 0\\
    \frac{\partial A_{max}}{\partial \widetilde{r}}|_{\widetilde{r}= 1} &= 0    
\end{align*}

\begin{align*}
    \frac{\partial A_{min}}{\partial \widetilde{r}}|_{\widetilde{r}= \sigma} &= 0\\
    \frac{\partial A_{min}}{\partial \widetilde{r}}|_{\widetilde{r}= 1} &= 0    
\end{align*}


\subsection{Minimum Boundary Fairing Function}

Looking at $A_{min}$ first, the polynomial is:

\begin{align*}
    A_{min} \left( \beta \right) &= 
    a + b \beta + c \beta^2 + d \beta^3                   \\
    A_{min} \left( \widetilde{r} \right) &= 
    a + b \left( \frac{\widetilde{r} - \sigma}{1 - \sigma} \right)+
    c\left( \frac{\widetilde{r} - \sigma}{1 - \sigma} \right)  ^2+
    d\left( \frac{\widetilde{r} - \sigma}{1 - \sigma} \right)^3                    \\
\end{align*}

Taking the derivative,
\begin{align*}
    A'_{min} \left( \widetilde{r} \right) &= 
    b \left( \frac{1}{1 - \sigma} \right)+
    2 c\left( \frac{1}{1 - \sigma} \right)\left( \frac{\widetilde{r} - \sigma}{1 - \sigma} \right)  +
    3 d\left( \frac{1}{1-\sigma} \right)\left( \frac{\widetilde{r} - \sigma}{1 - \sigma} \right)^2\\
    A'_{min} \left( \beta \right) &= 
    \left( \frac{1}{1 - \sigma} \right)
    \left[
    b +
    2 c \beta + 
    3 d \beta^2
    \right]
\end{align*}
Now we will use the conditons mentioned earlier as constraints to this system 
of equations
Using the possible values of $\widetilde{r}$,

\begin{align*}
    A_{min}(\sigma) &= a &= 1 \\
    A_{min}(1) &= a + b + c + d  &= 0 \\
    A'_{min}(\sigma) &=  b    &= 0 \\
    A'_{min}(1) &=  b + 2 c + 3 d    &= 0 \\
\end{align*}


which has the solution,

\begin{align*}
    a &= 1 \\
    b &= 0 \\
    c &= -3 \\
    d &= 2 
\end{align*}

giving the polynomial as: 

\begin{align*}
    A_{min} (\widetilde{r}) = 1 - 3 \left(  \frac{\widetilde{r} - \sigma }{ 1 - \sigma}\right)^2 +
    2 \left( \frac{\widetilde{r} - \sigma}{1 - \sigma} \right)^3 
\end{align*}


\subsection{Max boundary polynomial}
Following the same procedure for $A_{max}$ gives 
\begin{align*}
    A_{min} (\widetilde{r}) = 3 \left(  \frac{\widetilde{r} - \sigma }{ 1 - \sigma}\right)^2 -
    2 \left( \frac{\widetilde{r} - \sigma}{1 - \sigma} \right)^3
\end{align*}
\subsection{Corrected function} 

The corrected function is then, 

\begin{align*}
    f_{BCsImposed} (\widetilde{r}) &= 
    f_{MS}(\widetilde{r}) &+ A_{min} \Delta f_{minBC} + A_{max} \Delta f_{maxBC} \\
                                   &= 
    f_{MS}(\widetilde{r}) &+
    \left(
        1 - 3 \left(  \frac{\widetilde{r} - \sigma }{ 1 - \sigma}\right)^2 +
    2 \left( \frac{\widetilde{r} - \sigma}{1 - \sigma} \right)^3 
\right)
    \left[ \Delta f_{minBC} \right]\\ 
           & &+
    \left(
         3 \left(  \frac{\widetilde{r} - \sigma }{ 1 - \sigma}\right)^2- 
    2 \left( \frac{\widetilde{r} - \sigma}{1 - \sigma} \right)^3 
\right)
    \left[ \Delta f_{maxBC} \right] \\ 
              f_{BCsImposed} (\beta) &= f_{MS}(\beta) &+ \Delta f_{minBC} + \left(  3 \beta^2 - 2 \beta^3  \right)
    \left[  \Delta f_{maxBC} - \Delta f_{minBC} \right]
\end{align*}
Note that we're carrying the correction throughout the domain, as opposed to 
limiting the correction at a certain distance away from the boundary. The 
application of this correction ensures that there is no discontinuous derivatives
inside the domain; as suggested in Roach's MMS guidelines (insert ref) 


What is meant by ``just because $A_{min}$ and its first derivative go to zero
doesn't mean that the second derivatives''



\subsection{Symbolic Sanity Checks}
We want to ensure that $f_{BCsImposed}$ has the desired boundary conditions, 
$f_{minBC/maxBC}$ instead of the original boundary values that come along
for the ride in the manufactured solutions, $f_{MS}(\widetilde{r}=\sigma /1)$. 
In another iteration of this method, we will be changing the derivative values,
so let's check the values of $\frac{\partial f_{BCsImposed}}{\partial \widetilde{r}}$ 
to make sure those aren't effected unintentionally.

\subsubsection*{Symbolic Sanity Check 1}
The modified manufactured solution, $f_{BCsImposed}$ with the fairing functions
$A_{min}$ and $A_{max}$ substituted in is,
\begin{align*}
    f_{BCsImposed}(\widetilde{r}) =
    \left(
        3 \left(  \frac{\widetilde{r} - \sigma }{ 1 - \sigma}\right)^2- 
        2 \left( \frac{\widetilde{r} - \sigma}{1 - \sigma} \right)^3 
    \right)
    \left[ \Delta f_{maxBC} \right] .
\end{align*} 
Further simplification yields,
\begin{align*}
    f_{BCsImposed}(\widetilde{r} = \sigma) 
    &=
    \left(
        f_{MS}(\widetilde{r} = \sigma) +
        \Delta f_{minBC} +
        \left( 3\left(  \frac{\sigma - \sigma}{1 - \sigma} \right)^2- 
          2\left(  \frac{\sigma - \sigma}{1 - \sigma} \right)^3- 
        \right)
        \left[ \Delta f_{maxBC} - \Delta f_{minBC}  \right] 
    \right)\\
    &=  f_{MS}(\widetilde{r} = \sigma) + \Delta f_{minBC}\\
    &=  f_{MS}(\widetilde{r} = \sigma) + (f_{minBC} - f_{MS}(\widetilde{r} = \sigma)) \\
    &=  f_{minBC}
\end{align*} 


\begin{align*}
    f_{BCsImposed}(\widetilde{r} = 1 ) 
    &=
    \left(
        f_{MS}(\widetilde{r} = 1) +
        \Delta f_{minBC} +
        \left( 3\left(  \frac{1 - \sigma}{1 - \sigma} \right)^2- 
          2\left(  \frac{1 - \sigma}{1 - \sigma} \right)^3- 
        \right)
        \left[ \Delta f_{maxBC} - \Delta f_{minBC}  \right] 
    \right)\\
    &=  f_{MS}(\widetilde{r} = 1) + \Delta f_{maxBC}\\
    &=  f_{MS}(\widetilde{r} = 1) + (f_{maxBC} - f_{MS}(\widetilde{r} = 1)) \\
    &=  f_{maxBC}
\end{align*} 


\begin{align*}
    \frac{\partial}{\partial \widetilde{r}}\left(  f_{BCsImposed}(\widetilde{r}) =
    \left(
        3 \left(  \frac{\widetilde{r} - \sigma }{ 1 - \sigma}\right)^2- 
        2 \left( \frac{\widetilde{r} - \sigma}{1 - \sigma} \right)^3 
    \right)
    \left[ \Delta f_{maxBC} \right]\right) \\
    \frac{\partial f_{MS}}{\partial \widetilde{r}} + 
    \left( \frac{6}{1-\sigma} \right)
    \left( 
        \left( \frac{\widetilde{r} - \sigma}{1 - \sigma} \right) -
    \left( \frac{r - \sigma}{1 - \sigma} \right)^2 \right)
    \left( \Delta f_{maxBC} - \Delta f_{minBC} \right)
\end{align*} 


At $\widetilde{r} = \sigma$, the derivative is: 

\begin{align*}
    \frac{\partial f_{MS}}{\partial \widetilde{r}}|_{\sigma} \\
    \frac{\partial f_{MS}}{\partial \widetilde{r}}|_{1} 
\end{align*}


\subsection{Min boundary derivative polynomial}

The polynomial is of the form: 

\begin{align*}
    B_{min} \left( \beta \right) &= 
    a + b \beta + c \beta^2 + d \beta^3                   \\
    B_{min} \left( \widetilde{r} \right) &= 
    a + b \left( \frac{\widetilde{r} - \sigma}{1 - \sigma} \right)+
    c\left( \frac{\widetilde{r} - \sigma}{1 - \sigma} \right)  ^2+
    d\left( \frac{\widetilde{r} - \sigma}{1 - \sigma} \right)^3                    \\
\end{align*}
Taking the derivative,
\begin{align*}
    B'_{min} \left( \widetilde{r} \right) &= 
    b \left( \frac{1}{1 - \sigma} \right)+
    2 c\left( \frac{1}{1 - \sigma} \right)\left( \frac{\widetilde{r} - \sigma}{1 - \sigma} \right)  +
    3 d\left( \frac{1}{1-\sigma} \right)\left( \frac{\widetilde{r} - \sigma}{1 - \sigma} \right)^2\\
    B'_{min} \left( \beta \right) &= 
    \left( \frac{1}{1 - \sigma} \right)
    \left[
    b +
    2 c \beta + 
    3 d \beta^2
    \right]
\end{align*}


Applying the four constraints gives:
\begin{align*}
    a &= 0\\
    b &= \left( 1 - \sigma \right) \\
    a + b + c + d &= 0\\
    2 + 2c + 3d &= 0
\end{align*}
\begin{align*}
    c + d &= -b  \\
    2c + 3d &= -b
\end{align*}
\begin{align*}
    c &= -2b \\
    d &= b
\end{align*}

and the min boundary derivative polynomial is: 
\begin{align*}
    B_{min}\left( \widetilde{r} \right) &= 
    b \left( \frac{\widetilde{r} - \sigma }{1 - \sigma}\right) -
    2b\left( \frac{\widetilde{r} - \sigma }{1 - \sigma}\right) ^2 +
    b \left( \frac{\widetilde{r} - \sigma }{1 - \sigma}\right)^3 \\
    &=  \left( 1 - \sigma \right)
    \left( \left( \frac{\widetilde{r} - \sigma }{1 - \sigma}\right)  - 
    2\left( \frac{\widetilde{r} - \sigma }{1 - \sigma}\right)^2 +
    \left( \frac{\widetilde{r} - \sigma }{1 - \sigma}\right)^3\right)
\end{align*} 
\subsection{Polynomial function, max boundary derivative}
The polynomial is of the form:

The polynomial is of the form: 

\begin{align*}
    B_{max} \left( \beta \right) &= 
    a + b \beta + c \beta^2 + d \beta^3                   \\
    B_{max} \left( \widetilde{r} \right) &= 
    a + b \left( \frac{\widetilde{r} - \sigma}{1 - \sigma} \right)+
    c\left( \frac{\widetilde{r} - \sigma}{1 - \sigma} \right)  ^2+
    d\left( \frac{\widetilde{r} - \sigma}{1 - \sigma} \right)^3                    \\
\end{align*}
which has the derivative,


\begin{align*}
    B'_{max} \left( \widetilde{r} \right) &= 
    b \left( \frac{1}{1 - \sigma} \right)+
    2 c\left( \frac{1}{1 - \sigma} \right)\left( \frac{\widetilde{r} - \sigma}{1 - \sigma} \right)  +
    3 d\left( \frac{1}{1-\sigma} \right)\left( \frac{\widetilde{r} - \sigma}{1 - \sigma} \right)^2\\
    B'_{max} \left( \beta \right) &= 
    \left( \frac{1}{1 - \sigma} \right)
    \left[
    b +
    2 c \beta + 
    3 d \beta^2
    \right]
\end{align*}
Applying the four constraints gives:
\begin{align*}
    a &= 0 \\
    b &= 0 \\
    a + b + c + d &= 0 \\
    b + 2c + 3d &= (1-\sigma)
\end{align*} 

working this out: 

\begin{align*}
    c + d &=  0 \\
    2c + 3d &= (1 - \sigma)
\end{align*} 

gives 

\begin{align*}
    c &= -\left( 1 - \sigma\right)
    d &= \left( 1 - \sigma\right)
\end{align*}
and the max boundary derivative polynomial is:

\begin{align*}
    B_{max} \left( \widetilde{r} \right) &= 
    \left( 1 - \sigma \right) \left( 
        - \left( \frac{\widetilde{r}-\sigma}{1 - \sigma} \right)^2 +
        \left( \frac{\widetilde{r}-\sigma}{1-\sigma} \right)^3
    \right)
\end{align*}
\subsection{Putting it together}

The corrected function is then: 
\begin{align*}
    f_{BCsImposed} \left( \widetilde{r} \right) &= 
    f_{MS} + 
    B_{min}\left( \widetilde{r} \right) \Delta f'_{minBC} +
    B_{max}\left( \widetilde{r} \right) \Delta f'_{maxBC}
\end{align*}
\begin{align*}
    &= 
    f_{MS} + \\
    \left( 1 - \sigma \right) \left( 
        \left( \frac{\widetilde{r} - \sigma}{1 - \sigma} \right) 
        - \left( \frac{\widetilde{r} - \sigma}{1 - \sigma} \right) ^2
\right) \Delta f'_{minBC} + \\
\left( 1 - \sigma \right) \left( - \left( \frac{\widetilde{r} - \sigma}{1 - \sigma} \right)^2 + 
    \left( \frac{\widetilde{r} - \sigma}{1 - \sigma}  \right)^3
\right)
\left( \Delta f'_{minBC} + \Delta f'_{maxBC} \right) 
\end{align*}

\chapter{Results and Discusssion}
\section{Verification}
\subsection{Method of Manufactured Solution Results}
\section{Validation}
\subsection{Comparison to Kousen}
\subsection{Comparison to Maldanado}

%--------+----------------------------------------------------------+
%        |  \appendix                                  (IF NEEDED)  |
%        |                                                          |
%        |  See section 3.19 of "Read_Me_First_(v12).pdf"           |
%        |                                                          |
%        |  1) When you include the \appendix command a subsequent  |
%        |     \chapter{} command will not generate a chapter but   |
%        |     an appendix section.                                 |
%        |                                                          |
%        |  2) As is illustrated below, to generate a second or     |
%        |     third appendix you simply have to include            |
%        |     additional \chapter{} commands (i.e., you DO NOT     |
%        |     have to repeat the use of the \appendix command).    |
%        +----------------------------------------------------------+
\bibliography{references}
\bibliographystyle{unsrt}
\appendix
\chapter{Theory Appendix}

\chapter{Chapter 3 Appendix}
\section{Appendix A: Speed of Sound}

Sound wave is a pressure disturbance that moves with at a speed $a$

\begin{figure}[h!]
	\centering
	\includegraphics[width=0.3\linewidth]{screenshot001.png}
	\caption{}
	\label{fig:screenshot001}
\end{figure}

By applying a rectangular control volume around this pressure wave, we can apply our conservation equations. We are assuming that these properities are increasing by a small increment. This is why each variable is added by a infinitesimally small term.

Recalling the conservation of mass (continuity equation), $\dot{m} = constant$

\[\dot{m}_{left} = \dot{m}_{right}\]

Recalling he definition of density, $\rho = m/\bar{V}$ and rewriting $\bar{V} = uA$ (check the units)
\[\rho a \cancel{A}  = (\rho + d\rho)(a + da) \cancel{A}\]
Futher expanding gives,
\[\rho a   = (\rho a+ \rho da + a d\rho + da d\rho)\]

We say that $da d\rho$ is so small, we can assume it is zero. This is often referred to as "neglecting higher order terms (H.O.T)". The expression then becomes

\[\frac{da}{a} = -\frac{d \rho}{\rho}\]

For the momentum equation $P + \rho u^2 = constant$
\[P + \rho a^2  = P + dP +  (\rho + d\rho)(a + da)(a + da) \]

But we just said that $\rho a = (\rho + d\rho)(a + da)$


\[P + \rho a^2  = P + dP +  \rho a(a + da) \]
\[dP + \rho ada = a\]

Multiplying the second term by a and divide by a, this is essentially multiplying the second term by one.

\[dP + \rho a^2\frac{da}{a} = a\]

recalling the relation $\frac{da}{a} = -\frac{d \rho}{\rho}$

\[dp - a^2 d \rho = 0\]

                       \[a^2 = \frac{dp}{d\rho}\]

Since a sound wave is a very weak wave, when it travels through a medium, it only increases the pressure and density, etc. slightly. The effect of this is that friction  and heat transfer can be neglected. Since friction cannot be undone, we call this an irreversible process. Whenever there is no transfer of heat, it is called this adiabatic. Thus, the propagation of sound is an adiabatic, reversible process, otherwise called isentropic. Isentropic implies no increase in entropy, which is \textit{not} true in the presence of shock waves.

In the case of a thermally perfect gas, we can say $P = \rho R T$

For a callorically perfect gas we can say $pv^{\gamma} = constant$, where $v$ is volume per unit mass, or specific volume

Differentiating and recalling that $v = 1/\rho$

\[a = \sqrt{\frac{\gamma P}{\rho}} = \sqrt{\gamma R T}\]


\[dm = \left( \rho u\right)_2 - \left(\rho u\right)_1\]
\[D(mV) = \left(\rho u^2 + P \right)_2\]

For Steady flow,

\[\cancel{dm} = \left( \rho u\right)_2 - \left(\rho u\right)_1\]

\[ \left( \rho u\right)_1 = \left(\rho u\right)_2\]

Similarly, for the Momentum equation,

\[\left(\rho u^2 + P\right)_1 = \left(\rho u^2 + P\right)_2\]

Let us change the coordinate system motion for the traveling wave be independent of time, and thus corresponds to \textit{steady state} wave propagation. 
\[ u_1 = \bar{u} + a - \frac{1}{2} \partial u \]
\[ u_2 = \bar{u} + a + \frac{1}{2} \partial u \]


where, $\bar{u}$ is the average flow velocity and $a$ is the wave speed.


Substituting this back into the conservation of mass

\[\left(\rho u\right)_1 = \left(\rho u\right)_2 \]

\[ 
\left( \rho    - \frac{1}{2}\partial \rho \right) 
\left( \bar{u} - a - \frac{1}{2}\partial u\right) = 
\left( \rho    + \frac{1}{2}\partial \rho \right) 
\left( \bar{u} - a + \frac{1}{2}\partial u\right) 
\]

Further expanding

\[
\cancel{
	\rho \bar{u} - \rho a
} + 
\frac{1}{2} 
\left(
-\rho \partial u - \bar{u} - \bar{u} \partial \rho + a \partial \rho 
\right) +
\cancel{
	\frac{1}{4}\partial \rho \partial u 
}
= 
\cancel{
	\rho \bar{u} - \rho a
} + 
\frac{1}{2} 
\left(
\rho \partial u + \bar{u} - \bar{u} \partial \rho - a \partial \rho 
\right) +
\cancel{
	\frac{1}{4}\partial \rho \partial u 
}
\]
 
\[
  \frac{1}{2}\left( - \rho \partial u - \bar{u} \partial \rho + a \partial \rho\right) = 
  \frac{1}{2}\left(   \rho \partial u + \bar{u} \partial \rho - a \partial \rho\right)
\]

\[
\rho \partial u + u \partial \rho - a \partial \rho = 0 
\]

\[
\rho \partial u + \left( u - a\right)\partial \rho = 0
\]

Momentum Equation

\[
\left(\rho u^2 + P\right)_1 = 
\left(\rho u^2 + P\right)_2\]




\section{Appendix B: Isentropic Waves}

\[dU = dW + dQ\]

For adiabatic, reversible processes, the work done by a system with constant pressure and a change in volume is $-pdV$ and the change in heat energy is zero. Hence,

\[dU = -pdV\].

The change in enthalpy of such a system can be found by taking the derivative of its expression for a thermodynamic process 

\[H = U + pV\]

\[dH = dU + pdV + vdP\]

\[dH = -pdV + pdV + vdP\]

\[dH = vdP\]

The specific heats at constant pressure and constant volume 

\[\left(\frac{\partial U}{\partial T}\right)_v = C_v\]

\[\left(\frac{\partial H}{\partial T}\right)_p = C_p\]

\[\gamma = \frac{C_p}{C_v} = \frac{dH}{dU} = -\frac{VdP}{pdV} = -\frac{V}{dV}\frac{dP}{p}\]

Integrating both sides

\[\gamma \frac{dV}{V} = \frac{-dP}{P} \rightarrow \gamma \int \frac{1}{V} dV = - \int \frac{1}{P} dP\]

\[\gamma ln(V) + ln(P) = C\]

Using log rules

\[ln(V^\gamma) + ln(P) = C\]
 \[ln(pV^\gamma) = C \]
 \[pV^\gamma = e^C = C\]
 
 \[\frac{p}{\rho^\gamma}\]



%--------+----------------------------------------------------------+
%        |  \end{document}                              (REQUIRED)  |
%        |                                                          |
%        |  Details (if you can call them "details") are provided   |
%        |  in section 3.20 of "Read_Me_First_(v12).pdf"            |
%        +----------------------------------------------------------+

\end{document} %---> ---> ---> --->   DO NOT ALTER THIS COMMAND

%XXXXXXXXXXXXXXXXXXXXXXXXXXXXXXXXXXXXXXXXXXXXXXXXXXXXXXXXXXXXXXXXXXXX
%XXXXXXXXXXXXXXXXXXXXXXXXXXXXXXXXXXXXXXXXXXXXXXXXXXXXXXXXXXXXXXXXXXXX
%XXXXXXXXXXXXXXXXXXXXXXXXXXXXXXXXXXXXXXXXXXXXXXXXXXXXXXXXXXXXXXXXXXXX

