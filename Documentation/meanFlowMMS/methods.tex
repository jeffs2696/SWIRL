\section{Methods}

SWIRL is a linearized Euler equations of motion code that calculates the 
axial wavenumber and radial mode shapes from small unsteady disturbances in a mean flow. 
The mean flow varies along the axial and tangetial directons as a function of 
radius. The flow domain can either be a circular or annular duct, with or without
acoustic liner. SWIRL was originally written by Kousen [insert ref].

The SWIRL code requires two mean flow parameters as a function of radius, $M_x$
, and $M_{\theta}$. Afterwards, the speed of sound, $\widetilde{A}$ is calculated by 
integrating $M_{\theta}$ with respect to r. To verify that SWIRL is handling 
and returning the accompanying mean flow parameters, the error between the 
mean flow input and output variables are computed. Since the trapezoidal rule
is used to numerically integrate $M_{\theta}$, the discretization error and 
order of accuracy is computed. Since finite differencing schemes are to be used 
on the result of this integration, it is crucial to accompany the integration 
with methods of equal or less order of accuracy. This will be determined by 
applying another MMS on the eigenproblem which will also have an order of 
accuracy.

