
\subsection{Theory}

To relate the speed of sound to a given flow, the radial momentum equation
is used.  If the flow contains a swirling component,then the primitive variables 
are nonuniform through the flow, and mean flow assumptions are not valid. 
\begin{align*}
    \frac{\partial v_r}{\partial t} +
    v_r \frac{\partial v_r}{\partial r} + 
    \frac{v_{\theta}}{r} 
    \frac{\partial v_{\theta}}{\partial \theta} -
    \frac{v_{\theta}^2}{r} 
    v_x \frac{\partial v_r}{\partial x} =
    \frac{1}{\rho}\frac{\partial P}{\partial r}
\end{align*}

 To account to for this, the radial momentum is simplified by assuming the 
 flow is steady, the flow has no radial component. In addition, the viscous
 and body forces are neglected.  Then the radial pressure derivative term is
 set equal to the dynamic pressure term. Seperation of variables is applied.  

 \begin{align*}
     \frac{v_{\theta}^2}{r} = \frac{1}{\rho}\frac{\partial P}{\partial r} \\
 P = \int_{r}^{r_{max}} \frac{\rho V_{\theta}^2}{  r}
 \end{align*}

To show the work, we will start with the dimensional form of the equation and
differentiate both sides.  Applying separation of variables,
 \[
     \int_{r}^{r_{max}}
     \frac{\bar{\rho} v_{\theta}^2}{r} \partial r 
     =-\int_{P(r)}^{P(r_{max})}\partial p.
 \]

Since $\tilde{r} = r/r_{max}$,
\[r = \tilde{r}r_{max}.\]
Taking total derivatives (i.e. applying chain rule),
\[dr = d(\tilde{r}r_{max}) = d(\tilde{r})r_{max}, \]
Substituting these back in and evaluating the right hand side,
\[
    \int_{\tilde{r}}^{1} \frac{\bar{\rho} v_{\theta}^2}{\tilde{r}}\partial \tilde{r} 
    =P(1)-P(\tilde{r})
\]

For reference the minimum value of $\tilde{r}$ is,

\[\sigma = \frac{r_{max}}{r_{min}}\]

For the radial derivative, the definition of the speed of sound is utilized,
\[\frac{\partial A^2}{\partial r } =
\frac{\partial}{\partial r} \left( \frac{\gamma P}{\rho} \right).\]

Using the quotient rule, the definition of the speed of sound is extracted,

\begin{align*}
&= \frac{\partial P}{\partial r} \frac{\gamma \bar{\rho}}{\bar{\rho}^2} -
\left(
    \frac{\gamma P}{\bar{\rho}^2} 
\right) 
\frac{\partial \bar{\rho}}{\partial r}\\
&=  \frac{\partial P}{\partial r} \frac{\gamma }{\bar{\rho}} -
\left( \frac{A^2}{\bar{\rho}} \right) 
\frac{\partial \bar{\rho} }{\partial r}
\end{align*}

Using isentropic condition $ \partial P/A^2 = \partial \rho$, 

\begin{align*}
&= \frac{\partial P}{\partial r} \frac{\gamma }{\bar{\rho}} -
\left( \frac{1}{\bar{\rho}} \right) \frac{\partial  P }{\partial r}\\
\frac{\partial A^2}{\partial r} 
&= \frac{\partial P}{\partial r} \frac{\gamma - 1}{\bar{\rho}}  
\end{align*}

\begin{align*}
    \frac{\bar{\rho}}{\gamma -1}\frac{\partial A^2}{\partial r} &= \frac{\partial P}{\partial r} 
\end{align*}


Going back to the radial momentum equation, and rearranging the terms will simplify 
the expression. The following terms are defined to start the
nondimensionalization.  

\begin{align*}
    M_{\theta} &= \frac{V_{\theta}}{A} \\ 
    \widetilde{r} &= \frac{r}{r_{max}}  \\
    \widetilde{A} &= \frac{A}{A_{r,max}}  \\
    A &= \widetilde{A}{A_{r,max}} \\
    r &= \widetilde{r}{r_{max}}\\
    \frac{\partial}{\partial r} &=
    \frac{\partial \widetilde{r}}{\partial r} \frac{\partial}{\partial \widetilde{r}}\\
                                &= \frac{1}{r_{max}}\frac{\partial}{\partial \widetilde{r}}
\end{align*}
Dividing by $A$,
\begin{align*}
    \frac{M_{\theta}^2}{r}\left(\gamma - 1\right) 
 &= \frac{\partial A^2}{\partial r} \frac{1}{A^2}
\end{align*}

Now there is two options, either find the derivative of  $\bar{A}$ or the integral of
$M_{\theta}$ with respect to r.
\begin{enumerate}
    \item
%\begin{align*}
%\text{Integrating both sides } 
%\int_{r}^{r_{max}}\frac{M_{\theta}}{r}\left(\gamma - 1\right){\partial r}  &=\int_{A^2(r)}^{A^2(r_{max})}\frac{1}{A^2}  {\partial A^2}\\
%\int_{r}^{r_{max}}\frac{M^2_{\theta}}{r}\left(\gamma - 1\right){\partial r}  &=ln(A^2(r_{max})) - ln(A^2(r)) \\
%\int_{r}^{r_{max}}\frac{M^2_{\theta}}{r}\left(\gamma - 1\right){\partial r}  &=ln\left(\frac{A^2(r_{max})}{A^2(r)}\right) 
%\end{align*}
%
Defining non dimensional speed of sound $\tilde{A} = \frac{A(r)}{A(r_{max})}$
\begin{align*}
\int_{r}^{r_{max}}\frac{M_{\theta}}{r}\left(\gamma - 1\right){\partial r}  &=ln\left(\frac{1}{\tilde{A}^2}\right) \\
&= -2ln(\tilde{A})\\
\tilde{A}(r) &= exp\left[-\int_{r}^{r_{max}}\frac{M_{\theta}}{r}\frac{\left(\gamma - 1\right)}{2}{\partial r}\right] \\ \text{replacing r with }\tilde{r} \rightarrow \tilde{A}(r) &= exp\left[-\int_{r}^{r_{max}}\frac{M_{\theta}}{r}\frac{\left(\gamma - 1\right)}{2}{\partial r}\right]		\\
\tilde{A}(\tilde{r}) &= exp\left[\left(\frac{1 - \gamma}{2}\right)\int_{\tilde{r}}^{1}\frac{M_{\theta}}{\tilde{r}}{\partial \tilde{r}}\right]	
\end{align*}
\item Or we can differentiate
\end{enumerate}
Solving for $M_{\theta}$ ,
\begin{align*}
M_{\theta}^2 
&= \frac{\partial A^2}{\partial r} \frac{r}{A^2 \left(\gamma - 1\right)}
\end{align*}
Nondimensionalizing and substituting, 

\begin{align} 
    M_{\theta}^2
    \frac{\left( \gamma - 1 \right)}{\widetilde{r} r_{max}} &=
    \frac{1}{(\widetilde{A}A_{r,max})^2}\frac{A_{r,max}^2}{r_{max}}
    \frac{\partial \widetilde{A}^2}{\partial \widetilde{r}} \nonumber \\
    M_{\theta}^2     \frac{\left( \gamma - 1 \right)}{\widetilde{r} } &=
    \frac{1}{\widetilde{A}^2}
    \frac{\partial \widetilde{A}^2}{\partial \widetilde{r}} \nonumber \\
    M_{\theta} &= \sqrt{\frac{\widetilde{r}}{(\gamma-1) \widetilde{A}^2}
        \frac{\partial\widetilde{A}^2}{\partial \widetilde{r} }
    } \label{eq:Mthetabackcalculated}
\end{align}

% 3.1 Guidelines for Creating Manufactured Solutions states:
% \begin{enumerate}
%     \item  The manufactured solutions should be composed of smooth analytic 
%         functions 
%     \item The manufactured solutions should exercize every term in the governing
%         equation that is being tested,
%     \item The solution should have non trivial derivatives.  
%     \item The solution derivatives should be bounded by a small constant. In this case
%         this constant should prevent the function from becoming greater than 
%         one.
%     \item The solution should not prevent the code from running 
%     \item The solution should be defined on a connected subset of two- or three-
%         dimensional space to allow flexibility in chosing the domain of the PDE.
%         Section 3.3.1 provides more information about this.
%     \item The solution should coincide with the differential operators of the PDE.
%         For example, the flux term in Fourier's law of conduction requires T to 
%         be differentiable.
% \end{enumerate}
% With these guidelines, a function is specified for the speed of sound to conduct
% a method of manufactured solutions on SWIRL's speed of sound numerical 
% integration. This is checked by observing the tangential mach number 
% produced from the speed of sound and comparing that to the tangential mach number
% that has been analytically defined (See Equation \ref{eq:Mthetabackcalculated}).
