%        File: MMS_BC.tex
%     Created: Wed Oct 06 09:00 AM 2021 E
% Last Change: Wed Oct 06 09:00 AM 2021 E
%
\documentclass[a4paper]{article}
\usepackage{amsmath}
\usepackage{multicol}
\usepackage{biblatex}
\addbibresource{refs.bib}
\begin{document}
\section{Setting Boundary Condition Values Using a Fairing Function}

Given a specified analytical function, $f(\widetilde{r})$, where  

\begin{align*}
    \widetilde{r} = \frac{r - r_{min}}{r_{max} - r_{min}} 
\end{align*}

Substituting $r_{min}$ $r_{max}$ for $r$ gives,

\begin{align*}
    \widetilde{r}_{min} &= \frac{r_{min} - r_{min}}{r_{max} - r_{min}} = 0\\
    \widetilde{r}_{max} &= \frac{r_{max} - r_{min}}{r_{max} - r_{min}} = 1
\end{align*}

The goal is to set desired values at the boundaries of the specified analytical
function. First we define the values at the boundaries, i.e.
\begin{align*}
    f(\widetilde{r} &= \widetilde{r}_{min}) = f_{min}     \\
    f(\widetilde{r} &= \widetilde{r}_{max}) = f_{max}     
\end{align*}

Then, the change between our desired boundary condition value and the actual is,
To do so, a desired change in the boundary condition must be defined. 

\begin{align*}
    \Delta f_{min} =  (f_{min}) - (f_{min})_{desired}   \\
    \Delta f_{max} =  (f_{max}) - (f_{max})_{desired}   
\end{align*}

To ensure that the desired changes are imposed \textit{smoothly}. The smoothness 
of a function is measured by the number of continuous derivatives the desired function
has over the domain of the function. At the very minimum, a smooth function will be continuous and
hence differentiable everywhere. When generating manufactured solutions, smoothness
of the solution is often times assumed but is not guarenteed 
\cite{oberkampf2002verification}. (transition sentance)

Defining the faring function:

\begin{align*}
    f_{imposed}(\widetilde{r}) = 
    f(\widetilde{r}) + 
    A_{min}(\widetilde{r}) \Delta f_{min} +
    A_{max}(\widetilde{r}) \Delta f_{max}  
\end{align*}

In order for the imposed boundary conditions to work, the desired values must be 
such that,

\begin{align*}
    A_{min}(\widetilde{r}_{min}) &= 1 &A_{min}(\widetilde{r}_{max}) &= 0  \\
    A_{max}(\widetilde{r}_{max}) &= 1 &A_{max}(\widetilde{r}_{min}) &= 1 
\end{align*}

This assured that the opposite boundaries are not affected.(How?)

For simplicity lets define:

\begin{align*}
    A_{min}(\widetilde{r}) =
    1-
    A_{max}(\widetilde{r}) 
\end{align*}

so now only $A_{max}$ needs to be defined. 

As mentioned, the desired boundary condition need to allow the analytical function
to be differentiable, and as a consequence, it would be wise to also set those. 
In addition, different types of boundary conditions (such as Neumann) that would 
require this. 

\begin{align*}
    \frac{\partial A_{max} }{\partial \widetilde{r}}|_{\widetilde{r}_{min}} = 0 \\
    \frac{\partial A_{max} }{\partial \widetilde{r}}|_{\widetilde{r}_{max}} = 0 \\
\end{align*}

A straight forward choice would be 

\begin{align*}
    A_{max}(\widetilde{r}) = 
    3 \widetilde{r}^2 - 2 \widetilde{r}^3
\end{align*}

Note that the correction is carried from boundary to boundary, as opposed to 
applying the correction to only to a region near the boundaries. This ensures smooth 
derivatives in the interior domain.

\end{document}


