\section{Data Volume Class}

The data volume class is the 'backbone' of the STMA code, because it
provides a template for data storage in the code.

The basic idea is that the data volume class is used to define all variable
data in the code.  Then, all 'number crunching' routines, such as derivative
calculations, dissipation, fluxes, and message passing routines, will expect
the data that they are working with to be in DataVolumeClass format.

To easily allow this, the DataVolumeClass has all internal data declared PUBLIC.
The idea behind this is to allow data to be easily referred to as an object.

Next question:  can I have \emph{some} PRIVATE attributes in a data class and
some PUBLIC?

It looks like I'll have to put an internal data class into the DataVolumeClass
object to get the functionality I want.  I've bookmarked some F2003 articles
to read about (and try) putting procedures into derived data types.  This way
(I think) I can put a procedure isInitialized into the DataVolumeClass that
acts like a LOGICAL -- but actually calls a hidden routine using hidden data
to find out if the DVC is initialized properly.

Let's try this tomorrow.
