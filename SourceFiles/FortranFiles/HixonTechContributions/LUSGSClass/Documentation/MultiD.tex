\section{Viscous extension of the LU-SGS (Yoon and Jameson, 1988) Scheme}

The approach of interest is 
the inviscid LU-SGS method of Yoon and Jameson.  For viscous flows, we
add the effect of viscosity into the LHS operators.

To illustrate this method, we'll look
at the unsteady viscous 1-D equation in Cartesian coordinates:

\begin{eqnarray}
\frac{\partial Q}{\partial t} + F_{x} - R_{xx} &=& 0
\nonumber
\end{eqnarray}

We write the equation in implicit form (with $n+1$ denoting the
new time level and $i$ denoting the grid location):

\begin{eqnarray}
\left(\frac{\partial Q}{\partial t} \right)_i^{n+1} 
+ 
\left(F_x \right)_i^{n+1}  
- 
\left(R_{xx} \right)_i^{n+1} &=& 0
\nonumber
\end{eqnarray}

and introduce an implicit iteration at the new time level (with
$l+1$ denoting the new iteration level): 

\begin{eqnarray}
\left(\frac{\partial Q}{\partial t} \right)_i^{n+1,l+1} 
+ 
\left(F_x \right)_i^{n+1,l+1}  
- 
\left(R_{xx} \right)_i^{n+1,l+1} &=& 0
\nonumber
\end{eqnarray}

We define the flux Jacobians as:

\begin{eqnarray}
A^{n+1,l}_i &=& \left(\frac{\partial F}{\partial Q} \right)^{n+1,l}_i
\nonumber
\\
B^{n+1,l}_i &=& \left(\frac{\partial R}{\partial Q} \right)^{n+1,l}_i
\nonumber
\end{eqnarray}

and define the change in $Q$ between iteration levels as:

\begin{eqnarray}
\Delta Q_i &=& \frac{Q_i^{n+1,l+1} - Q_i^{n+1,l}}{\Delta \tau_i}
\nonumber
\end{eqnarray}

where $\Delta \tau$ is the 'distance' between iteration levels in the
iteration direction.  Note that I'm allowing $\Delta \tau$ to have different
values at different spatial locations.

At this point, I'm going to linearize about the old iteration level:

\begin{eqnarray}
\left(F_x \right)_i^{n+1,l+1}  
&\approx&
\left(F_x \right)_i^{n+1,l}  
+
\left(A^{n+1,l}_i \Delta Q_i \right)_x
\nonumber
\\
\left(R_{xx} \right)_i^{n+1,l+1}  
&\approx&
\left(R_{xx} \right)_i^{n+1,l}  
+
\left(B^{n+1,l}_i \Delta Q_i \right)_{xx}
\nonumber
\end{eqnarray}

which gives the equation as:

\begin{eqnarray}
\left\{ I + \Delta \tau_i \delta_x A_i^{n+1,l} 
- \Delta \tau_i \delta_{xx} B_i^{n+1,l} 
\right\} \Delta Q_i + \Delta \tau_i \left\{RHS \right\}_i^{n+1,l} &=& 0
\nonumber
\end{eqnarray}

where $\delta_x$ is a central differencing operator:

\begin{eqnarray}
\left(\delta_x A \Delta Q \right)_i^{n+1,l} &=&
\frac{
\left(A \Delta Q \right)_{i+1}^{n+1,l}
-\left(A \Delta Q \right)_{i-1}^{n+1,l}
}{ 2 \Delta x}
\nonumber
\end{eqnarray}

and

\begin{eqnarray}
\left(\delta_{xx} B \Delta Q \right)_i^{n+1,l} &=&
\frac{
\left(B \Delta Q \right)_{i+1}^{n+1,l}
-2 \left(B \Delta Q \right)_{i}^{n+1,l}
+\left(B \Delta Q \right)_{i-1}^{n+1,l}
}{ \Delta x^2}
\nonumber
\end{eqnarray}

We decompose the $A$ matrix into an $A^+$ matrix with only positive eigenvalues and
an $A^-$ matrix with only negative eigenvalues:

\begin{eqnarray}
A &=& A^+ + A^-
\nonumber
\end{eqnarray}

We define forward and backward differences as:

\begin{eqnarray}
\left(
\delta_x^+ A \Delta Q \right)_i^{n+1,l} &=&
\frac{
\left(A \Delta Q \right)_{i+1}^{n+1,l}
-\left(A \Delta Q \right)_{i}^{n+1,l}
}{ \Delta x}
\nonumber
\\
\left(
\delta_x^- A \Delta Q \right)_i^{n+1,l}&=&
\frac{
\left(A \Delta Q \right)_{i}^{n+1,l}
-\left(A \Delta Q \right)_{i-1}^{n+1,l}
}{ \Delta x}
\nonumber
\end{eqnarray}

For ease of use, we can define:

\begin{eqnarray}
A^{\pm} &=& \frac{1}{2} \left(A \pm \epsilon I \right)
\nonumber
\\
\epsilon &\ge& max \left|\lambda \left(A \right) \right|
\nonumber
\end{eqnarray}

where $\lambda \left(A \right)$ are the eigenvalues of the $A$ matrix.

The equation can be rewritten as:

\begin{eqnarray}
\left\{ I 
+ \Delta \tau_i \delta_x^- A^+ 
+ \Delta \tau_i \delta_x^+ A^- 
- \Delta \tau_i \delta_{xx} B 
\right\} \Delta Q_i 
+ \Delta \tau_i \left\{RHS \right\}^{n+1,l}_i &=& 0
\nonumber
\end{eqnarray}

Defining

\begin{eqnarray}
\alpha_i &=& \frac{\Delta \tau_i}{\Delta x}
\nonumber
\\
\beta_i &=& \frac{\Delta \tau_i}{\Delta x^2}
\nonumber
\end{eqnarray}

and expanding it out,

\begin{eqnarray}
\left(
\begin{array}{c}
\Delta Q_i 
+
\alpha_i
\left(
A^-_{i+1} \Delta Q_{i+1}
-A^-_{i} \Delta Q_{i}
+A^+_{i} \Delta Q_{i}
-A^+_{i-1} \Delta Q_{i-1}
\right)
\\
-
\beta_i
\left(
B_{i+1} \Delta Q_{i+1}
-2 B_{i} \Delta Q_{i}
+B_{i-1} \Delta Q_{i-1}
\right)
\end{array}
\right)
+ \Delta \tau_i \left\{RHS \right\}^{n+1,l}_i &=& 0
\nonumber
\end{eqnarray}

A symmetric Gauss-Seidel scheme is:

\begin{eqnarray}
\left(
\begin{array}{c}
\Delta Q^{\left(1 \right)}_i
+ \left(
\alpha_i \left(A^+_i - A^-_i \right)
+ 2 \beta_i B_i \right)
\Delta Q^{\left(1 \right)}_i
\\
-
\left( 
\alpha_i A^+_{i-1} 
+
\beta_i B_{i-1} 
\right)
\Delta Q^{\left(1 \right)}_{i-1}
\end{array}
\right)
+ \Delta \tau_i \left\{RHS \right\}^{n+1,l}_i &=& 0
\nonumber
\\
\left(
\begin{array}{c}
\Delta Q^{\left(2 \right)}_i
+ \left(\alpha_i \left(A^+_i - A^-_i \right)
+ 2 \beta_i B_i 
\right)
\Delta Q^{\left(2 \right)}_i
\\
+ \left( 
\alpha_i A^-_{i+1}  
-\beta_i B_{i+1} 
\right)
\Delta Q^{\left(2 \right)}_{i+1}
\\
- \left( 
 \alpha_i A^+_{i-1} 
+ \beta_i B_{i-1} 
\right)
\Delta Q^{\left(1 \right)}_{i-1}
\end{array}
\right)
+ \Delta \tau_i \left\{RHS \right\}^{n+1,l}_i &=& 0
\nonumber
\end{eqnarray}

Subtracting the first equation from the second gives:

\begin{eqnarray}
\left(
\begin{array}{c}
\left\{
I + \alpha_i \left(
A_i^+ 
-A_i^- 
\right)
+ 2 \beta_i B_i
\right\}
\Delta Q^{\left(2 \right)}_i
\\
+ \left( 
\alpha_i A^-_{i+1}  
-
\beta_i B_{i+1} 
\right)
\Delta Q^{\left(2 \right)}_{i+1}
\end{array}
\right)
&=&
\left\{
I + \alpha_i \left(
A_i^+ 
- A_i^- 
\right)
+ 2 \beta_i B_i
\right\}
\Delta Q^{\left(1 \right)}_i
\nonumber
\end{eqnarray}

The first stage can be rewritten as:

\begin{eqnarray}
\left(
\begin{array}{c}
\Delta Q^{\left(1 \right)}_i
+ \left(
\alpha_i \left(A^+_i - A^-_i \right)
+ 2 \beta_i B_i \right)
\Delta Q^{\left(1 \right)}_i
\\
-
\left( 
\alpha_i A^+_{i-1} 
+
\beta_i B_{i-1} 
\right)
\Delta Q^{\left(1 \right)}_{i-1}
\end{array}
\right)
+ \Delta \tau_i \left\{RHS \right\}^{n+1,l}_i &=& 0
\nonumber
\\
\left(
\begin{array}{c}
\Delta Q^{\left(1 \right)}_i
- \alpha_i A^-_i 
\Delta Q^{\left(1 \right)}_i
+ \beta_i B_i 
\Delta Q^{\left(1 \right)}_i
\\
+ \alpha_i
\left(
A^+_i 
\Delta Q^{\left(1 \right)}_{i}
- 
A^+_{i-1} 
\Delta Q^{\left(1 \right)}_{i-1}
\right)
\\
+ \beta_i 
\left(
B_i 
\Delta Q^{\left(1 \right)}_i
-
B_{i-1} 
\Delta Q^{\left(1 \right)}_{i-1}
\right)
\end{array}
\right)
+ \Delta \tau_i \left\{RHS \right\}^{n+1,l}_i &=& 0
\nonumber
\\
\left(
\left(
I
- \alpha_i A^-
+ \beta_i B
\right)
+ \alpha_i
\Delta x
\delta_x^-
A^+_i 
+ \beta_i 
\Delta x
\delta_x^-
B
\right)
\Delta Q^{\left(1 \right)}
&=&
- \Delta \tau_i \left\{RHS \right\} 
\nonumber
\end{eqnarray}

We then define:

\begin{eqnarray}
\Delta Q^{*}
&=&
\left(I + \alpha
\left(A^+ - A^- \right)
+ 2 \beta B
\right) 
\Delta Q^{\left(1 \right)}
\nonumber
\end{eqnarray}

and the second stage becomes:

\begin{eqnarray}
\left(
\begin{array}{c}
\left\{
I + \alpha_i \left(
A_i^+ 
-A_i^- 
\right)
+ 2 \beta_i B_i
\right\}
\Delta Q^{\left(2 \right)}_i
\\
+ \left( 
\alpha_i A^-_{i+1}  
-
\beta_i B_{i+1} 
\right)
\Delta Q^{\left(2 \right)}_{i+1}
\end{array}
\right)
&=&
\left\{
I + \alpha_i \left(
A_i^+ 
- A_i^- 
\right)
+ 2 \beta_i B_i
\right\}
\Delta Q^{\left(1 \right)}_i
\nonumber
\\
\left(
\begin{array}{c}
\left\{
I + \alpha_i 
A_i^+ 
+ \beta_i B_i
\right\}
\Delta Q^{\left(2 \right)}_i
\\
+ 
\alpha_i
\left( 
A^-_{i+1}  
\Delta Q^{\left(2 \right)}_{i+1}
-A_i^- 
\Delta Q^{\left(2 \right)}_i
\right)
\\
- 
\beta_i
\left( 
B_{i+1} 
\Delta Q^{\left(2 \right)}_{i+1}
-
B_i
\Delta Q^{\left(2 \right)}_i
\right)
\end{array}
\right)
&=&
\Delta Q^{*}_i
\nonumber
\\
\left(
\left\{
I + \alpha 
A^+ 
+ \beta B
\right\}
+ 
\alpha \Delta x
\delta^+_x
A^-
-
\beta \Delta x
\delta^+_x
B
\right)
\Delta Q^{\left(2 \right)}
&=&
\Delta Q^{*}
\nonumber
\end{eqnarray}

We can now write the scheme as:

\begin{eqnarray}
L D^{-1} U \Delta Q &=& 
-\Delta \tau \left\{RHS \right\}
\nonumber
\end{eqnarray}

where

\begin{eqnarray}
L &=& 
I - \alpha A^- + \beta B 
+ \alpha \Delta x \delta_x^- A^+
+ \beta \Delta x \delta_x^- B
\nonumber
\\
U &=& 
I + \alpha A^+ + \beta B
+ \alpha \Delta x \delta_x^+ A^-
- \beta \Delta x \delta_x^+ B
\nonumber
\\
D &=& 
I + \alpha \left(A^+ - A^- \right) + 2 \beta B
\nonumber
\end{eqnarray}

Note that L and U can each be solved by sweeping from one end of the grid
to the other. 

\subsection{Just to be clear...}

In 1D, the actual equation to solve on the first
sweep is:

\begin{eqnarray}
\left(
I 
+ \alpha \left(A^+ - A^- \right)_i 
+ 2 \beta B_i 
\right)
\Delta Q^*_i 
-\alpha_{i-1} A^+_{i-1} \Delta Q^*_{i-1}
-\beta_{i-1} B_{i-1}\Delta Q^*_{i-1}
&=& -\Delta \tau \left(RHS \right)_i
\nonumber
\end{eqnarray}

which becomes:

\begin{eqnarray}
\left(
I 
+ \alpha \epsilon_i
+ 2 \beta B_i 
\right)
\Delta Q^*_i 
-\alpha_{i-1} A^+_{i-1} \Delta Q^*_{i-1}
-\beta_{i-1} B_{i-1}\Delta Q^*_{i-1}
&=& -\Delta \tau \left(RHS \right)_i
\nonumber
\end{eqnarray}

The second sweep is:

\begin{eqnarray}
\left(
\begin{array}{c}
\left(
I 
+ \alpha \left(A^+ - A^- \right)_i 
+ 2 \beta B_i 
\right)
\Delta Q_i 
\\
+\alpha_{i+1} A^-_{i+1} \Delta Q_{i+1}
-\beta_{i+1} B_{i+1}\Delta Q_{i+1}
\end{array}
\right)
&=& 
\left(
I 
+ \alpha \epsilon_i
+ 2 \beta B_i 
\right)
\Delta Q^*_i 
\nonumber
\end{eqnarray}

This is solved as:

\begin{eqnarray}
\Delta Q^*_i 
&=&
\left(
I 
+ \alpha \epsilon_i
+ 2 \beta B_i 
\right)^{-1}
\left(
\begin{array}{c}
-\Delta \tau \left(RHS \right)_i
\\
+\left(\alpha_{i-1} A^+_{i-1} 
+\beta_{i-1} B_{i-1} \right)
\Delta Q^*_{i-1}
\end{array}
\right)
\nonumber
\\
\Delta Q_i 
&=&
\left(
I 
+ \alpha \epsilon_i
+ 2 \beta B_i 
\right)^{-1}
\left(
\begin{array}{c}
\left(
I 
+ \alpha \epsilon_i
+ 2 \beta B_i 
\right)
\Delta Q^*_{i}
\\
+\left(-\alpha_{i+1} A^-_{i+1} 
+\beta_{i+1} B_{i+1} \right)
\Delta Q_{i+1}
\end{array}
\right)
\nonumber
\end{eqnarray}

If we reverse the sweep directions, the scheme becomes:

\begin{eqnarray}
\Delta Q^*_i 
&=&
\left(
I 
+ \alpha \epsilon_i
+ 2 \beta B_i 
\right)^{-1}
\left(
\begin{array}{c}
-\Delta \tau \left(RHS \right)_i
\\
+\left(
-\alpha_{i+1} A^-_{i+1} 
+\beta_{i+1} B_{i+1} \right)
\Delta Q^*_{i+1}
\end{array}
\right)
\nonumber
\\
\Delta Q_i 
&=&
\left(
I 
+ \alpha \epsilon_i
+ 2 \beta B_i 
\right)^{-1}
\left(
\begin{array}{c}
\left(
I 
+ \alpha \epsilon_i
+ 2 \beta B_i 
\right)
\Delta Q^*_{i}
\\
+\left(\alpha_{i-1} A^+_{i-1} 
+\beta_{i-1} B_{i-1} \right)
\Delta Q_{i-1}
\end{array}
\right)
\nonumber
\end{eqnarray}

