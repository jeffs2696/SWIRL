%        File: ZGGEV_testing_doc.tex
%     Created: Thu Feb 09 11:00 AM 2023 E
% Last Change: Thu Feb 09 11:00 AM 2023 E
%

\documentclass[a4paper]{article}
\usepackage{amsmath}
\usepackage{mathtools}
\usepackage{verbatim}
\begin{document}
The current research direction is to conduct a test on the eigenvalue and eigenvector
output from SWIRL to ensure that LAPACK's generalized eigensolver for general matricies \verb|zggev.f|. Note the
slight difference in subroutine name \verb|zgeev.f| , which is intended for an unsymmetric general matrix, which in some cases 
is rectangular. 

\begin{align*}
    A {x} = \lambda B {x}
\end{align*}
where A is a $n x n$ matrix and $B$ is an identity matrix of equal size. 

Expanding,

\begin{align*}
    A {x} - \lambda B {x} = 0 \\
    (A - \lambda B) {x} = 0 
\end{align*}
therefore, eigenvectors of $A$ with the corresponding eigenvalue $\lambda$, if any
are the nontrivial solutions of the matrix equation $(A - \lambda B) {x} = 0$.
Since LAPACK uses iterative numerical techniques to solve for the eigenvalues
of the input matrix $A$, the matrix equation may not necessarily be zero if the resulting eigenvalue and
vector was substituted back into the equation. To account for this, the right 
hand side of the matrix equation is set to a variable $S$, which will be floating 
point precision for the correct eigenvalue and vector pair, given the LAPACK routine
has converged. 

The code provided in this directory computes the eigenvalues and vectors 
of a $3x3$ matrix and computes $S$ for a given eigenvalue/vector pair. The $L_2$ of $S$ is
also computed using the same \verb|SUBROUTINE| as in SWIRL. 

This code will also precondition the A matrix using the \verb|LWORK| variable by
first calling \verb|ZGGEV| with \verb|LWORK| set to minus one, and using the resulting 
$A$ and $B$ matrices in a second call. Otherwise, \verb|LWORK| is set to twice the order of the $A$ matrix,
\verb|N|. 
%
%For this particular problem $A$ is generated using the \verb|reshape| function\dots where as SWIRL does not 
%have this(?). It may have used an implied do loop instead\dots this needs to be
%checked.
%
%The $L_2$ of S is machine precision except when the \verb|eigenIndex| is 4,5, or 6
%but not 1-3 and 6-9\dots
%\begin{verbatim}
%    DO i = 1,N
%        WVN(i) = ALPHA(i)/BETA(i)
%    END DO
%    
%    S = MATMUL(A_before,VR(:,eigenIndex)) - MATMUL(B_before,VR(:,eigenIndex))*(WVN(eigenIndex))
%
%    CALL getL2Norm(L2,S)
%\end{verbatim}
%



\section{2 x 2 Example Problem}

The eigenvalues of the 2x2 matrix 
$A = \begin{bmatrix} 0 & 1 \\ -2 & -3 \end{bmatrix}$ can be found by solving the characteristic equation:
\begin{equation}
    \det(A - \lambda I) = 0
\end{equation}


where $I$ is the 2x2 identity matrix. 

The determinant of $A - \lambda I$ is given by:

\begin{align*}
    \det(A - \lambda I) &= \\
    \det\begin{bmatrix} 
        0 - \lambda & 1 \\ 
        -2 & -3 - \lambda 
    \end{bmatrix} 
                        &=  \\
    \det\begin{bmatrix} 
    -\lambda & 1 \\ 
    -2 & -3 - \lambda 
\end{bmatrix} &= \\
    (-\lambda)(-3 - \lambda) - (1)(-2) = \\
    \lambda^2 + 3\lambda + 2 
\end{align*}

So the characteristic equation is:

\begin{equation}
    \lambda^2 + 3\lambda + 2 = 0
\end{equation}

This equation can be solved using the quadratic formula:

\begin{equation}
    \lambda = \frac{-b \pm \sqrt{b^2 - 4ac}}{2a}
\end{equation}

where $a = 1$, $b = 3$, and $c = 2$.

So the eigenvalues are:
\begin{align*}
    \lambda = \frac{-(3) \pm \sqrt{3^2 - 4(1)(2)}}{2(1)} = \frac{-3 \pm \sqrt{9 - 8}}{2} = \frac{-3 \pm \sqrt{1}}{2} = \frac{-3 \pm 1}{2} = -2 \text{ or } -1 
\end{align*}

To find the eigenvectors, we need to solve the equation:

$(A - \lambda I)v = 0$

where $v$ is the eigenvector and $\lambda$ is the corresponding eigenvalue.

For $\lambda = -2$:

$\begin{bmatrix} 0 & 1 \\ -2 & -3 \end{bmatrix} \begin{bmatrix} x_1 \\ x_2 \end{bmatrix} = -2 \begin{bmatrix} x_1 \\ x_2 \end{bmatrix}$

which leads to the system of equations:
\begin{align*}
0x_1 + 1x_2 = -2x_1 \\  
-2x_1 - 3x_2 = -2x_2 
\end{align*}

Solving this system of equations, we find that $x_1 = 1$ and $x_2 = 2$, so the
eigenvector corresponding to $\lambda = -2$ is $\begin{bmatrix} 1 \\ 2 \end{bmatrix}$.

For $\lambda = -1$:

$\begin{bmatrix} 0 & 1 \\ -2 & -3 \end{bmatrix} \begin{bmatrix} x_1 \\ x_2 \end{bmatrix} = -1 \begin{bmatrix} x_1 \\ x_2 \end{bmatrix}$

which leads to the system of equations: 
\begin{align*}
    0x_1 + 1x_2 = -x_1 \\
    -2x_1 - 3x = -x_2
\end{align*}

Solving this system of equations, we find that $x_1 = 1$ and $x_2 = 1$, so the
eigenvector corresponding to $\lambda = -1$ is $\begin{bmatrix} 1 \\ -1 \end{bmatrix}$.

Now we have a basis to do our characteristic equation test with either pair!

Note that an eigensolver may chose different values for the eigenvectors than the
ones we chose. However, the ratio of $v_{1,1}$ to $v_{1,2}$ will be the same.  



 

\end{document}


