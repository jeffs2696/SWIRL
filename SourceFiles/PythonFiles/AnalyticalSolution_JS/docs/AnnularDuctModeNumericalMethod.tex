Bessel's Function,

\begin{equation}
    R(r) = A J_0 (k_r r) + B Y_0(k_r r ) 
    \label{eqn:besselsfun}
\end{equation}
where $J_0$ and $Y_0$ are Bessel's functions of the first 
and second kind, and $A$ and $B$ are arbitrary constants. 
Both functions reduce as $k_r r$ gets large. The Bessel function of the second
kind is unbounded as $k_r r$ goes to zero. 

For a hard-walled duct, the radial velocity component is zero at the 
boundaries $r_{min}$ and $r_{max}$. 

\begin{equation}
    \frac{dR(r)}{dr}\Bigr|_{r = r_{max}}  =%
    \frac{dR(r)}{dr}\Bigr|_{r = r_{min}}   = 0
    \label{eqn:besselBC}
\end{equation}
In the case of a hollow duct, there is no
minimum radius, therefore the wall boundary condition only applies at $r_{max}$.


Since the solution must be finite as $k_r r$ approaches zero, it can be observed
that $Y_0$ approaches infinity. Since this would yield in a trivial solution, the 
coefficient $B = 0$, which reduces \ref{eqn:besselsfun} to,

\begin{equation}
    R(r) = A J_0 (k_r r) 
    \label{eqn:besselsfunCylinder}
\end{equation}

Taking the derivative 
with respect to $r$ yields,

\begin{equation}
    \frac{dR(r)}{dr}\Bigr|_{r} = A J'_0 (k_r r)  = 0
    \label{eqn:besselsfunderivaTive}
\end{equation}
The boundary condition requires the Bessel function be zero at a hard wall.
The terms inside of the Bessel function would then correspond to values along 
the domain, $k_r r$,  which satisfy our equation.  Let $k_r r = \alpha_{m,n}$ 
where, $alpha$ represents the zeros of the Bessel function, and 
$m$ corresponds to the azimuthal mode order and $n$ represents the radial mode
order, i.e. the index for the number of zero crossings in the 
derivative of the Bessel function of the first kind. 

Therefore,


\begin{align}
    \frac{dR}{dr}\Bigr|_{r = r_{max}} = A J'_0 (k_r r_{max})  = 0 \\
    (k_r r_{max})  = 0 
\end{align}
Recalling $\alpha_{m,n}$


\begin{align}
    (k_r r_{max})  = \alpha_{m,n} \\
    k_r = \alpha_{m,n}/r_{max}
\end{align}

For annular ducts, $r_{min}$ is no longer zero, therefore $B$ cannot be removed
since $Y_0'$ has finite values as $k_r r $ increases.

\begin{equation}
    \frac{dR}{dr}\Bigr|_{r} = A J'_0 (k_r r) + B Y'_0(k_r r ) 
    \label{eqn:besselsfunderivaTive}
\end{equation}
Applying boundary conditions to both inner and outer walls at $r_{min}$ and $r_{max}$

\begin{equation}
    \frac{dR}{dr}\Bigr|_{r=r_{min}} = A J'_0 (k_r r_{min}) + B Y'_0(k_r r_{min}) 
\end{equation}

\begin{equation}
    \frac{dR}{dr}\Bigr|_{r=r_{max}} = A J'_0 (k_r r_{max}) + B Y'_0(k_r r_{max}) 
\end{equation}

defining $\beta = \frac{B}{A}$,

\begin{equation}
    \frac{dR}{dr}\Bigr|_{r=r_{min}} =  J'_0 (k_r r_{min}) + \beta Y'_0(k_r r_{min}) 
\end{equation}

\begin{equation}
    \frac{dR}{dr}\Bigr|_{r=r_{max}} =  J'_0 (k_r r_{max}) + \beta Y'_0(k_r r_{max}) 
\end{equation}

Defining,

\begin{align*}
    k_r^{n+1} = k_r^n + \Delta k_r \\
    \beta^{n+1} = \beta^n + \Delta \beta 
\end{align*}
And substituting into the boundary condition expressions,
\begin{equation}
    \frac{dR}{dr}\Bigr|_{r=r_{min}} =  J'_0 (k_r^{n+1} r_{min}) + \beta^{n+1} Y'_0(k_r^{n+1} r_{min}) 
\end{equation}

\begin{equation}
    \frac{dR}{dr}\Bigr|_{r=r_{max}} =  J'_0 (k_r^{n+1} r_{max}) + \beta^{n+1} Y'_0(k_r^{n+1} r_{max}) 
\end{equation}

