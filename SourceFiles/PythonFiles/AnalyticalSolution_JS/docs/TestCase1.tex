
\section{Analytical Test Case}
The results present a comparison of analytical and numerical solutions for a
uniform flow, hard-wall, cylindrical duct for six grids. The physical 
parameters for this test case are reporte in Table 1. The grid points were
doubled each iteration and a starting grid of 33 points was chosen. The 
first five radial modes were chosen in order to get a group of modes that were cut
on and cut off. The case number's are used as identifiers for each 
mode propagating up or downstream. In the event that a spurious mode is identified,
the case number will index it.  The number of cut off modes was also constrained
by setting a maximum value for the imaginary part of the axial wavenumber. 
While this magnitude is currently arbitrary, the value could be correlated to a
desired decay rate for the corresponding mode. Each of the axial wavenumber's 
radial mode for each grid in also reported. The same test case was ran twice using 
second and fourth order central schemes for the radial derivatives.  The 
approximate rate of convergence is also reported for both numerical schemes.
\begin{table}[!h]
    \centering
    \begin{tabular}{|l|l|}
        \hline
        $\sigma$ & \textit{0.0} \\ \hline
        $k$      & \textit{10}   \\ \hline
        $m$      & \textit{2}    \\ \hline
        $M_x$    & \textit{0.3}  \\ \hline
    \end{tabular}
    \caption{Validation test case parameters, Uniform Flow Hard-wall Cylindrical%
    Duct} 
\end{table}

\newpage


\foreach \i in {33,66,132,264,528,1056}
{
    \begin{figure}[p]
        \centering
        \includegraphics[width=\textwidth]%
        {../figures/second_order_wavenumber_grid_\i.pdf}
    \caption{ Second Order, Analytical Solution vs Numerical Approximation using \i\  points}
    \vspace{1in}
    \end{figure}
}
% \stop
\foreach \i in {33,66,132,264,528,1056}
{
    \begin{figure}
        \centering
        \includegraphics[width=\textwidth]
        {../figures/fourth_order_wavenumber_grid_\i.pdf}
    \caption{ Fourth Order, Analytical Solution vs Numerical Approximation using \i\  points}
    \vspace{1in}
    \end{figure}
}

\subsection{Propagating Radial Modes}

\subsubsection{Second Order, Radial Mode 0}
\newpage
\foreach \i in {0,...,1}
{
    \foreach \j in {33,66,132,264,528,1056} 
    {
        \begin{figure}
            \centering
            \includegraphics[width=\textwidth]
            {../figures/second_order_radial_mode_0_test_case_number_\i_grid_\j.pdf}
            \caption{Second Order, Case number \i, \j points}
            % \vspace{1in}
            \label{fig:analytical_bessel_function}
        \end{figure}
        \begin{figure}
            \centering
            \includegraphics[width=\textwidth]
            {../figures/second_order_radial_mode_error_0_test_case_number_\i_grid_\j.pdf}
            \caption{Second Order, Case number \i, \j points}
            % \vspace{1in}
            \label{fig:analytical_bessel_function}
        \end{figure}
    }
}
\clearpage

% \stop

\subsubsection{ Fourth Order, Radial Mode 0}
\newpage
\foreach \i in {0,...,1}
{
    \foreach \j in {33,66,132,264,528,1056} 
    {
        \begin{figure}
            \centering
            \includegraphics[width=\textwidth]
            {../figures/fourth_order_radial_mode_0_test_case_number_\i_grid_\j.pdf}
            \caption{Fourth Order, Case number \i, \j points}
            \label{fig:analytical_bessel_function}
        \end{figure}
        \begin{figure}
            \centering
            \includegraphics[width=\textwidth]
            {../figures/fourth_order_radial_mode_error_0_test_case_number_\i_grid_\j.pdf}
            \caption{Fourth Order, Case number \i, \j points}
            \label{fig:analytical_bessel_function}
        \end{figure}
    }
}
\clearpage
\subsubsection{Second Order, Radial Mode 1}
\newpage
\foreach \i in {0,...,1}
{
    \foreach \j in {33,66,132,264,528,1056} 
    {
        \begin{figure}
            \centering
            \includegraphics[width=\textwidth]
            {../figures/second_order_radial_mode_1_test_case_number_\i_grid_\j.pdf}
            \caption{Second Order, Case number \i, \j points}
            \label{fig:analytical_bessel_function}
        \end{figure}
        \begin{figure}
            \centering
            \includegraphics[width=\textwidth]
            {../figures/second_order_radial_mode_error_1_test_case_number_\i_grid_\j.pdf}
            \caption{Second Order, Case number \i, \j points}
            \label{fig:analytical_bessel_function}
        \end{figure}
    }
}

\clearpage
\subsubsection{Fourth Order, Radial Mode 1}
\newpage
\foreach \i in {0,...,1}
{
    \foreach \j in {33,66,132,264,528,1056} 
    {
        \begin{figure}
            \centering
            \includegraphics[width=\textwidth]
            {../figures/fourth_order_radial_mode_1_test_case_number_\i_grid_\j.pdf}
            \caption{Fourth Order, Case number \i, \j points}
            \label{fig:analytical_bessel_function}
        \end{figure}
        \begin{figure}
            \centering
            \includegraphics[width=\textwidth]
            {../figures/fourth_order_radial_mode_error_1_test_case_number_\i_grid_\j.pdf}
            \caption{Fourth Order, Case number \i, \j points}
            \label{fig:analytical_bessel_function}
        \end{figure}
    }
}

\clearpage
\subsubsection{Second Order, Radial Mode 2}
\newpage
\foreach \i in {0,...,1}
{
    \foreach \j in {33,66,132,264,528,1056} 
    {
        \begin{figure}
            \centering
            \includegraphics[width=\textwidth]
            {../figures/second_order_radial_mode_2_test_case_number_\i_grid_\j.pdf}
            \caption{Second Order, Case number \i, \j points}
            \label{fig:analytical_bessel_function}
        \end{figure}
        \begin{figure}
            \centering
            \includegraphics[width=\textwidth]
            {../figures/second_order_radial_mode_error_2_test_case_number_\i_grid_\j.pdf}
            \caption{Second Order, Case number \i, \j points}
            \label{fig:analytical_bessel_function}
        \end{figure}
    }
}

\clearpage
\subsubsection{Fourth Order, Radial Mode 2}
\newpage
\foreach \i in {0,...,1}
{
    \foreach \j in {33,66,132,264,528,1056} 
    {
        \begin{figure}
            \centering
            \includegraphics[width=\textwidth]
            {../figures/fourth_order_radial_mode_2_test_case_number_\i_grid_\j.pdf}
            \caption{Fourth Order, Case number \i, \j points}
            \label{fig:analytical_bessel_function}
        \end{figure}
        \begin{figure}
            \centering
            \includegraphics[width=\textwidth]
            {../figures/fourth_order_radial_mode_error_2_test_case_number_\i_grid_\j.pdf}
            \caption{Fourth Order, Case number \i, \j points}
            \label{fig:analytical_bessel_function}
        \end{figure}
    }
}


\clearpage
\subsubsection{Second Order, Radial Mode 3}
\newpage
\foreach \i in {0,...,1}
{
    \foreach \j in {33,66,132,264,528,1056} 
    {
        \begin{figure}
            \centering
            \includegraphics[width=\textwidth]
            {../figures/second_order_radial_mode_3_test_case_number_\i_grid_\j.pdf}
            \caption{Second Order, Case number \i, \j points}
            \label{fig:analytical_bessel_function}
        \end{figure}
        \begin{figure}
            \centering
            \includegraphics[width=\textwidth]
            {../figures/second_order_radial_mode_error_3_test_case_number_\i_grid_\j.pdf}
            \caption{Second Order, Case number \i, \j points}
            \label{fig:analytical_bessel_function}
        \end{figure}
    }
}

\clearpage
\subsubsection{Fourth Order, Radial Mode 3}
\newpage
\foreach \i in {0,...,1}
{
    \foreach \j in {33,66,132,264,528,1056} 
    {
        \begin{figure}
            \centering
            \includegraphics[width=\textwidth]
            {../figures/fourth_order_radial_mode_3_test_case_number_\i_grid_\j.pdf}
            \caption{Fourth Order, Case number \i, \j points}
            \label{fig:analytical_bessel_function}
        \end{figure}
        \begin{figure}
            \centering
            \includegraphics[width=\textwidth]
            {../figures/fourth_order_radial_mode_error_3_test_case_number_\i_grid_\j.pdf}
            \caption{Fourth Order, Case number \i, \j points}
            \label{fig:analytical_bessel_function}
        \end{figure}
    }
}



\clearpage
\subsubsection{Second Order, Radial Mode 4}
\newpage
\foreach \i in {0,...,1}
{
    \foreach \j in {33,66,132,264,528,1056} 
    {
        \begin{figure}
            \centering
            \includegraphics[width=\textwidth]
            {../figures/second_order_radial_mode_4_test_case_number_\i_grid_\j.pdf}
            \caption{Second Order, Case number \i, \j points}
            \label{fig:analytical_bessel_function}
        \end{figure}
        \begin{figure}
            \centering
            \includegraphics[width=\textwidth]
            {../figures/second_order_radial_mode_error_4_test_case_number_\i_grid_\j.pdf}
            \caption{Second Order, Case number \i, \j points}
            \label{fig:analytical_bessel_function}
        \end{figure}
    }
}

\clearpage
\subsubsection{Fourth Order, Radial Mode 4}
\newpage
\foreach \i in {0,...,1}
{
    \foreach \j in {33,66,132,264,528,1056} 
    {
        \begin{figure}
            \centering
            \includegraphics[width=\textwidth]
            {../figures/fourth_order_radial_mode_4_test_case_number_\i_grid_\j.pdf}
            \caption{Fourth Order, Case number \i, \j points}
            \label{fig:analytical_bessel_function}
        \end{figure}
        \begin{figure}
            \centering
            \includegraphics[width=\textwidth]
            {../figures/fourth_order_radial_mode_error_4_test_case_number_\i_grid_\j.pdf}
            \caption{Fourth Order, Case number \i, \j points}
            \label{fig:analytical_bessel_function}
        \end{figure}
    }
}


\section{Discussion}





