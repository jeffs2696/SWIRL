\documentclass[a4paper]{report}
\maxdeadcycles=200
\usepackage[section]{placeins}
\usepackage{afterpage}
\usepackage{mathtools}
\usepackage{amsmath}
\usepackage{pgffor}
\usepackage{booktabs}
\begin{document}

\begin{titlepage}
\title{
Research Report}


\author{ Jeff Severino
 \\
University of Toledo \\
Toledo, OH  43606 \\
email:  jeffrey.severino@rockets.utoledo.edu}

\maketitle

\end{titlepage}

\section{Current Research Direction}
This week the plan was to complete:

\begin{itemize}
    \item Wrap up the initial numerical analysis study on hard-walled cylindrical 
        duct acoustic mode convergence study by looking at the first five 
        radial modes; only the first cut-on radial mode for five grids were studied. 
    \item Start the same study on the annular duct case.  
\end{itemize}

\section{Research Performed This Week}
Last week, the convergence study for the cylindrical duct mode was completed
for the first cut-on radial mode for a given uniform flow. However, the issue was

Last week, the convergence study for the cylindrical duct mode was completed for
the first cut-on radial mode for a given uniform flow. However, the issue was that
the analytic computation and importing of numerical results coincided, which can be
problematic in convergence studies since the convergence rates can only be computed
once all the data from multiple grids is imported. Therefore, importing data was first
completed as a relatively quick fix. Then the corresponding analytic computations were
carried out, allowing the data to be pulled when needed instead of doing these two
steps simultaneously. 

Although these compuations are separate, the convergence rate for the next 
four radial modes need to be computed, the results for the first five grids are 
below. Note that the number of grids were doubled and not yet subtracted by one.


\begin{table}
    % \centering
    \makebox[\linewidth]{
        \input{../tables/L2_downstream_table.tex}
    }
    \caption{L2 error for downstream radial mode 1}
\end{table}
\begin{table}
    \makebox[\linewidth]{
        \begin{tabular}{lrrrrrrr}
 & $L_{2,k_x}$ & $L_{2,\bar{p}}$ & $L_{2,noLBC}$ & $L_{2,noRBC}$ & $L_{2,noBCS}$ & $L_{max}$ & $L_{max,location}$ \\
0 & 0.00000155279 & 0.00059786297 & 0.00019688319 & 0.00059696793 & 0.00016546650 & 0.00324886266 & 0 \\
1 & 0.00000028331 & 0.00010577879 & 0.00004184232 & 0.00010658661 & 0.00004216087 & 0.00079036919 & 0 \\
2 & 0.00000002131 & 0.00001964476 & 0.00000996508 & 0.00001971142 & 0.00000998706 & 0.00019476231 & 0 \\
3 & 0.00000000161 & 0.00000384159 & 0.00000243557 & 0.00000384595 & 0.00000243556 & 0.00004833159 & 0 \\
4 & 0.00000000011 & 0.00000079773 & 0.00000060218 & 0.00000079784 & 0.00000060188 & 0.00001203774 & 0 \\
5 & 0.00000000001 & 0.00000017589 & 0.00000014971 & 0.00000017586 & 0.00000014965 & 0.00000300377 & 0 \\
\end{tabular}

    }
    \caption{L2 error for upstream radial mode 1}
\end{table}

\begin{table}
    \makebox[\linewidth]{
        \begin{tabular}{lrrrr}
 & 0 & 1 & 2 & 3 \\
0 & 2.540732 & 2.921466 & 2.532691 & 2.436520 \\
1 & 2.520906 & 2.667102 & 2.520883 & 2.477816 \\
\end{tabular}

    }
    \caption{ROC error for downstream radial mode 1}
\end{table}
\begin{table}
    \makebox[\linewidth]{
        \begin{tabular}{lrrrr}
 & 0 & 1 & 2 & 3 \\
0 & 2.540732 & 2.921466 & 2.532691 & 2.436520 \\
\end{tabular}

    }
    \caption{ROC error for upstream radial mode 1}
\end{table}
\subsection{Discussion}
A total of five grids were studied, starting from 33 and doubling until 1056 for the
first cut on radial mode. The upstream and downstream mode pairs were used to compute
the L2, $L_{max}$, and convergence rates for a uniform flow in a cylindrical duct. The axial
wavenumber’s error starts with a magnitude of  e-6 and decreases to  $e-12$. A fourth
order is expected when computing the convergence rate since a fourth-order central
scheme was used. This behavior is shown in the upstream mode, with a rate of convergence
of  ~4.19. However, this behavior is less pronounced in the downstream case. The rate
increases to  3.7  but begins to decrease for the last grid pair. A grid spacing choice
should have been made such that the jumps are not as large so that the convergence
rates can be studied more carefully.                                                                                                                                                                                                                                                                                                                                                                                                                                                                                                                       For

For the first radial mode, the pressure distribution in the upstream and downstream directions 
do not reach fourth-order convergence as excpected. Using L’Hopital’s rule to obtain 
a value at the centerline causes the error between the numerical and analytic modes 
to be highest at the centerline gridpoint. L’Hopital’s rule’s effect is shown in upstream and downstream directions by identifying the
location of the highest error, $L_{max}$, for pressure radial mode comparisons. When computing
the rate of convergence for the  $L_{max}$ point, it converges to the second order. The
same trend is noticed when both boundaries are removed in the error calculation. When
looking at the difference between the excluding of boundaries, it seems that the convergence
rate is higher without the wall BC as opposed to being higher without the centerline
BC, but when computing the convergence rate of  $L_{max}$, a second-order rate is computed.         

For the pressure mode data, neither upstream or downstream reach fourth order convergence.
The use of the L'Hopital's rule to obtain a value at the centerline causes the 
error between the numerical and analytic modes to be highest at the centerline
gridpoint. This is shown in both upstream and downstream directions by identifying
the location of the highest error $L_{max}$. When computing the rate of convergence
for the $L_{max}$ point, it converges to second order. The same trend is noticed when 
both boundaries are removed in the error calculation. 

When looking at the difference between the excluding of boundaries, it seems that 
the convergence rate is higher without the wall BC as opposed to being higher without 
the centerline BC, but when computing the convergence rate of $L_{max}$, a second
order rate is computed. 

\section{Issues and Concerns}
The proper grid spacing needs to be used. The distance between grid points isn't
halving with the doubling of grid points because of the two points on the boundaries 
of the domain. As a result the grid should be ``doubled'' by following $2n - 1$ 
where $n$ is the number of grid points.

\section{Planned Research}
Now that the post processing routine allows for easier computation of more radial
modes, that will be completed first. Then, SWIRL will be revised with the  correct method of doubling grid points to preserve a halving of 
 grid spacing $\Delta r$ .  


\end{document}
