
\documentclass[a4paper]{article}
\usepackage{mathtools}
\usepackage{verbatim}
\usepackage{graphicx}
\usepackage{tabularx}
\usepackage{pgfplots}
\usepackage{adjustbox}
\usepackage{booktabs}
\makeatletter
\let\latex@xfloat=\@xfloat
\def\@xfloat #1[#2]{%
    \latex@xfloat #1[#2]%
    \def\baselinestretch{1}
    \@normalsize\normalsize
    \normalsize
}
\makeatother
\usepackage{amsmath}
\usepackage{mathtools}
\usepackage{epigraph}
\usepackage{cancel}
\usepackage{xcolor}
\newcommand\Ccancel[2][black]{\renewcommand\CancelColor{\color{#1}}\cancel{#2}}
%\usepackage{algorithm}
\usepackage{graphicx}
%\usepackage[noend]{algpseudocode}
%\usepackage{gnuplot-lua-tikz}
\usepackage[utf8]{inputenc}
\usepackage{pgfplots}
\usepackage{tabularx}
\DeclareUnicodeCharacter{2212}{−}
\usepgfplotslibrary{groupplots,dateplot}
\usetikzlibrary{patterns,shapes.arrows}
\pgfplotsset{compat=newest}
\begin{document}
\begin{titlepage}

    \title{
    Weekly Research Report}


    \author{ Jeffrey Severino \\
        University of Toledo \\
        Toledo, OH  43606 \\
    email: jseveri@rockets.utoledo.edu}


    \maketitle

\end{titlepage}
\section{Current Research Direction}
Validate annular duct modes. A result has been obtained but does not seem to 
have the correct boundary conditions.

\section{Research Performed This Week}
Documentation for the annular duct mode calculations is underway. The procedure
was presented in ``Aeroacoustic Analysis of Turbofan Noise Generation'' by 
Harold D. Meyer and Edmane Envia. In Chapter 4: Duct Acoustics, Section 
4.1 Normal Modes in an Annular Duct , along with the accompanying appendix 
section, Appendix B: Numerical Computation of the Normal Modes in An Annular 
Duct, the theoretical background discusses how the boundary conditions are
used to obtain the appropriate weighting factors and mode shapes. The FORTRAN 77
library codes have been copied over to SWIRL and a test routine using 
FORTRAN 90 was written. The test routine contains three main calls, \verb|anrt.f|,
\verb|eigen.f| and \verb|rmode.f|. 
The analytic radial mode shape is of the form,

\begin{equation}
    R(k_{r,mn} r) =  A J_m (k_{r,mn} r) + B Y_m (k_{r,nmn} r)
\end{equation}
The key to the numerical procedure is the following ``transcendental'' equation,
\begin{equation}
    \begin{vmatrix}
        J'_m(k_{r,mn} r_H) &Y'_m(k_{r,mn} r_H) \\
        J'_m(k_{r,mn} r_T) &Y'_m(k_{r,mn} r_T) 
    \end{vmatrix}   
    = 0
\end{equation}
 The non-dimensional roots $k_{r,mn} r_T$ are found using initial guess and then incrementing from there

 \begin{equation}
     k_{r,mn} = 
    \begin{cases}
        m & \text{if } n = 1\\
        k_{r,m(n-1)} r_T + \pi,              & \text{if }  n > 1.
    \end{cases}
 \end{equation}

 The estimate is refined by incrementing the value of $k_{r,mn}$ by $\pi/10$ until
 the determinant above changes sign. The step size is then halved and also changes 
 sign. This iterative process continues until the absolute value of the determinant 
 is reduced to a preassigned value (error tolerance?). The non dimensional versions
 of these equations are used in the FORTRAN 77 Code.

 Once $k_{r,mn}$ has been computed, the weighting factors $A$ and $B$ are assigned to 
 one of the following two sets of values ( \verb|eigen.f|)


 The remainder of the procedure is being documented along with directions on 
 how to pass in the correct inputs to the F77 calls
  
\section{Issues and Concerns}
Preliminary results appear to not have zero value derivatives at the boundaries
since there is an apparent non zero slope. The hypothesis is that the 
bessel functions are being truncated and not scaled. 

\section{Planned Research}
\begin{itemize}
    \item Complete investigation and documention on the numerical method.
    \item writing clear directions and examples will hopefully help me determine the 
        correct annular mode shapes.
\end{itemize}
\end{document}


