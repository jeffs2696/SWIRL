
% \section{Steady Flow Aerodynamic Model}
% \section{Modal Propagation Theroetical Background}

\subsection{Governing Equations for Compressible, Inviscid Flow}

The governing equations for an ideal polytropic gas with density, $\rho$, velocity, $\vec{V}$, and 
pressure, $p$ describe the conservation of mass, momentum, and energy for a 
given domain in Equations (\ref{eqn:CompressibleConservationOfMass},
\ref{eqn:CompressibleConservationOfMomentum},
\ref{eqn:CompressibleConservationOfEnergy}) respectively. 

\begin{align}
    \frac{D\rho}{Dt} = - \rho \nabla \cdot \vec{V} 
    \label{eqn:CompressibleConservationOfMass} \\
    \frac{D\vec{V}}{Dt} = - \frac{\nabla p}{\rho} + \vec{g} 
    \label{eqn:CompressibleConservationOfMomentum} \\
    \frac{Dp}{Dt} = - \gamma p \nabla \cdot \vec{V} 
    \label{eqn:CompressibleConservationOfEnergy} \\
\end{align}

where,
\begin{equation}
    \frac{D}{Dt} = \frac{\partial }{\partial t} + \nabla \cdot \vec{V}
    \label{eqn:MaterialDerivative}
\end{equation}

For this model, the domain is assumed to be uniformly cylindrical. Therefore the
flow is assumed to be asymmetric, then the radial velocity component is 
zero. With this considered, the velocity vector ,$\vec{V}$ in 
cylindrical coordinates become,
\begin{align}
    \vec{V}(r,\theta,x) &= v_x(r) \hat{e}_x + v_{\theta} (r) \hat{e}_{\theta} 
    \label{eqn:VelocityVector}
\end{align}
  
where $\hat{e}_x$ and $\hat{e}_{\theta}$ are unit vectors for the axial and 
tangential directions.The gradient operator ,$\nabla$ in cylindrical
coordinates, is 

\begin{align}
	\vec{\nabla} 
	&= \hat{e}_r \frac{\partial} {\partial r} + 
	\frac{1}{r} \hat{e}_{\theta}   
	\frac{\partial} {\partial \theta} + 
	\frac{\partial }{\partial z} \hat{e}_z = 0
    \label{eqn:NablaInCylindrical}
\end{align}

Expanding equations
(
\ref{eqn:CompressibleConservationOfMass},
\ref{eqn:CompressibleConservationOfMomentum},
\ref{eqn:CompressibleConservationOfEnergy}
) with the corresponding cylindrical expressions become, 

\begin{align}
\frac{\partial \rho}{\partial t} + %Conservation of mass
v_r \frac{\partial \rho}{\partial r} +
\frac{v_{\theta}   }{r}
\frac{\partial \rho}{\partial \theta} +
v_x \frac{\partial \rho}{\partial \theta} + 
\rho 
\left(
\frac{1}{r} \frac{\partial (rv_r)	}{\partial r} +
\frac{1}{r}	\frac{\partial v_{\theta}}{\partial \theta} +
\frac{\partial v_x}{\partial x}
\right) 
&= 0 \\% \label{ConservationOfMass} %%%%%%%%%%%%%%%%%%%%%%%%%%%%%%%%%%%%%%
\frac{\partial v_r}{\partial t} + 
v_r \frac{\partial v_r}{\partial r} +
\frac{v_{\theta}  }{r}
\frac{\partial v_r}{\partial \theta}- \frac{v_{\theta}^2}{r}+ 
v_x \frac{\partial v_r}{\partial x} 
&= -\frac{1}{\rho} \frac{\partial p}{\partial r}\\  
\frac{\partial v_{\theta}}{\partial t} + 
v_r \frac{\partial v_{\theta}}{\partial r} +
\frac{v_{\theta}}{r}
\frac{\partial v_{\theta}}{\partial \theta} +
\frac{v_r v_{\theta}}{r}+ 
v_x \frac{\partial v_{\theta}}{\partial x} 
&= -\frac{1}{\rho r} \frac{\partial p}{\partial \theta}\\ 
\frac{\partial v_{x}}{\partial t} + 
v_r 
\frac{\partial v_x}{\partial r} +
\frac{v_{\theta}}{r}
\frac{\partial v_x}{\partial \theta}+ 
v_x \frac{\partial v_x}{\partial x} 
&= 
-\frac{1}{\rho } 
\frac{\partial p}{\partial x}\\  
\frac{\partial p }{\partial t} +
v_r 
\frac{\partial p}{\partial r} +
\frac{v_{\theta}}{r}
\frac{\partial p}{\partial \theta} +
v_x \frac{\partial p}{\partial \theta} + 
\gamma p 
\left(
\frac{1}{r}\frac{\partial (rv_r)}{\partial r} +
\frac{1}{r}\frac{v_{\theta}}{\partial \theta} +
\frac{\partial v_x}{\partial x}
\right) &= 0
\end{align}

% For a steady flow, ($\partial/\partial t = 0$) , the compressible flow equations
% can be further reduced to,

% \begin{align}
%     \nabla (\vec{V} \rho) &=  0 \\
%     (\vec{V}\cdot \nabla) \vec{V} &=  0\\
%     \nabla S &= 0
% \end{align}

% where $S$ represents the entropy in the mean flow.

%\[\]
% Neglecting gravity and expanding and how is e reLated to p\ldotW

%The governing equations for an isentropic ideal gas are the conservation of 
%mass, momentum and energy respectively along with the the constitutive relation 
%for the speed of sound. For a cylindrical duct, the coordiate system consists
%of a radial, tangential and axial components.  

%% The following assumptions are utlized to simplify the model 

%% \begin{itemize}
%%     \item Steady flow does not fluctuate in time $\partial / \partial t = 0$
%% \end{itemize}

%% \begin{equation}
%%     \vec{V} \bar{\rho} = 0
%%     \label{eqn:consOfMass}
%% \end{equation}

%The following assumptions to simplify the aerodynamic model for the steady mean
%flow case,

%\begin{itemize}
%    \item No flow in the radial direction. Consequentially, the flow is 
%        axisymmetric along the downstream direction.
%    \item No surface or body forces
%    \item Isentropic conditions 
%\end{itemize}

%\subsubsection{Steady Flow}
%For steady flow, the continuity, momentum and entropy equations are

%\begin{align}
%    \nabla (\vec{V} \bar{\rho}) &=  0 \\
%(\vec{V}\cdot \nabla) \vec{V} &=  0\\
%\nabla S = 0
%\end{align}
%\[\]
 
%%Starting with the radial momentum equation, 
%%\begin{align*}
%%\frac{\partial v_r}{\partial t} + 
%%v_r \frac{\partial v_r}{\partial r} +
%%\frac{v_{\theta}  }{r}
%%\frac{\partial v_r}{\partial \theta}- \frac{v_{\theta}^2}{r}+ 
%%v_x \frac{\partial v_r}{\partial x} 
%%&= -\frac{1}{\rho} \frac{\partial p}{\partial r} 
%%\end{align*}
%%See Appendix for speed of sound derivation

