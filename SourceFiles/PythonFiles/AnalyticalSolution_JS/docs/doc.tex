%        File: doc.tex
%     Created: Mon Aug 08 10:00 AM 2022 E
% Last Change: Mon Aug 08 10:00 AM 2022 E
%
\documentclass[a4paper]{report}
\maxdeadcycles=200
\usepackage[section]{placeins}
\usepackage{afterpage}
\usepackage{mathtools}
\usepackage{amsmath}
\usepackage{pgffor}
\usepackage{booktabs}
\begin{document}
\begin{titlepage}
    \begin{center}
        \vspace*{1cm}

        \textbf{ Validation using the %
            Analytical Solution for Duct Mode Propagation in %
        Uniform Flow} 

        \vspace{0.5cm}
        Swirl Validation

        \vspace{1.5cm}

        \textbf{Jeff Severino}

        \vfill


        This document shows the analytical duct mode solution as well as a
        numerical comparison.
        \vspace{0.8cm}

        % \includegraphics[width=0.4\textwidth]{university}

        Mechanical, Industrial, Manufacturing Engineering Department\\
        University Of Toledo\\
        Toledo, OH\\
        \today

    \end{center}
\end{titlepage}

% \section{Introduction - Turbomachinery Noise}
% Turbomachinery noise generation occurs from pressure fluctuations from the series 
% of fans within it's annular duct. While the jet that is produced from this stream
% of air freely radiates to the observer, the pressure fluctuations 
% produced from the rotor may or may not propagate out of the inlet and exhaust and 
% radiate to the observer. The production of this propagation can be characterized
% by standing waves referred to as modes, in particular, duct modes because 
% the mode itself is dependent on the geometry of the column of air within the 
% annular duct, as well as the speed of the flow moving through it


% This document will provide the fundamental equations that describe sound propagation
% in ducted flow.

% \begin{itemize}
%     \item Introduce the governing equations for compressible inviscid flow and 
%         the assumptions used to obtain them
%     \item Demonstrate the linearization process and show the general equations
%         that describe duct modes.  
%     \item The analytic solution for cylindrical and annular ducts with and
%         without liner will be presented and used for validation against
%         numerical approximations.
%  \end{itemize}
% \newpage
% \section{Current Research Direction}
\section{Research Performed}
\section{Results}

\begin{table}
    % \centering
    \makebox[\linewidth]{
        \begin{tabular}{lrrrrrrr}
 & $L_{2,k_x}$ & $L_{2,\bar{p}}$ & $L_{2,noLBC}$ & $L_{2,noRBC}$ & $L_{2,noBCS}$ & $L_{max}$ & $L_{max,location}$ \\
0 & 0.00000155279 & 0.00059786297 & 0.00019688319 & 0.00059696793 & 0.00016546650 & 0.00324886266 & 0 \\
1 & 0.00000028328 & 0.00010577879 & 0.00004184232 & 0.00010658661 & 0.00004216087 & 0.00079036919 & 0 \\
2 & 0.00000002127 & 0.00001964476 & 0.00000996508 & 0.00001971142 & 0.00000998706 & 0.00019476230 & 0 \\
3 & 0.00000000162 & 0.00000384159 & 0.00000243557 & 0.00000384595 & 0.00000243556 & 0.00004833159 & 0 \\
4 & 0.00000000012 & 0.00000079773 & 0.00000060218 & 0.00000079784 & 0.00000060188 & 0.00001203774 & 0 \\
5 & 0.00000000001 & 0.00000017589 & 0.00000014971 & 0.00000017586 & 0.00000014965 & 0.00000300377 & 0 \\
\end{tabular}

    }
    \caption{L2 error for downstream radial mode 1}
\end{table}
\begin{table}
    \makebox[\linewidth]{
        \begin{tabular}{lrrrrrrr}
 & $L_{2,k_x}$ & $L_{2,\bar{p}}$ & $L_{2,noLBC}$ & $L_{2,noRBC}$ & $L_{2,noBCS}$ & $L_{max}$ & $L_{max,location}$ \\
0 & 0.00000155279 & 0.00059786297 & 0.00019688319 & 0.00059696793 & 0.00016546650 & 0.00324886266 & 0 \\
1 & 0.00000028331 & 0.00010577879 & 0.00004184232 & 0.00010658661 & 0.00004216087 & 0.00079036919 & 0 \\
2 & 0.00000002131 & 0.00001964476 & 0.00000996508 & 0.00001971142 & 0.00000998706 & 0.00019476231 & 0 \\
3 & 0.00000000161 & 0.00000384159 & 0.00000243557 & 0.00000384595 & 0.00000243556 & 0.00004833159 & 0 \\
4 & 0.00000000011 & 0.00000079773 & 0.00000060218 & 0.00000079784 & 0.00000060188 & 0.00001203774 & 0 \\
5 & 0.00000000001 & 0.00000017589 & 0.00000014971 & 0.00000017586 & 0.00000014965 & 0.00000300377 & 0 \\
\end{tabular}

    }
    \caption{L2 error for upstream radial mode 1}
\end{table}

\begin{table}
    \makebox[\linewidth]{
        \begin{tabular}{lrrrr}
 & 0 & 1 & 2 & 3 \\
0 & 2.540732 & 2.921466 & 2.532691 & 2.436520 \\
\end{tabular}

    }
    \caption{ROC error for downstream radial mode 1}
\end{table}
\begin{table}
    \makebox[\linewidth]{
        \begin{tabular}{lrrrr}
 & 0 & 1 & 2 & 3 \\
0 & 2.540732 & 2.921466 & 2.532691 & 2.436520 \\
\end{tabular}

    }
    \caption{ROC error for upstream radial mode 1}
\end{table}

% legend \\
% 0 - axial wavenumber error  \\
% 1 - pressure mode L2 \\
% 2 - pressure mode L2 without left \\
% 3 - pressure mode L2 without right   \\
% 4 - pressure mode L2 without boundaries   \\
% 5 - pressure mode LMax  \\
% 6 - pressure mode LMax location \\

% legend \\
% 0 - axial wavenumber ROC \\
% 1 - pressure mode ROC \\
% 2 - pressure mode ROC without left \\
% 3 - pressure mode ROC without right   \\
% 4 - pressure mode ROC without boundaries  \\
% 6 - pressure mode LMax ROC

\subsection{Discussion}

A total of five grids were studied, starting from 33 and doubling until 1056 for
the first cut on radial mode. The upstream and downstream mode pair were both used
to compute the L2, Lmax, and rate of convergence for a given uniform flow in 
a cylindrical duct. The axial wavenumber's error starts with a magnitude of $e-6$
and decreases to $e-12$. When computing the convergence rate, a fourth order is 
expected since a fourth order central scheme was used. This behavior is shown in
the upstream mode, with a rate of convergence of $~4.19$. However, this behaviour
is less pronounced in the downstream case. The rate increases to $3.7$ but begins
to decrease for the last grid pair. A grid spacing choice should have been made
such that the jumps are not as large so that the convergence rates can be 
studied more carefully.

For the pressure mode data, neither upstream or downstream reach fourth order convergence.
The use of the L'Hopital's rule to obtain a value at the centerline causes the 
error between the numerical and analytic modes to be highest at the centerline
gridpoint. This is shown in both upstream and downstream directions by identifying
the location of the highest error $L_{max}$. When computing the rate of convergence
for the $L_{max}$ point, it converges to second order. The same trend is noticed when 
both boundaries are removed in the error calculation. 

\section{Issues}
When looking at the difference between the excluding of boundaries, it seems that 
the convergence rate is higher without the wall BC as opposed to being higher without 
the centerline BC, but when computing the convergence rate of $LMax$, a second
order rate is computed. I could have the two collumns swapped. 

\section{Planned Research}
The formatting of the latex table doesn't show the Floating point values I am seeing but
discussing.

I only looked at the first radial mode at the moment. Once this study can be achieved 
for multiple radial modes simultaneously, the annular duct mode case will be studied
using the same outline.

% \begin{tabular}{lrrrrrrr}
 & $L_{2,k_x}$ & $L_{2,\bar{p}}$ & $L_{2,noLBC}$ & $L_{2,noRBC}$ & $L_{2,noBCS}$ & $L_{max}$ & $L_{max,location}$ \\
0 & 0.00000155279 & 0.00059786297 & 0.00019688319 & 0.00059696793 & 0.00016546650 & 0.00324886266 & 0 \\
1 & 0.00000028331 & 0.00010577879 & 0.00004184232 & 0.00010658661 & 0.00004216087 & 0.00079036919 & 0 \\
2 & 0.00000002131 & 0.00001964476 & 0.00000996508 & 0.00001971142 & 0.00000998706 & 0.00019476231 & 0 \\
3 & 0.00000000161 & 0.00000384159 & 0.00000243557 & 0.00000384595 & 0.00000243556 & 0.00004833159 & 0 \\
4 & 0.00000000011 & 0.00000079773 & 0.00000060218 & 0.00000079784 & 0.00000060188 & 0.00001203774 & 0 \\
5 & 0.00000000001 & 0.00000017589 & 0.00000014971 & 0.00000017586 & 0.00000014965 & 0.00000300377 & 0 \\
\end{tabular}

% \begin{tabular}{lrrrr}
 & 0 & 1 & 2 & 3 \\
0 & 2.540732 & 2.921466 & 2.532691 & 2.436520 \\
\end{tabular}

% \section{Theoretical Foundation for Duct Acoustics based on first principles}

% The pressure field within a duct is governed by the convective wave equation, a
% second order ODE as a function of radius. 


% The solution of the convective wave equation are eigenvalues and eigenvectors 
% which may or may not correspond to acoustic disturbances fall into two groups.  
% One group corresponding to the acoustics propagation and the other group 
% corresponding to the convection speed of the flow. Both are modes that are a
% result from the pressure distribution from within the cylindrical domain.  

% 
\section{Analytical Solution to Sound Propagation in ducted flows}
\subsection{Introduction}
The steps to get up to this point are described in the appendix, but for the
purposes of understanding this repository and its functions this document will 
start with the Pridmore-Brown equation. The general solution of this differential equation
will be assumed to be harmonic and used to present a eigenvalue problem which is
outlined here.    


\begin{equation}
    \frac{1}{A^2}\frac{D^2\tilde{p}}{Dt^2} -
    \nabla^2 \tilde{p} =
    2 \bar{\rho} \frac{d V_x}{d x} \frac{\partial  \tilde{v}_r}{ \partial x} 
    \label{eqn:KousensWaveEquation}
\end{equation}


Substituting the definitions for $\nabla$ and $\nabla^2$ and setting $\vec{V} =0$
in cylindrical coordinates gives,

\begin{align*} 
    \frac{1}{A^2}\left(
        \frac{\partial^2 \tilde{p}}{\partial t^2}
    + 
        \vec{V}\cdot \left(
            \frac{\partial\tilde{p}}{\partial t} + 
            \frac{1}{\tilde{r}}\frac{\partial \tilde{p} }{\partial \tilde{r}} +
            \frac{\partial \tilde{p}}{\partial \theta} +
            \frac{\partial \tilde{p}}{\partial x}  
        \right)  \right)-
        \left(
            \frac{\partial^2 \tilde{p}}{\partial t^2} + 
            \frac{1}{\tilde{r}}\frac{\partial \tilde{p}}{\partial r} +
            \frac{1}{\tilde{r}^2} \frac{\partial^2 \tilde{p}}{\partial \theta^2} + 
            \frac{\partial^2 \tilde{p}}{\partial x^2} 
        \right) &= 0  
\end{align*} 
Setting $\vec{V} = 0$,

\begin{align*} 
    \frac{1}{A^2}\left(
        \frac{\partial^2 \tilde{p}}{\partial t^2}
    \right) - 
        \left(
            \frac{\partial^2 \tilde{p}}{\partial t^2} + 
            \frac{1}{\tilde{r}}\frac{\partial \tilde{p}}{\partial  r}  +
            \frac{1}{\tilde{r}^2} \frac{\partial^2 \tilde{p}}{\partial \theta^2} + 
            \frac{\partial^2 \tilde{p}}{\partial x^2} 
        \right) &= 0  
\end{align*} 
Utilizing the relation, $\tilde{p} = p/\bar{\rho} A^2$,

\begin{align*} 
    \frac{1}{A^2}\left(
        \frac{\partial^2 {p}}{\partial t^2}
    \right) - 
        \left(
            \frac{\partial^2 {p}}{\partial t^2} + 
            \frac{1}{\tilde{r}}\frac{\partial p}{\partial r} +
            \frac{1}{\tilde{r}^2} \frac{\partial^2 p}{\partial \theta^2} + 
            \frac{\partial^2 p}{\partial x^2} 
        \right) &= 0  
\end{align*} 

The method of separation of variables requires an assumed solution as well as initial and boundary 
conditions. For a partial differential equation, the assumed solution can be a 
linear combination of solutions to a system of ordinary differential equations that
comprises the partial differential equation. Since $p$ is a function of four
variables, the solution is assumed to be a linear combination of four solutions.
Each solution is assumed to be Euler's identity, a common ansant for linear partial 
differential equations and boundary conditions.

\begin{equation}
    p(x,r,\theta,t) = X(x) R(r) \Theta(\theta) T(t)
\end{equation}

where, 

\begin{align*}
    X(x) &=
    A_1 e^{ik_x x} +
    B_1 e^{-ik_x x }\\
    \Theta(\theta) &=
    A_2 e^{i k_{\theta} \theta } +
    B_2 e^{-ik_{\theta} \theta }\\
    T(t) &=
    A_3 e^{i \omega t } +
    B_3 e^{-i\omega t  }
\end{align*}

and $A_1$,$A_2$,$A_3$,$B_1$,$B_2$,$B_3$ are arbitrary constants.  The next step 
is to rewrite Equation \ref{eqn:KousensWaveEquation} in terms of $X$, $R$, $\Theta$,
and $T$. To further simplify the result, each term is divided by $p$.
Before the substitution, the derivatives of the assumed solutions need to be
evaluated. The simplification of these derivatives are included in the Appendix.



\begin{equation}
    \frac{1}{A^2} \frac{1}{T}\frac{\partial^2 T}{\partial t^2} = 
    \frac{1}{R}\frac{\partial^2 R}{\partial r^2 } +
    \frac{1}{r}\frac{1}{R}\frac{\partial R}{\partial r}  + 
    \frac{1}{r^2}\frac{1}{\Theta}\frac{\partial \Theta}{\partial \theta} + 
    \frac{1}{X}\frac{\partial^2 X}{\partial x^2}
    \label{eqn:waveode}
\end{equation}

Notice that each term is only a function of its associated independent variable.
So, if we vary the time, only the term on the left-hand side can vary. However,
since none of the terms on the right-hand side depend on time, that means the
right-hand side cannot vary, which means that the ratio of time with its second
derivative is independent of time. The practical upshot is that each of these 
terms is constant, which has been shown. The wave numbers are the \textit{separation constants} 
that allow the PDE to be split into four separate ODE's. Substituting the separation constants 
into Equation (\ref{eqn:waveode}) gives, 


% \begin{equation}
%     -\frac{\omega^2}{A^2}  = 
%     \frac{1}{R}
%     \left(      
%     \frac{\partial^2 R}{\partial r^2 } +
%     \frac{1}{r}\frac{\partial R}{\partial r}  
% \right) -
%     \frac{k_{\theta}^2}{r^2}-  
%     k_x^2
%     \label{eqn:waveode2}
% \end{equation}
% where$\omega = k A$, i.e. the  dispersion relation , 

\begin{equation}
    \frac{1}{R}
    \left(      
    \frac{\partial^2 R}{\partial r^2 } +
    \frac{1}{r}\frac{\partial R}{\partial r}  
\right) -
    \frac{k_{\theta}^2}{r^2}-  
    k_x^2 + k^2 = 0
    \label{eqn:waveode3}
\end{equation}
The remaining terms are manipulated to follow the same form as \textit{Bessel's Differntial 
Equation} ,

% \begin{equation}
%     x^2 \frac{d^2 y}{dx^2} + x \frac{dy }{dx } + (x^2 - n^2) y = 0
%     \label{eqn:besselODE}
% \end{equation}

The general solution to Bessel's differential equation is a linear combination of
the Bessel functions of the first kind, $J_n(k_r r)$ and of the second kind, $Y_n(k_r r)$ 
\cite{wolphram:bessel}. The subscript $n$ refers to the order of Bessel's equation.

\begin{equation}
    R(r) = (AJ_n(k_r r) + BY_n(k_r r)) 
    \label{eqn:besselsolution}
\end{equation}
where the coefficients $A$ and $B$ are found after applying radial
boundary conditions. %and there is an exponential dependence. 

By rearranging Equation (\ref{eqn:waveode3}), a comparison can be made to Equation
(\ref{eqn:besselODE}) to show that the two equations are of the same form. 

The first step is to revisit the radial derivatives that have not been addressed.
As was done for the other derivative terms, the radial derivatives will also 
be set equal to a separation constant, $-k_r^2$. 

\begin{align}
    \underbrace{\frac{1}{R}
    \left(      
    \frac{\partial^2 R}{\partial r^2 } +
    \frac{1}{r}\frac{\partial R}{\partial r}  
\right) -
    \frac{k_{\theta}^2}{r^2}}_{-k_r^2}-  
    k_x^2 + k^2 = 0
    \label{eqn:wavenumber_without_kr}
\end{align}

% The reader may be curious as to why the tangential separation constant $k_{\theta}$ is 
% included within the definition of the radial separation constant. 

% Recall the ODE for the tangential direction, 

% \begin{align*}
%     \frac{\partial \Theta}{\partial \theta} \frac{1}{\Theta} = - k_{\theta}^2\\
%     \frac{\partial \Theta}{\partial \theta} \frac{1}{\Theta} + \Theta k_{\theta}^2 = 0 
% \end{align*}

% where the solution is more or less,

% \begin{align*}
%     \Theta(\theta) = e^{i k_{\theta} \theta}
% \end{align*}

% In order to have non trivial, sensible solutions, the value of $\Theta(0)$ and
% $\Theta(2\pi)$ need to be the same, and this needs to be true for any multiple 
% of $2\pi$ for a fixed r. Taking $\Theta$ to be one, a unit circle, it can be shown that the domain
% is only going to be an integer multiple. Therefore, there is an implied periodic
% azimuthal boundary condition, i.e. $0<\theta\leq 2 \pi$ and $k_{\theta}=m$. 

% There is a unique treatment for the radial derivatives.


% \begin{align*}
%     -k_r^2 =\frac{1}{R}
%     \left(      
%     \frac{\partial^2 R}{\partial r^2 } +
%     \frac{1}{r}\frac{\partial R}{\partial r}  
% \right) -
%     \frac{m^2}{r^2} 
% \end{align*}
% To further simplify, the chain rule is used to do a change of variables, $x = k_r r$
% \begin{align*}
%     \frac{\partial R}{\partial r} &= \frac{dR}{dx}\frac{dx}{dr}\\
%     &=
%     \frac{dR}{dx}\frac{d}{dr}\left( k_r r \right) \\
%     &=
%     \frac{dR}{dx} k_r 
% \end{align*} 


% \begin{align*}
%     \frac{\partial^2 R}{\partial r^2} &= \frac{d^2R}{dx^2}\left(\frac{dx}{dr}\right)^2 + 
%     \frac{dR}{dr}\frac{d^2x}{dr^2}\\
%     &=
%     \frac{d^2R}{dx^2}\frac{d}{dr} k_r^2 + k_r \frac{d^2r}{dr^2}\\
%     &=
%     \frac{d^2R}{dx^2}\frac{d}{dr} k_r^2
% \end{align*} 

% Substituting this into Equation (\ref{eqn:waveode3}),
% \begin{equation}
%     \left(\frac{d^2R}{dx^2}k_r^2 +
%     \frac{1}{r}\frac{d^2R}{dx^2}k_r\right) +
%     \left(k_r^2 - \frac{m^2}{r^2}\right)R
%     \label{eqn:waveode4}
% \end{equation}
% Dividing Equation \ref{eqn:waveode4} by $k_r^2$,

% \begin{equation}
%     \left(\frac{d^2R}{dx^2} +
%     \frac{1}{k_r r}\frac{d^2R}{dx^2}\right) +
%     \left(1  - \frac{m^2}{k_r^2 r^2}\right)R
%     \label{eqn:waveode5}
% \end{equation}

% \begin{equation}
%     \left(\frac{d^2R}{dx^2} +
%     \frac{1}{x^2}\frac{d^2R}{dx^2}\right) +
%     \left(1  - \frac{m^2}{x^2}\right)R
%     \label{eqn:waveode6}
% \end{equation}

% Multiplying Equation (\ref{eqn:waveode6}) by $x^2$ gives,

% \begin{equation}
%     \frac{d^2R}{dr^2}x^2 + 
%     \frac{dR}{dr}x + 
%     \left( x^2 - m^2 \right)R
%     \label{eqn:finalradialode}
% \end{equation}
% which matches the form of Bessel's equation

% Therefore, the solution goes from this,
% \begin{equation}
%     y(x) = AJ_n(x) + BY_n(x)
%     \label{eqn:besselsolution}
% \end{equation}
% to this,



\subsubsection{Hard Wall boundary condition}
\begin{align*}
    \frac{\partial p}{\partial r}|_{r = r_{min}}  =\frac{\partial p}{\partial r}|_{r = r_{max}} = 0 \rightarrow 
    \frac{\partial}{\partial r} \left( X\Theta T R \right) &= 0 \\
    X \Theta T\frac{\partial R}{\partial r}  &= 0 \\
    \frac{\partial R}{\partial r}  &= 0 
\end{align*}

where,


\begin{align*} 
    \frac{ \partial R}{\partial r}|_{r_{min}} &= AJ_n'(k_r r_{min}) + B Y_n' (k_r r_{min}) = 0 
    \rightarrow B = -A \frac{J_n'(k_r r_{min})}{Y_n'(k_r r_{min})}
\end{align*}


\begin{align*} 
    \frac{ \partial R}{\partial r} &= AJ_n'(k_r r_{max}) + B Y_n' (k_r r_{max}) = 0 \\
                                   &= AJ_n'(k_r r_{max}) - A\frac{J_n' (k_r r_{min})}{Y_n'(k_r r_{min})} Y_n' (k_r r_{max}) = 0 \\
                                   &= \frac{J_n'(k_r r_{min})}{J_n' (k_r r_{max})} - \frac{Y_n'(k_r r_{min})}{Y_n' (k_r r_{max})} = 0 
\end{align*}
where $k_r r$ are the zero crossings for the derivatives of the Bessel functions of the first and second kind.

In summary, the wave equation for no flow in a hollow duct with hard walls is obtained 
from Equation (\ref{eqn:wavenumber_without_kr}).
\begin{equation}
    k^2 = k_r^2 + k_x^2
    \label{eqn:wavenumber_equation}
\end{equation}


% Solving for the axial wavenumber gives,
% \section{Uniform Flow}
Following the same procedure, the axial wavenumber is,
% To get the same equation but for uniform flow, the same procedure can be followed.

% Starting with Equation 2.27 redimensionalized, 

% \begin{align*}
%     \frac{ d^2 \tilde{p}}{d \tilde{r}^2} +
%     \frac{1}{\tilde{r}} 
%     \frac{d \tilde{p}}{d \tilde{r}} + 
%     \frac{2 \bar{\gamma} \left( \frac{d m_x}{d \tilde{r}} \right)}
%     {\left( k - \bar{\gamma} m_x \right)}\frac{d \tilde{p}}{d \tilde{r}}+
%     \left[ \left( k - \bar{\gamma} m_x \right)^2 - \frac{m^2}{\tilde{r}^2}- 
%     \bar{\gamma}^2 \right] \tilde{p}
% \end{align*}

% Let's separate the new terms from the old ones, 

% \begin{align*}
%     \frac{ d^2 \tilde{p}}{d \tilde{r}^2} +
%     \frac{1}{\tilde{r}} 
%     \frac{d \tilde{p}}{d \tilde{r}} + 
%     \frac{2 \bar{\gamma} \left( \frac{d m_x}{d \tilde{r}} \right)}
%     {\left( k - \bar{\gamma} m_x \right)}\frac{d \tilde{p}}{d \tilde{r}}+
%     \left[ \left( k - \bar{\gamma} m_x \right)^2 - \frac{m^2}{\tilde{r}^2}- 
%     \bar{\gamma} \right] \tilde{p}
% \end{align*}


% Recalling the non-dimensional definitions,
% \begin{align*}
%     \tilde{p} &= \frac{p}{\bar{\rho} A^2} \\
%     \tilde{r} &= \frac{r}{r_T} \\
%     \frac{\partial \tilde{p}}{\partial \tilde{r}} &= 
%     \frac{ \partial \tilde{p}}{\partial r} \frac{\partial r}{ \partial \tilde{r}}  \\ 
%     &= \frac{ \partial \tilde{p}}{\partial r} \frac{\partial }{ \partial \tilde{r}} \left( \tilde{r} r_T \right) \\
%     &= 
%     \frac{ \partial \tilde{p}}{\partial r}  r_T \\
%     \frac{\partial^2 \tilde{p}}{\partial \tilde{r}^2} &= 
%     \frac{ \partial^2 \tilde{p}}{\partial r^2}  (r_T)^2+ 
%     \frac{ \partial \tilde{p}}{\partial r} \frac{\partial^2 r}{ \partial \tilde{r}^2} \\
%     &= \frac{ \partial^2 \tilde{p}}{\partial r^2}  (r_T)^2 
% \end{align*}

% \begin{align*}
%     \frac{\partial}{\partial r} \left( \frac{p}{\bar{\rho} A^2} \right) 
%     &=
%     \frac{\left(\frac{\partial}{\partial r} \left(  p\right) \bar{\rho} A^2 - 
%     \underbrace{\frac{\partial \bar{\rho}A^2}{\partial r}}_0 p \right)}{\left( \bar{\rho} A^2 \right)^2}\\ 
%     &= \frac{1}{\bar{\rho}A^2} \frac{\partial p}{\partial r}
% \end{align*}

% \begin{align*}
%     \frac{ d^2 \tilde{p}}{d \tilde{r}^2} +
%     \frac{1}{\tilde{r}} 
%     \frac{d \tilde{p}}{d \tilde{r}}- 
%     \frac{m^2}{\tilde{r}^2}\tilde{p}- 
%     \bar{\gamma}^2  \tilde{p}
%  + 
%     \frac{2 \bar{\gamma} \left( \frac{d M_x}{d \tilde{r}} \right)}
%     {\left( k - \bar{\gamma} M_x \right)}\frac{d \tilde{p}}{d \tilde{r}}+
%     \left( k - \bar{\gamma} M_x \right)^2\tilde{p} 
% \end{align*}

% If there is only uniform flow, then $dM_x/dr = 0$,

% \begin{align*}
%     \frac{ d^2 \tilde{p}}{d \tilde{r}^2} +
%     \frac{1}{\tilde{r}} 
%     \frac{d \tilde{p}}{d \tilde{r}}- 
%     \frac{m^2}{\tilde{r}^2}\tilde{p}- 
%     \bar{\gamma}^2  \tilde{p}
%  + 
%     \left( k - \bar{\gamma} M_x \right)^2\tilde{p} 
% \end{align*}

% Re-dimensionalizing,

% \begin{align*}
%     \frac{1}{\bar{\rho} A^2}\left[
%     \frac{ d^2 p}{d r} r_T^2+
%     \frac{r_T}{r} 
%     \frac{d p}{d r} r_T - 
%     \frac{m^2}{r^2}r_T^2 p - k_x^2r_T^2  p\right]
%     + \left( \frac{\omega }{A}r_T - k_x r_T M_x \right)^2p 
% \end{align*}

% Expanding the last term and substituting $\omega/A = k$

% \begin{align*}
%     \frac{1}{\bar{\rho} A^2}\left[
%     \frac{ d^2 p}{d r} r_T^2+
%     \frac{r_T}{r} 
%     \frac{d p}{d r} r_T - 
%     \frac{m^2}{r^2}r_T^2 p - k_x^2r_T^2  p\right]
%     +\left( r_T^2\left(
%         k^2 - 2 k k_x M_x + k_x^2 M_x^2 \right)
%     \right)p 
% \end{align*}
% Canceling out $r_T/\bar{\rho}A$ in every term


% \begin{align*}
%     \frac{ d^2 p}{d r} +
%     \frac{1}{r} 
%     \frac{d p}{d r} + \left[ 
%     k^2 - 2 k k_x M_x + k_x^2 M_x^2- \frac{m^2}{r^2}  - k_x^2\right]p 
% \end{align*}

% Continue here,


% Defining 

% $$- N^2 = k_x^2 M_x^2 - 2 k k_x M_x - k_x^2 $$
% $$-N^2 = -(1 -  M_x^2)k_x^2 - 2 k k_x M_x  $$
% $$-N^2 =  -\beta^2 k_x^2 - 2 k k_x M_x  $$


% \begin{align*}
%     \frac{ d^2 p}{d r} +
%     \frac{1}{r} 
%     \frac{d p}{d r} + \left[ 
%     k^2 - N^2 - \frac{m^2}{r^2}  \right]p 
% \end{align*}

% Let $k_r^2 = k^2 - N^2$


% \begin{align*}
%     \frac{ d^2 p}{d r} +
%     \frac{1}{r} 
%     \frac{d p}{d r} + \left[ 
%     k_r^2  - \frac{m^2}{r^2}  \right]p 
% \end{align*}

% Looking at the radial wavenumber,

% \begin{align*}
%     k_r^2 &= k^2 - N^2 \\
%           &= k^2-\beta^2 k_x^2 - 2 k k_x M_x \\
%     0 &=  -\beta ^2 k_x ^2 -  \left( 2M_x k \right)k_x +(k^2 - k_r^2)
% \end{align*}

% Where the roots to this equation are the axial wavenumber,


% Applying the quadratic formula and taking 

% \begin{align*}
%     A &= - \beta^2 \\
%     B &= - 2M_x k\\
%     C &= k^2 - k_2^2
% \end{align*} 

% Note B is negative when $M_x$ is positive,

% (I feel like N should change based on $M_x's$ sign)

\begin{align*}
    k_x &= \frac{2M_x k \pm \sqrt{4 M_x^2 k^2 + 4 \beta^2 \left( k^2 - k_r^2 \right)}}{-2\beta^2}\\
        &= \frac{-M_x k \pm \sqrt{k^2 - k_r^2}}{\beta^2}
\end{align*}

\section{Appendix}
\subsection{Simplification of the Assumed Solution}
\subsubsection{Temporal Derivatives}

\begin{align*}
    \frac{\partial p}{\partial t} 
    &=
    \frac{\partial }{\partial t}  \left( XR\Theta T \right) \\
    &=
    XR\Theta\frac{\partial T}{\partial t}  
\end{align*}


\begin{align*}
    \frac{1}{p}\frac{\partial p}{\partial t} 
    &=
    \frac{ 1}{X R \Theta T}  \left( XR\Theta\frac{\partial T}{\partial t} \right) \\
    &=\frac{ 1}{ T}\frac{\partial T}{\partial t}  
\end{align*}

\begin{align*}
    \frac{\partial^2 p}{\partial t^2} 
    &=
    \frac{\partial^2 }{\partial t^2}  \left( XR\Theta T \right) \\
    &=
    XR\Theta\frac{\partial^2 T}{\partial t^2}  
\end{align*}


\begin{align*}
    \frac{1}{p}\frac{\partial^2 p}{\partial t^2} 
    &=
    \frac{ 1}{X R \Theta T}  \left( XR\Theta\frac{\partial^2 T}{\partial t^2} \right) \\
    &=\frac{ 1}{ T}\frac{\partial^2 T}{\partial t^2}  
\end{align*}

\begin{align*}
    \frac{\partial T}{\partial t} &=
    \frac{\partial}{\partial t}
        \left( 
        A_3 e^{i \omega t} + B_3 e^{-i \omega t}
    \right)  \\
    &=
    \frac{\partial}{\partial t} \left(A_3 e^{i \omega t}  \right) +
    \frac{\partial}{\partial t} \left(B_3 e^{-i \omega t}  \right)\\ 
    &= i \omega A_3 e^{i \omega t} - i \omega B_3 e^{i \omega t} 
\end{align*}

\begin{align*}
    \frac{\partial^2 T}{\partial t^2} &=
    \frac{\partial^2}{\partial t^2}
        \left( 
        i \omega A_3 e^{i \omega t} + i \omega B_3 e^{-i \omega t}
    \right)  \\
    &=
    \frac{\partial^2}{\partial t^2} \left(i \omega A_3 e^{i \omega t}  \right) +
    \frac{\partial^2}{\partial t^2} \left(- i \omega B_3 e^{-i \omega t}  \right)\\ 
    &= (i \omega)^2 A_3 e^{i \omega t} - (i \omega)^2 B_3 e^{i \omega t} 
\end{align*}

\begin{align*}
    \frac{1}{T}\frac{\partial^2 T}{\partial t^2} 
    &=
    (i\omega)^2 \\
    &= -\omega^2
\end{align*}


\subsubsection{Radial Derivatives}
\begin{align*}
    \frac{\partial p}{\partial r} 
    &=
    \frac{\partial }{\partial r}  \left( XR\Theta T \right) \\
    &=
    X\Theta T\frac{\partial R}{\partial r}  
\end{align*}


\begin{align*}
    \frac{1}{p}\frac{\partial p}{\partial r} 
    &=
    \frac{ 1}{X R \Theta T}  \left( X\Theta T\frac{\partial R}{\partial r} \right) \\
    &=\frac{ 1}{ R}\frac{\partial R}{\partial r}  
\end{align*}

\begin{align*}
    \frac{\partial^2 p}{\partial r^2} 
    &=
    \frac{\partial^2 }{\partial r^2}  \left( XR\Theta T \right) \\
    &=
    X\Theta T\frac{\partial^2 R}{\partial r^2}  
\end{align*}


\begin{align*}
    \frac{1}{p}\frac{\partial^2 p}{\partial r^2} 
    &=
    \frac{ 1}{X R \Theta T}  \left( X\Theta T \frac{\partial^2 R}{\partial r^2} \right) \\
    &=\frac{ 1}{ R}\frac{\partial^2 R}{\partial r^2}  
\end{align*}
The radial derivatives will be revisited once the remaining derivatives are evaluated,

\subsubsection{Tangential Derivatives}

\begin{align*}
    \frac{\partial p}{\partial \theta } 
    &=
    \frac{\partial }{\partial t}  \left( XR\Theta T \right) \\
    &=
    XRT\frac{\partial \Theta}{\partial \theta}  
\end{align*}


\begin{align*}
    \frac{1}{p}\frac{\partial p}{\partial \theta} 
    &=
    \frac{ 1}{X R \Theta T}  \left( XR\Theta\frac{\partial T}{\partial \theta} \right) \\
    &=\frac{ 1}{ \Theta}\frac{\partial \Theta}{\partial \theta}  
\end{align*}

\begin{align*}
    \frac{\partial^2 p}{\partial \theta^2} 
    &=
    \frac{\partial^2 }{\partial \theta^2}  \left( XR\Theta T \right) \\
    &=
    XRT\frac{\partial^2 \Theta }{\partial \theta^2}  
\end{align*}


\begin{align*}
    \frac{1}{p}\frac{\partial^2 p}{\partial \theta^2} 
    &=
    \frac{ 1}{X R \Theta T}  \left( XRT\frac{\partial^2 \Theta}{\partial \theta^2} \right) \\
    &=\frac{ 1}{ \Theta}\frac{\partial^2 \Theta}{\partial \theta^2}  
\end{align*}

\begin{align*}
    \frac{\partial \Theta}{\partial \theta} &=
    \frac{\partial}{\partial \theta}
        \left( 
            A_2 e^{i k_{\theta} \theta} + B_2 e^{-i k_{\theta} \theta}
        \right)  \\
    &=
    \frac{\partial}{\partial \theta} \left(A_2 e^{i k_{\theta} \theta}  \right) +
    \frac{\partial}{\partial \theta} \left(B_2 e^{-i k_{\theta} \theta}  \right)\\ 
    &= i k_{\theta} A_2 e^{i k_{\theta} \theta} - i k_{\theta} B_2 e^{i k_{\theta} \theta} 
\end{align*}

\begin{align*}
    \frac{\partial^2 \Theta }{\partial \theta^2} &=
    \frac{\partial^2}{\partial \theta^2}
        \left( 
        i k_{\theta} A_2 e^{i k_{\theta} \theta} - i k_{\theta} B_2 e^{i k_{\theta} \theta} 
    \right)  \\
    &=
    \frac{\partial^2}{\partial \theta^2} \left(i k_{\theta} A_2 e^{i k_{\theta} \theta}  \right) +
    \frac{\partial^2}{\partial \theta^2} \left(- i k_{\theta} B_2 e^{-i k_{\theta} \theta}  \right)\\ 
    &= (i k_{\theta})^2 A_2 e^{i k_{\theta} \theta } - (i k_{\theta})^2 B_2 e^{i k_{\theta} \theta} 
\end{align*}

\begin{align*}
    \frac{1}{\Theta}\frac{\partial^2 \Theta}{\partial \theta^2} 
    &=
    (ik_{\theta})^2 \\
    &= -k_{\theta}^2
\end{align*}

\subsubsection{Axial Derivatives}

\begin{align*}
    \frac{\partial p}{\partial x} 
    &=
    \frac{\partial }{\partial x}  \left( XR\Theta T \right) \\
    &=
    R\Theta T \frac{\partial X}{\partial x}  
\end{align*}


\begin{align*}
    \frac{1}{p}\frac{\partial p}{\partial x} 
    &=
    \frac{ 1}{X R \Theta T}  \left( R\Theta\frac{\partial X}{\partial x} \right) \\
    &=\frac{ 1}{ X}\frac{\partial X}{\partial x}  
\end{align*}

\begin{align*}
    \frac{\partial^2 p}{\partial x^2} 
    &=
    \frac{\partial^2 }{\partial x^2}  \left( XR\Theta T \right) \\
    &=
    R\Theta T \frac{\partial^2 X}{\partial x^2}  
\end{align*}


\begin{align*}
    \frac{1}{p}\frac{\partial^2 p}{\partial x^2} 
    &=
    \frac{ 1}{X R \Theta T}  \left( R\Theta T \frac{\partial^2 X}{\partial x^2} \right) \\
    &=\frac{ 1}{ X}\frac{\partial^2 X}{\partial x^2}  
\end{align*}

\begin{align*}
    \frac{\partial X}{\partial x} &=
    \frac{\partial}{\partial t}
        \left( 
        A_3 e^{i k_x t} + B_3 e^{-i \omega t}
    \right)  \\
    &=
    \frac{\partial}{\partial t} \left(A_1 e^{i k_x x}  \right) +
    \frac{\partial}{\partial t} \left(B_1 e^{-i k_x x }  \right)\\ 
    &= i k_x A_1 e^{i k_x x } - i k_x B_1 e^{i k_x x} 
\end{align*}

\begin{align*}
    \frac{\partial^2 X}{\partial x^2} &=
    \frac{\partial^2}{\partial x^2}
        \left( 
        i k_x A_1 e^{i k_x x} + i k_x B_1 e^{-i k_x x}
    \right)  \\
    &=
    \frac{\partial^2}{\partial x^2} \left(i k_x A_1 e^{i k_x x}  \right) +
    \frac{\partial^2}{\partial x^2} \left(- i k_x B_1 e^{-i k_x x}  \right)\\ 
    &= (i k_x)^2 A_1 e^{i k_x x} - (i k_x)^2 B_1 e^{i k_x x} 
\end{align*}

\begin{align*}
    \frac{1}{X}\frac{\partial^2 X}{\partial x^2} 
    &=
    (i k_x)^2 \\
    &= -k_x^2
\end{align*}

Substituting this back into the Equation \ref{eqn:KousensWaveEquation} yields ,



\begin{align*} 
    \frac{1}{A^2}\left(
        \frac{\partial^2 {p}}{\partial t^2}
    \right) &= 
        \left(
            \frac{\partial^2 {p}}{\partial t^2} + 
            \frac{1}{\tilde{r}}\frac{\partial p}{\partial r} +
            \frac{1}{\tilde{r}^2} \frac{\partial^2 p}{\partial \theta^2} + 
            \frac{\partial^2 p}{\partial x^2} 
        \right) 
\end{align*} 


\begin{equation}
    \frac{1}{A^2} \frac{1}{T}\frac{\partial^2 T}{\partial t^2} = 
    \frac{1}{R}\frac{\partial^2 R}{\partial r^2 } +
    \frac{1}{r}\frac{1}{R}\frac{\partial R}{\partial r}  + 
    \frac{1}{r^2}\frac{1}{\Theta}\frac{\partial \Theta}{\partial \theta} + 
    \frac{1}{X}\frac{\partial^2 X}{\partial x^2}
    \label{eqn:waveode}
\end{equation}


%\section{Annular Duct Axial Wavenumber solution}



%This needs to be proven,


%In \cite{Amr2001}, the axial wavenumber for annular ducts is reported,

%\begin{equation*}
%    \frac{-(\omega - m M_{\theta})M_x \pm \sqrt{\left( \omega - m M_{\theta}^2 \right) - \beta\left( m^2 + \Gamma_{m,n} ^2 \right)}}{beta^2}
%\end{equation*}

%where 

%$\Gamma_{m,n}= \frac{n^2 \pi^2}{\left( r_{max} - r_{min} \right)^2}$

%%$\bibliographystyle{plain}
%%\bibliography{references}

% \subsection{ Analytic Solution: Axial wavenumbers and Pressure Modes}

% Modes can be categorized based on the sign of the axial wavenumber and if it is
% complex in value. For example, for the uniform axial flow case, propagating modes
% are defined by axial wavenumbers, $k_x$, that have a real-part only, yielding 
% the assumed fluctuation to resemble Euler's Formula ($e^{ik_x x}$). On the other 
% hand, if the $k_x$ is complex, then the mode will resemble an exponentially decaying
% function since the imaginary number cancels, leaving a minus sign in front of
% the axial wavenumber. These two distinctions are referred to as ``cut-on'' and 
% ``cut-off'' in the field of ducted sound propagation. Furthermore, the sign of 
% the imaginary part will change the direction of the mode's decay. If $k_x$ is 
% positive, the decay rate occurs in the negative direction. Conversely, if $k_x$ 
% is negative, the decay occurs in the positive direction. The axial wavenumber
% for uniform axial flow in a hollow duct is,

% \begin{equation}
%     k_x  = \frac{- M_x k \pm \sqrt{k^2 - ( 1 - M_x^2) k_{r,m,n}'^2 }}{\left( 1 - M_x^2 \right)}.
%     \label{eqn:ax_wavenumb}
% \end{equation}

% where $M_x$ is the axial Mach number, $k$ is the temporal (referred to as reduced)
% frequency, and $J_{m,n}'$ is the derivative of the Bessel function of the first kind.  
% The $\pm$ accounts for both upstream and downstream modes. See Appendix for detailed
% derivation.

% The condition for propagation is such that the axial wavenumber is larger than 
% a ``cut-off'' value

% \begin{equation}
%     k_{x,real}  = \frac{\pm M_x k }{\left( M_x^2 - 1 \right)}.
%     \label{eqn:cuton}
% \end{equation}

% Every term that is being raised to the one half i.e. square rooted must 
% be larger than zero to keep axial wavenumber from being imaginary. The mode 
% will propagate or decay based on this condition. 

% \subsection{Methods}
% Bessel's Function,

\begin{equation}
    R(r) = A J_0 (k_r r) + B Y_0(k_r r ) 
    \label{eqn:besselsfun}
\end{equation}
where $J_0$ and $Y_0$ are Bessel's functions of the first 
and second kind, and $A$ and $B$ are arbitrary constants. 
Both functions reduce as $k_r r$ gets large. The Bessel function of the second
kind is unbounded as $k_r r$ goes to zero. 

For a hard-walled duct, the radial velocity component is zero at the 
boundaries $r_{min}$ and $r_{max}$. 

\begin{equation}
    \frac{dR(r)}{dr}\Bigr|_{r = r_{max}}  =%
    \frac{dR(r)}{dr}\Bigr|_{r = r_{min}}   = 0
    \label{eqn:besselBC}
\end{equation}
In the case of a hollow duct, there is no
minimum radius, therefore the wall boundary condition only applies at $r_{max}$.


Since the solution must be finite as $k_r r$ approaches zero, it can be observed
that $Y_0$ approaches infinity. Since this would yield in a trivial solution, the 
coefficient $B = 0$, which reduces \ref{eqn:besselsfun} to,

\begin{equation}
    R(r) = A J_0 (k_r r) 
    \label{eqn:besselsfunCylinder}
\end{equation}

Taking the derivative 
with respect to $r$ yields,

\begin{equation}
    \frac{dR(r)}{dr}\Bigr|_{r} = A J'_0 (k_r r)  = 0
    \label{eqn:besselsfunderivaTive}
\end{equation}
The boundary condition requires the Bessel function be zero at a hard wall.
The terms inside of the Bessel function would then correspond to values along 
the domain, $k_r r$,  which satisfy our equation.  Let $k_r r = \alpha_{m,n}$ 
where, $alpha$ represents the zeros of the Bessel function, and 
$m$ corresponds to the azimuthal mode order and $n$ represents the radial mode
order, i.e. the index for the number of zero crossings in the 
derivative of the Bessel function of the first kind. 

Therefore,


\begin{align}
    \frac{dR}{dr}\Bigr|_{r = r_{max}} = A J'_0 (k_r r_{max})  = 0 \\
    (k_r r_{max})  = 0 
\end{align}
Recalling $\alpha_{m,n}$


\begin{align}
    (k_r r_{max})  = \alpha_{m,n} \\
    k_r = \alpha_{m,n}/r_{max}
\end{align}

For annular ducts, $r_{min}$ is no longer zero, therefore $B$ cannot be removed
since $Y_0'$ has finite values as $k_r r $ increases.

\begin{equation}
    \frac{dR}{dr}\Bigr|_{r} = A J'_0 (k_r r) + B Y'_0(k_r r ) 
    \label{eqn:besselsfunderivaTive}
\end{equation}
Applying boundary conditions to both inner and outer walls at $r_{min}$ and $r_{max}$

\begin{equation}
    \frac{dR}{dr}\Bigr|_{r=r_{min}} = A J'_0 (k_r r_{min}) + B Y'_0(k_r r_{min}) 
\end{equation}

\begin{equation}
    \frac{dR}{dr}\Bigr|_{r=r_{max}} = A J'_0 (k_r r_{max}) + B Y'_0(k_r r_{max}) 
\end{equation}

defining $\beta = \frac{B}{A}$,

\begin{equation}
    \frac{dR}{dr}\Bigr|_{r=r_{min}} =  J'_0 (k_r r_{min}) + \beta Y'_0(k_r r_{min}) 
\end{equation}

\begin{equation}
    \frac{dR}{dr}\Bigr|_{r=r_{max}} =  J'_0 (k_r r_{max}) + \beta Y'_0(k_r r_{max}) 
\end{equation}

Defining,

\begin{align*}
    k_r^{n+1} = k_r^n + \Delta k_r \\
    \beta^{n+1} = \beta^n + \Delta \beta 
\end{align*}
And substituting into the boundary condition expressions,
\begin{equation}
    \frac{dR}{dr}\Bigr|_{r=r_{min}} =  J'_0 (k_r^{n+1} r_{min}) + \beta^{n+1} Y'_0(k_r^{n+1} r_{min}) 
\end{equation}

\begin{equation}
    \frac{dR}{dr}\Bigr|_{r=r_{max}} =  J'_0 (k_r^{n+1} r_{max}) + \beta^{n+1} Y'_0(k_r^{n+1} r_{max}) 
\end{equation}


% \section{Discussion}


% 
\section{Analytical Test Case}
The results present a comparison of analytical and numerical solutions for a
uniform flow, hard-wall, cylindrical duct for six grids. The physical 
parameters for this test case are reporte in Table 1. The grid points were
doubled each iteration and a starting grid of 33 points was chosen. The 
first five radial modes were chosen in order to get a group of modes that were cut
on and cut off. The case number's are used as identifiers for each 
mode propagating up or downstream. In the event that a spurious mode is identified,
the case number will index it.  The number of cut off modes was also constrained
by setting a maximum value for the imaginary part of the axial wavenumber. 
While this magnitude is currently arbitrary, the value could be correlated to a
desired decay rate for the corresponding mode. Each of the axial wavenumber's 
radial mode for each grid in also reported. The same test case was ran twice using 
second and fourth order central schemes for the radial derivatives.  The 
approximate rate of convergence is also reported for both numerical schemes.
\begin{table}[!h]
    \centering
    \begin{tabular}{|l|l|}
        \hline
        $\sigma$ & \textit{0.0} \\ \hline
        $k$      & \textit{10}   \\ \hline
        $m$      & \textit{2}    \\ \hline
        $M_x$    & \textit{0.3}  \\ \hline
    \end{tabular}
    \caption{Validation test case parameters, Uniform Flow Hard-wall Cylindrical%
    Duct} 
\end{table}

\newpage


\foreach \i in {33,66,132,264,528,1056}
{
    \begin{figure}[p]
        \centering
        \includegraphics[width=\textwidth]%
        {../figures/second_order_wavenumber_grid_\i.pdf}
    \caption{ Second Order, Analytical Solution vs Numerical Approximation using \i\  points}
    \vspace{1in}
    \end{figure}
}
% \stop
\foreach \i in {33,66,132,264,528,1056}
{
    \begin{figure}
        \centering
        \includegraphics[width=\textwidth]
        {../figures/fourth_order_wavenumber_grid_\i.pdf}
    \caption{ Fourth Order, Analytical Solution vs Numerical Approximation using \i\  points}
    \vspace{1in}
    \end{figure}
}

\subsection{Propagating Radial Modes}

\subsubsection{Second Order, Radial Mode 0}
\newpage
\foreach \i in {0,...,1}
{
    \foreach \j in {33,66,132,264,528,1056} 
    {
        \begin{figure}
            \centering
            \includegraphics[width=\textwidth]
            {../figures/second_order_radial_mode_0_test_case_number_\i_grid_\j.pdf}
            \caption{Second Order, Case number \i, \j points}
            % \vspace{1in}
            \label{fig:analytical_bessel_function}
        \end{figure}
        \begin{figure}
            \centering
            \includegraphics[width=\textwidth]
            {../figures/second_order_radial_mode_error_0_test_case_number_\i_grid_\j.pdf}
            \caption{Second Order, Case number \i, \j points}
            % \vspace{1in}
            \label{fig:analytical_bessel_function}
        \end{figure}
    }
}
\clearpage

% \stop

\subsubsection{ Fourth Order, Radial Mode 0}
\newpage
\foreach \i in {0,...,1}
{
    \foreach \j in {33,66,132,264,528,1056} 
    {
        \begin{figure}
            \centering
            \includegraphics[width=\textwidth]
            {../figures/fourth_order_radial_mode_0_test_case_number_\i_grid_\j.pdf}
            \caption{Fourth Order, Case number \i, \j points}
            \label{fig:analytical_bessel_function}
        \end{figure}
        \begin{figure}
            \centering
            \includegraphics[width=\textwidth]
            {../figures/fourth_order_radial_mode_error_0_test_case_number_\i_grid_\j.pdf}
            \caption{Fourth Order, Case number \i, \j points}
            \label{fig:analytical_bessel_function}
        \end{figure}
    }
}
\clearpage
\subsubsection{Second Order, Radial Mode 1}
\newpage
\foreach \i in {0,...,1}
{
    \foreach \j in {33,66,132,264,528,1056} 
    {
        \begin{figure}
            \centering
            \includegraphics[width=\textwidth]
            {../figures/second_order_radial_mode_1_test_case_number_\i_grid_\j.pdf}
            \caption{Second Order, Case number \i, \j points}
            \label{fig:analytical_bessel_function}
        \end{figure}
        \begin{figure}
            \centering
            \includegraphics[width=\textwidth]
            {../figures/second_order_radial_mode_error_1_test_case_number_\i_grid_\j.pdf}
            \caption{Second Order, Case number \i, \j points}
            \label{fig:analytical_bessel_function}
        \end{figure}
    }
}

\clearpage
\subsubsection{Fourth Order, Radial Mode 1}
\newpage
\foreach \i in {0,...,1}
{
    \foreach \j in {33,66,132,264,528,1056} 
    {
        \begin{figure}
            \centering
            \includegraphics[width=\textwidth]
            {../figures/fourth_order_radial_mode_1_test_case_number_\i_grid_\j.pdf}
            \caption{Fourth Order, Case number \i, \j points}
            \label{fig:analytical_bessel_function}
        \end{figure}
        \begin{figure}
            \centering
            \includegraphics[width=\textwidth]
            {../figures/fourth_order_radial_mode_error_1_test_case_number_\i_grid_\j.pdf}
            \caption{Fourth Order, Case number \i, \j points}
            \label{fig:analytical_bessel_function}
        \end{figure}
    }
}

\clearpage
\subsubsection{Second Order, Radial Mode 2}
\newpage
\foreach \i in {0,...,1}
{
    \foreach \j in {33,66,132,264,528,1056} 
    {
        \begin{figure}
            \centering
            \includegraphics[width=\textwidth]
            {../figures/second_order_radial_mode_2_test_case_number_\i_grid_\j.pdf}
            \caption{Second Order, Case number \i, \j points}
            \label{fig:analytical_bessel_function}
        \end{figure}
        \begin{figure}
            \centering
            \includegraphics[width=\textwidth]
            {../figures/second_order_radial_mode_error_2_test_case_number_\i_grid_\j.pdf}
            \caption{Second Order, Case number \i, \j points}
            \label{fig:analytical_bessel_function}
        \end{figure}
    }
}

\clearpage
\subsubsection{Fourth Order, Radial Mode 2}
\newpage
\foreach \i in {0,...,1}
{
    \foreach \j in {33,66,132,264,528,1056} 
    {
        \begin{figure}
            \centering
            \includegraphics[width=\textwidth]
            {../figures/fourth_order_radial_mode_2_test_case_number_\i_grid_\j.pdf}
            \caption{Fourth Order, Case number \i, \j points}
            \label{fig:analytical_bessel_function}
        \end{figure}
        \begin{figure}
            \centering
            \includegraphics[width=\textwidth]
            {../figures/fourth_order_radial_mode_error_2_test_case_number_\i_grid_\j.pdf}
            \caption{Fourth Order, Case number \i, \j points}
            \label{fig:analytical_bessel_function}
        \end{figure}
    }
}


\clearpage
\subsubsection{Second Order, Radial Mode 3}
\newpage
\foreach \i in {0,...,1}
{
    \foreach \j in {33,66,132,264,528,1056} 
    {
        \begin{figure}
            \centering
            \includegraphics[width=\textwidth]
            {../figures/second_order_radial_mode_3_test_case_number_\i_grid_\j.pdf}
            \caption{Second Order, Case number \i, \j points}
            \label{fig:analytical_bessel_function}
        \end{figure}
        \begin{figure}
            \centering
            \includegraphics[width=\textwidth]
            {../figures/second_order_radial_mode_error_3_test_case_number_\i_grid_\j.pdf}
            \caption{Second Order, Case number \i, \j points}
            \label{fig:analytical_bessel_function}
        \end{figure}
    }
}

\clearpage
\subsubsection{Fourth Order, Radial Mode 3}
\newpage
\foreach \i in {0,...,1}
{
    \foreach \j in {33,66,132,264,528,1056} 
    {
        \begin{figure}
            \centering
            \includegraphics[width=\textwidth]
            {../figures/fourth_order_radial_mode_3_test_case_number_\i_grid_\j.pdf}
            \caption{Fourth Order, Case number \i, \j points}
            \label{fig:analytical_bessel_function}
        \end{figure}
        \begin{figure}
            \centering
            \includegraphics[width=\textwidth]
            {../figures/fourth_order_radial_mode_error_3_test_case_number_\i_grid_\j.pdf}
            \caption{Fourth Order, Case number \i, \j points}
            \label{fig:analytical_bessel_function}
        \end{figure}
    }
}



\clearpage
\subsubsection{Second Order, Radial Mode 4}
\newpage
\foreach \i in {0,...,1}
{
    \foreach \j in {33,66,132,264,528,1056} 
    {
        \begin{figure}
            \centering
            \includegraphics[width=\textwidth]
            {../figures/second_order_radial_mode_4_test_case_number_\i_grid_\j.pdf}
            \caption{Second Order, Case number \i, \j points}
            \label{fig:analytical_bessel_function}
        \end{figure}
        \begin{figure}
            \centering
            \includegraphics[width=\textwidth]
            {../figures/second_order_radial_mode_error_4_test_case_number_\i_grid_\j.pdf}
            \caption{Second Order, Case number \i, \j points}
            \label{fig:analytical_bessel_function}
        \end{figure}
    }
}

\clearpage
\subsubsection{Fourth Order, Radial Mode 4}
\newpage
\foreach \i in {0,...,1}
{
    \foreach \j in {33,66,132,264,528,1056} 
    {
        \begin{figure}
            \centering
            \includegraphics[width=\textwidth]
            {../figures/fourth_order_radial_mode_4_test_case_number_\i_grid_\j.pdf}
            \caption{Fourth Order, Case number \i, \j points}
            \label{fig:analytical_bessel_function}
        \end{figure}
        \begin{figure}
            \centering
            \includegraphics[width=\textwidth]
            {../figures/fourth_order_radial_mode_error_4_test_case_number_\i_grid_\j.pdf}
            \caption{Fourth Order, Case number \i, \j points}
            \label{fig:analytical_bessel_function}
        \end{figure}
    }
}


\section{Discussion}






\end{document}


