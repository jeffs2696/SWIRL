%        File: doc.tex
%     Created: Mon Aug 08 10:00 AM 2022 E
% Last Change: Mon Aug 08 10:00 AM 2022 E
%
\documentclass[a4paper]{report}
\maxdeadcycles=200
\usepackage[section]{placeins}
\usepackage{afterpage}
\usepackage{mathtools}
\usepackage{amsmath}
\usepackage{pgffor}
\usepackage{booktabs}
\begin{document}
\begin{titlepage}
    \begin{center}
        \vspace*{1cm}

        \textbf{ Validation using the %
            Analytical Solution for Duct Mode Propagation in %
        Uniform Flow} 

        \vspace{0.5cm}
        Swirl Validation

        \vspace{1.5cm}

        \textbf{Jeff Severino}

        \vfill


        This document shows the analytical duct mode solution as well as a
        numerical comparison.
        \vspace{0.8cm}

        % \includegraphics[width=0.4\textwidth]{university}

        Mechanical, Industrial, Manufacturing Engineering Department\\
        University Of Toledo\\
        Toledo, OH\\
        \today

    \end{center}
\end{titlepage}

% \section{Introduction - Turbomachinery Noise}
% Turbomachinery noise generation occurs from pressure fluctuations from the series 
% of fans within it's annular duct. While the jet that is produced from this stream
% of air freely radiates to the observer, the pressure fluctuations 
% produced from the rotor may or may not propagate out of the inlet and exhaust and 
% radiate to the observer. The production of this propagation can be characterized
% by standing waves referred to as modes, in particular, duct modes because 
% the mode itself is dependent on the geometry of the column of air within the 
% annular duct, as well as the speed of the flow moving through it


% This document will provide the fundamental equations that describe sound propagation
% in ducted flow.

% \begin{itemize}
%     \item Introduce the governing equations for compressible inviscid flow and 
%         the assumptions used to obtain them
%     \item Demonstrate the linearization process and show the general equations
%         that describe duct modes.  
%     \item The analytic solution for cylindrical and annular ducts with and
%         without liner will be presented and used for validation against
%         numerical approximations.
%  \end{itemize}
% \newpage
% \section{Current Research Direction}
\section{Research Performed}
\section{Results}

\begin{table}
    % \centering
    \makebox[\linewidth]{
        \input{../tables/L2_downstream_table.tex}
    }
    \caption{L2 error for downstream radial mode 1}
\end{table}
\begin{table}
    \makebox[\linewidth]{
        \begin{tabular}{lrrrrrrr}
 & $L_{2,k_x}$ & $L_{2,\bar{p}}$ & $L_{2,noLBC}$ & $L_{2,noRBC}$ & $L_{2,noBCS}$ & $L_{max}$ & $L_{max,location}$ \\
0 & 0.00000155279 & 0.00059786297 & 0.00019688319 & 0.00059696793 & 0.00016546650 & 0.00324886266 & 0 \\
1 & 0.00000028331 & 0.00010577879 & 0.00004184232 & 0.00010658661 & 0.00004216087 & 0.00079036919 & 0 \\
2 & 0.00000002131 & 0.00001964476 & 0.00000996508 & 0.00001971142 & 0.00000998706 & 0.00019476231 & 0 \\
3 & 0.00000000161 & 0.00000384159 & 0.00000243557 & 0.00000384595 & 0.00000243556 & 0.00004833159 & 0 \\
4 & 0.00000000011 & 0.00000079773 & 0.00000060218 & 0.00000079784 & 0.00000060188 & 0.00001203774 & 0 \\
5 & 0.00000000001 & 0.00000017589 & 0.00000014971 & 0.00000017586 & 0.00000014965 & 0.00000300377 & 0 \\
\end{tabular}

    }
    \caption{L2 error for upstream radial mode 1}
\end{table}

\begin{table}
    \makebox[\linewidth]{
        \begin{tabular}{lrrrr}
 & 0 & 1 & 2 & 3 \\
0 & 2.540732 & 2.921466 & 2.532691 & 2.436520 \\
1 & 2.520906 & 2.667102 & 2.520883 & 2.477816 \\
\end{tabular}

    }
    \caption{ROC error for downstream radial mode 1}
\end{table}
\begin{table}
    \makebox[\linewidth]{
        \begin{tabular}{lrrrr}
 & 0 & 1 & 2 & 3 \\
0 & 2.540732 & 2.921466 & 2.532691 & 2.436520 \\
\end{tabular}

    }
    \caption{ROC error for upstream radial mode 1}
\end{table}

% legend \\
% 0 - axial wavenumber error  \\
% 1 - pressure mode L2 \\
% 2 - pressure mode L2 without left \\
% 3 - pressure mode L2 without right   \\
% 4 - pressure mode L2 without boundaries   \\
% 5 - pressure mode LMax  \\
% 6 - pressure mode LMax location \\

% legend \\
% 0 - axial wavenumber ROC \\
% 1 - pressure mode ROC \\
% 2 - pressure mode ROC without left \\
% 3 - pressure mode ROC without right   \\
% 4 - pressure mode ROC without boundaries  \\
% 6 - pressure mode LMax ROC

\subsection{Discussion}

A total of five grids were studied, starting from 33 and doubling until 1056 for
the first cut on radial mode. The upstream and downstream mode pair were both used
to compute the L2, Lmax, and rate of convergence for a given uniform flow in 
a cylindrical duct. The axial wavenumber's error starts with a magnitude of $e-6$
and decreases to $e-12$. When computing the convergence rate, a fourth order is 
expected since a fourth order central scheme was used. This behavior is shown in
the upstream mode, with a rate of convergence of $~4.19$. However, this behaviour
is less pronounced in the downstream case. The rate increases to $3.7$ but begins
to decrease for the last grid pair. A grid spacing choice should have been made
such that the jumps are not as large so that the convergence rates can be 
studied more carefully.

For the pressure mode data, neither upstream or downstream reach fourth order convergence.
The use of the L'Hopital's rule to obtain a value at the centerline causes the 
error between the numerical and analytic modes to be highest at the centerline
gridpoint. This is shown in both upstream and downstream directions by identifying
the location of the highest error $L_{max}$. When computing the rate of convergence
for the $L_{max}$ point, it converges to second order. The same trend is noticed when 
both boundaries are removed in the error calculation. 

\section{Issues}
When looking at the difference between the excluding of boundaries, it seems that 
the convergence rate is higher without the wall BC as opposed to being higher without 
the centerline BC, but when computing the convergence rate of $LMax$, a second
order rate is computed. I could have the two collumns swapped. 

\section{Planned Research}
The formatting of the latex table doesn't show the Floating point values I am seeing but
discussing.

I only looked at the first radial mode at the moment. Once this study can be achieved 
for multiple radial modes simultaneously, the annular duct mode case will be studied
using the same outline.

% \begin{tabular}{lrrrrrrr}
 & $L_{2,k_x}$ & $L_{2,\bar{p}}$ & $L_{2,noLBC}$ & $L_{2,noRBC}$ & $L_{2,noBCS}$ & $L_{max}$ & $L_{max,location}$ \\
0 & 0.00000155279 & 0.00059786297 & 0.00019688319 & 0.00059696793 & 0.00016546650 & 0.00324886266 & 0 \\
1 & 0.00000028331 & 0.00010577879 & 0.00004184232 & 0.00010658661 & 0.00004216087 & 0.00079036919 & 0 \\
2 & 0.00000002131 & 0.00001964476 & 0.00000996508 & 0.00001971142 & 0.00000998706 & 0.00019476231 & 0 \\
3 & 0.00000000161 & 0.00000384159 & 0.00000243557 & 0.00000384595 & 0.00000243556 & 0.00004833159 & 0 \\
4 & 0.00000000011 & 0.00000079773 & 0.00000060218 & 0.00000079784 & 0.00000060188 & 0.00001203774 & 0 \\
5 & 0.00000000001 & 0.00000017589 & 0.00000014971 & 0.00000017586 & 0.00000014965 & 0.00000300377 & 0 \\
\end{tabular}

% \begin{tabular}{lrrrr}
 & 0 & 1 & 2 & 3 \\
0 & 2.540732 & 2.921466 & 2.532691 & 2.436520 \\
1 & 2.520906 & 2.667102 & 2.520883 & 2.477816 \\
\end{tabular}

% \section{Theoretical Foundation for Duct Acoustics based on first principles}

% The pressure field within a duct is governed by the convective wave equation, a
% second order ODE as a function of radius. 


% The solution of the convective wave equation are eigenvalues and eigenvectors 
% which may or may not correspond to acoustic disturbances fall into two groups.  
% One group corresponding to the acoustics propagation and the other group 
% corresponding to the convection speed of the flow. Both are modes that are a
% result from the pressure distribution from within the cylindrical domain.  

% \input{TheoreticalDerivation}
% \subsection{ Analytic Solution: Axial wavenumbers and Pressure Modes}

% Modes can be categorized based on the sign of the axial wavenumber and if it is
% complex in value. For example, for the uniform axial flow case, propagating modes
% are defined by axial wavenumbers, $k_x$, that have a real-part only, yielding 
% the assumed fluctuation to resemble Euler's Formula ($e^{ik_x x}$). On the other 
% hand, if the $k_x$ is complex, then the mode will resemble an exponentially decaying
% function since the imaginary number cancels, leaving a minus sign in front of
% the axial wavenumber. These two distinctions are referred to as ``cut-on'' and 
% ``cut-off'' in the field of ducted sound propagation. Furthermore, the sign of 
% the imaginary part will change the direction of the mode's decay. If $k_x$ is 
% positive, the decay rate occurs in the negative direction. Conversely, if $k_x$ 
% is negative, the decay occurs in the positive direction. The axial wavenumber
% for uniform axial flow in a hollow duct is,

% \begin{equation}
%     k_x  = \frac{- M_x k \pm \sqrt{k^2 - ( 1 - M_x^2) k_{r,m,n}'^2 }}{\left( 1 - M_x^2 \right)}.
%     \label{eqn:ax_wavenumb}
% \end{equation}

% where $M_x$ is the axial Mach number, $k$ is the temporal (referred to as reduced)
% frequency, and $J_{m,n}'$ is the derivative of the Bessel function of the first kind.  
% The $\pm$ accounts for both upstream and downstream modes. See Appendix for detailed
% derivation.

% The condition for propagation is such that the axial wavenumber is larger than 
% a ``cut-off'' value

% \begin{equation}
%     k_{x,real}  = \frac{\pm M_x k }{\left( M_x^2 - 1 \right)}.
%     \label{eqn:cuton}
% \end{equation}

% Every term that is being raised to the one half i.e. square rooted must 
% be larger than zero to keep axial wavenumber from being imaginary. The mode 
% will propagate or decay based on this condition. 

% \subsection{Methods}
% Bessel's Function,

\begin{equation}
    R(r) = A J_0 (k_r r) + B Y_0(k_r r ) 
    \label{eqn:besselsfun}
\end{equation}

% \section{Discussion}


% 
\section{Analytical Test Case}
The results present a comparison of analytical and numerical solutions for a
uniform flow, hard-wall, cylindrical duct for six grids. The physical 
parameters for this test case are reporte in Table 1. The grid points were
doubled each iteration and a starting grid of 33 points was chosen. The 
first five radial modes were chosen in order to get a group of modes that were cut
on and cut off. The case number's are used as identifiers for each 
mode propagating up or downstream. In the event that a spurious mode is identified,
the case number will index it.  The number of cut off modes was also constrained
by setting a maximum value for the imaginary part of the axial wavenumber. 
While this magnitude is currently arbitrary, the value could be correlated to a
desired decay rate for the corresponding mode. Each of the axial wavenumber's 
radial mode for each grid in also reported. The same test case was ran twice using 
second and fourth order central schemes for the radial derivatives.  The 
approximate rate of convergence is also reported for both numerical schemes.
\begin{table}[!h]
    \centering
    \begin{tabular}{|l|l|}
        \hline
        $\sigma$ & \textit{0.0} \\ \hline
        $k$      & \textit{10}   \\ \hline
        $m$      & \textit{2}    \\ \hline
        $M_x$    & \textit{0.3}  \\ \hline
    \end{tabular}
    \caption{Validation test case parameters, Uniform Flow Hard-wall Cylindrical%
    Duct} 
\end{table}

\newpage


\foreach \i in {33,66,132,264,528,1056}
{
    \begin{figure}[p]
        \centering
        \includegraphics[width=\textwidth]%
        {../figures/second_order_wavenumber_grid_\i.pdf}
    \caption{ Second Order, Analytical Solution vs Numerical Approximation using \i\  points}
    \vspace{1in}
    \end{figure}
}
% \stop
\foreach \i in {33,66,132,264,528,1056}
{
    \begin{figure}
        \centering
        \includegraphics[width=\textwidth]
        {../figures/fourth_order_wavenumber_grid_\i.pdf}
    \caption{ Fourth Order, Analytical Solution vs Numerical Approximation using \i\  points}
    \vspace{1in}
    \end{figure}
}

\subsection{Propagating Radial Modes}

\subsubsection{Second Order, Radial Mode 0}
\newpage
\foreach \i in {0,...,1}
{
    \foreach \j in {33,66,132,264,528,1056} 
    {
        \begin{figure}
            \centering
            \includegraphics[width=\textwidth]
            {../figures/second_order_radial_mode_0_test_case_number_\i_grid_\j.pdf}
            \caption{Second Order, Case number \i, \j points}
            % \vspace{1in}
            \label{fig:analytical_bessel_function}
        \end{figure}
        \begin{figure}
            \centering
            \includegraphics[width=\textwidth]
            {../figures/second_order_radial_mode_error_0_test_case_number_\i_grid_\j.pdf}
            \caption{Second Order, Case number \i, \j points}
            % \vspace{1in}
            \label{fig:analytical_bessel_function}
        \end{figure}
    }
}
\clearpage

% \stop

\subsubsection{ Fourth Order, Radial Mode 0}
\newpage
\foreach \i in {0,...,1}
{
    \foreach \j in {33,66,132,264,528,1056} 
    {
        \begin{figure}
            \centering
            \includegraphics[width=\textwidth]
            {../figures/fourth_order_radial_mode_0_test_case_number_\i_grid_\j.pdf}
            \caption{Fourth Order, Case number \i, \j points}
            \label{fig:analytical_bessel_function}
        \end{figure}
        \begin{figure}
            \centering
            \includegraphics[width=\textwidth]
            {../figures/fourth_order_radial_mode_error_0_test_case_number_\i_grid_\j.pdf}
            \caption{Fourth Order, Case number \i, \j points}
            \label{fig:analytical_bessel_function}
        \end{figure}
    }
}
\clearpage
\subsubsection{Second Order, Radial Mode 1}
\newpage
\foreach \i in {0,...,1}
{
    \foreach \j in {33,66,132,264,528,1056} 
    {
        \begin{figure}
            \centering
            \includegraphics[width=\textwidth]
            {../figures/second_order_radial_mode_1_test_case_number_\i_grid_\j.pdf}
            \caption{Second Order, Case number \i, \j points}
            \label{fig:analytical_bessel_function}
        \end{figure}
        \begin{figure}
            \centering
            \includegraphics[width=\textwidth]
            {../figures/second_order_radial_mode_error_1_test_case_number_\i_grid_\j.pdf}
            \caption{Second Order, Case number \i, \j points}
            \label{fig:analytical_bessel_function}
        \end{figure}
    }
}

\clearpage
\subsubsection{Fourth Order, Radial Mode 1}
\newpage
\foreach \i in {0,...,1}
{
    \foreach \j in {33,66,132,264,528,1056} 
    {
        \begin{figure}
            \centering
            \includegraphics[width=\textwidth]
            {../figures/fourth_order_radial_mode_1_test_case_number_\i_grid_\j.pdf}
            \caption{Fourth Order, Case number \i, \j points}
            \label{fig:analytical_bessel_function}
        \end{figure}
        \begin{figure}
            \centering
            \includegraphics[width=\textwidth]
            {../figures/fourth_order_radial_mode_error_1_test_case_number_\i_grid_\j.pdf}
            \caption{Fourth Order, Case number \i, \j points}
            \label{fig:analytical_bessel_function}
        \end{figure}
    }
}

\clearpage
\subsubsection{Second Order, Radial Mode 2}
\newpage
\foreach \i in {0,...,1}
{
    \foreach \j in {33,66,132,264,528,1056} 
    {
        \begin{figure}
            \centering
            \includegraphics[width=\textwidth]
            {../figures/second_order_radial_mode_2_test_case_number_\i_grid_\j.pdf}
            \caption{Second Order, Case number \i, \j points}
            \label{fig:analytical_bessel_function}
        \end{figure}
        \begin{figure}
            \centering
            \includegraphics[width=\textwidth]
            {../figures/second_order_radial_mode_error_2_test_case_number_\i_grid_\j.pdf}
            \caption{Second Order, Case number \i, \j points}
            \label{fig:analytical_bessel_function}
        \end{figure}
    }
}

\clearpage
\subsubsection{Fourth Order, Radial Mode 2}
\newpage
\foreach \i in {0,...,1}
{
    \foreach \j in {33,66,132,264,528,1056} 
    {
        \begin{figure}
            \centering
            \includegraphics[width=\textwidth]
            {../figures/fourth_order_radial_mode_2_test_case_number_\i_grid_\j.pdf}
            \caption{Fourth Order, Case number \i, \j points}
            \label{fig:analytical_bessel_function}
        \end{figure}
        \begin{figure}
            \centering
            \includegraphics[width=\textwidth]
            {../figures/fourth_order_radial_mode_error_2_test_case_number_\i_grid_\j.pdf}
            \caption{Fourth Order, Case number \i, \j points}
            \label{fig:analytical_bessel_function}
        \end{figure}
    }
}


\clearpage
\subsubsection{Second Order, Radial Mode 3}
\newpage
\foreach \i in {0,...,1}
{
    \foreach \j in {33,66,132,264,528,1056} 
    {
        \begin{figure}
            \centering
            \includegraphics[width=\textwidth]
            {../figures/second_order_radial_mode_3_test_case_number_\i_grid_\j.pdf}
            \caption{Second Order, Case number \i, \j points}
            \label{fig:analytical_bessel_function}
        \end{figure}
        \begin{figure}
            \centering
            \includegraphics[width=\textwidth]
            {../figures/second_order_radial_mode_error_3_test_case_number_\i_grid_\j.pdf}
            \caption{Second Order, Case number \i, \j points}
            \label{fig:analytical_bessel_function}
        \end{figure}
    }
}

\clearpage
\subsubsection{Fourth Order, Radial Mode 3}
\newpage
\foreach \i in {0,...,1}
{
    \foreach \j in {33,66,132,264,528,1056} 
    {
        \begin{figure}
            \centering
            \includegraphics[width=\textwidth]
            {../figures/fourth_order_radial_mode_3_test_case_number_\i_grid_\j.pdf}
            \caption{Fourth Order, Case number \i, \j points}
            \label{fig:analytical_bessel_function}
        \end{figure}
        \begin{figure}
            \centering
            \includegraphics[width=\textwidth]
            {../figures/fourth_order_radial_mode_error_3_test_case_number_\i_grid_\j.pdf}
            \caption{Fourth Order, Case number \i, \j points}
            \label{fig:analytical_bessel_function}
        \end{figure}
    }
}



\clearpage
\subsubsection{Second Order, Radial Mode 4}
\newpage
\foreach \i in {0,...,1}
{
    \foreach \j in {33,66,132,264,528,1056} 
    {
        \begin{figure}
            \centering
            \includegraphics[width=\textwidth]
            {../figures/second_order_radial_mode_4_test_case_number_\i_grid_\j.pdf}
            \caption{Second Order, Case number \i, \j points}
            \label{fig:analytical_bessel_function}
        \end{figure}
        \begin{figure}
            \centering
            \includegraphics[width=\textwidth]
            {../figures/second_order_radial_mode_error_4_test_case_number_\i_grid_\j.pdf}
            \caption{Second Order, Case number \i, \j points}
            \label{fig:analytical_bessel_function}
        \end{figure}
    }
}

\clearpage
\subsubsection{Fourth Order, Radial Mode 4}
\newpage
\foreach \i in {0,...,1}
{
    \foreach \j in {33,66,132,264,528,1056} 
    {
        \begin{figure}
            \centering
            \includegraphics[width=\textwidth]
            {../figures/fourth_order_radial_mode_4_test_case_number_\i_grid_\j.pdf}
            \caption{Fourth Order, Case number \i, \j points}
            \label{fig:analytical_bessel_function}
        \end{figure}
        \begin{figure}
            \centering
            \includegraphics[width=\textwidth]
            {../figures/fourth_order_radial_mode_error_4_test_case_number_\i_grid_\j.pdf}
            \caption{Fourth Order, Case number \i, \j points}
            \label{fig:analytical_bessel_function}
        \end{figure}
    }
}


\section{Discussion}






\end{document}


