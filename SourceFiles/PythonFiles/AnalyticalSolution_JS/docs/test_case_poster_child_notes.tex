\documentclass[a4paper]{article}
\usepackage{mathtools}
\usepackage{verbatim}
\usepackage{graphicx}
\usepackage{tabularx}
\usepackage{pgfplots}
\usepackage{adjustbox}
\usepackage{booktabs}
\makeatletter
\let\latex@xfloat=\@xfloat
\def\@xfloat #1[#2]{%
    \latex@xfloat #1[#2]%
    \def\baselinestretch{1}
    \@normalsize\normalsize
    \normalsize
}
\makeatother
\usepackage{amsmath}
\usepackage{mathtools}
\usepackage{epigraph}
\usepackage{cancel}
\usepackage{xcolor}
\newcommand\Ccancel[2][black]{\renewcommand\CancelColor{\color{#1}}\cancel{#2}}
\usepackage{algorithm}
\usepackage{graphicx}
\usepackage[noend]{algpseudocode}
\usepackage{gnuplot-lua-tikz}
\usepackage[utf8]{inputenc}
\usepackage{pgfplots}
\usepackage{tabularx}
\DeclareUnicodeCharacter{2212}{−}
\usepgfplotslibrary{groupplots,dateplot}
\usetikzlibrary{patterns,shapes.arrows}
\pgfplotsset{compat=newest}
\begin{document}
\begin{titlepage}

    \title{
    Research Report}



    \author{ Jeffrey Severino \\
        University of Toledo \\
        Toledo, OH  43606 \\
    email: jseveri@rockets.utoledo.edu}

    \maketitle

\end{titlepage}
\section{Current Research Direction}
Still going for the annular test case.
\section{Research Performed}

\begin{itemize}
    \item There is an additional function in \verb|V072| called \verb|anfu.f| which 
        is short for Annulus Function but was not investigated prior to yesterday's 
        report . I have now started seeing its output in my \verb|BesselFunctionCode.f90|,
        which has all of the same bessel function and its derivative subroutines 
        in Fortran 90. The intial radial mode plots look exponential when looking
        at a $r_min$ to $r_max$ from $\leq 1$ but the domain was changed to be larger,  
        periodic behavior can be observed in segements of the data as, $r$ increases.
         , so there is standing wave result from \verb|V072|'s libraries, but 
         does not compare to numerical results. 
    \item \verb|mode_test_case_poster_child.py| is a refactored version 
        of the analytical solution code used for the first cylindrical
        test case in this work. This was done to ensure that the ``Fundamentals
        of Duct Acoustics Toolkit '' (FDAT) functions were working properly 
        and so that the structure of a test routine can be outlined. The 
        normalization procedure from $Rienstra$ was indeed validated and documentation
        is underway. This also allows the normalization of any radial mode to be
        separate from the numerical study that is being conducted with SWIRL.
    \item Upon Further Thesis review, the convective wave equation that is 
        obtained by linearizing the governing equations for an 
        compressible, inviscid fluid is reffered to as the Pridmore-Brown, 
        Helmholtz , or Bessel's equation , depending on the further simplifications 
        that are made to obtain them. This distinction in my thesis would be useful 
        to the reader to observe how the governing equations change with changes
        in flow direction , which will in-turn assist in modal descriptions.
        The full derivation is included in the FDAT repository and is being revised 
        to be placed in the Thesis as well. The excluded steps in the main body
        are included in an appendix.  Are there too many steps presented?
\end{itemize}
\section{Issues and Concerns}
Visual inspection was used to compare numerical results to the analytical result
obtained from \verb|BesselFunctionCode.f90|, although the results are apparently different,
the two data sets will be super imposed to ensure that it is a true apples to 
apples comparison. This also proceeds to make progress.

\section{Planned Research}
\begin{itemize}
    \item Revisit \verb|anfu.f| to see how it can properly be used and if it
        is necessary, I wasn't using it initially.  
    \item Rerun Annular test cases in SWIRL using odd number of grid points because
        this will allow the user to se point by point convergence studies 
\end{itemize}
\end{document}


