
\section{Setting Boundary Condition Values Using a Fairing Function}
\subsection{Using $\beta$ as a scaling parameter}
Defining the nondimensional radius in the same way that SWIRL does:

\begin{align*}
    \widetilde{r} = \frac{r}{r_T}
\end{align*}
where $r_T$ is the outer radius of the annulus.

The hub-to-tip ratio is defined as:

\begin{align*}
    \sigma = \frac{r_H}{r_T}
     &= \widetilde{r}_H
\end{align*}
where $\widetilde{r}_H$ is the inner radius of the annular duct. The hub-to-tip
ratio can also be zero indicating the duct is hollow.

A useful and similar parameter is introduced, $\beta$, where $ 0 \leq \beta \leq 1$

\begin{align*}
    \beta &=
    \frac{r - r_H}{r_T - r_H}
\end{align*}
Dividing By $r_T$
\begin{align*}
    \beta &= 
    \frac{
        \frac{r}{r_T} - \frac{r_H}{r_T}
}{
        \frac{r_T}{r_T} - \frac{r_H}{r_T}
}\\
&= \frac{
    \widetilde{r} - \sigma 
}{
1 - \sigma
}
\end{align*}

Suppose a manufactured solution (MS) was decided. the goal is to change the
boundary conditions of the manufactured solution in such  way that allows us 
to adequately check the boundary conditions imposed on SWIRL. Defining the 
manufactured solution, $f_{MS}(\widetilde{r})$,   where
$\sigma \leq \widetilde{r} \leq 1$ and there are desired values of $f$ at the 
boundaries desired values are going to be denoted as $f_{minBC}$ and $f_{maxBC}$.
The desired changes in $f$ are defined as:

\begin{align*}
    \Delta f_{minBC} = f_{minBC} - f_{MS}(\widetilde{r} = \sigma)\\
    \Delta f_{maxBC} = f_{maxBC} - f_{MS}(\widetilde{r} = 1) 
\end{align*}
We'd like to impose these changes smoothly on the manufactured solution function.
To do this,the fairing functions, $A_{min}(\widetilde{r})$ and $A_{max}(\widetilde{r})$
where:
\begin{align*}
    f_{BCsImposed}(\widetilde{r}) = f_{MS}(\widetilde{r}) +
    A_{min}(\widetilde{r}) \Delta f_{minBC}  +  
    A_{max}(\widetilde{r}) \Delta f_{maxBC}  
\end{align*}
Then, in order to set the condition at the appropriate boundary, the following 
conditions are set,


\begin{align*}
    A_{min}(\widetilde{r} = \sigma) &= 1\\
    A_{min}(\widetilde{r} = 1) &= 0 \\
    A_{max}(\widetilde{r} = 1) &= 1 \\
    A_{max}(\widetilde{r} = \sigma) &= 0 
\end{align*}
If $A_{min}(\widetilde{r})$ is defined as a function of $A_{max}(\widetilde{r})$
then only $A_{max}(\widetilde{r})$ needs to be defined, therefore 
\begin{align*}
    A_{min}(\widetilde{r}) = 1 - A_{max}(\widetilde{r}) 
\end{align*}

It is also desirable to set the derivatives for the fairing function at the 
boundaries incase there are boundary conditions imposed on the derivatives of 
the fairing function.

\begin{align*}
    \frac{\partial A_{max}}{\partial \widetilde{r}}|_{\widetilde{r}= \sigma} &= 0\\
    \frac{\partial A_{max}}{\partial \widetilde{r}}|_{\widetilde{r}= 1} &= 0    
\end{align*}

\begin{align*}
    \frac{\partial A_{min}}{\partial \widetilde{r}}|_{\widetilde{r}= \sigma} &= 0\\
    \frac{\partial A_{min}}{\partial \widetilde{r}}|_{\widetilde{r}= 1} &= 0    
\end{align*}

