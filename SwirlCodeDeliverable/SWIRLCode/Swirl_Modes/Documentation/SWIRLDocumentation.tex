\documentclass[12pt]{article}

\usepackage{listings, color}    
\usepackage{textcomp}
\usepackage{fancyvrb}
\usepackage{verbatim}
\usepackage[table,dvipsnames]{xcolor}

\begin{document}

\begin{titlepage}
\title{
Documentation for the SWIRL code}


\author{
Ray Hixon \\
University of Toledo \\
Toledo, OH  43606 \\
email:  dhixon@eng.utoledo.edu}

\maketitle

\end{titlepage}

\section{Introduction}

This writeup is my notes for the SWIRL code, based on the Kousen technical report and
on the code itself.

\section{Notes on the Kousen report}

From the Summary:  there are two types of eigenvalues:  first, the acoustic modes (discrete values).
Second, the vortical modes, which are convecting with phase velocities that 'correspond to the
local mean flow velocities'.

From the Introduction:  when the mean flow is nonuniform (swirling or not), there is a question
about whether an exponential description of the axial disturbance is correct or not.  The
SWIRL code assumes an exponential function for the axial variation.

\section{Aerodynamic Model}

The cylindrical coordinate Euler equations are:

\begin{eqnarray}
\frac{\partial {\rho}}{\partial t} 
+ v_r \frac{\partial {\rho}}{\partial r}
+ \frac{v_{\theta}}{r} \frac{\partial {\rho}}{\partial \theta}
+ v_x \frac{\partial {\rho}}{\partial x}
+ {\rho}
\left(
\frac{1}{r} \frac{\partial \left(r v_r \right)}{\partial r}
+\frac{1}{r} \frac{\partial v_{\theta} }{\partial \theta}
+ \frac{\partial v_x }{\partial x}
\right)
&=&
0
\nonumber
\\
\frac{\partial v_r}{\partial t} 
+v_r \frac{\partial v_r}{\partial r}
+ \frac{v_{\theta}}{r}
\frac{\partial v_r}{\partial \theta}
-\frac{v_{\theta}^2}{r}
+v_x \frac{\partial v_r}{\partial x}
&=&
-\frac{1}{{\rho}} \frac{\partial p}{\partial r}
\nonumber
\\
\frac{\partial v_{\theta}}{\partial t} 
+v_r \frac{\partial v_{\theta}}{\partial r}
+ \frac{v_{\theta}}{r}
\frac{\partial v_{\theta}}{\partial \theta}
+\frac{v_r v_{\theta}}{r}
+v_x \frac{\partial v_{\theta}}{\partial x}
&=&
-\frac{1}{{\rho} r} \frac{\partial p}{\partial \theta}
\nonumber
\\
\frac{\partial v_x}{\partial t} 
+v_r \frac{\partial v_x}{\partial r}
+ \frac{v_{\theta}}{r}
\frac{\partial v_x}{\partial \theta}
+v_x \frac{\partial v_x}{\partial x}
&=&
-\frac{1}{{\rho}} \frac{\partial p}{\partial x}
\nonumber
\\
\frac{\partial p}{\partial t} 
+ v_r \frac{\partial p}{\partial r}
+ \frac{v_{\theta}}{r} \frac{\partial p}{\partial \theta}
+ v_x \frac{\partial p}{\partial x}
+ \gamma p
\left(
\frac{1}{r} \frac{\partial \left(r v_r \right)}{\partial r}
+\frac{1}{r} \frac{\partial v_{\theta} }{\partial \theta}
+ \frac{\partial v_x }{\partial x}
\right)
&=&
0
\nonumber
\end{eqnarray}

\subsection{Mean Flow}

For a steady axisymmetric mean flow with no radial velocity, the Euler equations simplify to:

\begin{eqnarray}
 \frac{\partial}{\partial x} \left(\overline{\rho} V_{x} \right)
&=&
0
\nonumber
\\
-\frac{V_{\theta}^2}{r}
&=&
-\frac{1}{\overline{\rho}} \frac{\partial P}{\partial r}
\nonumber
\\
 \frac{\partial}{\partial x} \left(V_{\theta} \right)
&=&
0
\nonumber
\\
 \frac{\partial}{\partial x} \left(V_{x} \right)
&=&
0
\nonumber
\end{eqnarray}

The radial momentum equation can be integrated to obtain the
mean pressure distribution:

\begin{eqnarray}
P &=& \int_{r_{min}}^{r_{max}} \frac{\overline{\rho} V_{\theta}^2}{r} dr
\nonumber
\end{eqnarray}

Defining

\begin{eqnarray}
\widetilde{r} &=& \frac{r}{r_{max}}
\nonumber
\\
dr &=& \frac{\partial r}{\partial \widetilde{r}} d \widetilde{r}
\nonumber
\\
&=& r_{max} d \widetilde{r}
\nonumber
\end{eqnarray}

and substituting in gives:

\begin{eqnarray}
P \left(\widetilde{r} \right) &=& 
P \left(1 \right)
-
\int_{\widetilde{r}}^{1} 
\frac{\overline{\rho} V_{\theta}^2}{\widetilde{r}} 
d \widetilde{r}
\nonumber
\end{eqnarray}

For reference, the minimum value of $\widetilde{r}$ is

\begin{eqnarray}
\sigma &=& \frac{r_{min}}{r_{max}}
\nonumber
\end{eqnarray}

For isentropic flow,

\begin{eqnarray}
S &=& \frac{P}{\overline{\rho}^{\gamma}}
\nonumber
\\
\vec{\nabla} S &=& 0
\nonumber
\\
\vec{\nabla} P &=& A^2 \vec{\nabla} \overline{\rho}
\nonumber
\end{eqnarray}

where $A$ is the local mean speed of sound:

\begin{eqnarray}
A^2 &=& \frac{\gamma P}{\overline{\rho}}
\nonumber
\end{eqnarray}

Then,

\begin{eqnarray}
\frac{\partial A^2}{\partial r}
&=&
\frac{\partial}{\partial r}
\left(
\frac{\gamma P}{\overline{\rho}}
\right)
\nonumber
\\
&=&
\frac{\gamma}{\overline{\rho}} \frac{\partial P}{\partial r}
-
\frac{\gamma P }{\overline{\rho}^2} \frac{\partial \overline{\rho}}{\partial r}
\nonumber
\\
&=&
\frac{\gamma}{\overline{\rho}} \frac{\partial P}{\partial r}
-
\frac{A^2}{\overline{\rho}} \frac{\partial \overline{\rho}}{\partial r}
\nonumber
\\
&=&
\frac{\gamma}{\overline{\rho}} \frac{\partial P}{\partial r}
-
\frac{1}{\overline{\rho}} \frac{\partial P}{\partial r}
\nonumber
\\
&=&
\frac{\gamma-1}{\overline{\rho}} \frac{\partial P}{\partial r}
\nonumber
\end{eqnarray}

Then,

\begin{eqnarray}
\frac{\partial P}{\partial r}
&=&
\frac{\overline{\rho} V_{\theta}^2}{r}
\nonumber
\\
\frac{\overline{\rho}}{\gamma - 1} \frac{\partial A^2}{\partial r}
&=&
\frac{\overline{\rho} V_{\theta}^2}{r}
\nonumber
\\
 \frac{\partial A^2}{\partial r}
&=&
\left(\gamma - 1 \right) \frac{V_{\theta}^2}{r}
\nonumber
\\
\frac{1}{A^2} \frac{\partial A^2}{\partial r}
&=&
\left(\gamma - 1 \right) \frac{M_{\theta}^2}{r}
\nonumber
\end{eqnarray}

Integrating, and referencing to the conditions at $r = r_{max}$,

\begin{eqnarray}
\ln{\left(A^2 \left(r \right) \right)}
&=&
\ln{\left(A^2 \left(r_{max} \right) \right)}
-\int_{r}^{r_{max}}
\left(\gamma - 1 \right) \frac{M_{\theta}^2}{r}
dr
\nonumber
\\
\ln{\left(A^2 \left(r \right) \right)}
-\ln{\left(A^2 \left(r_{max} \right) \right)}
&=&
-\int_{r}^{r_{max}}
\left(\gamma - 1 \right) \frac{M_{\theta}^2}{r}
dr
\nonumber
\\
\ln{
\left(
\frac{A^2 \left(r \right)} 
{A^2 \left(r_{max} \right)} 
\right)
}
&=&
-\int_{r}^{r_{max}}
\left(\gamma - 1 \right) \frac{M_{\theta}^2}{r}
dr
\nonumber
\end{eqnarray}

Defining

\begin{eqnarray}
\widetilde{A} \left(r \right) &=& \frac{A \left(r \right)}{A \left(r_{max} \right)}
\nonumber
\end{eqnarray}

gives:

\begin{eqnarray}
\ln{
\left(
\widetilde{A}^2 \left(r \right)
\right)
}
&=&
\left(1 - \gamma \right) 
\int_{r}^{r_{max}}
\frac{M_{\theta}^2}{r}
dr
\nonumber
\\
2 \ln{
\left(
\widetilde{A} \left(r \right)
\right)
}
&=&
\left(1 - \gamma \right) 
\int_{r}^{r_{max}}
\frac{M_{\theta}^2}{r}
dr
\nonumber
\\
\ln{
\left(
\widetilde{A} \left(r \right)
\right)
}
&=&
\frac{1 - \gamma}{2}
\int_{r}^{r_{max}}
\frac{M_{\theta}^2}{r}
dr
\nonumber
\\
\ln{
\left(
\widetilde{A} \left(\widetilde{r} \right)
\right)
}
&=&
\frac{1 - \gamma}{2}
\int_{\widetilde{r}}^{1}
\frac{M_{\theta}^2}{\widetilde{r}}
d \widetilde{r}
\nonumber
\end{eqnarray}

This gives the general result:

\begin{eqnarray}
\widetilde{A} \left(\widetilde{r} \right)
&=&
\exp{
\left(
\frac{1 - \gamma}{2}
\int_{\widetilde{r}}^{1}
\frac{M_{\theta}^2}{\widetilde{r}}
d \widetilde{r}
\right)
}
\nonumber
\end{eqnarray}

which is Eq. (2.6) in the Kousen paper.

\section{Linearized Perturbation Equations}

The linearized perturbation equations are:

\begin{eqnarray}
\frac{\partial v'_r}{\partial t} 
+ \frac{V_{\theta}}{r}
\frac{\partial v'_r}{\partial \theta}
-\frac{2 V_{\theta} v'_{\theta}}{r}
+V_x \frac{\partial v'_r}{\partial x}
&=&
-\frac{1}{\overline{\rho}} \frac{\partial p'}{\partial r}
+\frac{\rho'}{\overline{\rho}^2} \frac{\partial P}{\partial r}
\nonumber
\\
\frac{\partial v'_{\theta}}{\partial t} 
+v'_r \frac{\partial V_{\theta}}{\partial r}
+ \frac{V_{\theta}}{r}
\frac{\partial v'_{\theta}}{\partial \theta}
+\frac{v'_r V_{\theta}}{r}
+V_x \frac{\partial v'_{\theta}}{\partial x}
&=&
-\frac{1}{\overline{\rho} r} \frac{\partial p'}{\partial \theta}
\nonumber
\\
\frac{\partial v'_x}{\partial t} 
+v'_r \frac{\partial V_x}{\partial r}
+ \frac{V_{\theta}}{r}
\frac{\partial v'_x}{\partial \theta}
+V_x \frac{\partial v'_x}{\partial x}
&=&
-\frac{1}{\overline{\rho}} \frac{\partial p'}{\partial x}
\nonumber
\\
\frac{\partial p'}{\partial t} 
+ v'_r \frac{\partial P}{\partial r}
+ \frac{V_{\theta}}{r} \frac{\partial p'}{\partial \theta}
+ V_x \frac{\partial p'}{\partial x}
+ \gamma P
\left(
\frac{1}{r} \frac{\partial \left(r v'_r \right)}{\partial r}
+\frac{1}{r} \frac{\partial v'_{\theta} }{\partial \theta}
+ \frac{\partial v'_x }{\partial x}
\right)
&=&
0
\nonumber
\end{eqnarray}

Remembering that:

\begin{eqnarray}
\frac{\partial P}{\partial r} &=&
\frac{\overline{\rho} V_{\theta}^2}{r}
\nonumber
\\
\gamma P &=& \overline{\rho} A^2
\nonumber
\\
\rho' &=& \frac{1}{A^2} p'
\nonumber
\end{eqnarray}

and substituting in gives Eqs. (2.33-2.36) in the Kousen report:

\begin{eqnarray}
\frac{\partial v'_r}{\partial t} 
+ \frac{V_{\theta}}{r}
\frac{\partial v'_r}{\partial \theta}
+V_x \frac{\partial v'_r}{\partial x}
-\frac{2 V_{\theta}}{r} v'_{\theta}
&=&
-\frac{1}{\overline{\rho}} \frac{\partial p'}{\partial r}
+\frac{V_{\theta}^2}{\overline{\rho} r A^2} p'
\nonumber
\\
\frac{\partial v'_{\theta}}{\partial t} 
+ \frac{V_{\theta}}{r}
\frac{\partial v'_{\theta}}{\partial \theta}
+V_x \frac{\partial v'_{\theta}}{\partial x}
+
\left(
\frac{V_{\theta}}{r}
+\frac{\partial V_{\theta}}{\partial r}
\right) v'_r
&=&
-\frac{1}{\overline{\rho} r} \frac{\partial p'}{\partial \theta}
\nonumber
\\
\frac{\partial v'_x}{\partial t} 
+ \frac{V_{\theta}}{r}
\frac{\partial v'_x}{\partial \theta}
+V_x \frac{\partial v'_x}{\partial x}
+\frac{\partial V_x}{\partial r} v'_r
&=&
-\frac{1}{\overline{\rho}} \frac{\partial p'}{\partial x}
\nonumber
\\
\frac{1}{ \overline{\rho} A^2}
\left(
\frac{\partial p'}{\partial t} 
+ \frac{V_{\theta}}{r} \frac{\partial p'}{\partial \theta}
+ V_x \frac{\partial p'}{\partial x}
\right)
+\frac{V_{\theta}^2}{A^2 r}
v'_r
+ 
\frac{\partial v'_r}{\partial r}
+ 
\frac{v'_r}{r} 
+\frac{1}{r} \frac{\partial v'_{\theta} }{\partial \theta}
+ \frac{\partial v'_x }{\partial x}
&=&
0
\nonumber
\end{eqnarray}

Defining the perturbation variables as:

\begin{eqnarray}
v'_r &=& v_r \left(r \right) e^{i \left(k_x x + m \theta - \omega t \right)}
\nonumber
\\
v'_{\theta} &=& v_{\theta} \left(r \right) e^{i \left(k_x x + m \theta - \omega t \right)}
\nonumber
\\
v'_x &=& v_x \left(r \right) e^{i \left(k_x x + m \theta - \omega t \right)}
\nonumber
\\
p' &=& p \left(r \right) e^{i \left(k_x x + m \theta - \omega t \right)}
\nonumber
\end{eqnarray}

and substituting into the perturbation equations gives:

\begin{eqnarray}
\left(
-i \omega
+ \frac{i m V_{\theta}}{r}
+i k_x V_x 
\right) v_r
-\frac{2 V_{\theta}}{r} v_{\theta}
&=&
-\frac{1}{\overline{\rho}} \frac{\partial p}{\partial r}
+\frac{V_{\theta}^2}{\overline{\rho} r A^2} p
\nonumber
\\
\left(
-i \omega
+ \frac{i m V_{\theta}}{r}
+i k_x V_x 
\right) v_{\theta}
+
\left(
\frac{V_{\theta}}{r}
+\frac{\partial V_{\theta}}{\partial r}
\right) v_r
&=&
-\frac{i m}{\overline{\rho} r} p
\nonumber
\\
\left(
-i \omega
+ \frac{i m V_{\theta}}{r}
+i k_x V_x 
\right) v_x
+\frac{\partial V_x}{\partial r} v_r
&=&
-\frac{i k_x}{\overline{\rho}} p
\nonumber
\\
\frac{1}{ \overline{\rho} A^2}
\left(
-i \omega
+ \frac{i m V_{\theta}}{r} 
+ i k_x V_x 
\right) p
+\frac{V_{\theta}^2}{A^2 r}
v_r
+ 
\frac{\partial v_r}{\partial r}
+ 
\frac{v_r}{r} 
+\frac{i m }{r} v_{\theta} 
+ i k_x v_x
&=&
0
\nonumber
\end{eqnarray}

Defining

\begin{eqnarray}
r_T &=& r_{max}
\nonumber
\\
A_T &=& A \left(r_{max} \right)
\nonumber
\\
k &=& \frac{\omega r_T}{A_T}
\nonumber
\\
\overline{\gamma}
&=& k_x r_T
\nonumber
\\
\widetilde{r} &=& \frac{r}{r_T}
\nonumber
\\
\frac{\partial}{\partial r} 
&=& 
\frac{\partial \widetilde{r} }{\partial r} 
\frac{\partial}{\partial \widetilde{r}} 
\nonumber
\\
&=& 
\frac{1}{r_T}
\frac{\partial}{\partial \widetilde{r}} 
\nonumber
\\
V_{\theta} &=& M_{\theta} A
\nonumber
\\
V_{x} &=& M_{x} A
\nonumber
\\
\widetilde{A} &=& \frac{A}{A_T}
\nonumber
\\
v_x &=& \widetilde{v}_x A
\nonumber
\\
v_r &=& \widetilde{v}_r A
\nonumber
\\
v_{\theta} &=& \widetilde{v}_{\theta} A
\nonumber
\\
p &=& \widetilde{p} \overline{\rho} A^2
\nonumber
\end{eqnarray}

gives:

\begin{small}
\begin{eqnarray}
-i
\left(
k \frac{A_T}{r_T}
- \frac{ m M_{\theta} A}{\widetilde{r} r_T}
- \frac{\overline{\gamma}}{r_T} M_x A 
\right) \widetilde{v}_r A
-\frac{2 M_{\theta} A^2}{\widetilde{r} r_T} \widetilde{v}_{\theta} 
 = 
-\frac{1}{\overline{\rho}} 
\frac{1}{r_T}
\frac{\partial \left(\widetilde{p} \overline{\rho} A^2 \right) }{\partial \widetilde{r}} 
+\frac{M_{\theta}^2 A^2}{\overline{\rho} \widetilde{r} r_T A^2} \widetilde{p} \overline{\rho} A^2
\nonumber
\\
-i
\left(
k \frac{A_T}{r_T}
- \frac{ m M_{\theta} A}{\widetilde{r} r_T}
- \frac{\overline{\gamma}}{r_T} M_x A 
\right) \widetilde{v}_{\theta} A
+
\left(
\frac{M_{\theta} A}{\widetilde{r} r_T}
+
\frac{1}{r_T}
\frac{\partial \left(M_{\theta} A \right)}{\partial \widetilde{r}}
\right) \widetilde{v}_r A
 = 
-\frac{i m \overline{\rho} A^2}{\overline{\rho} \widetilde{r} r_T} \widetilde{p}
\nonumber
\\
-i
\left(
k \frac{A_T}{r_T}
- \frac{ m M_{\theta} A}{\widetilde{r} r_T}
- \frac{\overline{\gamma}}{r_T} M_x A
\right) \widetilde{v}_x A
+
\frac{1}{r_T}
\frac{\partial \left(M_x A \right)}{\partial \widetilde{r}} 
\widetilde{v}_r A
 = 
-\frac{i \overline{\gamma} \overline{\rho} A^2}{\overline{\rho} r_T} \widetilde{p}
\nonumber
\\
\frac{-i}{ \overline{\rho} A^2}
\left(
k \frac{A_T}{r_T}
- \frac{ m M_{\theta} A}{\widetilde{r} r_T} 
-  \frac{\overline{\gamma}}{r_T} M_x A
\right) \overline{\rho} A^2 \widetilde{p}
+\frac{M_{\theta}^2 A^2}{A^2 \widetilde{r} r_T}
\widetilde{v}_r A
+ 
\frac{1}{r_T}
\frac{\partial \left(\widetilde{v}_r A \right)}{\partial \widetilde{r}}
+ 
\frac{\widetilde{v}_r A}{\widetilde{r} r_T} 
+\frac{i m }{\widetilde{r} r_T} \widetilde{v}_{\theta} A
+ \frac{i \overline{\gamma}}{r_T} \widetilde{v}_x A
 = 
0
\nonumber
\end{eqnarray}
\end{small}%

which becomes

\begin{small}
\begin{eqnarray}
-i
\left(
 \frac{k}{ \widetilde{A}}
- \frac{ m M_{\theta} }{\widetilde{r}}
- \overline{\gamma} M_x  
\right) \widetilde{v}_r 
-\frac{2 M_{\theta}}{\widetilde{r}} \widetilde{v}_{\theta} 
 = 
-\frac{1}{\overline{\rho} A^2} 
\frac{\partial \left(\widetilde{p} \overline{\rho} A^2 \right) }{\partial \widetilde{r}} 
+
\frac{M_{\theta}^2 }{\widetilde{r}} 
\widetilde{p} 
\nonumber
\\
-i
\left(
\frac{k}{\widetilde{A}}
- \frac{ m M_{\theta} }{\widetilde{r} }
- \overline{\gamma} M_x  
\right) \widetilde{v}_{\theta} 
+
\left(
\frac{M_{\theta} }{\widetilde{r}}
+
\frac{1}{A}
\frac{\partial \left(M_{\theta} A \right)}{\partial \widetilde{r}}
\right) \widetilde{v}_r
 = 
-\frac{i m }{\widetilde{r} } \widetilde{p}
\nonumber
\\
-i
\left(
\frac{k}{\widetilde{A}}
- \frac{ m M_{\theta} }{\widetilde{r} }
- \overline{\gamma} M_x
\right) \widetilde{v}_x 
+
\frac{1}{A}
\frac{\partial \left(M_x A \right)}{\partial \widetilde{r}} 
\widetilde{v}_r
 = 
-i \overline{\gamma} \widetilde{p}
\nonumber
\\
-i
\left(
\frac{k}{\widetilde{A}}
- \frac{ m M_{\theta}}{\widetilde{r} } 
-  \overline{\gamma} M_x
\right) \widetilde{p}
+\frac{M_{\theta}^2 }{ \widetilde{r} }
\widetilde{v}_r 
+ 
\frac{1}{A}
\frac{\partial \left(\widetilde{v}_r A \right)}{\partial \widetilde{r}}
+ 
\frac{\widetilde{v}_r }{\widetilde{r}} 
+\frac{i m }{\widetilde{r}} \widetilde{v}_{\theta} 
+ i \overline{\gamma} \widetilde{v}_x 
 = 
0
\nonumber
\end{eqnarray}
\end{small}%

Expanding the mean flow derivatives gives:

\begin{small}
\begin{eqnarray}
-i
\left(
 \frac{k}{ \widetilde{A}}
- \frac{ m M_{\theta} }{\widetilde{r}}
- \overline{\gamma} M_x  
\right) \widetilde{v}_r 
-\frac{2 M_{\theta}}{\widetilde{r}} \widetilde{v}_{\theta} 
 = 
-
\frac{\partial \widetilde{p} }{\partial \widetilde{r}} 
-\frac{\widetilde{p}}{\overline{\rho} A^2} 
\frac{\partial \left(\overline{\rho} A^2 \right) }{\partial \widetilde{r}} 
+
\frac{M_{\theta}^2 }{\widetilde{r}} 
\widetilde{p} 
\nonumber
\\
-i
\left(
\frac{k}{\widetilde{A}}
- \frac{ m M_{\theta} }{\widetilde{r} }
- \overline{\gamma} M_x  
\right) \widetilde{v}_{\theta} 
+
\left(
\frac{M_{\theta} }{\widetilde{r}}
+
\frac{\partial M_{\theta} }{\partial \widetilde{r}}
+
\frac{M_{\theta}}{A}
\frac{\partial A }{\partial \widetilde{r}}
\right) \widetilde{v}_r
 = 
-\frac{i m }{\widetilde{r} } \widetilde{p}
\nonumber
\\
-i
\left(
\frac{k}{\widetilde{A}}
- \frac{ m M_{\theta} }{\widetilde{r} }
- \overline{\gamma} M_x
\right) \widetilde{v}_x 
+
\frac{\partial M_x }{\partial \widetilde{r}} 
\widetilde{v}_r
+
\frac{M_x}{A}
\frac{\partial A }{\partial \widetilde{r}} 
\widetilde{v}_r
 = 
-i \overline{\gamma} \widetilde{p}
\nonumber
\\
-i
\left(
\frac{k}{\widetilde{A}}
- \frac{ m M_{\theta}}{\widetilde{r} } 
-  \overline{\gamma} M_x
\right) \widetilde{p}
+\frac{M_{\theta}^2 }{ \widetilde{r} }
\widetilde{v}_r 
+ 
\frac{\partial \widetilde{v}_r }{\partial \widetilde{r}}
+ 
\frac{1}{A}
\frac{\partial A }{\partial \widetilde{r}}
\widetilde{v}_r
+ 
\frac{\widetilde{v}_r }{\widetilde{r}} 
+\frac{i m }{\widetilde{r}} \widetilde{v}_{\theta} 
+ i \overline{\gamma} \widetilde{v}_x 
 = 
0
\nonumber
\end{eqnarray}
\end{small}%

The mean flow derivatives are:

\begin{eqnarray}
\frac{1}{A} \frac{\partial A}{\partial \widetilde{r}}
&=&
\frac{r_T}{A^2} \left(A \frac{\partial A}{\partial r} \right)
\nonumber
\\
&=&
\frac{r_T}{2 A^2} \frac{\partial A^2}{\partial r}
\nonumber
\\
&=&
\frac{r_T }{A^2}
\left(\frac{\gamma - 1}{2} \right) \frac{V_{\theta}^2}{r }
\nonumber
\\
&=&
\left(\frac{\gamma - 1}{2} \right) \frac{M_{\theta}^2}{\widetilde{r}}
\nonumber
\end{eqnarray}

and

\begin{eqnarray}
\frac{1}{\overline{\rho} A^2}
\frac{\partial \left(\overline{\rho} A^2 \right) }{\partial \widetilde{r}} 
&=&
\frac{\gamma}{\overline{\rho} A^2}
 \frac{\partial P }{\partial \widetilde{r}}
\nonumber
\\
&=&
\frac{r_T \gamma}{\overline{\rho} A^2}
\gamma \frac{\partial P }{\partial r}
\nonumber
\\
&=&
\frac{r_T \gamma}{\overline{\rho} A^2}
\frac{\overline{\rho} V_{\theta}^2}{r}
\nonumber
\\
&=&
\frac{\gamma}{\overline{\rho} A^2}
\frac{\overline{\rho} A^2 M_{\theta}^2}{\widetilde{r}}
\nonumber
\\
&=&
\frac{\gamma M_{\theta}^2}{\widetilde{r}}
\nonumber
\end{eqnarray}

Substituting in,

\begin{small}
\begin{eqnarray}
-i
\left(
 \frac{k}{ \widetilde{A}}
- \frac{ m M_{\theta} }{\widetilde{r}}
- \overline{\gamma} M_x  
\right) \widetilde{v}_r 
-\frac{2 M_{\theta}}{\widetilde{r}} \widetilde{v}_{\theta} 
 = 
-
\frac{\partial \widetilde{p} }{\partial \widetilde{r}} 
-
\frac{\left(\gamma - 1 \right) M_{\theta}^2}{\widetilde{r}} \widetilde{p}
\nonumber
\\
-i
\left(
\frac{k}{\widetilde{A}}
- \frac{ m M_{\theta} }{\widetilde{r} }
- \overline{\gamma} M_x  
\right) \widetilde{v}_{\theta} 
+
\left(
\frac{M_{\theta} }{\widetilde{r}}
+
\frac{\partial M_{\theta} }{\partial \widetilde{r}}
+
\left(\frac{\gamma - 1}{2} \right) \frac{M_{\theta}^3}{\widetilde{r}}
\right) 
\widetilde{v}_r
 = 
-\frac{i m }{\widetilde{r} } \widetilde{p}
\nonumber
\\
-i
\left(
\frac{k}{\widetilde{A}}
- \frac{ m M_{\theta} }{\widetilde{r} }
- \overline{\gamma} M_x
\right) \widetilde{v}_x 
+
\left(
\frac{\partial M_x }{\partial \widetilde{r}} 
+
\left(\frac{\gamma - 1}{2} \right) \frac{M_x M_{\theta}^2}{\widetilde{r}}
\right)
\widetilde{v}_r
 = 
-i \overline{\gamma} \widetilde{p}
\nonumber
\\
-i
\left(
\frac{k}{\widetilde{A}}
- \frac{ m M_{\theta}}{\widetilde{r} } 
-  \overline{\gamma} M_x
\right) \widetilde{p}
+ 
\frac{\partial \widetilde{v}_r }{\partial \widetilde{r}}
+ 
\left(
\left(\frac{\gamma + 1}{2} \right) \frac{M_{\theta}^2}{\widetilde{r}}
+\frac{1 }{\widetilde{r}} 
\right)
\widetilde{v}_r
+\frac{i m }{\widetilde{r}} \widetilde{v}_{\theta} 
+ i \overline{\gamma} \widetilde{v}_x 
 = 
0
\nonumber
\end{eqnarray}
\end{small}%

which are Eq. (2.38-2.41) in Kousen's report.

Defining:

\begin{eqnarray}
\lambda &=& -i \overline{\gamma}
\nonumber
\end{eqnarray}

and defining:

\begin{eqnarray}
\left\{\vec{x} \right\} 
&=& 
\left\{
\begin{array}{l}
\widetilde{v}_r
\\
\widetilde{v}_{\theta}
\\
\widetilde{v}_x
\\
\widetilde{p}
\end{array}
\right\}
\nonumber
\end{eqnarray}

and writing the equation in matrix form gives:

\begin{tiny}
\begin{eqnarray}
\left[
\begin{array}{cccc}
-i
\left(
\frac{k}{\widetilde{A}}
- \frac{ m M_{\theta} }{\widetilde{r} }
\right) - \lambda M_x
&
-\frac{2 M_{\theta}}{\widetilde{r}}
&
0
&
\frac{\partial}{\partial \widetilde{r}}
+ \frac{\left(\gamma - 1 \right) M_{\theta}^2}{\widetilde{r}}
\\
\frac{M_{\theta} }{\widetilde{r}}
+
\frac{\partial M_{\theta} }{\partial \widetilde{r}}
+
\left(\frac{\gamma - 1}{2} \right) \frac{M_{\theta}^3}{\widetilde{r}}
&
-i
\left(
\frac{k}{\widetilde{A}}
- \frac{ m M_{\theta} }{\widetilde{r} }
\right) - \lambda M_x
&
0
&
\frac{i m}{\widetilde{r}}
\\
\frac{\partial M_x }{\partial \widetilde{r}} 
+
\left(\frac{\gamma - 1}{2} \right) \frac{M_x M_{\theta}^2}{\widetilde{r}}
&
0
&
-i
\left(
\frac{k}{\widetilde{A}}
- \frac{ m M_{\theta} }{\widetilde{r} }
\right) - \lambda M_x
&
-\lambda
\\
\frac{\partial}{\partial \widetilde{r}}
+ 
\left(\frac{\gamma + 1}{2} \right) \frac{M_{\theta}^2}{\widetilde{r}}
+\frac{1 }{\widetilde{r}} 
&
\frac{i m}{\widetilde{r}}
&
-\lambda
&
-i
\left(
\frac{k}{\widetilde{A}}
- \frac{ m M_{\theta} }{\widetilde{r} }
\right) - \lambda M_x
\end{array}
\right]
\left\{\vec{x} \right\} 
= 0
\nonumber
\end{eqnarray}
\end{tiny}%

Following the Kousen report, this equation can be rewritten as:

\begin{eqnarray}
\left[A \right]
\left\{\vec{x} \right\} 
&=&
\lambda
\left[B \right]
\left\{\vec{x} \right\} 
\nonumber
\end{eqnarray}

where

\begin{small}
\begin{eqnarray}
\left[A \right]
=
\left[
\begin{array}{cccc}
-i
\left(
\frac{k}{\widetilde{A}}
- \frac{ m M_{\theta} }{\widetilde{r} }
\right) 
&
-\frac{2 M_{\theta}}{\widetilde{r}}
&
0
&
\frac{\partial}{\partial \widetilde{r}}
+ \frac{\left(\gamma - 1 \right) M_{\theta}^2}{\widetilde{r}}
\\
\frac{M_{\theta} }{\widetilde{r}}
+
\frac{\partial M_{\theta} }{\partial \widetilde{r}}
+
\left(\frac{\gamma - 1}{2} \right) \frac{M_{\theta}^3}{\widetilde{r}}
&
-i
\left(
\frac{k}{\widetilde{A}}
- \frac{ m M_{\theta} }{\widetilde{r} }
\right) 
&
0
&
\frac{i m}{\widetilde{r}}
\\
\frac{\partial M_x }{\partial \widetilde{r}} 
+
\left(\frac{\gamma - 1}{2} \right) \frac{M_x M_{\theta}^2}{\widetilde{r}}
&
0
&
-i
\left(
\frac{k}{\widetilde{A}}
- \frac{ m M_{\theta} }{\widetilde{r} }
\right) 
&
0
\\
\frac{\partial}{\partial \widetilde{r}}
+ 
\left(\frac{\gamma + 1}{2} \right) \frac{M_{\theta}^2}{\widetilde{r}}
+\frac{1 }{\widetilde{r}} 
&
\frac{i m}{\widetilde{r}}
&
0
&
-i
\left(
\frac{k}{\widetilde{A}}
- \frac{ m M_{\theta} }{\widetilde{r} }
\right) 
\end{array}
\right]
\nonumber
\end{eqnarray}
\end{small}%

and

\begin{eqnarray}
\left[B \right]
=
\left[
\begin{array}{cccc}
M_x
&
0
&
0
&
0
\\
0
&
M_x
&
0
&
0
\\
0
&
0
&
M_x
&
1
\\
0
&
0
&
1
&
M_x
\end{array}
\right]
\nonumber
\end{eqnarray}

From the SWIRL code:  at $\widetilde{r} = 0$, 

\begin{tiny}
\begin{eqnarray}
\left[A \right]
=
\left[
\begin{array}{cccc}
-i
\left(
\frac{k}{\widetilde{A}}
- m M_{\theta}
\right) 
&
-2 M_{\theta}
&
0
&
\frac{\partial}{\partial \widetilde{r}}
+ 2 \left(\gamma - 1 \right) M_{\theta}
 \frac{\partial M_{\theta}}{\partial \widetilde{r}}
\\
\frac{\partial M_{\theta} }{\partial \widetilde{r}}
+
\left(1 + 3 \left(\frac{\gamma - 1}{2} \right) M_{\theta}^2 \right)
\frac{\partial M_{\theta} }{\partial \widetilde{r}}
&
-i
\left(
\frac{k}{\widetilde{A}}
- m M_{\theta}
\right) 
&
0
&
0
\\
\frac{\partial M_x }{\partial \widetilde{r}} 
+
\left(\frac{\gamma - 1}{2} \right) M_{\theta} 
\left(
M_{\theta}
\frac{\partial M_x }{\partial \widetilde{r}} 
+2 M_x
\frac{\partial M_{\theta} }{\partial \widetilde{r}} 
\right)
&
0
&
-i
\left(
\frac{k}{\widetilde{A}}
- m M_{\theta}
\right) 
&
0
\\
\frac{\partial}{\partial \widetilde{r}}
+ 
\left(\gamma + 1 \right) M_{\theta}
\frac{\partial M_{\theta} }{\partial \widetilde{r}} 
&
0
&
0
&
-i
\left(
\frac{k}{\widetilde{A}}
- m M_{\theta}
\right) 
\end{array}
\right]
\nonumber
\end{eqnarray}
\end{tiny}%

\section{Divergence and Vorticity?}

The divergence and vorticity of the perturbation velocity in cylindrical
coordinates is:

\begin{eqnarray}
\vec{\nabla} \cdot \vec{v'} &=&
\frac{1}{r}
\frac{\partial}{\partial r} \left( r v'_r \right)
+ \frac{1}{r} \frac{\partial v'_{\theta}}{\partial \theta}
+ \frac{\partial v'_{x}}{\partial x}
\nonumber
\\
&=&
\frac{1}{r_T}
\left(
\frac{1}{\widetilde{r}}
\frac{\partial}{\partial \widetilde{r}} 
\left( \widetilde{r} v'_r \right)
+ 
 \frac{1}{\widetilde{r}} 
\frac{\partial v'_{\theta}}{\partial \theta}
+ 
 \frac{\partial v'_{x}}{\partial \widetilde{x}}
\right)
\nonumber
\\
\vec{\nabla} \times \vec{v'}
&=&
\left(
\begin{array}{r}
\left(
\frac{1}{r} \frac{\partial v'_x}{\partial \theta}
-\frac{\partial v'_{\theta}}{\partial x} 
\right) \vec{e}_r
\\
+\left(
\frac{\partial v'_r}{\partial x}
-\frac{\partial v'_x}{\partial r}
\right) \vec{e}_{\theta}
\\
+
\frac{1}{r} 
\left(
  \frac{\partial}{\partial r} \left(r v'_{\theta} \right)
- \frac{\partial v'_r}{\partial \theta}
\right) \vec{e}_x
\end{array}
\right)
\nonumber
\\
&=&
\frac{1}{r_T}
\left(
\begin{array}{r}
\left(
\frac{1}{\widetilde{r}} \frac{\partial v'_x}{\partial \theta}
-\frac{\partial v'_{\theta}}{\partial \widetilde{x}} 
\right) \vec{e}_r
\\
+\left(
\frac{\partial v'_r}{\partial \widetilde{x}}
-\frac{\partial v'_x}{\partial \widetilde{r}}
\right) \vec{e}_{\theta}
\\
+
\frac{1}{\widetilde{r}} 
\left(
  \frac{\partial}{\partial \widetilde{r}} \left(\widetilde{r} v'_{\theta} \right)
- \frac{\partial v'_r}{\partial \theta}
\right) \vec{e}_x
\end{array}
\right)
\nonumber
\end{eqnarray}

Remembering the definitions:

\begin{eqnarray}
v'_r &=& v_r \left(r \right) e^{i \left(k_x x + m \theta - \omega t \right)}
\nonumber
\\
v'_{\theta} &=& v_{\theta} \left(r \right) e^{i \left(k_x x + m \theta - \omega t \right)}
\nonumber
\\
v'_x &=& v_x \left(r \right) e^{i \left(k_x x + m \theta - \omega t \right)}
\nonumber
\\
p' &=& p \left(r \right) e^{i \left(k_x x + m \theta - \omega t \right)}
\nonumber
\end{eqnarray}

with the nondimensional counterparts:

\begin{eqnarray}
v'_r &=& A \widetilde{v}_r \left(\widetilde{r} \right) e^{i \left(\overline{\gamma} \widetilde{x} + m \theta - \omega t \right)}
\nonumber
\\
v'_{\theta} &=& A \widetilde{v}_{\theta} \left(\widetilde{r} \right) e^{i \left(\overline{\gamma} \widetilde{x} + m \theta - \omega t \right)}
\nonumber
\\
v'_x &=& A \widetilde{v}_x \left(\widetilde{r} \right) e^{i \left(\overline{\gamma} \widetilde{x} + m \theta - \omega t \right)}
\nonumber
\\
p' &=& \overline{\rho} A^2 \widetilde{p} \left(\widetilde{r} \right) 
e^{i \left(\overline{\gamma} \widetilde{x} + m \theta - \omega t \right)}
\nonumber
\end{eqnarray}

where:

\begin{eqnarray}
\overline{\gamma} &=& k_x r_T
\nonumber
\end{eqnarray}

The axial derivatives are:

\begin{eqnarray}
\frac{\partial v'_r}{\partial x}
&=&
\frac{1}{r_T}
\frac{\partial v'_r}{\partial \widetilde{x}}
\nonumber
\\
&=&
i \overline{\gamma} \widetilde{v}_r
\left(
\frac{A}{r_T}
e^{i \left(\overline{\gamma} \widetilde{x} + m \theta - \omega t \right)}
\right)
\nonumber
\\
\frac{\partial v'_{\theta}}{\partial x}
&=&
i \overline{\gamma} \widetilde{v}_{\theta}
\left(
\frac{A}{r_T}
e^{i \left(\overline{\gamma} \widetilde{x} + m \theta - \omega t \right)}
\right)
\nonumber
\\
\frac{\partial v'_x}{\partial x}
&=&
i \overline{\gamma} \widetilde{v}_x
\left(
\frac{A}{r_T}
e^{i \left(\overline{\gamma} \widetilde{x} + m \theta - \omega t \right)}
\right)
\nonumber
\end{eqnarray}

The azimuthal derivatives are:

\begin{eqnarray}
\frac{1}{r}
\frac{\partial v'_r}{\partial \theta}
&=&
\frac{i m \widetilde{v}_r}{\widetilde{r}}
\left(
\frac{A}{r_T}
e^{i \left(\overline{\gamma} \widetilde{x} + m \theta - \omega t \right)}
\right)
\nonumber
\\
\frac{1}{r}
\frac{\partial v'_{\theta}}{\partial \theta}
&=&
\frac{i m \widetilde{v}_{\theta}}{\widetilde{r}}
\left(
\frac{A}{r_T}
e^{i \left(\overline{\gamma} \widetilde{x} + m \theta - \omega t \right)}
\right)
\nonumber
\\
\frac{1}{r}
\frac{\partial v'_x}{\partial \theta}
&=&
\frac{i m \widetilde{v}_x}{\widetilde{r}}
\left(
\frac{A}{r_T}
e^{i \left(\overline{\gamma} \widetilde{x} + m \theta - \omega t \right)}
\right)
\nonumber
\end{eqnarray}

To do the radial derivatives, note that:

\begin{eqnarray}
\frac{\partial A}{\partial r}
&=&
\frac{A}{r_T} 
\left(\frac{\gamma-1}{2} \right) \frac{M^2_{\theta}}{\widetilde{r}}
\nonumber
\end{eqnarray}

The radial derivatives (remembering that $A = A\left(r \right)$!) are:

\begin{eqnarray}
\frac{\partial v'_r}{\partial r}
&=&
\left(
\widetilde{v'_r}
\frac{\partial A}{\partial r}
+
\frac{A}{r_T}
\frac{\partial v'_r}{\partial \widetilde{r}} 
\right)
e^{i \left(\overline{\gamma} \widetilde{x} + m \theta - \omega t \right)}
\nonumber
\\
&=&
\left(
\left(\frac{\gamma-1}{2} \right)
\frac{M_{\theta}^2}{\widetilde{r}}
\widetilde{v_r}
+
\frac{\partial \widetilde{v}_r}{\partial \widetilde{r}} 
\right)
\left(
\frac{A}{r_T}
e^{i \left(\overline{\gamma} \widetilde{x} + m \theta - \omega t \right)}
\right)
\nonumber
\\
\frac{\partial v'_{\theta}}{\partial r}
&=&
\left(
\left(\frac{\gamma-1}{2} \right)
\frac{M_{\theta}^2}{\widetilde{r}}
\widetilde{v}_{\theta}
+
\frac{\partial \widetilde{v}_{\theta}}{\partial \widetilde{r}} 
\right)
\left(
\frac{A}{r_T}
e^{i \left(\overline{\gamma} \widetilde{x} + m \theta - \omega t \right)}
\right)
\nonumber
\\
\frac{\partial v'_x}{\partial r}
&=&
\left(
\left(\frac{\gamma-1}{2} \right)
\frac{M_{\theta}^2}{\widetilde{r}}
\widetilde{v}_x
+
\frac{\partial \widetilde{v}_x}{\partial \widetilde{r}} 
\right)
\left(
\frac{A}{r_T}
e^{i \left(\overline{\gamma} \widetilde{x} + m \theta - \omega t \right)}
\right)
\nonumber
\end{eqnarray}

gives:

\begin{eqnarray}
\vec{\nabla} \cdot \vec{v'} &=&
\left(
\frac{ \widetilde{v}_r}{\widetilde{r}}
+
\left(
\frac{\gamma-1}{2}
\right)
\frac{
M_{\theta}^2
 \widetilde{v}_r
}{\widetilde{r}}
+ \frac{\partial \widetilde{v}_r}{\partial \widetilde{r}}
+ \frac{i m}{\widetilde{r}} \widetilde{v}_{\theta}
+ i \overline{\gamma} \widetilde{v}_x
\right)
\left(
\frac{A}{r_T}
e^{i \left(\overline{\gamma} \widetilde{x} + m \theta - \omega t \right)}
\right)
\nonumber
\\
\vec{\nabla} \times \vec{v'}
&=&
\left(
\begin{array}{r}
\left(
\frac{i m \widetilde{v}_x}{\widetilde{r}}
-
i \overline{\gamma} \widetilde{v}_{\theta}
\right) \vec{e}_r
\\
+\left(
i \overline{\gamma} \widetilde{v}_r
-
\left(
\left(\frac{\gamma-1}{2} \right)
\frac{M_{\theta}^2}{\widetilde{r}}
\widetilde{v}_x
+
\frac{\partial \widetilde{v}_x}{\partial \widetilde{r}} 
\right)
\right) \vec{e}_{\theta}
\\
+
\left(
\frac{  \widetilde{v}_{\theta}}{\widetilde{r}}
+ 
\left(
\left(\frac{\gamma-1}{2} \right)
\frac{M_{\theta}^2}{\widetilde{r}}
\widetilde{v}_{\theta}
+
\frac{\partial \widetilde{v}_{\theta}}{\partial \widetilde{r}} 
\right)
-\frac{i m \widetilde{v}_r}{\widetilde{r}}
\right) \vec{e}_x
\end{array}
\right)
\left(
\frac{A}{r_T}
e^{i \left(\overline{\gamma} \widetilde{x} + m \theta - \omega t \right)}
\right)
\nonumber
\end{eqnarray}

If all three vorticity components are zero, the perturbations will
be irrotational.  From the first two velocity components, this gives:

\begin{eqnarray}
\widetilde{v}_{\theta}
&=&
\frac{m}{\overline{\gamma} \widetilde{r}} \widetilde{v}_{x}
\nonumber
\\
\widetilde{v}_r
&=&
\frac{-i}{ \overline{\gamma}}
\left(
\left(\frac{\gamma-1}{2} \right)
\frac{M_{\theta}^2}{\widetilde{r}}
\widetilde{v}_x
+
\frac{\partial \widetilde{v}_x}{\partial \widetilde{r}} 
\right)
\nonumber
\end{eqnarray}

As a test, the last vorticity component is:

\begin{eqnarray}
\frac{  \widetilde{v}_{\theta}}{\widetilde{r}}
+ 
\left(\frac{\gamma-1}{2} \right)
\frac{M_{\theta}^2}{\widetilde{r}}
\widetilde{v}_{\theta}
+
\frac{\partial \widetilde{v}_{\theta}}{\partial \widetilde{r}} 
-\frac{i m}{\widetilde{r}}
\widetilde{v}_r
&=&
\left(
\begin{array}{r}
\frac{m}{\overline{\gamma}\widetilde{r}}
\left(
\frac{1}{\widetilde{r}}
+ 
\left(\frac{\gamma-1}{2} \right)
\frac{M_{\theta}^2}{\widetilde{r}}
\right)
\widetilde{v}_{x}
\\
+
\frac{\partial }{\partial \widetilde{r}} 
\left(
\frac{m}{\overline{\gamma} \widetilde{r}} \widetilde{v}_{x}
\right)
\\
-\frac{ m}
{ \overline{\gamma}
\widetilde{r}}
\left(
\left(\frac{\gamma-1}{2} \right)
\frac{M_{\theta}^2}{\widetilde{r}}
\widetilde{v}_x
+
\frac{\partial \widetilde{v}_x}{\partial \widetilde{r}} 
\right)
\end{array}
\right)
\nonumber
\\
&=&
\left(
\begin{array}{r}
\frac{m}{\overline{\gamma}\widetilde{r}}
\left(
\frac{1}{\widetilde{r}}
+ 
\left(\frac{\gamma-1}{2} \right)
\frac{M_{\theta}^2}{\widetilde{r}}
\right)
\widetilde{v}_{x}
\\
-
\frac{m}{\overline{\gamma} \widetilde{r}^2} \widetilde{v}_{x}
\\
+
\frac{m}{\overline{\gamma} \widetilde{r}} 
\frac{\partial \widetilde{v}_{x}
 }{\partial \widetilde{r}} 
\\
-\frac{ m}
{ \overline{\gamma}
\widetilde{r}}
\left(
\left(\frac{\gamma-1}{2} \right)
\frac{M_{\theta}^2}{\widetilde{r}}
\widetilde{v}_x
+
\frac{\partial \widetilde{v}_x}{\partial \widetilde{r}} 
\right)
\end{array}
\right)
\nonumber
\\
&=& 0
\nonumber
\end{eqnarray}

showing that these are the irrotational relations between the
perturbation velocities.

\subsection{Velocity decomposition}

Let's decompose the perturbation velocity field into
irrotational and divergence-free components:

\begin{eqnarray}
\widetilde{v}_x &=& 
\widetilde{v}_{x,\omega}
+\widetilde{v}_{x,\phi}
\nonumber
\\
\widetilde{v}_r &=& 
\widetilde{v}_{r,\omega}
+\widetilde{v}_{r,\phi}
\nonumber
\\
\widetilde{v}_{\theta} &=& 
\widetilde{v}_{\theta,\omega}
+\widetilde{v}_{\theta,\phi}
\nonumber
\end{eqnarray}

with the relations:

\begin{eqnarray}
\widetilde{v}_{\theta,\phi}
&=&
\frac{m}{\overline{\gamma} \widetilde{r}} \widetilde{v}_{x,\phi}
\nonumber
\\
\widetilde{v}_{r,\phi}
&=&
\frac{-i}{ \overline{\gamma}}
\left(
\left(\frac{\gamma-1}{2} \right)
\frac{M_{\theta}^2}{\widetilde{r}}
\widetilde{v}_{x,\phi}
+
\frac{\partial \widetilde{v}_{x,\phi}}{\partial \widetilde{r}} 
\right)
\nonumber
\end{eqnarray}

The velocity divergence gives one relation for the rotational
components of the perturbation velocities:

\begin{eqnarray}
\widetilde{v}_{x,\omega}
&=&
\frac{i}{\overline{\gamma}}
\left(
\frac{ \widetilde{v}_{r,\omega}}{\widetilde{r}}
+
\left(
\frac{\gamma-1}{2}
\right)
\frac{
M_{\theta}^2
 \widetilde{v}_{r,\omega}
}{\widetilde{r}}
+ \frac{\partial \widetilde{v}_{r,\omega}}{\partial \widetilde{r}}
\right)
- \frac{m}{\overline{\gamma} \widetilde{r}} \widetilde{v}_{\theta,\omega}
\nonumber
\\
&=&
i
\left(
\frac{ 1 + \Gamma}{\overline{\gamma} \widetilde{r}}
\right)
 \widetilde{v}_{r,\omega}
+ 
\frac{i}{\overline{\gamma}}
\frac{\partial \widetilde{v}_{r,\omega}}{\partial \widetilde{r}}
- \frac{m}{\overline{\gamma} \widetilde{r}} \widetilde{v}_{\theta,\omega}
\nonumber
\end{eqnarray}

As a check, the velocity perturbations
are put into the velocity divergence equation:

\begin{eqnarray}
\vec{\nabla} \cdot \vec{v'} &=&
\left(
\frac{ \widetilde{v}_r}{\widetilde{r}}
+
\left(
\frac{\gamma-1}{2}
\right)
\frac{
M_{\theta}^2
 \widetilde{v}_r
}{\widetilde{r}}
+ \frac{\partial \widetilde{v}_r}{\partial \widetilde{r}}
+ \frac{i m}{\widetilde{r}} \widetilde{v}_{\theta}
+ i \overline{\gamma} \widetilde{v}_x
\right)
\nonumber
\\
&=&
\left(
\begin{array}{r}
\frac{ \widetilde{v}_{r,\phi}}{\widetilde{r}}
+
\left(
\frac{\gamma-1}{2}
\right)
\frac{
M_{\theta}^2
 \widetilde{v}_{r,\phi}
}{\widetilde{r}}
+ \frac{\partial \widetilde{v}_{r,\phi}}{\partial \widetilde{r}}
+ \frac{i m}{\widetilde{r}} \widetilde{v}_{\theta,\phi}
+ i \overline{\gamma} \widetilde{v}_{x,\phi}
\\
+\frac{ \widetilde{v}_{r,\omega}}{\widetilde{r}}
+
\left(
\frac{\gamma-1}{2}
\right)
\frac{
M_{\theta}^2
 \widetilde{v}_{r,\omega}
}{\widetilde{r}}
+ \frac{\partial \widetilde{v}_{r,\omega}}{\partial \widetilde{r}}
+ \frac{i m}{\widetilde{r}} \widetilde{v}_{\theta,\omega}
+ i \overline{\gamma} \widetilde{v}_{x,\omega}
\end{array}
\right)
\nonumber
\\
&=&
\left(
\begin{array}{r}
\frac{1}{\widetilde{r}}
\left(
1
+
\left(
\frac{\gamma-1}{2}
\right)
M_{\theta}^2
\right)
\frac{-i}{ \overline{\gamma}}
\left(
\left(\frac{\gamma-1}{2} \right)
\frac{M_{\theta}^2}{\widetilde{r}}
\widetilde{v}_{x,\phi}
+
\frac{\partial \widetilde{v}_{x,\phi}}{\partial \widetilde{r}} 
\right)
\\
+ \frac{\partial }{\partial \widetilde{r}}
\left(
\frac{-i}{ \overline{\gamma}}
\left(
\left(\frac{\gamma-1}{2} \right)
\frac{M_{\theta}^2}{\widetilde{r}}
\widetilde{v}_{x,\phi}
+
\frac{\partial \widetilde{v}_{x,\phi}}{\partial \widetilde{r}} 
\right)
\right)
\\
+ \frac{i m^2}
{\overline{\gamma} \widetilde{r}^2} \widetilde{v}_{x,\phi}
\\
+ i \overline{\gamma} \widetilde{v}_{x,\phi}
\\
+\frac{ \widetilde{v}_{r,\omega}}{\widetilde{r}}
+
\left(
\frac{\gamma-1}{2}
\right)
\frac{
M_{\theta}^2
 \widetilde{v}_{r,\omega}
}{\widetilde{r}}
+ \frac{\partial \widetilde{v}_{r,\omega}}{\partial \widetilde{r}}
+ \frac{i m}{\widetilde{r}} \widetilde{v}_{\theta,\omega}
\\
+ i \overline{\gamma} 
\left(
\frac{i}{\overline{\gamma}}
\left(
\frac{ \widetilde{v}_{r,\omega}}{\widetilde{r}}
+
\left(
\frac{\gamma-1}{2}
\right)
\frac{
M_{\theta}^2
 \widetilde{v}_{r,\omega}
}{\widetilde{r}}
+ \frac{\partial \widetilde{v}_{r,\omega}}{\partial \widetilde{r}}
\right)
- \frac{m}{\overline{\gamma} \widetilde{r}} \widetilde{v}_{\theta,\omega}
\right)
\end{array}
\right)
\nonumber
\\
&=&
\left(
\begin{array}{r}
\frac{-i}{\overline{\gamma} \widetilde{r}^2}
\left(
1
+
\left(
\frac{\gamma-1}{2}
\right)
M_{\theta}^2
\right)
\left(
\left(\frac{\gamma-1}{2} \right)
M_{\theta}^2
\right)
\widetilde{v}_{x,\phi}
\\
-\frac{i}{\overline{\gamma} \widetilde{r}}
\left(
1
+
\left(
\frac{\gamma-1}{2}
\right)
M_{\theta}^2
\right)
\left(
\frac{\partial \widetilde{v}_{x,\phi}}{\partial \widetilde{r}} 
\right)
\\
-\frac{i}{ \overline{\gamma}}
\frac{\partial }{\partial \widetilde{r}}
\left(
\left(\frac{\gamma-1}{2} \right)
\frac{M_{\theta}^2}{\widetilde{r}}
\widetilde{v}_{x,\phi}
\right)
\\
-\frac{i}{ \overline{\gamma}}
\frac{\partial^2 \widetilde{v}_{x,\phi}}{\partial \widetilde{r}^2} 
\\
+ i \overline{\gamma} 
\left(
1 
+ 
\frac{m^2}
{\overline{\gamma}^2 \widetilde{r}^2} 
\right) \widetilde{v}_{x,\phi}
\\
+\frac{ \widetilde{v}_{r,\omega}}{\widetilde{r}}
+
\left(
\frac{\gamma-1}{2}
\right)
\frac{
M_{\theta}^2
 \widetilde{v}_{r,\omega}
}{\widetilde{r}}
+ \frac{\partial \widetilde{v}_{r,\omega}}{\partial \widetilde{r}}
+ \frac{i m}{\widetilde{r}} \widetilde{v}_{\theta,\omega}
\\
-
\left(
\frac{ \widetilde{v}_{r,\omega}}{\widetilde{r}}
+
\left(
\frac{\gamma-1}{2}
\right)
\frac{
M_{\theta}^2
 \widetilde{v}_{r,\omega}
}{\widetilde{r}}
+ \frac{\partial \widetilde{v}_{r,\omega}}{\partial \widetilde{r}}
\right)
- \frac{i m}{\widetilde{r}} \widetilde{v}_{\theta,\omega}
\end{array}
\right)
\nonumber
\\
&=&
\left(
\begin{array}{r}
i \overline{\gamma} 
\left(
\left(
1 
+ 
\frac{m^2}
{\overline{\gamma}^2 \widetilde{r}^2} 
\right) 
-\frac{1}{\overline{\gamma}^2 \widetilde{r}^2}
\left(
1
+
\left(
\frac{\gamma-1}{2}
\right)
M_{\theta}^2
\right)
\left(
\left(\frac{\gamma-1}{2} \right)
M_{\theta}^2
\right)
\right)
\widetilde{v}_{x,\phi}
\\
-\frac{i}{\overline{\gamma} \widetilde{r}}
\left(
1
+
\left(
\frac{\gamma-1}{2}
\right)
M_{\theta}^2
\right)
\left(
\frac{\partial \widetilde{v}_{x,\phi}}{\partial \widetilde{r}} 
\right)
\\
+\frac{i}{ \overline{\gamma}}
\left(
\left(\frac{\gamma-1}{2} \right)
\frac{M_{\theta}^2}{\widetilde{r}^2}
\widetilde{v}_{x,\phi}
\right)
\\
-\frac{i}{ \overline{\gamma}}
\frac{\partial M_{\theta} }{\partial \widetilde{r}}
\left(
\left(\frac{\gamma-1}{2} \right)
\frac{2 M_{\theta}}{\widetilde{r}}
\widetilde{v}_{x,\phi}
\right)
\\
-\frac{i}{ \overline{\gamma} \widetilde{r}}
\left(
\left(\frac{\gamma-1}{2} \right)
M_{\theta}^2
\frac{\partial 
\widetilde{v}_{x,\phi}
 }{\partial \widetilde{r}}
\right)
\\
-\frac{i}{ \overline{\gamma}}
\frac{\partial^2 \widetilde{v}_{x,\phi}}{\partial \widetilde{r}^2} 
\end{array}
\right)
\nonumber
\\
&=&
\left(
\begin{array}{r}
i 
\left(
\overline{\gamma} 
+\frac{1 }{\overline{\gamma} \widetilde{r}^2}
\left(
m^2
- \Gamma^2
-\widetilde{r}
\frac{\partial \Gamma}{\partial \widetilde{r}}
\right)
\right)
\widetilde{v}_{x,\phi}
\\
-\frac{i}{\overline{\gamma} \widetilde{r}}
\left(
1
+
2 \Gamma
\right)
\frac{\partial \widetilde{v}_{x,\phi}}{\partial \widetilde{r}} 
\\
-\frac{i}{ \overline{\gamma}}
\frac{\partial^2 \widetilde{v}_{x,\phi}}{\partial \widetilde{r}^2} 
\end{array}
\right)
\nonumber
\end{eqnarray}

where

\begin{eqnarray}
\Gamma &=&
\left(
\frac{\gamma-1}{2}
\right)
M_{\theta}^2
\nonumber
\end{eqnarray}

Note that the divergence of velocity can be written solely in
terms of $\widetilde{v}_{x,\phi}$:

\begin{eqnarray}
\left(
\frac{ \widetilde{v}_r}{\widetilde{r}}
+
\left(
\frac{\gamma-1}{2}
\right)
\frac{
M_{\theta}^2
 \widetilde{v}_r
}{\widetilde{r}}
+ \frac{\partial \widetilde{v}_r}{\partial \widetilde{r}}
+ \frac{i m}{\widetilde{r}} \widetilde{v}_{\theta}
+ i \overline{\gamma} \widetilde{v}_x
\right)
&=&
\Phi_0 
\widetilde{v}_{x,\phi}
+\Phi_1 
\frac{\partial
\widetilde{v}_{x,\phi}
}{\partial \widetilde{r}} 
+\Phi_2 
\frac{\partial^2
\widetilde{v}_{x,\phi}
}{\partial \widetilde{r}^2} 
\nonumber
\end{eqnarray}

where

\begin{eqnarray}
\Phi_0 
&=&
i 
\left(
\overline{\gamma} 
+\frac{1 }{\overline{\gamma} \widetilde{r}^2}
\left(
m^2
- \Gamma^2
-\widetilde{r}
\frac{\partial \Gamma}{\partial \widetilde{r}}
\right)
\right)
\nonumber
\\
\Phi_1 
&=&
-\frac{i}{\overline{\gamma} \widetilde{r}}
\left(
1
+
2 \Gamma
\right)
\nonumber
\\
\Phi_2 
&=&
-\frac{i}{ \overline{\gamma}}
\nonumber
\end{eqnarray}

\subsection{Back to business...}

The SWIRL code equations are:

\begin{small}
\begin{eqnarray}
-i
\left(
 \frac{k}{ \widetilde{A}}
- \frac{ m M_{\theta} }{\widetilde{r}}
- \overline{\gamma} M_x  
\right) \widetilde{v}_r 
-\frac{2 M_{\theta}}{\widetilde{r}} \widetilde{v}_{\theta} 
 = 
-
\frac{\partial \widetilde{p} }{\partial \widetilde{r}} 
-
\frac{\left(\gamma - 1 \right) M_{\theta}^2}{\widetilde{r}} \widetilde{p}
\nonumber
\\
-i
\left(
\frac{k}{\widetilde{A}}
- \frac{ m M_{\theta} }{\widetilde{r} }
- \overline{\gamma} M_x  
\right) \widetilde{v}_{\theta} 
+
\left(
\frac{M_{\theta} }{\widetilde{r}}
+
\frac{\partial M_{\theta} }{\partial \widetilde{r}}
+
\left(\frac{\gamma - 1}{2} \right) \frac{M_{\theta}^3}{\widetilde{r}}
\right) 
\widetilde{v}_r
 = 
-\frac{i m }{\widetilde{r} } \widetilde{p}
\nonumber
\\
-i
\left(
\frac{k}{\widetilde{A}}
- \frac{ m M_{\theta} }{\widetilde{r} }
- \overline{\gamma} M_x
\right) \widetilde{v}_x 
+
\left(
\frac{\partial M_x }{\partial \widetilde{r}} 
+
\left(\frac{\gamma - 1}{2} \right) \frac{M_x M_{\theta}^2}{\widetilde{r}}
\right)
\widetilde{v}_r
 = 
-i \overline{\gamma} \widetilde{p}
\nonumber
\\
-i
\left(
\frac{k}{\widetilde{A}}
- \frac{ m M_{\theta}}{\widetilde{r} } 
-  \overline{\gamma} M_x
\right) \widetilde{p}
+ 
\frac{\partial \widetilde{v}_r }{\partial \widetilde{r}}
+ 
\left(
\left(\frac{\gamma + 1}{2} \right) \frac{M_{\theta}^2}{\widetilde{r}}
+\frac{1 }{\widetilde{r}} 
\right)
\widetilde{v}_r
+\frac{i m }{\widetilde{r}} \widetilde{v}_{\theta} 
+ i \overline{\gamma} \widetilde{v}_x 
 = 
0
\nonumber
\end{eqnarray}
\end{small}%

Defining:

\begin{eqnarray}
\alpha &=&
 \frac{k}{ \widetilde{A}}
- \frac{ m M_{\theta} }{\widetilde{r}}
- \overline{\gamma} M_x  
\nonumber
\end{eqnarray}

and expanding the pressure equation to isolate the divergence
of velocity term gives:

\begin{small}
\begin{eqnarray}
-i \alpha \widetilde{v}_r 
-\frac{2 M_{\theta}}{\widetilde{r}} \widetilde{v}_{\theta} 
&=&
-
\frac{\partial \widetilde{p} }{\partial \widetilde{r}} 
-
\frac{\left(\gamma - 1 \right) M_{\theta}^2}{\widetilde{r}} \widetilde{p}
\nonumber
\\
-i
\alpha
 \widetilde{v}_{\theta} 
+
\left(
\frac{M_{\theta} }{\widetilde{r}}
+
\frac{\partial M_{\theta} }{\partial \widetilde{r}}
+
\left(\frac{\gamma - 1}{2} \right) \frac{M_{\theta}^3}{\widetilde{r}}
\right) 
\widetilde{v}_r
&=&
-\frac{i m }{\widetilde{r} } \widetilde{p}
\nonumber
\\
-i
\alpha
\widetilde{v}_x 
+
\left(
\frac{\partial M_x }{\partial \widetilde{r}} 
+
\left(\frac{\gamma - 1}{2} \right) \frac{M_x M_{\theta}^2}{\widetilde{r}}
\right)
\widetilde{v}_r
&=&
-i \overline{\gamma} \widetilde{p}
\nonumber
\\
-i
\alpha
 \widetilde{p}
+
\frac{M_{\theta}^2}{\widetilde{r}}
\widetilde{v}_r
+ 
\left(
\Phi_0 
\widetilde{v}_{x,\phi}
+\Phi_1 
\frac{\partial
\widetilde{v}_{x,\phi}
}{\partial \widetilde{r}} 
+\Phi_2 
\frac{\partial^2
\widetilde{v}_{x,\phi}
}{\partial \widetilde{r}^2} 
\right)
&=&
0
\nonumber
\end{eqnarray}
\end{small}%

Rewriting this in terms of the rotational and irrotational velocities,


\begin{eqnarray}
-i \alpha 
\left(
\widetilde{v}_{r,\phi} 
+\widetilde{v}_{r,\omega} 
\right)
-\frac{2 M_{\theta}}{\widetilde{r}} 
\left(
\widetilde{v}_{\theta,\phi} 
+\widetilde{v}_{\theta,\omega} 
\right)
&=&
-
\frac{\partial \widetilde{p} }{\partial \widetilde{r}} 
-
\frac{2 \Gamma}{\widetilde{r}} \widetilde{p}
\nonumber
\\
-i
\alpha
\left(
\widetilde{v}_{\theta,\phi} 
+\widetilde{v}_{\theta,\omega} 
\right)
+
\left(
\left(1 + \Gamma \right)
\frac{M_{\theta} }{\widetilde{r}}
+
\frac{\partial M_{\theta} }{\partial \widetilde{r}}
\right) 
\left(
\widetilde{v}_{r,\phi} 
+\widetilde{v}_{r,\omega} 
\right)
&=&
-\frac{i m }{\widetilde{r} } \widetilde{p}
\nonumber
\\
-i
\alpha
\left(
\widetilde{v}_{x,\phi} 
+\widetilde{v}_{x,\omega} 
\right)
+
\left(
\frac{\partial M_x }{\partial \widetilde{r}} 
+
\Gamma \frac{M_x }{\widetilde{r}}
\right)
\left(
\widetilde{v}_{r,\phi} 
+\widetilde{v}_{r,\omega} 
\right)
&=&
-i \overline{\gamma} \widetilde{p}
\nonumber
\\
-i
\alpha
 \widetilde{p}
+
\frac{M_{\theta}^2}{\widetilde{r}}
\left(
\widetilde{v}_{r,\phi} 
+\widetilde{v}_{r,\omega} 
\right)
+ 
\left(
\Phi_0 
\widetilde{v}_{x,\phi}
+\Phi_1 
\frac{\partial
\widetilde{v}_{x,\phi}
}{\partial \widetilde{r}} 
+\Phi_2 
\frac{\partial^2
\widetilde{v}_{x,\phi}
}{\partial \widetilde{r}^2} 
\right)
&=&
0
\nonumber
\end{eqnarray}

Substituting in:

\begin{eqnarray}
\left(
\begin{array}{r}
-i \alpha 
\left(
\frac{-i}{ \overline{\gamma}}
\left(
\frac{\Gamma}{\widetilde{r}}
\widetilde{v}_{x,\phi}
+
\frac{\partial \widetilde{v}_{x,\phi}}{\partial \widetilde{r}} 
\right)
+\widetilde{v}_{r,\omega} 
\right)
\\
-\frac{2 M_{\theta}}{\widetilde{r}} 
\left(
\frac{m}{\overline{\gamma} \widetilde{r}} \widetilde{v}_{x,\phi}
+\widetilde{v}_{\theta,\omega} 
\right)
\end{array}
\right)
&=&
-
\frac{\partial \widetilde{p} }{\partial \widetilde{r}} 
-
\frac{2 \Gamma}{\widetilde{r}} \widetilde{p}
\nonumber
\\
\left(
\begin{array}{r}
-i
\alpha
\left(
\frac{m}{\overline{\gamma} \widetilde{r}} \widetilde{v}_{x,\phi}
+\widetilde{v}_{\theta,\omega} 
\right)
\\
+
\left(
\left(1 + \Gamma \right)
\frac{M_{\theta} }{\widetilde{r}}
+
\frac{\partial M_{\theta} }{\partial \widetilde{r}}
\right) 
\left(
\frac{-i}{ \overline{\gamma}}
\left(
\frac{\Gamma}{\widetilde{r}}
\widetilde{v}_{x,\phi}
+
\frac{\partial \widetilde{v}_{x,\phi}}{\partial \widetilde{r}} 
\right)
+\widetilde{v}_{r,\omega} 
\right)
\end{array}
\right)
&=&
-\frac{i m }{\widetilde{r} } \widetilde{p}
\nonumber
\\
\left(
\begin{array}{r}
-i
\alpha
\left(
\widetilde{v}_{x,\phi} 
+
i
\left(
\frac{ 1 + \Gamma}{\overline{\gamma} \widetilde{r}}
\right)
 \widetilde{v}_{r,\omega}
+ 
\frac{i}{\overline{\gamma}}
\frac{\partial \widetilde{v}_{r,\omega}}{\partial \widetilde{r}}
- \frac{m}{\overline{\gamma} \widetilde{r}} \widetilde{v}_{\theta,\omega}
\right)
\\
+
\left(
\frac{\partial M_x }{\partial \widetilde{r}} 
+
\Gamma \frac{M_x }{\widetilde{r}}
\right)
\left(
\frac{-i}{ \overline{\gamma}}
\left(
\frac{\Gamma}{\widetilde{r}}
\widetilde{v}_{x,\phi}
+
\frac{\partial \widetilde{v}_{x,\phi}}{\partial \widetilde{r}} 
\right)
+\widetilde{v}_{r,\omega} 
\right)
\end{array}
\right)
&=&
-i \overline{\gamma} \widetilde{p}
\nonumber
\\
\left(
\begin{array}{r}
-i
\alpha
 \widetilde{p}
\\
+
\frac{M_{\theta}^2}{\widetilde{r}}
\left(
\frac{-i}{ \overline{\gamma}}
\left(
\frac{\Gamma}{\widetilde{r}}
\widetilde{v}_{x,\phi}
+
\frac{\partial \widetilde{v}_{x,\phi}}{\partial \widetilde{r}} 
\right)
+\widetilde{v}_{r,\omega} 
\right)
\\
+ 
\left(
\Phi_0 
\widetilde{v}_{x,\phi}
+\Phi_1 
\frac{\partial
\widetilde{v}_{x,\phi}
}{\partial \widetilde{r}} 
+\Phi_2 
\frac{\partial^2
\widetilde{v}_{x,\phi}
}{\partial \widetilde{r}^2} 
\right)
\end{array}
\right)
&=&
0
\nonumber
\end{eqnarray}

Defining:

\begin{eqnarray}
\Theta &=&
\left(1 + \Gamma \right)
\frac{M_{\theta} }{\widetilde{r}}
+
\frac{\partial M_{\theta} }{\partial \widetilde{r}}
\nonumber
\\
\tau_{\phi}
&=&
\frac{\Gamma}{\widetilde{r}}
\widetilde{v}_{x,\phi}
+
\frac{\partial \widetilde{v}_{x,\phi}}{\partial \widetilde{r}} 
\nonumber
\\
\tau_{\omega}
&=&
\frac{\Gamma}{\widetilde{r}}
\widetilde{v}_{r,\omega}
+
\frac{\partial \widetilde{v}_{r,\omega}}{\partial \widetilde{r}} 
\nonumber
\\
S
&=&
\Gamma \frac{M_x }{\widetilde{r}}
+
\frac{\partial M_x }{\partial \widetilde{r}} 
\nonumber
\end{eqnarray}

and gathering the rotational and irrotational components together:

\begin{eqnarray}
\left(
\begin{array}{r}
\frac{\alpha}{ \overline{\gamma}} \tau_{\phi}
-\frac{2 m M_{\theta}}{\overline{\gamma} \widetilde{r}^2} 
\widetilde{v}_{x,\phi}
\\
-i \alpha \widetilde{v}_{r,\omega} 
-\frac{2 M_{\theta}}{\widetilde{r}} 
\widetilde{v}_{\theta,\omega} 
\end{array}
\right)
&=&
-
\frac{\partial \widetilde{p} }{\partial \widetilde{r}} 
-
\frac{2 \Gamma}{\widetilde{r}} \widetilde{p}
\nonumber
\\
\left(
\begin{array}{r}
-\frac{i \Theta}{ \overline{\gamma}}
\tau_{\phi}
-
\frac{i m \alpha}{\overline{\gamma} \widetilde{r}} 
\widetilde{v}_{x,\phi}
\\
+
\Theta
\widetilde{v}_{r,\omega} 
-i
\alpha
\widetilde{v}_{\theta,\omega} 
\end{array}
\right)
&=&
-\frac{i m }{\widetilde{r} } \widetilde{p}
\nonumber
\\
\left(
\begin{array}{r}
-i \alpha
\widetilde{v}_{x,\phi} 
+ 
\frac{\alpha}{\overline{\gamma}}
\tau_{\omega}
+
\frac{ \alpha }{\overline{\gamma}
\widetilde{r}}
 \widetilde{v}_{r,\omega}
+i 
 \frac{m \alpha}{\overline{\gamma} \widetilde{r}} \widetilde{v}_{\theta,\omega}
\\
-i
\frac{S}{ \overline{\gamma}}
\tau_{\phi}
+ S \widetilde{v}_{r,\omega} 
\end{array}
\right)
&=&
-i \overline{\gamma} \widetilde{p}
\nonumber
\\
\left(
\begin{array}{r}
-i
\alpha
 \widetilde{p}
\\
-i
\frac{M_{\theta}^2}{\overline{\gamma} \widetilde{r}}
\tau_{\phi}
\\
+ 
\Phi_0 
\widetilde{v}_{x,\phi}
+\Phi_1 
\frac{\partial
\widetilde{v}_{x,\phi}
}{\partial \widetilde{r}} 
+\Phi_2 
\frac{\partial^2
\widetilde{v}_{x,\phi}
}{\partial \widetilde{r}^2} 
\\
- i
\frac{M_{\theta}^2}
{
\widetilde{r}
}
\widetilde{v}_{r,\omega} 
\end{array}
\right)
&=&
0
\nonumber
\end{eqnarray}

I can't help but notice these groupings:

\begin{eqnarray}
A &=& 
\frac{1}{\overline{\gamma}} \tau_{\phi}
-i \widetilde{v}_{r,\omega}
\nonumber
\\
B &=&
\frac{m}{\overline{\gamma} \widetilde{r}} \widetilde{v}_{x,\phi}
+ \widetilde{v}_{\theta,\omega}
\nonumber
\end{eqnarray}

which gives:

\begin{eqnarray}
\alpha A
-\frac{2 M_{\theta}}{\widetilde{r}} B 
&=&
-
\frac{\partial \widetilde{p} }{\partial \widetilde{r}} 
-
\frac{2 \Gamma}{\widetilde{r}} \widetilde{p}
\nonumber
\\
i \Theta A
-
i \alpha B
&=&
-\frac{i m }{\widetilde{r} } \widetilde{p}
\nonumber
\\
\left(
\begin{array}{r}
-i \alpha
\widetilde{v}_{x,\phi} 
+ 
\frac{\alpha}{\overline{\gamma}}
\tau_{\omega}
+
\frac{ \alpha }{\overline{\gamma}
\widetilde{r}}
 \widetilde{v}_{r,\omega}
+i 
 \frac{m \alpha}{\overline{\gamma} \widetilde{r}} \widetilde{v}_{\theta,\omega}
\\
-i
S A
\end{array}
\right)
&=&
-i \overline{\gamma} \widetilde{p}
\nonumber
\\
\left(
\begin{array}{r}
-i
\alpha
 \widetilde{p}
\\
-i
\frac{M_{\theta}^2}{\widetilde{r}}
A
\\
+ 
\Phi_0 
\widetilde{v}_{x,\phi}
+\Phi_1 
\frac{\partial
\widetilde{v}_{x,\phi}
}{\partial \widetilde{r}} 
+\Phi_2 
\frac{\partial^2
\widetilde{v}_{x,\phi}
}{\partial \widetilde{r}^2} 
\end{array}
\right)
&=&
0
\nonumber
\end{eqnarray}

I can't help but notice these groupings:


\section{The SWIRL Code}

In this section, the Fortran2008 version of the SWIRL code is discussed.

The main routine is 'swirl.f90'.  The routines called (when Chebyshev polynomials
are used) are:

\begin{enumerate}
\item{input}
\item{grid}
\item{derivs}
\item{smachAndSndspd}
\item{rmach}
\item{machout}
\item{global}
\item{boundary}
\item{analysis}
\item{output}
\end{enumerate}

\subsection{input}

The subroutine 'input' is used to obtain the input data for the code.
Input reads a NAMELIST input from the file 'input.data'.  The NAMELIST
entries are:

\begin{enumerate}
\item{MM: circumferential mode number ($m$ in report).}
\item{NPTS: number of radial mesh points.}
\item{SIG: hub-to-duct radius ratio ($\sigma$ in report).}
\item{AKRE: real part of nondimensional frequency ($k = \frac{\omega r_T}{A_T}$ in report).}
\item{AKIM: imaginary part of nondimensional frequency ($k = \frac{\omega r_T}{A_T}$ in report).}
\item{IX: this parameter is not actually used in the code.}
\item{NX: this parameter is not actually used in the code.}
\item{IR: axial velocity distribution flag (used in rmachModule):
\begin{itemize}
\item{IR = 0: uniform axial flow velocity: 
\begin{eqnarray}
V_x &=& RMAX
\nonumber
\end{eqnarray}
}.
\item{IR = 1: linear shear: 
\begin{eqnarray}
V_x &=& \left(RMAX - SLOPE \left(\widetilde{r}-1.0 \right) \right)
\nonumber
\end{eqnarray}
}
\item{IR = 2: user axial velocity input from file 'mach.input'.  See rmachModule for
file format.
}
\item{IR = 3: uniform flow plus sine wave boundary layers of thickness $SLOPE$.
}
\item{IR = 4: uniform flow plus linear boundary layers of thickness $SLOPE$.
}
\item{IR = 5: uniform flow plus $\frac{1}{7}$ power boundary layers.
}
\item{IR = 6: hyperbolic secant profile.
}
\item{IR = 7: laminar mean flow:
\begin{eqnarray}
V_x &=& 
\frac{-4 RMAX}{\left(1-\sigma \right)^2}
\left(
\widetilde{r}^2
- \left(1+\sigma \right) \widetilde{r}
+ \sigma 
\right)
\nonumber
\end{eqnarray}
}
\item{IR = 8: wavy sinusoid for axial Mach number:
\begin{eqnarray}
M_x &=& 
\frac{RMAX + SLOPE}{2}
+
\frac{RMAX-SLOPE}{2} \sin{\left(4 \pi \left(\frac{2 \left(\widetilde{r}-\sigma \right)}{1-\sigma} - 1 \right) \right)}
\nonumber
\end{eqnarray}
}
\item{IR = 9: Hagen-Poiseuille flow:
\begin{eqnarray}
K &=&
\frac{\sigma^2-1}{\ln{\sigma}} 
\nonumber
\\
V_x &=& 
RMAX \left(
\frac{
1 + K \ln{r} - \widetilde{r}^2
}{
1 + \frac{K}{2} \ln{\frac{K}{2}} - \frac{K}{2}
}
\right)
\nonumber
\end{eqnarray}
}
\end{itemize}
}
\item{RMAX: maximum axial Mach number value.}
\item{SLOPE: slope of linear axial Mach number distribution.}
\item{IS: swirl Mach number distribution flag (0-6). See smach for details.
\begin{itemize}
\item{IS = 0: no swirl.
}
\item{IS = 1: solid-body swirl with angular velocity ANGOM.

\begin{eqnarray}
v_{\theta} &=& \left(ANGOM \right) \widetilde{r}
\nonumber
\end{eqnarray}

}
\item{IS = 2: free-vortex swirl with strength $GAM$.

\begin{eqnarray}
v_{\theta} &=& \frac{GAM}{ \widetilde{r}}
\nonumber
\end{eqnarray}

}
\item{IS = 3: combined solid-body and free-vortex swirl.

\begin{eqnarray}
v_{\theta} &=& \left(ANGOM \right) \widetilde{r}
+\frac{GAM}{ \widetilde{r}}
\nonumber
\end{eqnarray}

}
\item{IS = 4: Stability test case:

\begin{eqnarray}
v_{\theta} &=& \frac{1}{ \widetilde{r}^2}
\nonumber
\end{eqnarray}
}
\item{IS = 5: user-input azimuthal velocity profile from file 'swrl.input'. 
See smachAndSndspdModule for file format.
}
\item{IS = 6: constant swirl:

\begin{eqnarray}
v_{\theta} &=& ANGOM
\nonumber
\end{eqnarray}
}
\item{IS = 7: trailing line vortex (NOT COMPLETED IN CODE):

\begin{eqnarray}
v_{\theta}
&=&
\frac{GAM}{\widetilde{r}} \left(1 - e^{-\widetilde{r}^2} \right)
\nonumber
\end{eqnarray}
}
\end{itemize}
}
\item{ANGOM: magnitude of solid-body swirl.}
\item{GAM: magnitude of free-vortex swirl.}
\item{IREPEAT: this parameter is not actually used in the code.}
\item{IFD: flag to choose method for derivative calculation:
\begin{itemize}
\item{ IFD=0: Chebyshev polynomials}
\item{ IFD$\neq$ 0: Finite differencing}
\end{itemize}
}
\item{ITEST: this parameter is not actually used in the code.
}
\item{ETAHR: real part of hub liner admittance ($\eta_H$ in report).}
\item{ETAHI: imaginary part of hub liner admittance ($\eta_H$ in report).}
\item{ETADR: real part of duct liner admittance ($\eta_D$ in report).}
\item{ETADI: imaginary part of duct liner admittance ($\eta_D$ in report).}
\item{ED2: second derivative smoothing coefficient.}
\item{ED4: fourth derivative smoothing coefficient.}
\item{ICMPR: flag to perform consistency test on selected modes (currently does
not work):
\begin{itemize}
\item{ ICMPR=1: Perform consistency test.}
\item{ ICMPR $\neq$ 1: Do not perform consistency test.}
\end{itemize}
 
}
\end{enumerate}

\subsection{grid}

\subsection{derivs}

\subsection{smachAndSndspd}

\subsection{rmach}

\subsection{machout}

\subsection{globalM}

\subsection{boundary}

\subsection{analysis}

In this module, the eigendecomposition is actually performed.

The convected wavenumbers are computed as:

\begin{eqnarray}
\lambda_{cvct} \left(\widetilde{r} \right) &=& 
\frac{1}{M_x} 
\left(\frac{k}{\widetilde{A}} - \frac{m M_{\theta}}{\widetilde{r}} \right)
\nonumber
\end{eqnarray}

(noting that this is a range of convection speeds, due to the different
axial and azimuthal mean velocities)

The routine ZGEGV is used to obtain the eigenvalues and eigenvectors.
ZGEGV is a LAPACK routine that calculates the generalized eigenvalues
$\left(\alpha, \beta \right)$ and the generalized eigenvector $\vec{r}$
that satisfy the equation:

\begin{eqnarray}
\left(\left[A \right] - w \left[B \right] \right) \vec{r} &=& 0
\nonumber
\\
w &=& \frac{\alpha}{\beta}
\nonumber
\end{eqnarray}

For SWIRL, the result is:

\begin{eqnarray}
\lambda &=& \frac{\alpha}{\beta}
\nonumber
\\
\overline{\gamma} &=& \frac{i \alpha}{\beta}
\nonumber
\\
k_x &=& \frac{1}{r_t} \frac{i \alpha}{\beta}
\nonumber
\end{eqnarray}

The $\overline{\gamma}$ values are printed to the display,
with:

\begin{eqnarray}
\ldots
\end{eqnarray}


\subsection{output}

\section{SWIRL Verification cases}

In the Kousen report, there are a number of verification
cases presented.  In this section, the inputs for running
those cases are determined.

\subsection{Cylinder, Uniform Flow with Liner (Table 4.3)}

In Table 4.3, a comparison is shown for the case of a 
cylinder (no centerbody) with uniform flow.  For this
test case, 

\begin{eqnarray}
m &=& 2
\nonumber
\\
k &=& \frac{\omega r_T}{A_T} = -1
\nonumber
\\
M_x &=& 0.5
\nonumber
\\
\eta_T &=& 0.72 + 0.42 i
\nonumber
\end{eqnarray}

Recall that in the SWIRL code, $A_T$ is the speed of
sound at the outer radius, $r_T$.  Also, $\eta_T$ is the 
admittance of the liner on the outer wall.  The azimuthal
mode number is $m$.

The input deck for this test case is:

\begin{tiny}
\VerbatimInput{input.data.Table4.3}
\end{tiny}%

The resulting $\gamma$ values are given in 'gam.acc'.  The
data from left to right are:

\begin{enumerate}
\item{Eigenvalue ID number, j.}
\item{Real part of the nondimensional axial wavenumber $\overline{\gamma}_j$}
\item{Imaginary part of the nondimensional axial wavenumber $\overline{\gamma}_j$}
\item{Real part of $\frac{k}{\overline{\gamma}_j}$}
\item{Imaginary part of $\frac{k}{\overline{\gamma}_j}$}
\item{Real part of $\frac{\overline{\gamma}_j}{k}$}
\item{Imaginary part of $\frac{\overline{\gamma}_j}{k}$}
\end{enumerate}

(I have no idea why the last two columns are needed.)

The output in 'gam.acc' for this test case is:

\begin{tiny}
\VerbatimInput{gam.acc.Table4.3}
\end{tiny}%

Comparing to the data in Table 4.3 of the paper, the first eigenvalues are
well predicted.  However, there are a number of eigenvalues that lie
between the tabulated results for 
$\gamma_8^{+}$ and
$\gamma_9^{+}$, and between $\gamma_8^{-}$ and
$\gamma_9^{-}$.

\begin{table}
 \centering
 \begin{tabular}{c | r | r | r | r}
 \hline
 $\gamma^{\pm}_n$ & Kousen Ref. [15] & Kousen report & current & index  \\
 \hline
 $\gamma_0^{+}$ & $ 0.620 - 5.014  i $ & $ 0.6195 - 5.0139 i$ & $ 0.61954  - 5.01386 i$ & 60  \\
 $\gamma_1^{+}$ & $-5.820 - 3.897  i $ & $-5.8195 - 3.8968 i$ & $-5.81953  - 3.89677 i$ & 58  \\
 $\gamma_2^{+}$ & $ 0.445 - 9.187  i $ & $ 0.4453 - 9.1868 i$ & $ 0.44533  - 9.18684 i$ & 59  \\
 $\gamma_3^{+}$ & $ 0.453 - 13.062 i $ & $ 0.4539 - 13.062 i$ & $ 0.45389  - 13.0615 i$ & 57  \\
 $\gamma_4^{+}$ & $ 0.480 - 16.822 i $ & $ 0.4795 - 16.822 i$ & $ 0.47952  - 16.8216 i$ & 55  \\
 $\gamma_5^{+}$ & $ 0.503 - 20.531 i $ & $ 0.5029 - 20.531 i$ & $ 0.50287  - 20.5307 i$ & 51  \\
 $\gamma_6^{+}$ & $ 0.522 - 24.213 i $ & $ 0.5220 - 24.213 i$ & $ 0.52202  - 24.2129 i$ & 50  \\
 $\gamma_7^{+}$ & $ 0.538 - 27.880 i $ & $ 0.5376 - 27.880 i$ & $ 0.53754  - 27.8800 i$ & 48  \\
 $\gamma_8^{+}$ & $ 0.550 - 31.537 i $ & $ 0.5502 - 31.537 i$ & $ 0.55024  - 31.5368 i$ & 47  \\
 $\gamma_9^{+}$ & $ 0.589 - 49.75  i $ & $ 0.5891 - 49.754 i$ & $ 0.58745  - 49.7669 i$ & 33  \\ \hline
 $\gamma_0^{-}$ & $ 0.410 + 1.290  i $ & $ 0.4101 + 1.2904 i$ & $ 0.41009  + 1.29037 i$ & 64  \\
 $\gamma_1^{-}$ & $ 1.259 + 6.085  i $ & $ 1.2595 + 6.0852 i$ & $ 1.25949  + 6.08517 i$ & 63  \\
 $\gamma_2^{-}$ & $ 1.146 + 9.668  i $ & $ 1.1457 + 9.6679 i$ & $ 1.14567  + 9.66787 i$ & 62  \\
 $\gamma_3^{-}$ & $ 1.022 + 13.315 i $ & $ 1.0218 + 13.315 i$ & $ 1.02183  + 13.3150 i$ & 61  \\
 $\gamma_4^{-}$ & $ 0.943 + 16.977 i $ & $ 0.9425 + 16.977 i$ & $ 0.94250  + 16.9767 i$ & 56  \\
 $\gamma_5^{-}$ & $ 0.891 + 20.635 i $ & $ 0.8908 + 20.635 i$ & $ 0.89075  + 20.6353 i$ & 54  \\
 $\gamma_6^{-}$ & $ 0.855 + 24.288 i $ & $ 0.8549 + 24.288 i$ & $ 0.85490  + 24.2883 i$ & 53  \\
 $\gamma_7^{-}$ & $ 0.829 + 27.937 i $ & $ 0.8288 + 27.937 i$ & $ 0.82877  + 27.9369 i$ & 52  \\
 $\gamma_8^{-}$ & $ 0.809 + 31.581 i $ & $ 0.8089 + 31.581 i$ & $ 0.80891  + 31.5812 i$ & 49  \\
 $\gamma_9^{-}$ & $ 0.755 + 49.77  i $ & $ 0.7547 + 49.772 i$ & $ 0.75658  + 49.7851 i$ & 39  \\ \hline
 \end{tabular}
 \caption{Table 4.3 data}
 \label{Table43}
\end{table}

\subsection{Cylinder, Shear Flow (Table 4.4)}

This test case is given in Ref. 10 by Shankar.  The data corresponds to the results
given in Table 1 of Ref. 10.  This is a hard-wall duct ($\eta = 0$), with an axial
Mach number profile given by:

\begin{eqnarray}
b &=& r_T - r_H
\nonumber
\\
\widetilde{r} &=& \frac{r}{b}
\nonumber
\\
M \left(\widetilde{r} \right)
&=&
M_0 \left(1 - \widetilde{r} \right)^{\frac{1}{7}}
\nonumber
\\
M_0 &=& 0.3
\nonumber
\\
m &=& 0
\nonumber
\\
kb &=& 20
\nonumber
\end{eqnarray}

Next, we translate Shankar's notation into the SWIRL notation.  Comparing
Shankar and SWIRL,

\begin{eqnarray}
p_n \left(r, x, t \right)
&=&
a_n \phi_n \left( r \right)
e^{i k \left(\beta_n x - c t \right)}
\nonumber
\\
\widetilde{p}_n \left(r, x, \theta, t \right)
&=&
p_n \left(r \right) e^{i \left(k_x x + m \theta - \omega t \right)}
\nonumber
\end{eqnarray}

Translating this into SWIRL inputs:

\begin{eqnarray}
\left(kb \right)_{Shankar} &=& \frac{\omega}{A_T} r_T
\nonumber
\\
&=& k_{SWIRL}
\nonumber
\end{eqnarray}

The input deck for this test case is:

\begin{tiny}
\VerbatimInput{input.data.Table4.4}
\end{tiny}%

For the cylinder case of Shankar, the results in Table 4.4 are for
$\frac{\overline{\gamma}_n}{k}$ in order to compare with the $\beta_n$
results:

\begin{table}
 \centering
 \begin{tabular}{ |l | c | c | c | c | c |}
 \hline
 $\gamma^{\pm}_n$ & Shankar $\beta_n$  & $\frac{\overline{\gamma}_n}{k}$ (16 points) & index & $\frac{\overline{\gamma}_n}{k}$ (32 points) & index  \\
 \hline
 $\gamma_0^{-}$    & $  0.81500 $            & $  0.816009 $             & 45 & $ 0.815493             $ & 69 \\
 $\gamma_1^{-}$    & $  0.76944 $            & $  0.769595 $             & 47 & $ 0.769520             $ & 71 \\
 $\gamma_2^{-}$    & $  0.72751 $            & $  0.727753 $             & 46 & $ 0.727642             $ & 70 \\
 $\gamma_3^{-}$    & $  0.65329 $            & $  0.653542 $             & 44 & $ 0.653472             $ & 68 \\
 $\gamma_4^{-}$    & $  0.54028 $            & $  0.540499 $             & 43 & $ 0.540524             $ & 67 \\
 $\gamma_5^{-}$    & $  0.36933 $            & $  0.369465 $             & 42 & $ 0.369746             $ & 66 \\
 $\gamma_6^{-}$    & $  0.06361 $            & $  0.064613 $             & 41 & $ 0.064754             $ & 65 \\
 $\gamma_7^{-}$    & $ -0.28313 + 0.48807 i$ & $ -0.278964 + 0.487173 i$ & 39 & $-0.282743 + 0.486526 i$ & 63 \\
 $\gamma_8^{-}$    & $ -0.28357 + 0.80635 i$ & $ -0.289882 + 0.854887 i$ & 29 & $-0.283251 + 0.804397 i$ & 62 \\
 $\gamma_9^{-}$    & $ -0.28410 + 1.05658 i$ & $ -0.263260 + 1.090093 i$ & 27 & $-0.283670 + 1.053847 i$ & 57 \\ 
 $\gamma_{10}^{-}$ & $ -0.28410 + 1.27947 i$ & $ -0.315161 + 1.655351 i$ & 25 & $-0.284027 + 1.275635 i$ & 50 \\ \hline
 \end{tabular}
 \caption{Table 4.4 data}
 \label{Table44}
\end{table}

The output in 'gam.acc' for this test case (using 16 points) is:

\begin{tiny}
\VerbatimInput{gam.acc.Table4.4.16pt}
\end{tiny}%

The output in 'gam.acc' for this test case (using 32 points) is:

\begin{tiny}
\VerbatimInput{gam.acc.Table4.4.32pt}
\end{tiny}%

\subsection{Annulus, Shear Flow (Table 4.5)}

This test case is given in Ref. 10 by Shankar.   In order to
get the correct input data,

\begin{eqnarray}
\widetilde{r}_i  
&=& 6.0
\nonumber
\\
\frac{r_i}{b}
&=& 6.0
\nonumber
\\
\frac{r_i}{r_o - r_i}
&=& 6.0
\nonumber
\\
r_i
&=& 6 r_o - 6 r_i
\nonumber
\\
7 r_i
&=& 6 r_o 
\nonumber
\\
\frac{r_i}{r_o}
&=& \frac{6}{7} 
\nonumber
\\
\sigma &=& \frac{6}{7}
\nonumber
\\
b &=& \frac{1}{7}
\nonumber
\\
k b &=& 10
\nonumber
\\
k &=& 70
\nonumber
\end{eqnarray}

The axial Mach number profile is given by:

\begin{eqnarray}
M \left(\frac{r}{b} \right)
&=&
M_0 \left(1 - 2 \left| \frac{r_i - r}{b} + 0.5 \right| \right)^{\frac{1}{7}}
\nonumber
\\
M_0 &=& 0.3
\nonumber
\end{eqnarray}

The input data for SWIRL is then:

\begin{tiny}
\VerbatimInput{input.data.Table4.5}
\end{tiny}%

The output in 'gam.acc' for this test case is:

\begin{tiny}
\VerbatimInput{gam.acc.Table4.5}
\end{tiny}%

The comparison with the original data is given in the table.  Note that, in order to
make sure of a valid comparison, 64 radial grid points were run.

\begin{table}
 \centering
 \begin{tabular}{ |l | c | c | c | c |}
 \hline
 $\gamma^{\pm}_n$ & Shankar $\beta_n$  & Kousen report & $\frac{\overline{\gamma}_n}{k}$ (current) & index  \\
 \hline
 $\gamma_0^{-}$    & $  0.79293 $            & $  0.79353 $            & $ 0.793478             $ & 134 \\
 $\gamma_1^{-}$    & $  0.75075 $            & $  0.75292 $            & $ 0.752847             $ & 132 \\
 $\gamma_2^{-}$    & $  0.57143 $            & $  0.57320 $            & $ 0.573124             $ & 131 \\
 $\gamma_3^{-}$    & $ -0.00969 $            & $  0.16437 $            & $ 0.164294             $ & 130 \\
 $\gamma_4^{-}$    & $ -0.28733 + 0.73219 i$ & $ -0.28357 + 0.73425 i$ & $-0.283637 + 0.734325 i$ & 127 \\
 $\gamma_5^{-}$    & $ -0.29118 + 1.21721 i$ & $ -0.28622 + 1.2198  i$ & $-0.286225 + 1.219841 i$ & 125 \\
 $\gamma_6^{-}$    & $ -0.29248 + 1.62569 i$ & $ -0.28766 + 1.6281  i$ & $-0.287553 + 1.628282 i$ & 122 \\
 $\gamma_7^{-}$    & $ -0.29519 + 2.00221 i$ & $ -0.28947 + 2.0055  i$ & $-0.289320 + 2.005620 i$ & 120 \\
 $\gamma_8^{-}$    & $ -0.29567 + 2.36511 i$ & $ -0.29035 + 2.3683  i$ & $-0.289988 + 2.368514 i$ & 114 \\
 $\gamma_9^{-}$    & $ -0.29768 + 2.71665 i$ & $ -0.29167 + 2.7209  i$ & $-0.291333 + 2.720846 i$ & 112 \\ 
 $\gamma_{10}^{-}$ & $ -0.29776 + 3.06414 i$ & $ -0.29243 + 3.0679  i$ & $-0.291720 + 3.068215 i$ & 110 \\ \hline
 \end{tabular}
 \caption{Table 4.5 data}
 \label{Table45}
\end{table}

In the data, note the propagating mode at index 133.  This needs to be investigated further.

\subsection{Lined Annulus, Shear Flow (Table 4.6)}

This test case is given in Ref. 10 by Shankar.   In order to
get the correct input data,

\begin{eqnarray}
\widetilde{r}_i  
&=& 2.0
\nonumber
\\
\frac{r_i}{b}
&=& 2.0
\nonumber
\\
\frac{r_i}{r_o - r_i}
&=& 2.0
\nonumber
\\
r_i
&=& 2 r_o - 2 r_i
\nonumber
\\
3 r_i
&=& 2 r_o 
\nonumber
\\
\frac{r_i}{r_o}
&=& \frac{2}{3} 
\nonumber
\\
\sigma &=& \frac{2}{3}
\nonumber
\\
b &=& \frac{1}{3}
\nonumber
\\
k b &=& 10
\nonumber
\\
k &=& 30
\nonumber
\end{eqnarray}

The axial Mach number profile is again given by:

\begin{eqnarray}
M \left(\frac{r}{b} \right)
&=&
M_0 \left(1 - 2 \left| \frac{r_i - r}{b} + 0.5 \right| \right)^{\frac{1}{7}}
\nonumber
\\
M_0 &=& 0.3
\nonumber
\end{eqnarray}

The admittance is given as:

\begin{eqnarray}
\eta &=& 0.3 + 0.1 i
\nonumber
\end{eqnarray}


It is not clear from Shankar whether both walls are lined or only one.
Running the code for each case, it appears that the outer wall is lined
and the inner wall is hard.

The input data for SWIRL is then:

\begin{tiny}
\VerbatimInput{input.data.Table4.6}
\end{tiny}%

The output in 'gam.acc' for this test case is:

\begin{tiny}
\VerbatimInput{gam.acc.Table4.6}
\end{tiny}%

The comparison with the original data is given in the table.  Note that, in order to
make sure of a valid comparison, 64 radial grid points were run (though the 32 point
dataset looked good too).

\begin{table}
 \centering
 \begin{tabular}{ |l | c | c | c | c |}
 \hline
 $\gamma^{\pm}_n$ & Shankar $\beta_n$  & Kousen report & $\frac{\overline{\gamma}_n}{k}$ (current) & index  \\
 \hline
 $\gamma_0^{-}$    & $  0.78698 + 0.00400 i$ & $  0.78093 + 0.00913 i$ & $ 0.787228 + 0.003757 i$ & 136 \\
 $\gamma_1^{-}$    & $  0.73438 + 0.02541 i$ & $  0.75079 + 0.03387 i$ & $ 0.735936 + 0.026332 i$ & 135 \\
 $\gamma_2^{-}$    & $  0.55840 + 0.03148 i$ & $  0.57267 + 0.03246 i$ & $ 0.560121 + 0.031371 i$ & 134 \\
 $\gamma_3^{-}$    & $  0.14308 + 0.07638 i$ & $  0.16875 + 0.06982 i$ & $ 0.143097 + 0.072855 i$ & 131 \\
 $\gamma_4^{-}$    & $ -0.23900 + 0.74173 i$ & $ -0.23734 + 0.72727 i$ & $-0.235625 + 0.743714 i$ & 130 \\
 $\gamma_5^{-}$    & $ -0.26149 + 1.21973 i$ & $ -0.25993 + 1.2120  i$ & $-0.257309 + 1.221354 i$ & 129 \\
 $\gamma_6^{-}$    & $ -0.26996 + 1.62627 i$ & $ -0.26860 + 1.6207  i$ & $-0.265988 + 1.627490 i$ & 124 \\
 $\gamma_7^{-}$    & $ -0.27669 + 2.00192 i$ & $ -0.27468 + 1.9983  i$ & $-0.272158 + 2.003568 i$ & 118 \\
 $\gamma_8^{-}$    & $ -0.27974 + 2.36439 i$ & $ -0.27813 + 2.3612  i$ & $-0.275516 + 2.365756 i$ & 116 \\
 $\gamma_9^{-}$    & $ -0.28359 + 2.71568 i$ & $ -0.28147 + 2.7139  i$ & $-0.278985 + 2.717521 i$ & 114 \\ 
 $\gamma_{10}^{-}$ & $ -0.28502 + 3.06304 i$ & $ -0.28361 + 3.0610  i$ & $-0.280811 + 3.064511 i$ & 112 \\ \hline
 \end{tabular}
 \caption{Table 4.6 data}
 \label{Table46}
\end{table}

In the data, note the (very nearly) propagating mode at index 137.  This needs to be investigated further -- are
these modes real?



% \documentclass[12pt]{article}
\begin{document}
\section{SWIRL code revisited}

Going back to the original perturbation equations:

\begin{eqnarray}
\left(
-i \omega
+ \frac{i m V_{\theta}}{r}
+i k_x V_x 
\right) v_r
-\frac{2 V_{\theta}}{r} v_{\theta}
&=&
-\frac{1}{\overline{\rho}} \frac{\partial p}{\partial r}
+\frac{V_{\theta}^2}{\overline{\rho} r A^2} p
\nonumber
\\
\left(
-i \omega
+ \frac{i m V_{\theta}}{r}
+i k_x V_x 
\right) v_{\theta}
+
\left(
\frac{V_{\theta}}{r}
+\frac{\partial V_{\theta}}{\partial r}
\right) v_r
&=&
-\frac{i m}{\overline{\rho} r} p
\nonumber
\\
\left(
-i \omega
+ \frac{i m V_{\theta}}{r}
+i k_x V_x 
\right) v_x
+\frac{\partial V_x}{\partial r} v_r
&=&
-\frac{i k_x}{\overline{\rho}} p
\nonumber
\\
\frac{1}{ \overline{\rho} A^2}
\left(
-i \omega
+ \frac{i m V_{\theta}}{r} 
+ i k_x V_x 
\right) p
+\frac{V_{\theta}^2}{A^2 r}
v_r
+ 
\frac{\partial v_r}{\partial r}
+ 
\frac{v_r}{r} 
+\frac{i m }{r} v_{\theta} 
+ i k_x v_x
&=&
0
\nonumber
\end{eqnarray}

and defining a nondimensionalization based on the tip quantities:

\begin{eqnarray}
\omega &=& \widehat{\omega} \frac{A_T}{r_T}
\nonumber
\\
k_x &=& \frac{\widehat{k}_x}{ r_T}
\nonumber
\\
\overline{\rho} &=& 
\left(\widehat{\overline{\rho}} \right) \overline{\rho}_T
\nonumber
\\
A &=& \widetilde{A} A_T
\nonumber
\\
V_{x} &=& \widehat{V}_{x} A_T
\nonumber
\\
V_{\theta} &=& \widehat{V}_{\theta} A_T
\nonumber
\\
v_{x} &=& \widehat{v}_{x} A_T
\nonumber
\\
v_{r} &=& \widehat{v}_{r} A_T
\nonumber
\\
v_{\theta} &=& \widehat{v}_{\theta} A_T
\nonumber
\\
p &=& \widehat{p} \overline{\rho}_T A_T
\nonumber
\\
r &=& \widetilde{r} r_T
\nonumber
\\
\frac{\partial}{\partial r} &=& \frac{1}{r_T} \frac{\partial}{\partial \widetilde{r}}
\nonumber
\end{eqnarray}

Substituting in,

\begin{small}
\begin{eqnarray}
\left(
-i \widehat{\omega} 
+ \frac{i m \widehat{V}_{\theta}}{\widetilde{r}}
+i \widehat{k}_x \widehat{V}_x 
\right) 
\widehat{v}_r 
\frac{A_T^2}{r_T}
-\frac{2 \widehat{V}_{\theta}}{\widetilde{r}} \widehat{v}_{\theta}
\frac{A_T^2}{r_T}
 = 
-\frac{1}{\widehat{\overline{\rho}}} \frac{\partial \widehat{p}}{\partial \widetilde{r}}
\frac{A_T^2}{r_T}
+\frac{\widehat{V}_{\theta}^2}{\widehat{\overline{\rho}} \widetilde{r} \widetilde{A}^2} 
\widehat{p}
\frac{A_T^2}{r_T}
\nonumber
\\
\left(
-i \widehat{\omega} 
+ \frac{i m \widehat{V}_{\theta}}{\widetilde{r}}
+i \widehat{k}_x \widehat{V}_x 
\right) 
\widehat{v}_{\theta} 
\frac{A_T^2}{r_T}
+
\left(
\frac{\widehat{V}_{\theta}}{\widetilde{r}}
+\frac{\partial \widehat{V}_{\theta}}{\partial \widetilde{r}}
\right) \widehat{v}_r
\frac{A_T^2}{r_T}
 = 
-\frac{i m}{\widehat{\overline{\rho}} \widetilde{r}} \widehat{p}
\frac{A_T^2}{r_T}
\nonumber
\\
\left(
-i \widehat{\omega}
+ \frac{i m \widehat{V}_{\theta}}{\widetilde{r}}
+i \widehat{k}_x \widehat{V}_x 
\right) 
\widehat{v}_{x} 
\frac{A_T^2}{r_T}
+\frac{\partial \widehat{V}_x}{\partial \widetilde{r}} \widehat{v}_r
\frac{A_T^2}{r_T}
 = 
-\frac{i \widehat{k}_x}{\widehat{\overline{\rho}}} \widehat{p}
\frac{A_T^2}{r_T}
\nonumber
\\
\frac{1}{ \widehat{\overline{\rho}} \widetilde{A}^2}
\left(
-i \widehat{\omega}
+ \frac{i m \widehat{V}_{\theta}}{\widetilde{r}}
+i \widehat{k}_x \widehat{V}_x 
\right) 
\widehat{p} 
\frac{A_T}{r_T}
+\frac{\widehat{V}_{\theta}^2}{\widetilde{A}^2 \widetilde{r}}
\widehat{v}_r
\frac{A_T}{r_T}
+ 
\frac{\partial \widehat{v}_r}{\partial \widetilde{r}}
\frac{A_T}{r_T}
+ 
\frac{\widehat{v}_r}{\widetilde{r}} 
\frac{A_T}{r_T}
+\frac{i m }{\widetilde{r}} \widehat{v}_{\theta} 
\frac{A_T}{r_T}
+ i \widehat{k}_x \widehat{v}_x
\frac{A_T}{r_T}
 = 
0
\nonumber
\end{eqnarray}
\end{small}%

Simplifying the equation and rearranging gives:

\begin{small}
\begin{eqnarray}
\left(
-i \widehat{\omega} 
+ \frac{i m \widehat{V}_{\theta}}{\widetilde{r}}
+i \widehat{k}_x \widehat{V}_x 
\right) 
\widehat{v}_r 
-\frac{2 \widehat{V}_{\theta}}{\widetilde{r}} \widehat{v}_{\theta}
 = 
-\frac{1}{\widehat{\overline{\rho}}} \frac{\partial \widehat{p}}{\partial \widetilde{r}}
+\frac{\widehat{V}_{\theta}^2}{\widehat{\overline{\rho}} \widetilde{r} \widetilde{A}^2} 
\widehat{p}
\nonumber
\\
\left(
-i \widehat{\omega} 
+ \frac{i m \widehat{V}_{\theta}}{\widetilde{r}}
+i \widehat{k}_x \widehat{V}_x 
\right) 
\widehat{v}_{\theta} 
+
\left(
\frac{\widehat{V}_{\theta}}{\widetilde{r}}
+\frac{\partial \widehat{V}_{\theta}}{\partial \widetilde{r}}
\right) \widehat{v}_r
 = 
-\frac{i m}{\widehat{\overline{\rho}} \widetilde{r}} \widehat{p}
\nonumber
\\
\left(
-i \widehat{\omega}
+ \frac{i m \widehat{V}_{\theta}}{\widetilde{r}}
+i \widehat{k}_x \widehat{V}_x 
\right) 
\widehat{v}_{x} 
+\frac{\partial \widehat{V}_x}{\partial \widetilde{r}} \widehat{v}_r
 = 
-\frac{i \widehat{k}_x}{\widehat{\overline{\rho}}} \widehat{p}
\nonumber
\\
\left(
-i \widehat{\omega}
+ \frac{i m \widehat{V}_{\theta}}{\widetilde{r}}
+i \widehat{k}_x \widehat{V}_x 
\right) 
\widehat{p} 
+
\widehat{\overline{\rho}} \widetilde{A}^2
\left(
\frac{\widehat{V}_{\theta}^2}{\widetilde{A}^2 \widetilde{r}}
\widehat{v}_r
+ 
\frac{\partial \widehat{v}_r}{\partial \widetilde{r}}
+ 
\frac{\widehat{v}_r}{\widetilde{r}} 
+\frac{i m }{\widetilde{r}} \widehat{v}_{\theta} 
+ i \widehat{k}_x \widehat{v}_x
\right)
 = 
0
\nonumber
\end{eqnarray}
\end{small}%

\subsection{Some basic definitions}

In cylindrical coordinates,

\begin{eqnarray}
\vec{\nabla} \cdot \vec{V}
&=&
\frac{\partial v_r}{\partial r}
+ \frac{v_r}{r}
+ \frac{1}{r}
\frac{\partial v_{\theta}}{\partial \theta}
+\frac{\partial v_x}{\partial x}
\nonumber
\\
&=&
\frac{\partial V_r}{\partial r}
+ \frac{V_r}{r}
+\frac{\partial v'_r}{\partial r}
+ \frac{v'_r}{r}
+ \frac{1}{r}
\frac{\partial v'_{\theta}}{\partial \theta}
+\frac{\partial v'_x}{\partial x}
\nonumber
\\
&=&
\frac{\partial v'_r}{\partial r}
+ \frac{v'_r}{r}
+ \frac{1}{r}
\frac{\partial v'_{\theta}}{\partial \theta}
+\frac{\partial v'_x}{\partial x}
\nonumber
\\
&=&
\frac{\partial v_r}{\partial r}
+ \frac{v_r}{r}
+ \frac{i m}{r} v_{\theta}
+i k_x v_x
\nonumber
\end{eqnarray}

\begin{eqnarray}
\vec{\nabla} \times \vec{V}
&=&
\left(
\begin{array}{r}
\left(
\frac{1}{r}
\frac{\partial V_x}{\partial \theta}
-
\frac{\partial V_{\theta}}{\partial x}
\right) \vec{e}_r
\\
+\left(
\frac{\partial V_r}{\partial x}
-
\frac{\partial V_x}{\partial r}
\right) \vec{e}_{\theta}
\\
+\left(
\frac{V_{\theta}}{r}
+\frac{\partial V_{\theta}}{\partial r}
-
\frac{1}{r}
\frac{\partial V_{r}}{\partial \theta}
\right) \vec{e}_x
\end{array}
\right)
\nonumber
\\
&=&
\left(
\begin{array}{r}
\left(
\frac{i m}{r}
v_x
-
i k_x v_{\theta}
\right) \vec{e}_r
\\
+\left(
i k_x v_r
-
\frac{\partial v_x}{\partial r}
\right) \vec{e}_{\theta}
\\
+\left(
\frac{v_{\theta}}{r}
+\frac{\partial v_{\theta}}{\partial r}
-
\frac{i m}{r}
v_{r}
\right) \vec{e}_x
\end{array}
\right)
\nonumber
\end{eqnarray}

\begin{eqnarray}
\vec{\nabla} \phi
&=&
\left(
\begin{array}{r}
\frac{\partial \phi}{\partial r} \vec{e}_r
\\
+\frac{1}{r} \frac{\partial \phi}{\partial \theta} \vec{e}_{\theta}
\\
+\frac{\partial \phi}{\partial z} \vec{e}_z
\end{array}
\right)
\nonumber
\end{eqnarray}

Let's define:

\begin{eqnarray}
\phi &=& \kappa \left(r \right) e^{i \left(k_x x + m \theta - \omega t \right)}
\nonumber
\\
\vec{v}_{\phi} &=& \vec{\nabla} \phi
\nonumber
\\
&=& 
\left(
\begin{array}{r}
\frac{\partial \kappa}{\partial r} e^{i \left(k_x x + m \theta - \omega t \right)} \vec{e}_r
\\
+\frac{i m}{r} \kappa e^{i \left(k_x x + m \theta - \omega t \right)} \vec{e}_{\theta}
\\
+i k_x \kappa e^{i \left(k_x x + m \theta - \omega t \right)} \vec{e}_x
\end{array}
\right)
\nonumber
\\
&=& 
\left(
\begin{array}{r}
\frac{\partial \kappa}{\partial r} e^{i \left(k_x x + m \theta - \omega t \right)} \vec{e}_r
\\
+\frac{i m \kappa}{r} e^{i \left(k_x x + m \theta - \omega t \right)} \vec{e}_{\theta}
\\
+i k_x \kappa e^{i \left(k_x x + m \theta - \omega t \right)} \vec{e}_x
\end{array}
\right)
\nonumber
\end{eqnarray}

and

\begin{eqnarray}
\vec{\psi} &=& 
\left(
\begin{array}{r}
\alpha \left(r \right) e^{i \left(k_x x + m \theta - \omega t \right)} \vec{e}_r
\\
+
\widehat{\beta} \left(r \right) e^{i \left(k_x x + m \theta - \omega t \right)} \vec{e}_{\theta}
\\
+
\tau \left(r \right) e^{i \left(k_x x + m \theta - \omega t \right)} \vec{e}_x
\end{array}
\right)
\nonumber
\end{eqnarray}

\begin{eqnarray}
\vec{v}_{\psi}
&=&
\vec{\nabla} \times \vec{\psi}
\nonumber
\\
&=&
\left(
\begin{array}{r}
\left(
\frac{i m}{r}
\tau
-
i k_x \widehat{\beta}
\right) 
e^{i \left(k_x x + m \theta - \omega t \right)}
\vec{e}_r
\\
+\left(
i k_x \alpha
-
\frac{\partial \tau}{\partial r}
\right) 
e^{i \left(k_x x + m \theta - \omega t \right)}
\vec{e}_{\theta}
\\
+\left(
\frac{\widehat{\beta}}{r}
+\frac{\partial \widehat{\beta}}{\partial r}
-
\frac{i m}{r}
\alpha
\right) 
e^{i \left(k_x x + m \theta - \omega t \right)} 
\vec{e}_x
\end{array}
\right)
\nonumber
\end{eqnarray}

Defining the perturbation velocities as:

\begin{eqnarray}
\vec{v} &=& 
\vec{v}_{\phi}
+\vec{v}_{\psi}
\nonumber
\\
&=&
\left(
\begin{array}{r}
\left(
\frac{\partial \kappa}{\partial r}
+
\frac{i m}{r}
\tau
-
i k_x \widehat{\beta}
\right) 
e^{i \left(k_x x + m \theta - \omega t \right)}
\vec{e}_r
\\
+
\left(
\frac{i m}{r} \kappa
+
i k_x \alpha
-
\frac{\partial \tau}{\partial r}
\right) 
e^{i \left(k_x x + m \theta - \omega t \right)}
\vec{e}_{\theta}
\\
+
\left(
\frac{\widehat{\beta}}{r}
+\frac{\partial \widehat{\beta}}{\partial r}
-
\frac{i m}{r}
\alpha
+ i k_x \kappa
\right) 
e^{i \left(k_x x + m \theta - \omega t \right)} 
\vec{e}_x
\end{array}
\right)
\nonumber
\end{eqnarray}

The perturbation pressure is defined as:

\begin{eqnarray}
\widehat{p} &=& p \left(r \right) 
e^{i \left(k_x x + m \theta - \omega t \right)} 
\nonumber
\end{eqnarray}

\subsection{Trying it out}

Nondimensionalizing and substituting into the perturbation equations gives:

\begin{small}
\begin{eqnarray}
\left(
\begin{array}{r}
\left(
-i \widehat{\omega} 
+ \frac{i m \widehat{V}_{\theta}}{\widetilde{r}}
+i \widehat{k}_x \widehat{V}_x 
\right) 
\left(
\frac{\partial \widehat{\kappa}}{\partial \widetilde{r}}
+
\frac{i m}{\widetilde{r}}
\widehat{\tau}
-
i \widehat{k}_x \widehat{\beta}
\right) 
\\
-\frac{2 \widehat{V}_{\theta}}{\widetilde{r}} 
\left(
\frac{i m}{\widetilde{r}} \widehat{\kappa}
+
i \widehat{k}_x \widehat{\alpha}
-
\frac{\partial \widehat{\tau}}{\partial \widetilde{r}}
\right) 
\end{array}
\right)
 = 
-\frac{1}{\widehat{\overline{\rho}}} \frac{\partial \widehat{p}}{\partial \widetilde{r}}
+\frac{\widehat{V}_{\theta}^2}{\widehat{\overline{\rho}} \widetilde{r} \widetilde{A}^2} 
\widehat{p}
\nonumber
\\
\left(
\begin{array}{r}
\left(
-i \widehat{\omega} 
+ \frac{i m \widehat{V}_{\theta}}{\widetilde{r}}
+i \widehat{k}_x \widehat{V}_x 
\right) 
\left(
\frac{i m}{\widetilde{r}} \widehat{\kappa}
+
i \widehat{k}_x \widehat{\alpha}
-
\frac{\partial \widehat{\tau}}{\partial \widetilde{r}}
\right) 
\\
+
\left(
\frac{\widehat{V}_{\theta}}{\widetilde{r}}
+\frac{\partial \widehat{V}_{\theta}}{\partial \widetilde{r}}
\right) 
\left(
\frac{\partial \widehat{\kappa}}{\partial \widetilde{r}}
+
\frac{i m}{\widetilde{r}}
\widehat{\tau}
-
i \widehat{k}_x \widehat{\beta}
\right) 
\end{array}
\right)
 = 
-\frac{i m}{\widehat{\overline{\rho}} \widetilde{r}} \widehat{p}
\nonumber
\\
\left(
\begin{array}{r}
\left(
-i \widehat{\omega}
+ \frac{i m \widehat{V}_{\theta}}{\widetilde{r}}
+i \widehat{k}_x \widehat{V}_x 
\right) 
\left(
\frac{\widehat{\beta}}{\widetilde{r}}
+\frac{\partial \widehat{\beta}}{\partial \widetilde{r}}
-
\frac{i m}{\widetilde{r}}
\widehat{\alpha}
+ i \widehat{k}_x \widehat{\kappa}
\right) 
\\
+\frac{\partial \widehat{V}_x}{\partial \widetilde{r}} 
\left(
\frac{\partial \widehat{\kappa}}{\partial \widetilde{r}}
+
\frac{i m}{\widetilde{r}}
\widehat{\tau}
-
i \widehat{k}_x \widehat{\beta}
\right) 
\end{array}
\right)
 = 
-\frac{i \widehat{k}_x}{\widehat{\overline{\rho}}} \widehat{p}
\nonumber
\\
\left(
-i \widehat{\omega}
+ \frac{i m \widehat{V}_{\theta}}{\widetilde{r}}
+i \widehat{k}_x \widehat{V}_x 
\right) 
\widehat{p} 
+
\widehat{\overline{\rho}} \widetilde{A}^2
\left(
\begin{array}{r}
\frac{\widehat{V}_{\theta}^2}{\widetilde{A}^2 \widetilde{r}}
\left(
\frac{\partial \widehat{\kappa}}{\partial \widetilde{r}}
+
\frac{i m}{\widetilde{r}}
\widehat{\tau}
-
i \widehat{k}_x \widehat{\beta}
\right) 
\\
+ 
\frac{\partial}{\partial \widetilde{r}}
\left(
\frac{\partial \widehat{\kappa}}{\partial \widetilde{r}}
+
\frac{i m}{\widetilde{r}}
\widehat{\tau}
-
i \widehat{k}_x \widehat{\beta}
\right) 
\\
+ 
\frac{1}{\widetilde{r}} 
\left(
\frac{\partial \widehat{\kappa}}{\partial \widetilde{r}}
+
\frac{i m}{\widetilde{r}}
\widehat{\tau}
-
i \widehat{k}_x \widehat{\beta}
\right) 
\\
+\frac{i m }{\widetilde{r}} 
\left(
\frac{i m}{\widetilde{r}} \widehat{\kappa}
+
i \widehat{k}_x \widehat{\alpha}
-
\frac{\partial \widehat{\tau}}{\partial \widetilde{r}}
\right) 
\\
+ i \widehat{k}_x 
\left(
\frac{\widehat{\beta}}{\widetilde{r}}
+\frac{\partial \widehat{\beta}}{\partial \widetilde{r}}
-
\frac{i m}{\widetilde{r}}
\widehat{\alpha}
+ i \widehat{k}_x \widehat{\kappa}
\right) 
\end{array}
\right)
 = 
0
\nonumber
\end{eqnarray}
\end{small}%

\subsubsection{Pressure equation}

Working on the pressure equation gives:

\begin{eqnarray}
\left(
-i \widehat{\omega}
+ \frac{i m \widehat{V}_{\theta}}{\widetilde{r}}
+i \widehat{k}_x \widehat{V}_x 
\right) 
\widehat{p} 
+
\widehat{\overline{\rho}} \widetilde{A}^2
\left(
\begin{array}{r}
\frac{\widehat{V}_{\theta}^2}{\widetilde{A}^2 \widetilde{r}}
\left(
\frac{\partial \widehat{\kappa}}{\partial \widetilde{r}}
+
\frac{i m}{\widetilde{r}}
\widehat{\tau}
-
i \widehat{k}_x \widehat{\beta}
\right) 
\\
+ 
\left(
\frac{\partial^2 \widehat{\kappa}}{\partial \widetilde{r}^2}
-
\frac{i m}{\widetilde{r}^2}
\widehat{\tau}
+
\frac{i m}{\widetilde{r}}
\frac{\partial \widehat{\tau}}{\partial \widetilde{r}}
-
i \widehat{k}_x 
\frac{\partial \widehat{\beta}}{\partial \widetilde{r}}
\right) 
\\
+ 
\left(
\frac{1}{\widetilde{r}} 
\frac{\partial \widehat{\kappa}}{\partial \widetilde{r}}
+
\frac{i m}{\widetilde{r}^2}
\widehat{\tau}
-
\frac{i \widehat{k}_x}{\widetilde{r}} 
\widehat{\beta}
\right) 
\\
+
\left(
\left(\frac{i m}{\widetilde{r}} \right)^2 \widehat{\kappa}
+
\frac{i m }{\widetilde{r}} 
i \widehat{k}_x \widehat{\alpha}
-
\frac{i m }{\widetilde{r}} 
\frac{\partial \widehat{\tau}}{\partial \widetilde{r}}
\right) 
\\
+ 
\left(
i \widehat{k}_x 
\frac{\widehat{\beta}}{\widetilde{r}}
+
i \widehat{k}_x 
\frac{\partial \widehat{\beta}}{\partial \widetilde{r}}
-
\frac{i m}{\widetilde{r}}
i \widehat{k}_x 
\widehat{\alpha}
+ 
\left(i \widehat{k}_x \right)^2 \widehat{\kappa}
\right) 
\end{array}
\right)
 = 
0
\nonumber
\\
\left(
-i \widehat{\omega}
+ \frac{i m \widehat{V}_{\theta}}{\widetilde{r}}
+i \widehat{k}_x \widehat{V}_x 
\right) 
\widehat{p} 
+
\widehat{\overline{\rho}} \widetilde{A}^2
\left(
\begin{array}{r}
\frac{\widehat{V}_{\theta}^2}{\widetilde{A}^2 \widetilde{r}}
\left(
\frac{\partial \widehat{\kappa}}{\partial \widetilde{r}}
+
\frac{i m}{\widetilde{r}}
\widehat{\tau}
-
i \widehat{k}_x \widehat{\beta}
\right) 
\\
+ 
\left(
\frac{\partial^2 \widehat{\kappa}}{\partial \widetilde{r}^2}
+ 
\frac{1}{\widetilde{r}} 
\frac{\partial \widehat{\kappa}}{\partial \widetilde{r}}
+
\left(\frac{i m}{\widetilde{r}} \right)^2 \widehat{\kappa}
+ 
\left(i \widehat{k}_x \right)^2 \widehat{\kappa}
\right)
\end{array}
\right)
 = 
0
\nonumber
\\
\left(
-i \widehat{\omega}
+ \frac{i m \widehat{V}_{\theta}}{\widetilde{r}}
+i \widehat{k}_x \widehat{V}_x 
\right) 
\widehat{p} 
+
\left(
\begin{array}{r}
\frac{\overline{\rho} \widehat{V}_{\theta}^2}{\widetilde{r}}
\left(
\frac{\partial \widehat{\kappa}}{\partial \widetilde{r}}
+
\frac{i m}{\widetilde{r}}
\widehat{\tau}
-
i \widehat{k}_x \widehat{\beta}
\right) 
\\
+ 
\widehat{\overline{\rho}} \widetilde{A}^2
\left(
\frac{\partial^2 \widehat{\kappa}}{\partial \widetilde{r}^2}
+ 
\frac{1}{\widetilde{r}} 
\frac{\partial \widehat{\kappa}}{\partial \widetilde{r}}
+
\left(\frac{i m}{\widetilde{r}} \right)^2 \widehat{\kappa}
+ 
\left(i \widehat{k}_x \right)^2 \widehat{\kappa}
\right)
\end{array}
\right)
 = 
0
\nonumber
\\
\left(
-i \widehat{\omega}
+ \frac{i m \widehat{V}_{\theta}}{\widetilde{r}}
+i \widehat{k}_x \widehat{V}_x 
\right) 
\widehat{p} 
+
\left(
\begin{array}{r}
\frac{\overline{\rho} \widehat{V}_{\theta}^2}{\widetilde{r}}
\widehat{v}_r
\\
+ 
\widehat{\overline{\rho}} \widetilde{A}^2
\left(
\frac{\partial^2 \widehat{\kappa}}{\partial \widetilde{r}^2}
+ 
\frac{1}{\widetilde{r}} 
\frac{\partial \widehat{\kappa}}{\partial \widetilde{r}}
+
\left(\frac{i m}{\widetilde{r}} \right)^2 \widehat{\kappa}
+ 
\left(i \widehat{k}_x \right)^2 \widehat{\kappa}
\right)
\end{array}
\right)
 = 
0
\nonumber
\end{eqnarray}

\subsubsection{Vorticity equations: Radial vorticity}

The radial component of vorticity is:

\begin{eqnarray}
\omega_r &=&
\frac{im}{r} 
v_x 
- 
i k_x 
v_{\theta}
\nonumber
\\
&=&
\left(
\frac{im}{r} 
\left(
\frac{\widehat{\beta}}{\widetilde{r}}
+\frac{\partial \widehat{\beta}}{\partial \widetilde{r}}
-
\frac{i m}{\widetilde{r}}
\widehat{\alpha}
+ i \widehat{k}_x \widehat{\kappa}
\right) 
-
i k_x 
\left(
\frac{i m}{\widetilde{r}} \widehat{\kappa}
+
i \widehat{k}_x \widehat{\alpha}
-
\frac{\partial \widehat{\tau}}{\partial \widetilde{r}}
\right) 
\right)
e^{i \left(k_x x + m \theta - \omega t \right)} 
\nonumber
\\
&=&
\left(
\left(\widehat{k}_x^2 + \frac{m^2}{\widetilde{r}^2} \right) \widehat{\alpha}
+
\frac{im}{r} 
\left(
\frac{\widehat{\beta}}{\widetilde{r}}
+\frac{\partial \widehat{\beta}}{\partial \widetilde{r}}
\right)
+i \widehat{k}_x 
\frac{\partial \widehat{\tau}}{\partial \widetilde{r}}
\right)
e^{i \left(k_x x + m \theta - \omega t \right)} 
\nonumber
\end{eqnarray}

The radial vorticity amplitude is defined as:

\begin{eqnarray}
\widehat{\omega}_r
&=&
\left(\widehat{k}_x^2 + \frac{m^2}{\widetilde{r}^2} \right) \widehat{\alpha}
+
\frac{im}{r} 
\left(
\frac{\widehat{\beta}}{\widetilde{r}}
+\frac{\partial \widehat{\beta}}{\partial \widetilde{r}}
\right)
+i \widehat{k}_x 
\frac{\partial \widehat{\tau}}{\partial \widetilde{r}}
\nonumber
\end{eqnarray}

The radial vorticity equation is then

\begin{small}
\begin{eqnarray}
\left(
\begin{array}{r}
\frac{i m}{\widetilde{r}}
\left(
-i \widehat{\omega}
+ \frac{i m \widehat{V}_{\theta}}{\widetilde{r}}
+i \widehat{k}_x \widehat{V}_x 
\right) 
\left(
\frac{\widehat{\beta}}{\widetilde{r}}
+\frac{\partial \widehat{\beta}}{\partial \widetilde{r}}
-
\frac{i m}{\widetilde{r}}
\widehat{\alpha}
+ i \widehat{k}_x \widehat{\kappa}
\right) 
\\
+
\frac{i m}{\widetilde{r}}
\frac{\partial \widehat{V}_x}{\partial \widetilde{r}} 
\left(
\frac{\partial \widehat{\kappa}}{\partial \widetilde{r}}
+
\frac{i m}{\widetilde{r}}
\widehat{\tau}
-
i \widehat{k}_x \widehat{\beta}
\right) 
\\
-i \widehat{k}_x
\left(
-i \widehat{\omega} 
+ \frac{i m \widehat{V}_{\theta}}{\widetilde{r}}
+i \widehat{k}_x \widehat{V}_x 
\right) 
\left(
\frac{i m}{\widetilde{r}} \widehat{\kappa}
+
i \widehat{k}_x \widehat{\alpha}
-
\frac{\partial \widehat{\tau}}{\partial \widetilde{r}}
\right) 
\\
-i \widehat{k}_x
\left(
\frac{\widehat{V}_{\theta}}{\widetilde{r}}
+\frac{\partial \widehat{V}_{\theta}}{\partial \widetilde{r}}
\right) 
\left(
\frac{\partial \widehat{\kappa}}{\partial \widetilde{r}}
+
\frac{i m}{\widetilde{r}}
\widehat{\tau}
-
i \widehat{k}_x \widehat{\beta}
\right) 
\end{array}
\right)
 = 
-\frac{i \widehat{k}_x}{\widehat{\overline{\rho}}} 
\frac{i m}{\widetilde{r}}
\widehat{p}
+i \widehat{k}_x \frac{i m}{\widehat{\overline{\rho}} \widetilde{r}} \widehat{p}
\nonumber
\\
\left(
\begin{array}{r}
\left(
-i \widehat{\omega}
+ \frac{i m \widehat{V}_{\theta}}{\widetilde{r}}
+i \widehat{k}_x \widehat{V}_x 
\right) 
\widehat{\omega}_r
\\
+
\left(
\frac{i m}{\widetilde{r}}
\frac{\partial \widehat{V}_x}{\partial \widetilde{r}} 
-i \widehat{k}_x
\left(
\frac{\widehat{V}_{\theta}}{\widetilde{r}}
+\frac{\partial \widehat{V}_{\theta}}{\partial \widetilde{r}}
\right) 
\right)
\left(
\frac{\partial \widehat{\kappa}}{\partial \widetilde{r}}
+
\frac{i m}{\widetilde{r}}
\widehat{\tau}
-
i \widehat{k}_x \widehat{\beta}
\right) 
\end{array}
\right)
 = 
0
\nonumber
\\
\left(
\begin{array}{r}
\left(
-i \widehat{\omega}
+ \frac{i m \widehat{V}_{\theta}}{\widetilde{r}}
+i \widehat{k}_x \widehat{V}_x 
\right) 
\widehat{\omega}_r
\\
+
\left(
\frac{i m}{\widetilde{r}}
\frac{\partial \widehat{V}_x}{\partial \widetilde{r}} 
-i \widehat{k}_x
\left(
\frac{\widehat{V}_{\theta}}{\widetilde{r}}
+\frac{\partial \widehat{V}_{\theta}}{\partial \widetilde{r}}
\right) 
\right)
\widehat{v}_r
\end{array}
\right)
 = 
0
\nonumber
\end{eqnarray}
\end{small}%

\subsubsection{Azimuthal vorticity}

The azimuthal vorticity is:

\begin{eqnarray}
\widehat{\omega}_{\theta}
&=&
i \widehat{k}_x 
\widehat{v}_r
- \frac{\partial \widehat{v}_x}{\partial \widetilde{r}}
\nonumber
\\
&=&
\left(
\begin{array}{r}
i \widehat{k}_x 
\left(
\frac{\partial \widehat{\kappa}}{\partial \widetilde{r}}
+
\frac{i m}{\widetilde{r}}
\widehat{\tau}
-
i \widehat{k}_x \widehat{\beta}
\right) 
\\
- \frac{\partial}{\partial \widetilde{r}}
\left(
\frac{\widehat{\beta}}{r}
+\frac{\partial \widehat{\beta}}{\partial r}
-
\frac{i m}{\widetilde{r}}
\widehat{\alpha}
+ i \widehat{k}_x \widehat{\kappa}
\right) 
\end{array}
\right)
\nonumber
\\
&=&
\left(
\begin{array}{r}
i \widehat{k}_x 
\left(
\frac{i m}{r}
\tau
-
i k_x \widehat{\beta}
\right) 
\\
- \frac{\partial}{\partial \widetilde{r}}
\left(
\frac{\widehat{\beta}}{r}
+\frac{\partial \widehat{\beta}}{\partial r}
-
\frac{i m}{\widetilde{r}}
\widehat{\alpha}
\right) 
\end{array}
\right)
\nonumber
\end{eqnarray}

The azimuthal vorticity equation is:

\begin{small}
\begin{eqnarray}
\left(
\begin{array}{r}
i \widehat{k}_x
\left(
-i \widehat{\omega} 
+ \frac{i m \widehat{V}_{\theta}}{\widetilde{r}}
+i \widehat{k}_x \widehat{V}_x 
\right) 
\widehat{v}_r 
-
i \widehat{k}_x
\frac{2 \widehat{V}_{\theta}}{\widetilde{r}} \widehat{v}_{\theta}
\\
-\frac{\partial}{\partial \widetilde{r}}
\left(
\left(
-i \widehat{\omega}
+ \frac{i m \widehat{V}_{\theta}}{\widetilde{r}}
+i \widehat{k}_x \widehat{V}_x 
\right) 
\widehat{v}_{x} 
\right)
-\frac{\partial}{\partial \widetilde{r}}
\left(
\frac{\partial \widehat{V}_x}{\partial \widetilde{r}} \widehat{v}_r
\right)
\end{array}
\right)
 = 
\left(
\begin{array}{r}
-
i \widehat{k}_x
\frac{1}{\widehat{\overline{\rho}}} \frac{\partial \widehat{p}}{\partial \widetilde{r}}
+
i \widehat{k}_x
\frac{\widehat{V}_{\theta}^2}{\widehat{\overline{\rho}} \widetilde{r} \widetilde{A}^2} 
\widehat{p}
\\
+\frac{\partial}{\partial \widetilde{r}}
\left(
\frac{i \widehat{k}_x}{\widehat{\overline{\rho}}} \widehat{p}
\right)
\end{array}
\right)
\nonumber
\\
\left(
\begin{array}{r}
i \widehat{k}_x
\left(
-i \widehat{\omega} 
+ \frac{i m \widehat{V}_{\theta}}{\widetilde{r}}
+i \widehat{k}_x \widehat{V}_x 
\right) 
\widehat{v}_r 
\\
-
i \widehat{k}_x
\frac{2 \widehat{V}_{\theta}}{\widetilde{r}} \widehat{v}_{\theta}
\\
-
\left(
\frac{i m}{\widetilde{r}}
\frac{\partial \widehat{V}_{\theta}
}{\partial \widetilde{r}}
- 
\frac{i m \widehat{V}_{\theta}}{\widetilde{r}^2}
+i \widehat{k}_x 
\frac{\partial
\widehat{V}_x 
}{\partial \widetilde{r}}
\right) 
\widehat{v}_{x} 
\\
-
\left(
-i \widehat{\omega}
+ \frac{i m \widehat{V}_{\theta}}{\widetilde{r}}
+i \widehat{k}_x \widehat{V}_x 
\right) 
\frac{\partial
\widehat{v}_{x} 
}{\partial \widetilde{r}}
\\
-
\frac{\partial^2 \widehat{V}_x}{\partial \widetilde{r}^2} \widehat{v}_r
-
\frac{\partial \widehat{V}_x}{\partial \widetilde{r}} 
\frac{\partial
\widehat{v}_r
}{\partial \widetilde{r}}
\end{array}
\right)
 = 
\left(
\begin{array}{r}
-
i \widehat{k}_x
\frac{1}{\widehat{\overline{\rho}}} \frac{\partial \widehat{p}}{\partial \widetilde{r}}
+
i \widehat{k}_x
\frac{\widehat{V}_{\theta}^2}{\widehat{\overline{\rho}} \widetilde{r} \widetilde{A}^2} 
\widehat{p}
\\
-\frac{\partial
\widehat{\overline{\rho}}
}{\partial \widetilde{r}}
\frac{i \widehat{k}_x}{
\widehat{\overline{\rho}}^2
} \widehat{p}
\\
+
\frac{i \widehat{k}_x}{\widehat{\overline{\rho}}} 
\frac{\partial
\widehat{p}
}{\partial \widetilde{r}}
\end{array}
\right)
\nonumber
\\
\left(
\begin{array}{r}
i \widehat{k}_x
\left(
-i \widehat{\omega} 
+ \frac{i m \widehat{V}_{\theta}}{\widetilde{r}}
+i \widehat{k}_x \widehat{V}_x 
\right) 
\widehat{v}_r 
\\
-
i \widehat{k}_x
\frac{2 \widehat{V}_{\theta}}{\widetilde{r}} \widehat{v}_{\theta}
\\
-
\left(
\frac{i m}{\widetilde{r}}
\frac{\partial \widehat{V}_{\theta}
}{\partial \widetilde{r}}
- 
\frac{i m \widehat{V}_{\theta}}{\widetilde{r}^2}
+i \widehat{k}_x 
\frac{\partial
\widehat{V}_x 
}{\partial \widetilde{r}}
\right) 
\widehat{v}_{x} 
\\
-
\left(
-i \widehat{\omega}
+ \frac{i m \widehat{V}_{\theta}}{\widetilde{r}}
+i \widehat{k}_x \widehat{V}_x 
\right) 
\frac{\partial
\widehat{v}_{x} 
}{\partial \widetilde{r}}
\\
-
\frac{\partial^2 \widehat{V}_x}{\partial \widetilde{r}^2} \widehat{v}_r
-
\frac{\partial \widehat{V}_x}{\partial \widetilde{r}} 
\frac{\partial
\widehat{v}_r
}{\partial \widetilde{r}}
\end{array}
\right)
 = 
\left(
\begin{array}{r}
i \widehat{k}_x
\left(
\frac{\widehat{V}_{\theta}^2}{\widehat{\overline{\rho}} \widetilde{r} \widetilde{A}^2} 
-\frac{\partial
\widehat{\overline{\rho}}
}{\partial \widetilde{r}}
\frac{1}{
\widehat{\overline{\rho}}^2
} 
\right)
\widehat{p}
\end{array}
\right)
\nonumber
\\
\left(
\begin{array}{r}
i \widehat{k}_x
\left(
-i \widehat{\omega} 
+ \frac{i m \widehat{V}_{\theta}}{\widetilde{r}}
+i \widehat{k}_x \widehat{V}_x 
\right) 
\widehat{v}_r 
\\
-
i \widehat{k}_x
\frac{2 \widehat{V}_{\theta}}{\widetilde{r}} \widehat{v}_{\theta}
\\
-
\left(
\frac{i m}{\widetilde{r}}
\frac{\partial \widehat{V}_{\theta}
}{\partial \widetilde{r}}
- 
\frac{i m \widehat{V}_{\theta}}{\widetilde{r}^2}
+i \widehat{k}_x 
\frac{\partial
\widehat{V}_x 
}{\partial \widetilde{r}}
\right) 
\widehat{v}_{x} 
\\
-
\left(
-i \widehat{\omega}
+ \frac{i m \widehat{V}_{\theta}}{\widetilde{r}}
+i \widehat{k}_x \widehat{V}_x 
\right) 
\frac{\partial
\widehat{v}_{x} 
}{\partial \widetilde{r}}
\\
-
\frac{\partial^2 \widehat{V}_x}{\partial \widetilde{r}^2} \widehat{v}_r
-
\frac{\partial \widehat{V}_x}{\partial \widetilde{r}} 
\frac{\partial
\widehat{v}_r
}{\partial \widetilde{r}}
\end{array}
\right)
 = 
\left(
\begin{array}{r}
i \widehat{k}_x
\left(
\frac{\widehat{\overline{\rho}}\widehat{V}_{\theta}^2}{\widetilde{r}} 
\frac{1}{\widehat{\overline{\rho}^2} \widetilde{A}^2} 
-\frac{\partial
\widehat{\overline{\rho}}
}{\partial \widetilde{r}}
\frac{1}{
\widehat{\overline{\rho}}^2
} 
\right)
\widehat{p}
\end{array}
\right)
\nonumber
\\
\left(
\begin{array}{r}
i \widehat{k}_x
\left(
-i \widehat{\omega} 
+ \frac{i m \widehat{V}_{\theta}}{\widetilde{r}}
+i \widehat{k}_x \widehat{V}_x 
\right) 
\widehat{v}_r 
\\
-
i \widehat{k}_x
\frac{2 \widehat{V}_{\theta}}{\widetilde{r}} \widehat{v}_{\theta}
\\
-
\left(
\frac{i m}{\widetilde{r}}
\frac{\partial \widehat{V}_{\theta}
}{\partial \widetilde{r}}
- 
\frac{i m \widehat{V}_{\theta}}{\widetilde{r}^2}
+i \widehat{k}_x 
\frac{\partial
\widehat{V}_x 
}{\partial \widetilde{r}}
\right) 
\widehat{v}_{x} 
\\
-
\left(
-i \widehat{\omega}
+ \frac{i m \widehat{V}_{\theta}}{\widetilde{r}}
+i \widehat{k}_x \widehat{V}_x 
\right) 
\frac{\partial
\widehat{v}_{x} 
}{\partial \widetilde{r}}
\\
-
\frac{\partial^2 \widehat{V}_x}{\partial \widetilde{r}^2} \widehat{v}_r
-
\frac{\partial \widehat{V}_x}{\partial \widetilde{r}} 
\frac{\partial
\widehat{v}_r
}{\partial \widetilde{r}}
\end{array}
\right)
 = 
\left(
\begin{array}{r}
i \widehat{k}_x
\left(
\frac{\partial \widehat{P}}{\partial \widetilde{r}} 
\frac{1}{\widehat{\overline{\rho}^2} \widetilde{A}^2} 
-\frac{\partial
\widehat{\overline{\rho}}
}{\partial \widetilde{r}}
\frac{1}{
\widehat{\overline{\rho}}^2
} 
\right)
\widehat{p}
\end{array}
\right)
\nonumber
\\
\left(
\begin{array}{r}
i \widehat{k}_x
\left(
-i \widehat{\omega} 
+ \frac{i m \widehat{V}_{\theta}}{\widetilde{r}}
+i \widehat{k}_x \widehat{V}_x 
\right) 
\widehat{v}_r 
\\
-
i \widehat{k}_x
\frac{2 \widehat{V}_{\theta}}{\widetilde{r}} \widehat{v}_{\theta}
\\
-
\left(
\frac{i m}{\widetilde{r}}
\frac{\partial \widehat{V}_{\theta}
}{\partial \widetilde{r}}
- 
\frac{i m \widehat{V}_{\theta}}{\widetilde{r}^2}
+i \widehat{k}_x 
\frac{\partial
\widehat{V}_x 
}{\partial \widetilde{r}}
\right) 
\widehat{v}_{x} 
\\
-
\left(
-i \widehat{\omega}
+ \frac{i m \widehat{V}_{\theta}}{\widetilde{r}}
+i \widehat{k}_x \widehat{V}_x 
\right) 
\frac{\partial
\widehat{v}_{x} 
}{\partial \widetilde{r}}
\\
-
\frac{\partial^2 \widehat{V}_x}{\partial \widetilde{r}^2} \widehat{v}_r
-
\frac{\partial \widehat{V}_x}{\partial \widetilde{r}} 
\frac{\partial
\widehat{v}_r
}{\partial \widetilde{r}}
\end{array}
\right)
 = 
\left(
\begin{array}{r}
i \widehat{k}_x
\left(
\widetilde{A}^2
\frac{\partial \widehat{\overline{\rho}}}{\partial \widetilde{r}} 
\frac{1}{\widehat{\overline{\rho}^2} \widetilde{A}^2} 
-\frac{\partial
\widehat{\overline{\rho}}
}{\partial \widetilde{r}}
\frac{1}{
\widehat{\overline{\rho}}^2
} 
\right)
\widehat{p}
\end{array}
\right)
\nonumber
\\
\left(
\begin{array}{r}
\left(
-i \widehat{\omega} 
+ \frac{i m \widehat{V}_{\theta}}{\widetilde{r}}
+i \widehat{k}_x \widehat{V}_x 
\right) 
\widehat{\omega}_{\theta} 
\\
-
i \widehat{k}_x
\frac{2 \widehat{V}_{\theta}}{\widetilde{r}} \widehat{v}_{\theta}
\\
-
\left(
\frac{i m}{\widetilde{r}}
\left(
\frac{\partial \widehat{V}_{\theta}
}{\partial \widetilde{r}}
- 
\frac{\widehat{V}_{\theta}}{\widetilde{r}}
\right)
+i \widehat{k}_x 
\left(
\frac{\partial
\widehat{V}_x 
}{\partial \widetilde{r}}
\right)
\right) 
\widehat{v}_{x} 
\\
-
\frac{\partial^2 \widehat{V}_x}{\partial \widetilde{r}^2} \widehat{v}_r
-
\frac{\partial \widehat{V}_x}{\partial \widetilde{r}} 
\frac{\partial
\widehat{v}_r
}{\partial \widetilde{r}}
\end{array}
\right)
 = 
0
\nonumber
\end{eqnarray}
\end{small}%
\end{document}


\end{document}

