\documentclass[12pt]{uthesis-v12}  %---> DO NOT ALTER THIS COMMAND
\begin{document} %---> %---> %---> %---> DO NOT ALTER THIS COMMAND

%--------+----------------------------------------------------------+
%        |  \title{}                                    (REQUIRED)  |
%        |  \author{}                                   (REQUIRED)  |
%        |                                                          |
%        |  See section 3.1 of "Read_Me_First_(v12).pdf"            |
%        |                                                          |
%        |  Also see section 2.2 of above "Read Me" file for the    |
%        |  proper use of the invisible tilde ("~") character when  |
%        |  entering a middle initial in the \author command.       |
%        +----------------------------------------------------------+

\title{A Game-Theoretic Approach to a General Equilibrium Model
       \protect\\ with Asymmetric Price Information and No Goods}

\author{Elmer J.~Fudd}

%--------+----------------------------------------------------------+
%        |  \copyrightpage{}                            (REQUIRED)  |
%        |                                                          |
%        |  See section 3.2 of "Read_Me_First_(v12).pdf"            |
%        |                                                          |
%        |  1) You must enter either "yes" or "no" in this          |
%        |      command.  Inputting "yes" produces a copyright      |
%        |      notification page as the second page and inputting  |
%        |      "no" produces a blank second page.                  |
%        |  2) Input to this command is case sensitive.             |
%        |  3) Default: the "yes" option.                           |
%        +----------------------------------------------------------+

\copyrightpage{yes}

%--------+----------------------------------------------------------+
%        |  \mydocument{}                               (REQUIRED)  |
%        |                                                          |
%        |  See section 3.3 of "Read_Me_First_(v12).pdf"            |
%        |                                                          |
%        |  1) Input to this command is limited to the following    |
%        |     three options: a) Dissertation                       |
%        |                    b) Thesis                             |
%        |                    c) Project                            |
%        |  2) Input to this command is case-sensitive.             |
%        +----------------------------------------------------------+

\mydocument{Dissertation}

%--------+----------------------------------------------------------+
%        |  \degree{}{}                                 (REQUIRED)  |
%        |                                                          |
%        |  See section 3.4 of "Read_Me_First_(v12).pdf"            |
%        |                                                          |
%        |  You need to provide two distinct inputs into this       |
%        |  command:                                                |
%        |     1) In the first set of braces you need to specify    |
%        |        the *exact* degree you will receive. Some         |
%        |        examples are: -) Masters of Arts                  |
%        |                      -) Masters of Science               |
%        |                      -) Doctor of Philosophy             |
%        |     2) In the second set of braces you need to state the |
%        |        *specific* discipline or area for that degree     |
%        |        (e.g., Economics, Education, Engineering, etc.).  |
%        |  Students should consult their advisor if they have any  |
%        |  questions about this information.                       |
%        +----------------------------------------------------------+

\degree{Masters of Arts}{Economics}

%--------+----------------------------------------------------------+
%        |  \conferraldate{}{}                          (REQUIRED)  |
%        |                                                          |
%        |  See section 3.5 of "Read_Me_First_(v12).pdf"            |
%        |                                                          |
%        |  In the two set of braces enter the month and then the   |
%        |  year your degree will be *conferred* by the university. |
%        +----------------------------------------------------------+

\conferraldate{May}{2012}

%--------+----------------------------------------------------------+
%        |  \advisor{}                                  (REQUIRED)  |
%        |                                                          |
%        |  See section 3.6.2 of "Read_Me_First_(v12).pdf"          |
%        |                                                          |
%        |  1) Also see section 2.2 of "Read_Me_First_(v12).pdf"    |
%        |     for the proper use of the invisible tilde ("~")      |
%        |     character when entering a middle initial or the      |
%        |     abbreviation of an academic title (e.g., Dr.) in     |
%        |     the \advisor{} command.                              |
%        |  2) Also see section 3.6.1. for consistent presentation  |
%        |     of title page signature lines.                       |
%        +----------------------------------------------------------+

\advisor{Dr.~Roy Hinkley}

%--------+----------------------------------------------------------+
%        |  Committee Member Signature Commands         (OPTIONAL)  |
%        |                                                          |
%        |  See section 3.6.3 of "Read_Me_First_(v12).pdf"          |
%        |                                                          |
%        |  1) Use the commands below to provide signature lines    |
%        |     for your "other" committee members;                  |
%        |        --> you must list your other committee members    |
%        |            in alphabetic order by last name              |
%        |        --> to do this, use the commands below in the     |
%        |            order presented below.                        |
%        |  2) You can choose to include none, some, or all of the  |
%        |     "XXXmember" commands below --- based on the number   |
%        |     committee members you have; simply delete (or        |
%        |     comment-out) any of the commands below that are not  |
%        |     needed.                                              |
%        |  3) Do not include the name of your committee chair or   |
%        |     the Graduate Dean in the commands listed below.      |
%        |     Their signature lines are generated by the           |
%        |     \advisor{} and \graduatedean{}{} commands.           |
%        |  4) You cannot use any of the commands below more than   |
%        |     once. (For details on this issue, see section 3.6.3  |
%        |     of "Read_Me_First_(v12).pdf".)                       |
%        |  5) Also see section 2.2 of "Read_Me_First_(v12).pdf"    |
%        |     for the proper use of the invisible tilde ("~")      |
%        |     character when entering a middle initial or the      |
%        |     abbreviation of an academic title (e.g., Dr.) in     |
%        |     the commands below.                                  |
%        |  6) See section 3.6.1. for consistent presentation of    |
%        |     title page signature lines.                          |
%        |                                                          |
%        |  I know I shouldn't have to say this, but enough         |
%        |  students over the years have made the same mistake      |
%        |  that I'm forced to state:                               |
%        |                                                          |
%        |      THE NAMES USED IN THE FOLLOWING COMMANDS ARE        |
%        |      SILLY NAMES I'VE USED AS EXAMPLES ONLY.  THEY       |
%        |      ARE NOT THE ACTUAL NAMES OF YOUR COMMITTEE          |
%        |      MEMBERS.  REPLACE THE SILLY NAMES BELOW WITH        |
%        |      THE NAMES OF YOUR ACTUAL COMMITTEE MEMBERS.         |
%        |                                                          |
%        +----------------------------------------------------------+

  \secondmember{Dr.~Anita Bath}
   \thirdmember{Dr.~Chris P.~Bacon}
  \fourthmember{Dr.~Adam Baum}
   \fifthmember{Dr.~Corey O.~Graff}
   \sixthmember{Dr.~Hugh Jass}
 \seventhmember{Dr.~Noah Lott}
  \eighthmember{Dr.~Jean Poole}

%--------+----------------------------------------------------------+
%        |  \graduatedean{}{}                           (REQUIRED)  |
%        |                                                          |
%        |  See section 3.6.4 of "Read_Me_First_(v12).pdf"          |
%        |                                                          |
%        |  1) THE NAME AND TITLE PROVIDED BELOW ARE THOSE OF THE   |
%        |     ACTUAL GRADUATE DEAN AT THE TIME THIS DOCUMENT WAS   |
%        |     CONSTRUCTED (January 2012). Contact the Graduate     |
%        |     College to determine whether this information is     |
%        |     correct at the time you submit your document.        |
%        |  2) Section 2.2 of "Read_Me_First_(v12).pdf" describes   |
%        |     the proper use of the invisible tilde ("~")          |
%        |     character when entering a middle initial or the      |
%        |     abbreviation of an academic title (e.g., Dr.) in     |
%        |     the \graduatedean{} command.                         |
%        |  3) See section 3.6.1. for consistent presentation of    |
%        |     title page signature lines.                          |
%        +----------------------------------------------------------+

\graduatedean{Dr.~Patricia R.~Komuniecki}{Dean}

%--------+----------------------------------------------------------+
%        |  \maketitle                                  (REQUIRED)  |
%        |                                                          |
%        |  See section 3.7 of "Read_Me_First_(v12).pdf"            |
%        |                                                          |
%        |  This is a required LaTeX command; to be brief, bad      |
%        |  things will happen if this command is not included      |
%        |  in your document at this particular location.           |
%        +----------------------------------------------------------+

\maketitle  %---->  ----->  ---->  ---->   DO NOT ALTER THIS COMMAND

%--------+----------------------------------------------------------+
%        |  Abstract Page Environment                   (REQUIRED)  |
%        |                                                          |
%        |  See section 3.8 of "Read_Me_First_(v12).pdf"            |
%        +----------------------------------------------------------+

\begin{abstractpage}
[Insert the abstract to your work here]
\end{abstractpage}

%--------+----------------------------------------------------------+
%        |  Dedication Page Environment                 (OPTIONAL)  |
%        |                                                          |
%        |  See section 3.9 of "Read_Me_First_(v12).pdf"            |
%        |                                                          |
%        |  If both a dedication page and an acknowledgements page  |
%        |  are included in the document, the dedication page must  |
%        |  proceed the acknowledgements page.                      |
%        +----------------------------------------------------------+

\begin{dedication}
\noindent [Insert your dedication here]
\end{dedication}

%--------+----------------------------------------------------------+
%        |  Acknowledgments Page Environment            (OPTIONAL)  |
%        |                                                          |
%        |  See section 3.10 of "Read_Me_First_(v12).pdf"           |
%        |                                                          |
%        |  If both a dedication page and an acknowledgements page  |
%        |  are included in the document, the dedication page must  |
%        |  proceed the acknowledgements page.                      |
%        +----------------------------------------------------------+

\begin{acknowledgments}
\noindent [Insert your acknowledgements here]
\end{acknowledgments}

%--------+----------------------------------------------------------+
%        |  \tableofcontents                            (REQUIRED)  |
%        |  \listoftables                            (CONDITIONAL)  |
%        |  \listoffigures                           (CONDITIONAL)  |
%        |                                                          |
%        |  See sections 3.11 & 3.12 of "Read_Me_First_(v12).pdf"   |
%        |                                                          |
%        |  1) You must include the \tableofcontents command in     |
%        |     your document: the UT Manual requires every          |
%        |     dissertation/thesis to have a detailed table of      |
%        |     contents.                                            |
%        |  2) Including the \listoftables and \listoffigures       |
%        |     commands is "conditional."  See sections 3.12 of     |
%        |     "Read_Me_First_(v12).pdf" for additional details.    |
%        +----------------------------------------------------------+

\tableofcontents  %----->  ----->  ---->  DO NOT ALTER THIS COMMAND
\listoftables \listoffigures

%--------+----------------------------------------------------------+
%        |  \captionformat{}                            (REQUIRED)  |
%        |                                                          |
%        |  See section 3.12.2 of "Read_Me_First_(v12).pdf"         |
%        |                                                          |
%        |  1) You are required to choose between the "hang" or     |
%        |     "align" option for this command.                     |
%        |  2) Input to this command is case sensitive.             |
%        |  3) Default: ``hang'' option.                            |
%        +----------------------------------------------------------+

\captionformat{hang}

%--------+----------------------------------------------------------+
%        |  List of Abbreviations Environment           (OPTIONAL)  |
%        |                                                          |
%        |  See section 3.13 of "Read_Me_First_(v12).pdf"           |
%        |                                                          |
%        |  1) This is an optional section; consult your advisor    |
%        |     to determine whether you need/want to include this   |
%        |     section in your document.                            |
%        |  2) If you do not want a List of Abbreviations simply    |
%        |     delete the material below (and these instructions).  |
%        |  3) If you do want a List of Abbreviations simply        |
%        |     replace the silly material below with the            |
%        |     information relevant to your document.               |
%        |     a. Within the "listofabbreviations" environment      |
%        |        below you must use a separate \abbreviation{}{}   |
%        |        command for each entry in your List of            |
%        |        Abbreviations.                                    |
%        |     b. As the examples below demonstrate, the            |
%        |        information within the first set of braces is     |
%        |        the abbreviation and the information in the       |
%        |        second set of braces is the definition of that    |
%        |        abbreviation.                                     |
%        +----------------------------------------------------------+

\begin{listofabbreviations}

 \abbreviation{ABBREV}{This is where you provide a brief
                       definition of the abbreviation ``ABBREV''}
     \abbreviation{BB}{B.B.~King}
    \abbreviation{BSE}{Bovine Spongiform Encephalopathy (Mad Cow
                       Disease)}
     \abbreviation{CB}{L.D. Caskey and J.D. Beazley, \emph{Attic
                       Vase Paintings in the Museum of Fine
                       Arts}, Boston (Oxford 1931--1963)}
    \abbreviation{GLE}{Gauss' law for electricity: $\nabla\cdot E
                       = \displaystyle\frac{\rho}{\varepsilon_0}
                       = 4\pi k \rho$}
    \abbreviation{HHS}{Department of Health and Human Services}
    \abbreviation{IaR}{I am root}

\end{listofabbreviations}

%--------+----------------------------------------------------------+
%        |  List of Symbols Environment                 (OPTIONAL)  |
%        |                                                          |
%        |  See section 3.14 of "Read_Me_First_(v12).pdf"           |
%        |                                                          |
%        |  1) This is an optional section; consult your advisor    |
%        |     to determine whether you need/want to include this   |
%        |     section in your document.                            |
%        |  2) If you do not want a List of Symbols simply delete   |
%        |     the material below (and these instructions).         |
%        |  3) If you do want a List of Symbols simply replace the  |
%        |     silly material below with the information relevant   |
%        |     to your document.                                    |
%        |       a. Within the "listofsymbols" environment below    |
%        |          you must use a separate \emblem{}{} command     |
%        |          for each entry in your List of Symbols.         |
%        |       b. As the examples below show, insert your symbol  |
%        |          within the first set of braces in the           |
%        |          \emblem{}{} command, and its definition within  |
%        |          the second set of braces.                       |
%        |       c. Use the \emblemskip command to insert a blank   |
%        |          line between different categories of symbols:   |
%        |          -) such additional spacing is required between  |
%        |             different categories of symbols;             |
%        |          -) see "Read_Me_First_(v12).pdf" for details.   |
%        +----------------------------------------------------------+

\begin{listofsymbols}

 \emblem{$\ddagger$}{the degree to which the flayrod has gone out of
                    skew on tredel}
\emblem{$\triangle$}{the ratio of the M2 monetary aggregate to the
                    Monetary Base}

       \emblemskip

  \emblem{$\alpha$}{angle of rotation around internal rotation axis}
   \emblem{$\beta$}{the number of people named ``Bob''}

       \emblemskip

         \emblem{Q}{Tobin's q; the ratio of the market value of
                    installed capital to the replacement cost of
                    capital}
         \emblem{Y}{Gross Domestic Product (adjusted for inflation)}

\end{listofsymbols}

%--------+----------------------------------------------------------+
%        |  Preface Environment                         (OPTIONAL)  |
%        |                                                          |
%        |  See section 3.15 of "Read_Me_First_(v12).pdf"           |
%        +----------------------------------------------------------+

\begin{preface}
[Insert your preface here]
\end{preface}

%XXXXXXXXXXXXXXXXXXXXXXXXXXXXXXXXXXXXXXXXXXXXXXXXXXXXXXXXXXXXXXXXXXXX
%XXXXXXXXXXXXXXXXXXXXXXXXXXXXXXXXXXXXXXXXXXXXXXXXXXXXXXXXXXXXXXXXXXXX
%XXXXXXXXXXXXXXXXXXXXXXXXXXXXXXXXXXXXXXXXXXXXXXXXXXXXXXXXXXXXXXXXXXXX
%XXXXXXXXXXXXXXXXXXXXXXXXXXXXXXXXXXXXXXXXXXXXXXXXXXXXXXXXXXXXXXXXXXXX

%--------+----------------------------------------------------------+
%        |  \makebody                                   (REQUIRED)  |
%        |                                                          |
%        |  See section 3.16 of "Read_Me_First_(v12).pdf"           |
%        |                                                          |
%        |  This is a *required* UThesis command; again, bad        |
%        |  things will happen if this command is not included in   |
%        |  your document at this particular location --- see the   |
%        |  file "Read_Me_First_(v12).pdf" for details.             |
%        +----------------------------------------------------------+

\makebody   %------->  ------->  ------->  DO NOT ALTER THIS COMMAND

%XXXXXXXXXXXXXXXXXXXXXXXXXXXXXXXXXXXXXXXXXXXXXXXXXXXXXXXXXXXXXXXXXXXX
%XXXXXXXXXXXXXXXXXXXXXXXXXXXXXXXXXXXXXXXXXXXXXXXXXXXXXXXXXXXXXXXXXXXX
%XXXXXXXXXXXXXXXXXXXXXXXXXXXXXXXXXXXXXXXXXXXXXXXXXXXXXXXXXXXXXXXXXXXX
%XXXXXXXXXXXXXXXXXXXXXXXXXXXXXXXXXXXXXXXXXXXXXXXXXXXXXXXXXXXXXXXXXXXX

%--------+----------------------------------------------------------+
%        |  \chapter{}                                  (REQUIRED)  |
%        |                                                          |
%        |  See section 3.17 of "Read_Me_First_(v12).pdf"           |
%        |                                                          |
%        |  For guidance on using the commands \chapter{},          |
%        |  \section{}, \subsection{}, \subsubsection{}, etc., see  |
%        |  Leslie Lamport's "LaTeX: A Document Preparation         |
%        |  System." Addison Wesley: Reading Massachusetts, 1985.   |
%        +----------------------------------------------------------+

\chapter{Insert the Heading to Chapter 1}

[ Insert your text to chapter 1 here.  A pretend example of a silly
table is provided below (i.e., Table~\ref{SILLY}).~]

%        +----------------------------------------------------------+
%        |                                                          |
%        |  IMPORTANT:                                              |
%        |                                                          |
%        |  Delete the following material.  I included this         |
%        |  material here as silly table to provide an example in   |
%        |  the List of Tables page for this template file.         |
%        |                                                          |
%        +----------------------------------------------------------+

     \vfill
     \begin{table}[ht]
     \caption{A silly glossary for research reports.\label{SILLY}}
     \begin{center}
     \begin{tabular}{p{2.75in}|p{2.5in}}
      \rule[-0.5em]{0pt}{1.75em} \bf When Professors write \ldots
      & \bf they REALLY mean \ldots \\ \hline\hline
      %--------+------------------------------------
      \rule[-0.5em]{0pt}{1.75em} Typical results are shown \ldots
      & The best results are shown \ldots \\ \hline
      %--------+------------------------------------
      \rule[-0.5em]{0pt}{1.75em} It is generally believed that
      \ldots & A couple of other guys think so too  \\ \hline
      %--------+------------------------------------
      \rule[-0.5em]{0pt}{1.75em}Thanks to Al K.~Seltzer for
      assistance and to I.P. Daly for valuable discussions &
      Seltzer did the work and Daly explained
      what it meant \\
     \end{tabular}
     \end{center}
     \end{table}
     \vfill

%XXXXXXXXXXXXXXXXXXXXXXXXXXXXXXXXXXXXXXXXXXXXXXXXXXXXXXXXXXXXXXXXXXXX
%XXXXXXXXXXXXXXXXXXXXXXXXXXXXXXXXXXXXXXXXXXXXXXXXXXXXXXXXXXXXXXXXXXXX
%XXXXXXXXXXXXXXXXXXXXXXXXXXXXXXXXXXXXXXXXXXXXXXXXXXXXXXXXXXXXXXXXXXXX
%XXXXXXXXXXXXXXXXXXXXXXXXXXXXXXXXXXXXXXXXXXXXXXXXXXXXXXXXXXXXXXXXXXXX

\chapter{Insert the Heading to Chapter 2}

[~Insert your text to chapter 2 here. A pretend example of a silly
figure is provided below (i.e., Figure~\ref{DOWD}).~]

%        +----------------------------------------------------------+
%        |                                                          |
%        |  IMPORTANT:                                              |
%        |                                                          |
%        |  Delete the following material.  I included this         |
%        |  material here as a silly figure to provide an example   |
%        |  in the List of Figures page for this template file.     |
%        |                                                          |
%        +----------------------------------------------------------+

      \vfill
      \begin{figure}[ht]
      \begin{center}
      \begin{tabular}{|cp{5.5in}c|} \hline
      & Let's pretend that instead of this text I provided a figure
      of myself smoking a most excellent cigar.  As long as we're
      pretending, let's assume that figure shows that I have a really
      big smile on my face. & \\
      & & \\[-0.75em]
      & OK, the truth is that I wanted to provide an example of a
      figure to contrast the text in the figure's caption to the text
      entered as a entry in the \emph{List of Figures}. This
      difference is controlled through the
      $\backslash$\texttt{caption[\,]\{\}} command. Appendix~B of
      ``Read Me First (v12).pdf'' provides a primer for this issue. &
      \\ \hline
      \end{tabular}
      \caption[Dr.~Dowd enjoying a wonderful cigar.] {Dr.~Dowd
              enjoying a wonderful cigar. From the smile on his face
              we are left to wonder if the cigar is an H.~Upmann
              \emph{Corona Imperial} or a Punch \emph{Rothchild}
              (with a double maduro wrapper, of course). \label{DOWD}
              }
      \end{center}
      \end{figure}
      \vfill

%XXXXXXXXXXXXXXXXXXXXXXXXXXXXXXXXXXXXXXXXXXXXXXXXXXXXXXXXXXXXXXXXXXXX
%XXXXXXXXXXXXXXXXXXXXXXXXXXXXXXXXXXXXXXXXXXXXXXXXXXXXXXXXXXXXXXXXXXXX
%XXXXXXXXXXXXXXXXXXXXXXXXXXXXXXXXXXXXXXXXXXXXXXXXXXXXXXXXXXXXXXXXXXXX
%XXXXXXXXXXXXXXXXXXXXXXXXXXXXXXXXXXXXXXXXXXXXXXXXXXXXXXXXXXXXXXXXXXXX

%--------+----------------------------------------------------------+
%        |  \myreferences{}                          (CONDITIONAL)  |
%        |                                                          |
%        |  See section 3.18 of "Read_Me_First_(v12).pdf"           |
%        |                                                          |
%        |  That section of the READ ME file describes two options  |
%        |  for listing the works cited in your document: first,    |
%        |  manually entering your list of references and, second,  |
%        |  using BibTeX to generate that list.                     |
%        |                                                          |
%        |  1) If you manually enter your reference list, first     |
%        |     include the \myreferences command, as illustrated    |
%        |     below.  Second, use the "referencelist" environment  |
%        |     described below.  For this, note that the UT Manual  |
%        |     requires double-spacing *between* references in      |
%        |     that list. However, it also states that *within*     |
%        |     individual references the spacing may be single- or  |
%        |     double-spaced.  Because of this, UThesis provides    |
%        |     two options for the "referencelist" environment:     |
%        |                                                          |
%        |            \begin{referencelist}{ENTER OPTION HERE}      |
%        |            \item ...                                     |
%        |            \end{referencelist}                           |
%        |                                                          |
%        |     a. Replacing "ENTER OPTION HERE" above with the      |
%        |        text "single" will generate a reference list      |
%        |        with single-spaced entries in that list but       |
%        |        double-spacing between those entries. An example  |
%        |        of this is provided below.                        |
%        |     b. Alternatively, using the "double" option will     |
%        |        generate a list with double-spaced entries and    |
%        |        double-spacing between entries. An example of     |
%        |        this is also provided below.                      |
%        |                                                          |
%        |     Note that input to these options is case sensitive   |
%        |     and the default setting is the "double" option.      |
%        |                                                          |
%        |  2) If you instead choose to use BibTeX to generate the  |
%        |     reference list, then you *cannot* include either     |
%        |     the "\myreferences" command or the "referencelist"   |
%        |     environment in your document. In this case you must  |
%        |                                                          |
%        |     a. delete the \myreferences command and the two      |
%        |        examples of the "referencelist"  environment      |
%        |        below;                                            |
%        |     b. then you must add and locate appropriately all    |
%        |        necessary BibTeX commands within your document    |
%        |        (e.g., \bibliographystyle{}, \citationstyle{},    |
%        |        \bibliography{}, etc.).                           |
%        +----------------------------------------------------------+

\myreferences

                                          %%%%%%%%%%%%%%%%%%%%%%
                                          %% DELETE THIS WHEN
 \hfill \fbox{First example: references   %% CONSTRUCTING YOUR
 generated by the ``single'' option}      %% OWN REFERENCE LIST
                                          %%%%%%%%%%%%%%%%%%%%%%

\begin{referencelist}{single}
\item \textbf{Friedman, Milton}, ``The Role of Monetary Policy,''
      \emph{American Economic Review}, March 1968, 58(1), 1--17.

\item \textbf{Keynes, John Maynard}, \emph{The General Theory of
      Employment, Interest, and Money}, New York: Harcourt Brace
      Jovanovic, 1936.

\item \textbf{Smith, Adam}, \emph{An Inquiry into the Nature and
      Causes of the Wealth of Nations}, Edwin Cannan, ed.,
      London: Methuen \& Co., Ltd. 1904.

\item \textbf{Tobin, James}, ``A Dynamic Aggregative Model,''
      \emph{Journal of Political Economy,} April 1955, 63(2),
      103--115.
\end{referencelist}

                                          %%%%%%%%%%%%%%%%%%%%%%
                                          %% DELETE THIS WHEN
 \hfill \fbox{Second example: references  %% CONSTRUCTING YOUR
 generated by the ``double'' option}      %% OWN REFERENCE LIST
                                          %%%%%%%%%%%%%%%%%%%%%%

\begin{referencelist}{double}
\item \textbf{Friedman, Milton}, ``The Role of Monetary Policy,''
      \emph{American Economic Review}, March 1968, 58(1), 1--17.

\item \textbf{Keynes, John Maynard}, \emph{The General Theory of
      Employment, Interest, and Money}, New York: Harcourt Brace
      Jovanovic, 1936.

\item \textbf{Smith, Adam}, \emph{An Inquiry into the Nature and
      Causes of the Wealth of Nations}, Edwin Cannan, ed.,
      London: Methuen \& Co., Ltd. 1904.

\item \textbf{Tobin, James}, ``A Dynamic Aggregative Model,''
      \emph{Journal of Political Economy,} April 1955, 63(2),
      103--115.
\end{referencelist}

%XXXXXXXXXXXXXXXXXXXXXXXXXXXXXXXXXXXXXXXXXXXXXXXXXXXXXXXXXXXXXXXXXXXX
%XXXXXXXXXXXXXXXXXXXXXXXXXXXXXXXXXXXXXXXXXXXXXXXXXXXXXXXXXXXXXXXXXXXX
%XXXXXXXXXXXXXXXXXXXXXXXXXXXXXXXXXXXXXXXXXXXXXXXXXXXXXXXXXXXXXXXXXXXX
%XXXXXXXXXXXXXXXXXXXXXXXXXXXXXXXXXXXXXXXXXXXXXXXXXXXXXXXXXXXXXXXXXXXX

%--------+----------------------------------------------------------+
%        |  \appendix                                  (IF NEEDED)  |
%        |                                                          |
%        |  See section 3.19 of "Read_Me_First_(v12).pdf"           |
%        |                                                          |
%        |  1) When you include the \appendix command a subsequent  |
%        |     \chapter{} command will not generate a chapter but   |
%        |     an appendix section.                                 |
%        |                                                          |
%        |  2) As is illustrated below, to generate a second or     |
%        |     third appendix you simply have to include            |
%        |     additional \chapter{} commands (i.e., you DO NOT     |
%        |     have to repeat the use of the \appendix command).    |
%        +----------------------------------------------------------+

\appendix

\chapter{Insert the Heading to Appendix A}

[Insert text to Appendix A (if appendix is needed)]

%====================================================================

\chapter{Insert the Heading to Appendix B}

[Insert text to Appendix B (if appendix is needed)]

%XXXXXXXXXXXXXXXXXXXXXXXXXXXXXXXXXXXXXXXXXXXXXXXXXXXXXXXXXXXXXXXXXXXX
%XXXXXXXXXXXXXXXXXXXXXXXXXXXXXXXXXXXXXXXXXXXXXXXXXXXXXXXXXXXXXXXXXXXX
%XXXXXXXXXXXXXXXXXXXXXXXXXXXXXXXXXXXXXXXXXXXXXXXXXXXXXXXXXXXXXXXXXXXX
%XXXXXXXXXXXXXXXXXXXXXXXXXXXXXXXXXXXXXXXXXXXXXXXXXXXXXXXXXXXXXXXXXXXX

%--------+----------------------------------------------------------+
%        |  \end{document}                              (REQUIRED)  |
%        |                                                          |
%        |  Details (if you can call them "details") are provided   |
%        |  in section 3.20 of "Read_Me_First_(v12).pdf"            |
%        +----------------------------------------------------------+

\end{document} %---> ---> ---> --->   DO NOT ALTER THIS COMMAND

%XXXXXXXXXXXXXXXXXXXXXXXXXXXXXXXXXXXXXXXXXXXXXXXXXXXXXXXXXXXXXXXXXXXX
%XXXXXXXXXXXXXXXXXXXXXXXXXXXXXXXXXXXXXXXXXXXXXXXXXXXXXXXXXXXXXXXXXXXX
%XXXXXXXXXXXXXXXXXXXXXXXXXXXXXXXXXXXXXXXXXXXXXXXXXXXXXXXXXXXXXXXXXXXX
%XXXXXXXXXXXXXXXXXXXXXXXXXXXXXXXXXXXXXXXXXXXXXXXXXXXXXXXXXXXXXXXXXXXX
